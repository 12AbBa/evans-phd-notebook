\chapter{Whittaker models}
\section{Local uniqueness}
Let $F$ be a non-Archimedean local field,
$G = \GL_2(F)$,
$N = \left\{ \begin{bmatrix}
  1 & x \\ 0 & 1
\end{bmatrix} \right\}$,
and $\psi_F$ a fixed additive character.
Since $N \cong F$, we can think of $\psi_F$ as a one-dimensional representation of $N$.

Recall that we have the \alert{induced representation}
\[
  \Ind^G_N \psi_F
  =
  \left\{ W \colon G \to \CC \mid
    W\left( \begin{bmatrix} 1 & x \\ 0 & 1 \end{bmatrix} \cdot g\right)
  = \psi_F(x) W(g) \; \forall x \in F, g \in \GL_2(F)
  \right\}.
\]
\begin{definition}
  A \alert{Whittaker model} $\WW$ of $(V, \pi)$
  is a subrepresentation of $\Ind^G_N \psi_F$ isomorphic to $\WW$.
\end{definition}

\begin{theorem}
  [Local uniqueness]
  An irreducible admissible representation of $\GL_2(F)$
  has at most one Whittaker model.
\end{theorem}
\begin{proof}
  By Frobenius reciprocity reduce to fact about functional.
\end{proof}

\section{Global uniqueness}
If $K$ is a global field,
a \alert{Whittaker model} is a subspace of
\[
  \left\{ W \colon G \to \CC \mid
    W\left( \begin{bmatrix} 1 & x \\ 0 & 1 \end{bmatrix} \cdot g\right)
    = \psi_{\AA/K}(x) W(g) \; \forall x \in K, g \in \GL_2(K)
  \right\}
\]
closed under translation by $G$.
We also assume the functions in $\WW$ are smooth, $K$-finite,
and of moderate growth.

\begin{theorem}
  [Global uniqueness]
  An irreducible admissible representation $\pi$ of $G = \GL_2(\AA)$
  has a Whittaker model if and only if each $\pi_v$ has a Whittaker model.
  If so it is unique and equals the sums of
  $g \mapsto \prod_v W_v(g_v)$,
  where $W_v \in \WW_v$ and for almost all $v$ we have $W_v$ equal
  to the spherical element of $\WW_v$, normalized to equal $1$ on $K_v = \GL_2(\OO_v)$.
\end{theorem}

\section{Automorphic cuspidal representations have Whittaker models}
\begin{theorem}
  [Whittaker model for an automorphic cuspidal representation]
  Let $V$ be an automorphic \emph{cuspidal} representation.
  For each $\phi \in V$ define
  \[ W_\phi(g) = \int_{\AA/K} \phi\left( \begin{bmatrix}
      1 & x \\ & 1
  \end{bmatrix} g\right) \psi_K(-x) \; dx.
  \]
  The space $\WW = \left\{ W_\phi \mid \phi \in V \right\}$ is a Whittaker model.
  It satisfies Fourier expansion
  \[ \phi(g) = \sum_{\alpha \in K^\times}
    W_\phi \left( \begin{bmatrix} \alpha \\ & 1 \end{bmatrix} g\right).  \]
\end{theorem}
The multiplicity one theorem follows from this.

\section{The zeta integral}
Given an automorphic cuspidal $\phi$
and a central quasicharacter $\xi$ we define
\begin{align*}
  Z(s, \phi, \xi)
  &\defeq \int_{\AA^\times/K^\times}
  \phi\left( \begin{bmatrix} y \\ & 1 \end{bmatrix} \right)
  |y|^{s-\half} \xi(y) \; d^\times y \\
  &= \sum_{\alpha \in K^\times} \int_{\AA^\times/K^\times}
  W_\phi\left( \begin{bmatrix} y \\ & 1 \end{bmatrix} \right)
  |y|^{s-\half} \xi(y) \; d^\times y \\
  &= \int_{\AA^\times}
  W_\phi\left( \begin{bmatrix} y \\ & 1 \end{bmatrix} \right)
  |y|^{s-\half} \xi(y) \; d^\times y.
\end{align*}
If $\phi = \bigotimes_v \phi_v$ and $W_\phi = \bigotimes_v W_v$
then $Z(s, \phi, \xi) = \prod_v Z_v(s, W_v, \xi_v)$.

In general, if $F$ is a non-Archimedean local field
then most irreps of $F$ are isomorphic to $\pi(\chi_1, \chi_2)$,
the so-called principle series.
Then $\alpha_1 = \chi_1(\varpi)$ and $\alpha_2 = \chi_2(\varpi)$
are called the Satake parameters.

\begin{theorem}
  If $v$ is unramified then
  \[
    W_v \left( \begin{bmatrix}
      y \\ & 1
  \end{bmatrix}\right)
  =
  \begin{cases}
    q^{-m/2} \cdot \frac{\alpha_1^{m+1}-\alpha_2^{m+1}}{\alpha_1-\alpha_2} & m \ge 0 \\
    0 & m < 0
  \end{cases}
\]
where $m = \ord_v(y)$.
Consequently, a direct calculation shows
\[ Z_v(s, W_v, \xi_v) = L_v(s, \pi_v, \xi_v)
  = \left( 1 - \alpha_1 \xi(\varpi) q^{-s} \right)\inv
  \left( 1 - \alpha_2 \xi(\varpi) q^{-s} \right)\inv. \]
\end{theorem}
Since the Satake parameters of the contragredient $\wh{\pi_v}$
are exactly $\alpha_1\inv$, $\alpha_2\inv$, it follows that
\[ L_v(s, \wh \pi_v, \xi_v\inv) = L_v(s, \pi_v, \omega_v\inv \xi_v\inv) \]

Using this, we get a functional equation
\[ L(s, \pi, \xi)=  L(1-s, \wh\pi, \xi\inv) \cdot \text{gamma crap}. \]
Dubious: discuss
