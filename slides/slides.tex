\documentclass[11pt]{beamer}
\usepackage[nosetup]{evan}
\usepackage{derivative}
\usepackage{mathrsfs}
\usepackage{booktabs}
\usepackage{multirow}
\usetheme{Malaysia}

\usepackage{fontspec}
\directlua{luaotfload.add_fallback
  ("evans_fallbacks",
    {
      "NotoColorEmoji:mode=harf;",
      "Source Han Sans TW:style=Regular;",
      "Noto Serif CJK SC:style=Regular;",
    }
  )}
\setmainfont{lmroman10-regular}[
  BoldFont=lmroman10-bold,
  ItalicFont=lmroman10-italic,
  BoldItalicFont=lmroman10-bolditalic,
  SlantedFont=lmromanslant10-regular,
  BoldSlantedFont=lmromanslant10-bold,
  SmallCapsFont=lmromancaps10-regular,
  RawFeature={fallback=evans_fallbacks}
]
\setsansfont{lmsans10-regular}[
  BoldFont=lmsans10-bold,
  ItalicFont=lmsans10-oblique,
  BoldItalicFont=lmsans10-boldoblique,
  RawFeature={fallback=evans_fallbacks}
]

\setbeamercovered{transparent}

\DeclareMathOperator{\BC}{BC}
\DeclareMathOperator{\Int}{Int}
\DeclareMathOperator{\Orb}{Orb}
\DeclareMathOperator{\Sh}{Sh}
\DeclareMathOperator{\U}{U}
\newcommand{\HH}{\mathcal{H}}
\newcommand{\VV}{\mathbb{V}}
\newcommand{\G}{\mathrm{G}}
\renewcommand{\H}{\mathrm{H}}
\renewcommand{\OO}{O}
\newcommand{\guv}{{(\gamma, \uu, \vv^\top)}}
\newcommand{\rs}{_{\text{rs}}}
\newcommand{\uu}{\mathbf{u}}
\newcommand{\vv}{\mathbf{v}}
\newcommand{\oneV}{\mathbf{1}_{\OO_F^n \times (\OO_F^n)^\vee}}

\title[Semi-Lie AFL for Hecke]{Semi-Lie Arithmetic Fundamental Lemma for full spherical Hecke Algebra}
\subtitle{Mass Tech Learning Seminar}
\author{Evan Chen}
\date{4 November 2024}

\begin{document}

\begin{frame}
  \maketitle
\end{frame}

\begin{frame}
  \frametitle{Notation}
  \begin{itemize}
  \ii $E/F$ a quadratic field extension of local fields
  \ii $q$ residue characteristic of $F$.
  \ii $\VV_n^-$ split $E/F$-Hermitian space of dimension $n$
  \ii $\VV_n^+$ non-split $E/F$-Hermitian space of dimension $n$
  \ii Symmetric space:
  \[ S_n(F) \coloneqq \left\{ \gamma \in \GL_n(E) \mid \gamma\bar\gamma = 1_n \right\} . \]
  \end{itemize}
  Hecke algebras:
  \begin{itemize}
  \ii $\HH(\U(\VV_n^-)) \coloneqq \QQ[K \backslash \U(\VV_n^-) \slash K]$
  is the Hecke algebra of compactly supported $K$-bi-invariant functions,
  where $K \coloneqq \U(\VV_n^-) \cap \GL_n(\OO_E)$ is hyperspecial maximal compact subgroup.
  \ii $\HH(\GL_n(E)) \coloneqq \QQ[K' \backslash \GL_n(E) \slash K']$, $K' \coloneqq \GL_n(\OO_E)$.
  \ii $\HH(S_n(F)) \coloneqq \mathcal C^\infty_{\text{c}} (S_n(F))^{K'}$,
  is an $\HH(\GL_n(E))$-module.
  \end{itemize}
\end{frame}


\begin{frame}
  \frametitle{Overview}
  \centering
  \begin{tabular}{lp{12em}p{8em}}
    \toprule
    \textbf{Conjecture} & \textbf{Lemma} & \textbf{Generalization} \\
    \midrule
    GGP & Fundamental lemma & Leslie (2023) \\ & (Jacquet-Rallis 2011) \\
    \midrule
    \multirow{2}{*}[-1em]{Arith GGP} & AFL for $\U(n) \times \U(n+1)$ (Zhang 2012) & Li-Rapoport-Zhang (2024) \\ \cline{2-3}
    & $\iff$ AFL for $\U(n) \times \U(n)$ (Liu 2021) & 🤔 \\
    \bottomrule
  \end{tabular}
  \begin{block}{Table of contents for talk}
    \begin{itemize}
    \ii Discuss first, second, third column, each from top to bottom.
    \ii Things in first column and first row will be cursory historical overview,
      not defined or made precise at all.
    \end{itemize}
  \end{block}
\end{frame}

\begin{frame}
  \frametitle{What GGP looks like, extremely roughly}
  \begin{itemize}
  \ii Started with Waldspurger 1985 formula
    relating automorphic period to $L(1/2)$.
  \ii Generalized by Gross-Prasad in 1992 and 1994,
  \ii Generalized further to Gan-Gross-Prossad conjecture (GGP).
  \end{itemize}
  %Let $k$ be a number field.
  \begin{exampleblock}{Global GGP conjecture, extremely roughly}
    Let $\H \subset \G$ be ``spherical'' pair of reductive groups,
    $\pi$ a cuspidal automorphic representation of $\G$.
    Define the period integral
    \[ \mathscr P(\phi) = \int_{\H(k) \backslash \H(\mathbb A_k)} \phi(h) \odif h \]
    for automorphic forms $\phi \in \pi$.
    Then the following are equivalent:
    \begin{enumerate}
      \ii $\mathscr P(\phi) \neq 0$ for some $\phi \in \pi$.
      \ii $\Hom_{\H(\mathbb A_k)}(\pi, \CC) \neq 0$ and $L(\half, \pi , R) \neq 0$
      for certain $R$.
    \end{enumerate}
  \end{exampleblock}
\end{frame}

\begin{frame}
  \frametitle{Jacquet-Rallis fundamental lemma (2011)}
  \begin{itemize}
  \ii $(\VV_n^-)^\flat$ codimension one subspace in $\VV_n^-$.
  \ii $G'^\flat \coloneqq \GL_{n-1}(E)$, $G' \coloneqq \GL_n(E)$,
    $G^\flat \coloneqq \U((\VV_n^-)^\flat)(F)$, $G \coloneqq \U(\VV_n^-)(F)$.
  \ii $K'^\flat$, $K'$, $K^\flat$, $K$ maximal hyperspecial compact subgroups.
  \end{itemize}
  \begin{block}{Fundamental lemma}
    For certain ``matching''
    $\gamma \in G'^\flat \times G'$ and $g \in G^\flat \times G$
    we get
   \[ \Orb(\gamma, \mathbf{1}_{K'^\flat \times K'}) = \omega(\gamma) \Orb(g, \mathbf{1}_{K^\flat \times K}) \]
  \end{block}
  Now proved completely; a local proof was given by Beuzart-Plessis (2021)
  while a global proof was given for large characteristic by W.\ Zhang (2021).
\end{frame}

\begin{frame}
  \frametitle{What Arithmetic GGP looks like, extremely roughly}
  \begin{itemize}
  \ii Started with Gross-Zagier (1986), relating height of Heegner point on modular curve to $L'(s)$ at $s = 1$
  \ii Yuan-Zhang-Zhang (2013) proved generalization to Shimura curve over totally real field.
  \ii Arith.~GGP further generalizes to higher dimensional Shimura variety.
  \end{itemize}
  %Beginning from a spherical pair $(\H,\G)$,
  %upgrade to Shimura data $(\H, X_\H) \to (\G, X_\G)$.
  \begin{exampleblock}{Arithmetic GGP conjecture, extremely roughly}
    Let $\pi$ be a tempered cuspidal automorphic representation of $\G(\mathbb A_k)$
    appearing in the cohomology $H^\bullet(\Sh(\G))$.
    The following are equivalent:
    \begin{enumerate}
      \ii A certain height pairing
      \[ \mathscr{P}_{\Sh(\H)} \colon \opname{Ch}^{n-1}(\Sh(\G))_0 \to \CC \]
      does not vanish on $\pi_f$-isotypic component.
      \ii $\Hom_{\H(\mathbb{A}_k)}(\pi_f, \CC) \neq 0$
      and $L'(\half, \pi, R) \neq 0$ for certain $R$.
    \end{enumerate}
  \end{exampleblock}
\end{frame}

\begin{frame}
  \frametitle{Arithmetic fundamental lemma (W.\ Zhang 2012)}
  Analogous statement to Jacquet-Rallis fundamental lemma.
  For simplicity, stating the inhomogeneous version (the full version is more general).
  \begin{theorem}
  [Inhomogeneous AFL]
  For matching regular semisimple elements
  $g \in \U(\VV_n^+)\rs \longleftrightarrow \gamma \in S_n(F)\rs$:
  \[
    \Int\left( (1,g), \mathbf{1}_{K'^\flat} \otimes \mathbf{1}_{K'} \right) \log q
    = -\omega(\gamma) \left. \pdv{}{s} \right\rvert_{s=0} \Orb(\gamma, \mathbf{1}_K, s).
  \]
  \end{theorem}
  \begin{itemize}
  \ii I won't talk about $\Int$ or transfer factor $\omega(\gamma)$ in this talk.
  \ii I'll tell you what matching and orbital look like.
  \end{itemize}
\end{frame}

\begin{frame}
  \frametitle{Regular semisimple definition}
  \[ \begin{bmatrix} A & \uu \\ \vv^\top & d \end{bmatrix} \in \GL_n(E) \]
  is \textbf{regular semi-simple} if
  $\left[ \vv^\top A^{i+j-2} \uu \right]_{i,j=1}^{n-1}$
  is non-singular.
  \begin{itemize}
    % \ii The matrix $A$ is $(n-1) \times (n-1)$.
    \ii Write $\GL_n(E)\rs$ to denote semisimple elements.
    \ii Equivalent to requiring that, under the action of conjugation by $\GL_{n-1}(E)$:
    the matrix has trivial stabilizer; and
    the $\GL_{n-1}(\ol E)$-orbit is a Zariski-closed subset of $\GL_n(\ol E)$.
  \end{itemize}
  Regular semisimple elements are acted on by conjugation under $\GL_{n-1}(E)$
  (viewed as a subset of $\GL_n(E)$ by $g \mapsto \begin{bmatrix} g & 0 \\ & 1 \end{bmatrix}$.)
\end{frame}

\begin{frame}
  \frametitle{Matching definition}
  %Orbits are classified by the following invariants:
  %\begin{itemize}
  %  \ii The matrices $A_1$ and $A_2$ have the same characteristic polynomial;
  %  \ii We have $\vv_1^\top A_1^i \uu_1 = \vv_2^\top A_2^i \uu_2$
  %  for every $i = 0, 1, \dots, n-2$; and
  %  \ii We have $d_1 = d_2$.
  %\end{itemize}
  We say $\gamma \in S_n(F)\rs$ \textbf{matches} the element $g \in \U(\VV_n^\pm)\rs$
  if $g$ is conjugate to $\gamma$ by an element of $\GL_{n-1}(E)$.
  Write this as
  \[ g \in \U(\VV_n^\pm)\rs \longleftrightarrow \gamma \in S_n(F)\rs. \]
  \begin{block}{Matching}
    \[ [S_n(F)]\rs \xrightarrow{\sim} [\U(\VV_n^-)]\rs \amalg [\U(\VV_n^+)]\rs. \]
    For $\gamma \in S_n(F)$, to tell which case we're in, let
    \[ \Delta \coloneqq \det \left[ \vv^\top A^{i+j-2} \uu \right]_{i,j=1}^{n-1} \neq 0. \]
    We get $-$ if $v(\Delta)$ is even
    and $+$ if $v(\Delta)$ is odd.
  \end{block}
\end{frame}

\begin{frame}
  \frametitle{Weighted orbital integral definition}
  For $\gamma \in S_n(F)$, $\phi \in \HH(S_n(F))$, and $s \in \CC$:
  \[ \Orb(\gamma, \phi, s) \coloneqq
    \int_{h \in \GL_{n-1}(F)} \phi(h\inv \gamma h) (-1)^{v(\det h)}
    \left\lvert \det(h) \right\rvert_F^{-s} \odif h. \]
  For comparison, the unweighted orbital for $f \in \HH(\U(\VV_n^-))$ is:
  \[ \Orb^{\U(\VV_n^-)}(g, f) \coloneqq \int_{\U(\VV_n^-)} f(x^{-1}gx) \odif x \]
  \begin{theorem}
    [Relative fundamental lemma; Leslie 2023]
    Suppose $\phi$ and $f$ are related by \emph{base change}, and $\gamma \longleftrightarrow g$, then:
    \[
      \omega \Orb(\phi, \gamma, 0)
      = \begin{cases}
        0 & v(\Delta) \text{ odd} \\
        \Orb^{\U(\VV_n^-)}(g, f) & v(\Delta) \text{ even}.
      \end{cases}
    \]
  \end{theorem}
\end{frame}

\begin{frame}
  \frametitle{Semi-Lie version of AFL by Yifeng Liu (2021)}
  \begin{theorem}
    [AFL, semi-Lie version]
    If $(g, u) \in (\U(\VV_n^+) \times \VV_n^+)\rs \longleftrightarrow
      (\gamma, \uu, \vv^\top) \in (S_n(F) \times \OO_F^n \times (\OO_F^n)^\vee)\rs$
    then
    \begin{align*}
      &\phantom= \Int\left( (g,u), \mathbf{1}_{K'} \right) \log q \\
      &= -\omega\guv \left. \pdv{}{s} \right\rvert_{s=0}
      \Orb(\guv, \mathbf{1}_K \otimes \oneV, s).
    \end{align*}
  \end{theorem}
  Equivalent to Zhang's proposed Bessel-period AFL for $\U(n) \times \U(n+1)$;
  in fact the interplay is used in the induction proof of both.
\end{frame}

\begin{frame}
  \frametitle{Matching (analogous)}
  Matching is defined in the same way via
  $(g,u) \mapsto \begin{bmatrix} g & u \\ u^\ast & 0 \end{bmatrix} \in \GL_{n+1}(E)$
  and $\guv \mapsto \begin{bmatrix} \gamma & \uu \\ \vv^\top & 0 \end{bmatrix} \in \GL_{n+1}(E)$.

  \begin{block}{Matching in semi-Lie case}
  \[ [S_n(F) \times \OO_F^n \times (\OO_F^n)^\vee]\rs \xrightarrow{\sim} [\U(\VV_n^-) \times \VV_n^-]\rs \amalg [\U(\VV_n^+) \times \VV_n^+]\rs. \]
  To see which one, define
  \[ \Delta \coloneqq \det \left[ \vv^\top \gamma^{i+j-2} \uu \right]_{i,j=1}^n \neq 0. \]
  We get $-$ if $v(\Delta)$ is even
  and $+$ if $v(\Delta)$ is odd.
  \end{block}
\end{frame}

\begin{frame}
  \frametitle{Weighted orbital integral (definition)}
  For $\guv \in S_n(F) \times \OO_F^n \times (\OO_F^n)^\vee$,,
  $\phi \in \HH(S_n(F))$, and $s \in \CC$,
  \begin{align*}
    & \Orb((\gamma, \uu, \vv^\top), \phi \otimes \oneV, s) \\
    &\coloneqq \int_{h \in H} \phi(h\inv \gamma h) \oneV(h \uu, \vv^\top h^{-1})
    \eta(h) \left\lvert \det(h) \right\rvert_F^{-s} \odif h.
  \end{align*}
  Unweighted version, where $f \in \HH(\U(\VV_n^-))$,
  and $\Lambda_n$ is a self-dual lattice:
  \[
    \Orb^{\U(\VV_n^-) \times \VV_n^-}\left( (g,u), f \otimes \mathbf{1}_{\Lambda_n} \right)
    \coloneqq \int_{\U(\VV_n^-)} f(x\inv g x) \mathbf{1}_{\Lambda_n}(x^{-1} u) \odif x.
  \]
  \begin{exampleblock}{Conjecture}
    If $\phi$ and $f$ are related by \emph{base change},
    $\guv \longleftrightarrow (g,u)$, then:
    \[ \omega\guv \Orb(\dots)
      =
      \begin{cases}
        0 & v(\Delta) \text{ odd} \\
        \Orb^{\U(\VV_n^-) \times \VV_n^-}\left( (g,u), f \otimes \mathbf{1}_{\Lambda_n} \right)
        & v(\Delta) \text{ even}.
      \end{cases}
    \]
  \end{exampleblock}
\end{frame}


\begin{frame}
  \frametitle{Leslie's generalization of the fundamental lemma}
  As before $G'^\flat \coloneqq \GL_{n-1}(E)$, $G' \coloneqq \GL_n(E)$,
  $G^\flat \coloneqq \U((\VV_n^-)^\flat)(F)$, $G \coloneqq \U(\VV_n^-)(F)$.
  \begin{theorem}[Leslie 2023]
    Suppose \alert{$\varphi'$ and $\varphi$ are related by base change}; then still
    for certain ``matching'' $\gamma \in G'^\flat \times G'$ and $g \in G^\flat \times G$:
    \[ \Orb(\gamma, \alert{\varphi'}) = \omega(\gamma) \Orb(g, \alert{\varphi}). \]
  \end{theorem}
  The original Jacquet-Rallis fundamental lemma is the special case
  \begin{align*}
    \varphi' = \mathbf{1}_{K'^\flat \times K'} &\in \HH(\GL_{n-1}(E)) \otimes \HH(\GL_n(E)) \\
    \varphi = \mathbf{1}_{K^\flat \times K} &\in \HH(\U((\VV_n^-)^\flat)) \otimes \HH(\U(\VV_n^-)).
  \end{align*}
\end{frame}

\begin{frame}
  \frametitle{AFL conjectured for the full spherical Hecke algebra}
  \begin{exampleblock}{Conjecture (Li-Rapoport-Zhang 2024)}
    Let \alert{$f \in \HH(\U(\VV_n^-))$} and \alert{$\phi \in \HH(S_n(F))$}.
    Suppose \alert{$f$ and $\phi$ are related by base change}.
    Then for matching $g \in \U(\VV_n^+)\rs \longleftrightarrow \gamma \in S_n(F)\rs$:
    we have
    \[ \Int\left( (1,g), \mathbf{1}_{K'^\flat} \otimes \alert{f} \right) \log q
    = -\omega(\gamma) \left. \pdv{}{s} \right\rvert_{s=0} \Orb(\gamma, \alert{\phi}, s). \]
  \end{exampleblock}
  AFL is special case $f = \mathbf{1}_{K'}$ and $\phi = \mathbf{1}_K$.
  \begin{theorem}
    [Li-Rapoport-Zhang  2024]
    True for $n = 2$.
  \end{theorem}
  Tried to mimic the proof for $n = 3$ of original AFL from W.\ Zhang 2012.
  \begin{itemize}
  \ii Was able to make the base change explicit enough to use for $n = 3$.
  \ii Was able to compute the orbital integral (surprisingly nice).
  \ii However, didn't understand geometric side enough to compute it.
  \end{itemize}
\end{frame}

\begin{frame}
  \frametitle{An example of an orbital integral in the $n=3$ case}
  TODO
\end{frame}

\begin{frame}
  \frametitle{Semi-Lie version of AFL conjecture}
  \begin{exampleblock}{Conjecture}
    Let \alert{$f \in \HH(\U(\VV_n^-))$} and \alert{$\phi \in \HH(S_n(F))$}.
    Suppose \alert{$f$ and $\phi$ are related by base change}.
    If $(g, u) \in (\U(\VV_n^+) \times \VV_n^+)\rs \longleftrightarrow
      (\gamma, \uu, \vv^\top) \in (S_n(F) \times \OO_F^n \times (\OO_F^n)^\vee)\rs$
    then
    \begin{align*}
      &\phantom= \Int\left( (g,u), \alert{f} \right) \log q \\
      &= -\omega\guv \left. \pdv{}{s} \right\rvert_{s=0}
      \Orb(\guv, \alert{\phi} \otimes \oneV, s).
    \end{align*}
  \end{exampleblock}
\end{frame}

\begin{frame}
  \frametitle{An example of an orbital integral in the $n=2$ case}
  TODO
\end{frame}

\end{document}
