\documentclass[11pt,colorlinks]{amsart}
\pdfoutput=1
% Preamble - Evan's thesis version

%%% Load packages
\usepackage{hyperref}
\usepackage[obeyFinal,textsize=scriptsize,shadow,loadshadowlibrary]{todonotes}
\usepackage[shortlabels]{enumitem}
\usepackage[usenames,dvipsnames,svgnames]{xcolor}
\usepackage{amsmath}
\usepackage{amssymb}
\usepackage{amsthm}
\usepackage{booktabs}
\usepackage[nameinlink]{cleveref}
\usepackage{derivative}
\usepackage{graphicx}
\usepackage{mathdots}
\usepackage{mathrsfs}
\usepackage{mathtools}
\usepackage{microtype}

\usepackage{asymptote}
\usepackage{tikz-cd}
\usetikzlibrary{decorations.pathmorphing}

\allowdisplaybreaks

\usepackage[backend=biber,backref=true,style=alphabetic]{biblatex}

\newtheorem{theorem}{Theorem}[section]
\newtheorem{lemma}[theorem]{Lemma}
\newtheorem{proposition}[theorem]{Proposition}
\newtheorem{corollary}[theorem]{Corollary}

\theoremstyle{definition}
\newtheorem{assume}[theorem]{Assumption}
\newtheorem{definition}[theorem]{Definition}
\newtheorem{example}[theorem]{Example}
\newtheorem{ques}[theorem]{Question}
\newtheorem{claim}[theorem]{Claim}
\newtheorem{conjecture}[theorem]{Conjecture}
\newtheorem{remark}[theorem]{Remark}
\newtheorem{question}[theorem]{Question}

%%% Macros
\providecommand{\ol}{\overline}
\providecommand{\ul}{\underline}
\providecommand{\wt}{\widetilde}
\providecommand{\wh}{\widehat}
\providecommand{\eps}{\varepsilon}
\providecommand{\half}{\frac{1}{2}}
\providecommand{\inv}{^{-1}}
\newcommand{\dang}{\measuredangle} %% Directed angle
\providecommand{\CC}{\mathbb C}
\providecommand{\FF}{\mathbb F}
\providecommand{\NN}{\mathbb N}
\providecommand{\QQ}{\mathbb Q}
\providecommand{\RR}{\mathbb R}
\providecommand{\ZZ}{\mathbb Z}
\providecommand{\ts}{\textsuperscript}
\providecommand{\dg}{^\circ}
\providecommand{\ii}{\item}
\newcommand{\surjto}{\twoheadrightarrow}

\DeclareMathOperator*{\Arch}{ARCH}
\DeclareMathOperator{\BC}{BC}
\DeclareMathOperator{\End}{End}
\DeclareMathOperator{\Gal}{Gal}
\DeclareMathOperator{\GL}{GL}
\DeclareMathOperator{\Hom}{Hom}
\DeclareMathOperator{\Int}{Int}
\DeclareMathOperator{\Lie}{Lie}
\DeclareMathOperator{\Mat}{Mat}
\DeclareMathOperator{\Norm}{N}
\DeclareMathOperator{\Orb}{Orb}
\DeclareMathOperator{\SL}{SL}
\DeclareMathOperator{\Sat}{Sat}
\DeclareMathOperator{\Spec}{Spec}
\DeclareMathOperator{\Spf}{Spf}
\DeclareMathOperator{\Sym}{Sym}
\DeclareMathOperator{\U}{U}
\DeclareMathOperator{\Vol}{Vol}
\DeclareMathOperator{\antidiag}{antidiag}
\DeclareMathOperator{\diag}{diag}
\DeclareMathOperator{\id}{id}
\DeclareMathOperator{\rproj}{proj}

\newcommand{\cc}{\mathbf c}
\newcommand{\nn}{\mathbf n}
\newcommand{\uu}{\mathbf u}
\newcommand{\vv}{\mathbf v}
\newcommand{\HH}{\mathcal H}
\newcommand{\EE}{\mathbb E}
\newcommand{\TT}{\mathbb T}
\newcommand{\VV}{\mathbb V}
\newcommand{\XX}{\mathbb X}

\newcommand{\RZ}{\mathcal N} % for RZ space
\newcommand{\ZD}{\mathcal Z} % Cartier divisor

\newcommand{\BG}{\mathbf{G}}
\newcommand{\BK}{\mathbf{K}}
\newcommand{\BM}{\mathbf{M}}
\newcommand{\BN}{\mathbf{N}}
\newcommand{\BP}{\mathbf{P}}
\newcommand{\BS}{\mathbf{S}}

\newcommand{\guv}{{(\gamma, \uu, \vv^\top)}}
\newcommand{\oneV}{\mathbf{1}_{\OO_F^n \times (\OO_F^n)^\vee}}
\newcommand{\rs}{_{\mathrm{rs}}}

\newcommand{\OO}{O} % to avoid confusion with structure sheaf I guess
\newcommand{\SO}{\mathcal{O}} % sheaf O

\newcommand{\jiao}{\mathop{\otimes}^{\mathbf{L}}} % this is a cute macro name (交)
% found in arXiv:2402.1767v1 while trying to search for where \mathbb{L} was defined
% (answer: it's not, this entire glyph is just generally used for derived tensor products)

\hypersetup{pdfauthor={Evan Chen},pdftitle={Semi-Lie arithmetic fundamental lemma for the full spherical Hecke algebra}}

\usepackage{amsaddr}
\addbibresource{refs.bib}

\title[Semi-Lie Hecke AFL]
{Semi-Lie arithmetic fundamental lemma for the full spherical Hecke algebra}

\author{Evan Chen}
\date{\today}
\address{Department of Mathematics, Massachusetts Institute of Technology}
\email{evanchen@alum.mit.edu}
\subjclass[2010]{11G18, 11F70}
\keywords{arithmetic fundamental lemma}

\begin{document}

\begin{abstract}
  As an analog to the Jacquet-Rallis fundamental lemma that appears in the
  relative trace formula approach to the Gan-Gross-Prasad conjectures,
  the arithmetic fundamental lemma was proposed by Wei Zhang and used in an approach
  to the arithmetic Gan-Gross-Prasad conjectures.
  The Jacquet-Rallis fundamental lemma was recently generalized by Spencer Leslie
  to a statement holding for the full spherical Hecke algebra.
  In the same spirit, Li, Rapoport, and Zhang
  have recently formulated a conjectural generalization of the arithmetic
  fundamental lemma to the full spherical Hecke algebra.
  This paper formulates another analogous conjecture for the semi-Lie version
  of the arithmetic fundamental lemma proposed by Yifeng Liu.
  Then this paper produces explicit formulas for particular cases
  of the weighted orbital integrals in the two conjectures mentioned above,
  and proves the first non-trivial case of the conjecture.
\end{abstract}

\maketitle

\tableofcontents
\newpage

\chapter{Introduction}
Throughout this whole paper, $p > 2$ is a prime,
$F$ is a finite extension of $\QQ_p$,
and $E/F$ is an unramified quadratic field extension.

\section{Brief history and motivation for the arithmetic fundamental lemma}
The primary motivation for this paper arises from
the study of conjectured variants of the arithmetic fundamental lemma
for spherical Hecke algebras proposed in \cite{ref:AFLspherical}.
This section briefly provides an overview of the historical context
that led to the formulation of these conjectures.
This history is also summarized in \Cref{fig:history}.

Because this subsection is meant for motivation only, in this survey we do not give
complete definitions or statements, being content to outline a brief gist.
A more detailed account can be found in \cite{ref:survey}.

\begin{figure}[ht]
  \centering
  \begin{tikzcd}
      \text{\cite{ref:waldspurger}} \ar[d] \\
      \text{\cite{ref:GP1,ref:GP2}} \ar[d] \\
    \text{GGP \cite{ref:GGP}}
      \ar[d, dotted, leftrightarrow, "\text{analog}"]
      & \ar[l, Rightarrow, "\text{used to prove}"] \text{FL \cite{ref:JR}}
        \ar[d, dotted, leftrightarrow, "\text{analog}"] \ar[r]
      & \text{\cite{ref:leslie}}
        \ar[d, dotted, leftrightarrow, "\text{analog}"] \\
    \text{Arith.\ GGP \cite{ref:GGP}}
      & \ar[l, Rightarrow, "\text{used to prove}"] \text{AFL \cite{ref:AFL}} \ar[r]
      & \text{\cite{ref:AFLspherical}} \\
    \text{\cite{ref:GZshimura}} \ar[u] \\
    \text{\cite{ref:gross_zagier}} \ar[u]
  \end{tikzcd}
  \caption{The history behind the fundamental lemma and its arithmetic counterpart.
    Unlabeled arrows denote generalizations.}
  \label{fig:history}
\end{figure}

\subsection{The GGP conjectures, and the fundamental lemma of Jacquet-Rallis}
In modern arithmetic geometry, a common theme is that there are deep connections
between geometric data with the values of related $L$-functions.

This story begins with a result of
Waldspurger \cite{ref:waldspurger} which showed a formula
relating the nonvanishing of an automorphic period integral
to the central value of the same $L$-functions.
Later, a conjecture that generalizes Waldspurger's formula
was proposed by Gross-Prasad in \cite{ref:GP1,ref:GP2}.
This was further generalized to a series of conjectures
now known as the Gan-Gross-Prasad (GGP) conjectures,
which were proposed in 2012 in \cite{ref:GGP};
they generalize the Gross-Prasad conjecture to different classical groups.
Specifically, the GGP conjecture predict the nonvanishing of a period integral
based on the values of the $L$-function of a certain cuspidal automorphic representation.

In 2011, Jacquet-Rallis \cite{ref:JR} proposed an approach to the Gross-Prasad conjectures
for unitary groups via a relative trace formula (RTF).
The idea is to compare a RTF for the general linear group to one for a unitary group.
This approach relies on a so-called \emph{fundamental lemma},
which links values of certain orbital integrals
over two reductive groups over a non-Archimedean local field.

Let's be a bit more precise about what this fundamental lemma says.
Let $\VV_n^+$ denote the split $E/F$-Hermitian space of dimension $n$ (unique up to isomorphism),
fix a unit vector $w_0$ in it,
and let $(\VV_n^+)^\flat$ be the orthogonal complement of the span of $w_0$.
Let $G'^\flat \coloneqq \GL_{n-1}(E)$, $G' \coloneqq \GL_n(E)$,
$G^\flat \coloneqq \U((\VV_n^+)^\flat)(F)$ and $G \coloneqq \U(\VV_n^+)(F)$.
For certain
\[ \gamma \in G'^\flat \times G', \qquad g \in G^\flat \times G \]
the Jacquet-Rallis fundamental lemma proposes a relation between two orbital integrals.
Specifically, it supplies a relation between
\begin{itemize}
\item the orbital integral of $\gamma$ with respect to
  the indicator function $\mathbf{1}_{K'^\flat \times K'}$
  of the natural hyperspecial compact subgroup
  \[ K'^\flat \times K' \subset G'^\flat \times G' = \GL_{n-1}(E) \times \GL_n(E); \]
  and
\item the orbital integral of $g$ with respect to
  the indicator function $\mathbf{1}_{K^\flat \times K}$
  of the natural hyperspecial compact subgroup
  \[ K^\flat \times K \subset G^\flat \times G = \U((\VV_n^+)^\flat)(F) \times \U(\VV_n^+)(F). \]
\end{itemize}
In other words, it states that
\begin{equation}
  \Orb(\gamma, \mathbf{1}_{K'^\flat \times K'}) = \omega(\gamma) \Orb(g, \mathbf{1}_{K^\flat \times K})
  \label{eq:old_FL}
\end{equation}
where $\omega(\gamma)$ is a suitable \emph{transfer factor}.
The fundamental lemma has since been proved completely;
a local proof was given by Beuzart-Plessis \cite{ref:BeuzartPlessis}
while a global proof was given for large characteristic by W.\ Zhang \cite{ref:Wei2021}.

\subsection{The arithmetic GGP conjectures, and the arithmetic fundamental lemma}
At around the same time Waldspurger's formula was published,
Gross-Zagier \cite{ref:gross_zagier} proved a formula
relating the height of Heegner points
on certain modular curves to the derivative at $s=1$ of certain $L$-functions.
The Gross-Zagier formula was then generalized over several decades,
culminating in \cite{ref:GZshimura} where the formula is established
for Shimura curves over arbitrary totally real fields.

An arithmetic analogue of the original Gan-Gross-Prasad conjectures,
which we henceforth refer to as \emph{arithmetic GGP} \cite{ref:GGP},
can then be formulated, further generalizing Gross-Zagier's formula.
Here the modular curves in Gross-Zagier
are replaced with higher dimensional Shimura varieties.
Rather than the period integrals considered previously,
one instead takes intersection numbers of cycles on some Shimura varieties.
Specifically, if one considers the Shimura variety associated to a classical group,
the arithmetic GGP conjecture predicts a relation between intersection numbers
on this Simura variety with the central derivative of automorphic $L$-functions.

By analogy to the work Jacquet-Rallis \cite{ref:JR},
the arithmetic GGP conjectures should have a corresponding
\emph{arithmetic fundamental lemma} (henceforth AFL),
which was proposed by W.\ Zhang \cite[Conjecture 2.9]{ref:AFL}.
The arithmetic fundamental lemma then relates the derivative
of the weighted orbital integral with respect to the indicator function
$\mathbf{1}_{K'^\flat \times K'} \in \HH(G'^\flat \times G, K'^\flat \times K')$, that is
\[ \left. \pdv{}{s} \right\rvert_{s=0} \Orb(\gamma, \mathbf{1}_{K'^\flat \times K'}, s) \]
for $\gamma \in G'^\flat \times G'$,
to arithmetic intersection numbers on a certain Rapoport-Zink formal moduli space.
The AFL in \cite{ref:AFL} has since been proven over $p$-adic fields for any prime $p$ in
Mihatsch-Zhang \cite{ref:MZ2021}, W.\ Zhang \cite{ref:Wei2021}, Z.\ Zhang \cite{ref:Zhiyu}.

\subsection{The semi-Lie version of the AFL proposed by Liu}
There is another different version of the AFL, proposed by Yifeng Liu in
\cite[Conjecture 1.12]{ref:liuFJ},
which is often referred to as the \emph{semi-Lie version} of the AFL.
Its statement has been shown to be equivalent to AFL,
\cite[Remark 1.13]{ref:liuFJ} (and is thus now proven).
In contrast, the original AFL proposed by Zhang in \cite[Conjecture 2.9]{ref:AFL}
is sometimes referred to as the \emph{group version}.

A more detailed account of this equivalence is described in \cite[\S1.4]{ref:liuFJ}.

\subsection{Generalizations of FL and AFL to the full spherical Hecke algebra}
Recently it was shown by Leslie \cite{ref:leslie} that in fact
\eqref{eq:old_FL} holds in greater generality where the indicator function
$\mathbf{1}_{K^\flat \times K}$ can be replaced by any element in the spherical
Hecke algebra $\varphi \in \HH(G'^\flat \times G', K'^\flat \times K')$.
In that case, $\mathbf{1}_{K'^\flat \times K}$ needs to be replaced
by the corresponding element $\varphi'$ under a certain base change homomorphism
\begin{align*}
  \HH(G'^\flat \times G', K'^\flat \times K') &\to \HH(G^\flat \times G, K^\flat \times K) \\
  \varphi' &\mapsto \varphi
\end{align*}

In that case, the identity \eqref{eq:old_FL} still hold as
\begin{equation}
  \Orb(\gamma, \varphi') = \omega(\gamma) \Orb(g, \varphi).
  \label{eq:eq:leslie_FL}
\end{equation}
To complete the analogy illustrated in \Cref{fig:history},
there should thus be a generalization of the AFL in which
$\mathbf{1}_{K'^\flat \times K}$ is replaced by any element of the Hecke algebra
$\HH(G'^\flat \times G, K'^\flat \times K)$.
This formula is proposed by \cite{ref:AFLspherical},
and is the primary focus of this paper; we discuss it in the next section.

\section{Formulation of AFL conjectures to the full spherical Hecke algebra}
\subsection{The inhomogeneous version of the arithmetic fundamental lemma for spherical Hecke algebras proposed by Li-Rapoport-Zhang}
In contrast to the vague motivational cheerleading in the previous section,
starting in this section we will give more precise statements,
even though we will necessarily need to reference definitions appearing in later sections.

Retain the notation $G' \coloneqq \GL_n(E)$, and $G \coloneqq \U(\VV_n^+)(F)$,
with $K' \subset G'$ and $K \subset G$ the natural hyperspecial compact subgroups.
Also, let $q$ denote the residue characteristic of $F$.
Moreover, define the symmetric space
\[ S_n(F) \coloneqq \left\{ g \in \GL_n(E) \mid g \bar{g} = \id_n \right\}. \]
Finally, let $\VV_n^-$ be the non-split Hermitian space of dimension $n$
(unique up to isomorphism),
and let $\VV_n^+$ be the split one (again unique up to isomorphism).
Denote by $\U(\VV_n^-)$ the corresponding unitary groups.

For concreteness, we focus on the inhomogeneous version
of the arithmetic fundamental lemma, which is \cite[Conjecture 6.2.1]{ref:AFLspherical}.
This allows us to deal with just $G'$ instead of $G'^\flat \times G'$, etc.,
so that the Hecke algebra $\HH(G'^\flat \times G', K'^\flat \times K')$
can be replaced by the simpler one $\HH(\GL_n(E)) \coloneqq \HH(G', K')$.
Similarly, $\HH(G^\flat \times G, K^\flat \times K)$
can be replaced by the simpler $\HH(\U(\VV_n^+)) \coloneqq \HH(G, K)$.

The AFL conjecture provides a bridge between a geometric left-hand side
(given by an intersection number)
and an analytic right-hand side (given by an weighted orbital integral).
Stating it requires several pieces of data.
We only mention these pieces by name here, with definitions given later:
\begin{itemize}
  \ii On the geometric side, we have an intersection number.
  It uses the following ingredients.
  \begin{itemize}
    \ii We choose a regular semisimple element $g \in \U(\VV_n^-)\rs$.
    (The notation $\U(\VV_n^-)\rs$ denotes the regular semisimple elements of $\U(\VV_n^-)$, etc.
    The notion of regular semisimple is defined in \Cref{def:regular}.)
    \ii We choose a function $f \in \HH(\GL_n(E))$ from the spherical Hecke algebra,
    defined in \Cref{ch:background}.
    \ii We define a certain \emph{intersection number} $\Int((g,u), f)$
    in \Cref{def:intersection_number_inhomog}.
    These intersection numbers take place in a Rapoport-Zink space
    described in \Cref{ch:geo}.
  \end{itemize}

  \ii On the analytic side, we have an weighted orbital integral.
  It uses the following ingredients.
  \begin{itemize}
    \ii We choose a regular semisimple element $\gamma \in S_n(F)\rs$.
    \ii We choose a test function $\phi$ which comes
    from a certain $\HH(\GL_n(E))$-module that we will denote $\HH(S_n(F))$.
    This module $\HH(S_n(F))$ is defined in \Cref{ch:background}.
    \ii The weighted orbital integral $\Orb(\gamma, \phi, s)$
    is itself defined in \Cref{def:orbital0}.
    (It is connected to an unweighted orbital integral on the unitary group
    according to \Cref{thm:rel_fundamental_lemma}.)
    \ii There is also an extra transfer factor $\omega \in \{\pm1\}$
    which we define in \Cref{ch:transf}.
  \end{itemize}

  \ii We need a way to connect the inputs between the two parts of our conjecture.
  Specifically, $f$ and $\phi$ need to be linked, and $g$ and $\gamma$ need to be linked.
  This is done as follows.
  \begin{itemize}
    \ii Once the regular semisimple element $g \in \U(\VV_n^-)\rs$ is chosen,
    we require $\gamma \in S_n(F)\rs$ to be a \emph{matching} element.
    This matching is defined in \Cref{def:matching_inhomog}.
    (Alternatively, one could imagine picking $\gamma \in S_n(F)\rs$ first
    and finding corresponding $g$;
    it turns out $\gamma$ will match an element of $\U(\VV_n^\pm)\rs$ is general,
    and the conjecture is only formulated in the case where $g \in \U(\VV_n^-)$).

    \ii Once $f \in \HH(\U(\VV_n^+))$ is chosen, we select
    \[ \phi = (\BC_{S_n}^{\eta^{n-1}})^{-1}(f) \]
    to be the image of $f$ under a \emph{base change}.
    This base change is defined and then calculated explicitly for $n = 3$ in \Cref{ch:satake}.
  \end{itemize}
\end{itemize}
With all our protagonists now having names and references,
we can now state the conjecture proposed in \cite{ref:AFLspherical}.

\begin{conjecture}
  [Inhomogeneous version of the AFL for the full spherical Hecke algebra,
  {\cite[Conjecture 6.2.1]{ref:AFLspherical}}]
  \label{conj:inhomog}
  Let $f \in \HH(\U(\VV_n^+))$ be any element of the Hecke algebra,
  and let $\phi = (\BC_{S_n}^{\eta^{n-1}})^{-1}(f) \in \HH(S_n(F))$ be its image
  under base change as defined in \Cref{ch:satake}.
  Then for matching (as defined in \Cref{def:matching_inhomog}) regular semisimple elements
  \[ g \in \U(\VV_n^-)\rs \longleftrightarrow \gamma \in S_n(F)\rs \]
  we have
  \begin{equation}
    \Int\left( (1,g), \mathbf{1}_{K^\flat} \otimes f \right) \log q
    = -\omega(\gamma) \left. \pdv{}{s} \right\rvert_{s=0} \Orb(\gamma, \phi, s)
    \label{eq:inhomog}
  \end{equation}
  where the weighted orbital integral $\Orb(\dots)$ is defined in \Cref{def:orbital0},
  the transfer factor $\omega$ is defined in \Cref{ch:transf},
  and the intersection number $\Int(\dots)$ is defined in \Cref{ch:geo}.
\end{conjecture}

At present, the (inhomogeneous) AFL is the case where $f = \mathbf{1}_K$,
and is thus proven.
Note that in the case of interest where $\gamma \in S_n(F)\rs$
matches an element of $\U(\VV_n^-)\rs$ (rather than $\U(\VV_n^+)\rs$),
the actual value of $\Orb(\gamma, \phi, s)$ at $s = 0$ is always zero
by \Cref{thm:rel_fundamental_lemma};
so the conjecture instead looks at the first derivative at $s = 0$.

The generalized conjecture is also proved in full for
$n = 2$ in \cite[Theorem 1.0.1]{ref:AFLspherical}
(in that reference, our $n$ denotes the $n+1$ in \emph{loc.\ cit.}).
The part of the calculation involving weighted orbital integral has two parts:
\begin{itemize}
  \ii The calculation makes $\BC_{S_{n}}^{\eta^{n-1}}$
  completely explicit in a natural basis for $n = 2$.
  The result is \cite[Lemma 7.1.1]{ref:AFLspherical}.

  \ii The calculation makes explicit the value of the weighted orbital integral
  \[ \Orb(\gamma, \phi, s) \]
  for any $\gamma \in S_n(F)\rs$ and $\phi \in \HH(S_n(F))$,
  in terms of invariants of $\gamma$ and a decomposition of $\phi$ in a natural basis.
  The result is \cite[Proposition 7.3.2]{ref:AFLspherical}.
\end{itemize}
Combining these two (hence obtaining the right-hand side of \eqref{eq:inhomog})
with a calculation of intersection numbers in \cite[Corollary 7.4.3]{ref:AFLspherical}
(which is the left-hand side of \eqref{eq:inhomog})
shows that \Cref{conj:inhomog} holds for $n = 2$,
cf.\ \cite[Theorem 7.5.1]{ref:AFLspherical}.

\subsection{A proposed arithmetic fundamental lemma for spherical Hecke algebras in the semi-Lie case}
The primary focus of this paper is an analogous conjecture to \Cref{conj:inhomog}
for the semi-Lie version (also called the Fourier-Jacobi case).
It serves to complete the analogy given in \Cref{tab:semi_lie_analogy}.

\begin{table}[ht]
  \centering
  \begin{tabular}{lll}
    \toprule
    Version & AFL for $\mathbf{1}$ (now proven) & Full spherical Hecke \\
    \midrule
    Group & \cite[Conjecture 2.9]{ref:AFL} & \cite[Conjecture 6.2.1]{ref:AFLspherical} \\
    Semi-Lie & \cite[Conjecture 1.12]{ref:liuFJ} & \Cref{conj:semi_lie_spherical} \\
    \bottomrule
  \end{tabular}
  \caption{Table showing the analogy between the proposed
    \Cref{conj:semi_lie_spherical} and the existing conjectures.}
  \label{tab:semi_lie_analogy}
\end{table}

In this variation, as in \cite{ref:liuFJ},
rather than matching $g \in \U(\VV_n^-)\rs$ to $\gamma \in S_n(F)\rs$,
we consider an augmented space larger than $\U(\VV_n^-)$ and $S_n(F)$.
Specifically, one considers a matching between tuples of regular semisimple elements
\[ (g, u) \in (\U(\VV_n^-) \times \VV_n^-)\rs
  \longleftrightarrow (\gamma, \uu, \vv^\top) \in (S_n(F) \times V'_n(F))\rs \]
where $V'_n(F) = F^n \times (F^n)^\vee$ (defined in \Cref{def:matching_semi_lie})
consists of pairs of column vectors and row vectors of length $n$,
and the space $\VV_n^-$ is defined in \Cref{def:VV_n_nonsplit}.
The notion of \emph{matching} is defined in \Cref{def:matching_semi_lie} as well.
Meanwhile, we still use the same test functions $f$ and $\phi$,
as we did for \cite[Conjecture 6.2.1]{ref:AFLspherical}.
Finally, we also update the definition of intersection number
to accommodate the new term $u$ in \Cref{def:intersection_number_semi_lie_spherical}.

\begin{conjecture}
  [Semi-Lie version of the AFL for the full spherical Hecke algebra]
  Let $f \in \HH(\U(\VV_n^+))$ be any element of the Hecke algebra,
  and let $\phi = (\BC_{S_n}^{\eta^{n-1}})^{-1}(f) \in \HH(S_n(F))$ be its image
  under base change defined in \Cref{ch:satake}.
  Then for matching (as defined in \Cref{def:matching_semi_lie}) regular semisimple elements
  \[ (g, u) \in (\U(\VV_n^-) \times \VV_n^-)\rs \longleftrightarrow
    \guv \in (S_n(F) \times V'_n(F))\rs \]
  we have
  \begin{equation}
    \Int\left( (g,u), f \right) \log q \\
    = -\omega\guv \left. \pdv{}{s} \right\rvert_{s=0}
    \Orb(\guv, \phi \otimes \oneV, s)
  \end{equation}
  where the orbital integral $\Orb(\dots)$ is defined in \Cref{def:orbitalFJ},
  the transfer factor is defined in \Cref{ch:transf},
  and the intersection number $\Int(\dots)$ is defined in \Cref{ch:geo}.
  \label{conj:semi_lie_spherical}
\end{conjecture}
Note that in this version the new orbital integral $\Orb(\guv, \phi \otimes \oneV, s)$
is defined similarly.
However, as far as we know, no analog of \Cref{thm:rel_fundamental_lemma}
(linking it to an orbital integral on the unitary side) appears in the literature.
Thus we record the corresponding statement as \Cref{conj:rel_fundamental_lemma_semilie}.
Like before, \Cref{conj:rel_fundamental_lemma_semilie}
predicts that $\Orb(\guv, \phi \otimes \oneV, 0) = 0$ in the case of interest,
which in this case can be checked independently.

\begin{remark}
  For $n = 1$, the Hecke algebra $\HH(S_n(F))$ is trivial
  and therefore \Cref{conj:semi_lie_spherical}
  becomes a special case of the known result \cite{ref:liuFJ}.
  Therefore $n=2$ is the first case of \Cref{conj:semi_lie_spherical} worth examining.
\end{remark}

\subsection{A proposed conjecture on the kernel of indicator functions on which the orbital has identically vanishing derivative}
\label{sec:intro_large_kernel}

In \cite{ref:AFLspherical}, it was observed that for $n = 2$
there was a rather large space of $\phi \in \HH(S_2(F))$ such that
\[ \left. \pdv{}{s} \right\rvert_{s=0}
  \Orb \left(\gamma, \phi, s \right) = 0 \]
held identically across all $\gamma \in S_2(F)$.
In fact, the space of such $\phi$ has codimension $2$
as a vector subspace of $\HH(S_2(F))$ for $n = 2$.
They thus stated a conjecture the kernel was ``large'' for general $n$
as \cite[Conjecture 1.0.2]{ref:AFLspherical}.

It is therefore natural to ask whether a similar large kernel result
could hold for the analogous orbital integral in the semi-Lie case.
In fact, even for $n=2$ the behavior is somewhat different.
We propose the following conjecture, which we have proved for $n = 2$
(see \Cref{thm:semi_lie_ker_trivial} and \Cref{thm:semi_lie_ker_huge} momentarily).
\begin{conjecture}
  \label{conj:kernel_semi_lie}
  Let $n \ge 2$.
  \begin{enumerate}
  \item[(a)]
  Choose any $\gamma \in S_n(F)$ which could appear in a triple $\guv$
  matching some element of $\U(\VV_n^-) \times \VV_n^-$.
  Suppose $\phi \in \HH(S_n(F))$ satisfies
  \[ \left. \pdv{}{s} \right\rvert_{s=0}
    \Orb \left(\guv, \phi \otimes \oneV, s \right) = 0 \]
  for every such $(\uu, \vv^\top) \in V'_n(F)$,
  Then in fact $\Orb \left(\guv, \phi \otimes \oneV, s \right) = 0$ for all $s \in \CC$.

  \item[(b)]
  Fix a choice of $(\uu, \vv^\top) \in V'_n(F)$.
  Consider all $\gamma \in S_n(F)\rs$ for which $\guv$
  matches with an element of $(\U(\VV_n^-) \times \VV_n^-)\rs$.
  Consider the map of vector spaces
  \begin{align*}
    \HH(S_n(F))
    &\to C^\infty( \left\{ \text{elements of } S_n(F)\rs \text{ matching as above} \right\} ) \\
      \phi &\mapsto \left( \gamma \mapsto \left. \pdv{}{s} \right\rvert_{s=0}
      \Orb \left(\guv, \phi \otimes \oneV, s \right) \right).
  \end{align*}
  Then this map has large kernel,
  in the sense it is not contained in any maximal ideal of $\HH(S_n(F))$.
  \end{enumerate}
\end{conjecture}

\section{Results}
Most of the results here are dedicated toward the semi-Lie version of the AFL,
which is the new contribution provided by this paper.
But in \Cref{sec:results_group_AFL} we mention some other results
we proved for the group version of the AFL.

\subsection{Formulas for the orbital side of the semi-Lie AFL conjecture for $n=2$}
\label{sec:semi_lie_2_intro_params}
The main case of interest in this thesis is the new conjectured AFL
for the spherical Hecke algebra in the semi-Lie situation in the
specific case $n = 2$ where one can provide evidence for the conjecture.
On the orbital side, the various ingredients can be described concretely
in the following way:
\begin{itemize}
  \ii The Hecke algebra $\HH(S_2(F))$ has a natural basis of
  indicator functions $\mathbf{1}_{K', \le r}$ for each $r \ge 0$;
  see \Cref{ch:orbitalFJ0} for a definition.

  \ii Suppose $\guv \in (S_2(F) \times V'_2(F))\rs$
  pairs with an element of $(\U(\VV_2^-) \times \VV_2^-)\rs$.
  Then under the action $\GL_2(F)$ we may assume $\guv$ is of the form
  \[
    \guv = \left( \begin{pmatrix} a & b \\ c & d \end{pmatrix},
      \begin{pmatrix} 0 \\ 1 \end{pmatrix},
      \begin{pmatrix} 0 & e \end{pmatrix} \right)
    \in (S_2(F) \times V_2'(F))\rs
  \]
  (that is, we can find an orbit representative of this form).
  The parameters $a$, $b$, $c$, $d$ need to satisfy certain dependencies
  for the matching to hold; these are described in \Cref{ch:orbitalFJ0}.
\end{itemize}

Then we were able to derive the following fully explicit formula.
See \Cref{sec:proof_semi_lie_formula} for some concrete examples and illustrations.
\begin{theorem}
  \label{thm:semi_lie_formula}
  Let $\guv$ be as above.
  If $v(e) < 0$ or $v(b) + v(c) < -2r$, the entire orbital integral is $0$.
  Otherwise define
  \[ \nn_\guv(k) \coloneqq \min\left( \left\lfloor \tfrac{k + (v(b)+r)}{2} \right\rfloor,
    \left\lfloor \tfrac{(2v(e)+v(c)+r)-k}{2} \right\rfloor, N \right) \]
  where
  \[ N \coloneqq \min \left(
      v(e), \tfrac{v(b)+v(c)-1}{2} + r,
      v(d-a) + r \right). \]
  Also, if $v(d-a) < v(e) - r$ and $v(b) + v(c) > 2v(d-a)$, then additionally define
  \begin{align*}
    \cc_\guv(k) &= \min\big( k - (2v(d-a)-v(b)+r), \\
      &\qquad (2v(e)+v(c)-2v(d-a)-r)-k, v(e)-v(d-a)-r \big).
  \end{align*}
  Otherwise define $\cc_\guv(k) = 0$.
  Then we have
  \begin{align*}
    &\phantom= \Orb(\guv, \mathbf{1}_{K'_{S, \le r}} \otimes \oneV, s) \\
    &= \sum_{k = -(v(b)+r)}^{2v(e)+v(c)+r} (-1)^k
    \left( 1 + q + q^2 + \dots + q^{\nn_\guv(k)} \right) (q^s)^k \\
    &+ \sum_{k = 2v(d-a)-v(b)+r}^{2v(e)+v(c)-2v(d-a)-r} (-1)^k \cc_\guv(k) q^{v(d-a) + r} (q^s)^k.
  \end{align*}
\end{theorem}

Differentiating this yields the following result:
\begin{corollary}
  \label{cor:semi_lie_derivative_single}
  Let $r$, $\guv$ and $N$ be as in \Cref{thm:semi_lie_formula}
  Also define $\varkappa \coloneqq v(e) - (v(d-a)+r)$.
  Then
  \begin{align*}
    \frac{(-1)^{v(c)+r}}{\log q}
    &\partial \Orb(\guv, \mathbf{1}_{K'_{S, \le r}}) \\
    &= \sum_{j=0}^N \left( \frac{2v(e)+v(b)+v(c)+1}{2} + r - 2j \right) \cdot q^j \\
    & - q^{v(d-a)+r} \cdot
    \begin{cases}
      \frac{\varkappa}{2} & \text{if }\varkappa \equiv 0 \pmod 2 \\
      \left( v(e)+\frac{v(b)+v(c)}{2}-2v(d-a)-r \right) - \frac{\varkappa}{2}
      & \text{if }\varkappa \equiv 1 \pmod 2 \\
    \end{cases}
  \end{align*}
  where the second term is only present when $\varkappa \ge 0$ and $v(b)+v(c)>2v(d-a)$.
\end{corollary}

The formula simplifies even further if one considers instead
$\mathbf{1}_{K'_{S, \le r}} + \mathbf{1}_{K'_{S, \le (r-1)}}$;
and indeed we will see that this particular combination comes up naturally
with a special role later as well.
\begin{corollary}
  \label{cor:semi_lie_combo}
  Let $r$, $\guv$, $N$, $\varkappa$ be as in \Cref{cor:semi_lie_derivative_single}.
  Assume $r \ge 1$ and define
  \begin{align*}
    C &\coloneqq
    \begin{cases}
      \frac{\varkappa-1}{2}
        & \text{if } \varkappa > 0 \text{ is odd}
          \text{ and } v(b) + v(c) > 2v(d-a)  \\
      \frac{\varkappa+v(b)+v(c)-2v(d-a)-1}{2}
        & \text{if } \varkappa \ge 0 \text{ is even}
          \text{ and } v(b) + v(c) > 2v(d-a)  \\
      v(e) - N
        & \text{if } v(e) \ge \frac{v(b)+v(c)-1}{2} + r
        \text{ and } 2v(d-a) > v(b) + v(c) \\
      0 & \text{otherwise}
    \end{cases} \\
    C' &\coloneqq
    \begin{cases}
      C + 1 & \text{if } \varkappa \ge 0 \text{ and } v(b)+v(c) > 2v(d-a) \\
      0 & \text{otherwise}.
    \end{cases}
  \end{align*}
  Then
  \begin{align*}
    \frac{(-1)^{v(c)+r}}{\log q} &
    \partial\Orb(\guv, \mathbf{1}_{K'_{S, \le r}} + \mathbf{1}_{K'_{S, \le (r-1)}}) \\
    &= (q^N + q^{N-1} + \dots + 1) + C q^N + C' q^{N-1}
  \end{align*}
\end{corollary}

\begin{example}
  We show some examples of \Cref{cor:semi_lie_combo}:
  \begin{itemize}
  \ii When $r=5$, $v(b) = -20$, $v(c) = 37$, $v(e) = 35$ and $v(d-a) > \frac{v(b)+v(c)}{2} = 8.5$
  the derivative in \Cref{cor:semi_lie_combo} equals
  \[ \log q \cdot (23q^{13} + q^{12} + q^{11} + q^{10} + q^9 + \dots + q + 1). \]
  \ii When $r = 6$, $v(b) = 10$, $v(c) = 5$, $v(e) = 7$, $v(d-a) > v(e)-r = 1$,
  the derivative in \Cref{cor:semi_lie_combo} equals
  \[ -\log q \cdot (q^7 + q^6 + q^5 + \dots + q + 1). \]
  \ii When $r = 8$, $v(b) = -101$, $v(c) = 1000$, $v(e) = 29$, $v(d-a) = 11$,
  the derivative in \Cref{cor:semi_lie_combo} equals
  \[ \log q \cdot (444 q^{19} + 445q^{18} + q^{17} + q^{16} + q^{15} + \dots + q + 1). \]
  \end{itemize}
\end{example}

\subsection{Kernel results for the semi-Lie orbital integral when $n=2$}
As we mentioned our earlier conjecture \Cref{conj:kernel_semi_lie}
is true for $n = 2$.
More precisely, we have the following two theorems.

\begin{theorem}
  \label{thm:semi_lie_ker_trivial}
  Consider any $\gamma = \begin{pmatrix} a & b \\ c & d \end{pmatrix} \in S_2(F)$
  as in \Cref{sec:semi_lie_2_intro_params}.
  Assume $v(b)+v(c)>0$ (so that the orbital integrals are not all zero.)
  Then there don't exist any nonzero functions $\phi \in \HH(S_2(F))$ such that
  \[ \left. \pdv{}{s} \right\rvert_{s=0}
    \Orb \left(
      \left( \gamma, \begin{pmatrix} 0 \\ 1 \end{pmatrix}, \begin{pmatrix} 0 & \varpi^i \end{pmatrix} \right),
      \phi \otimes \oneV, s
    \right) = 0 \]
  holds for every integer $i \ge 0$.
  Thus part (a) of \Cref{conj:kernel_semi_lie} holds for $n = 2$.
\end{theorem}

\begin{theorem}
  \label{thm:semi_lie_ker_huge}
  Fix a choice of $(\uu, \vv^\top) \in V_2'(F)$ with $\vv^\top \uu \neq 0$.
  Consider all $\gamma \in S_2(F)$ for which $\guv$
  matches with an element of $(\U(\VV_n^-) \times \VV_n^-)\rs$.
  Then the set of $\phi \in \HH(S_2(F))$ such that
  \[
    \left. \pdv{}{s} \right\rvert_{s=0}
    \Orb \left( \guv, \phi \otimes \oneV, s \right) = 0
  \]
  holds for all such $\gamma$ is a $\QQ$-vector subspace of $\HH(S_2(F))$
  whose codimension is at most $v(\uu \vv^\top) + 2$.
  Moreover, the set of such $\phi$ is not contained in any maximal ideal of $\HH(S_2(F))$.
\end{theorem}

\begin{remark}
  In each case the Hecke algebra is isomorphic as a $\QQ$-algebra to $\QQ[T]$
  for a single variable $T = Y+Y^{-1}$.
  So actually it's mildly surprising that we get a result on finite codimension.
  In general, a finite codimension vector subspace of $\QQ[T]$ could
  be contained in a maximal ideal,
  such as the codimension one subspace $T \mathbb Q[T] \subset \QQ[T]$.
  Conversely, a finite \emph{dimension} vector subspace such as
  the one-dimensional space $\QQ \subseteq \QQ[T]$ is not contained in any maximal ideal.

  Thus neither of being finite codimension and generating all of $\QQ[T]$ imply each other.
  However, the author's opinion is that being finite codimension should be more surprising,
  since even a finite-dimensional $\QQ$-vector subspace of $\QQ[T]$
  (indeed, any subspace containing $1$) could still generate the entire ring $\QQ[T]$.

  In particular, we do not currently have a reason to expect that for larger $n$
  the codimension of the kernel will still be finite in $\HH(S_n(F))$,
  i.e.\ an analogous finite-codimension conjecture
  to \Cref{conj:large_kernel_group} seems possibly too optimistic.
  Nonetheless, it might still be interesting to consider other different ways
  of formalizing the notion of ``large kernel'' in \Cref{conj:large_kernel_group}
  and \Cref{conj:kernel_semi_lie}.
\end{remark}

\subsection{The geometric side of the semi-Lie AFL conjecture for $n=2$}
On the geometric side, we were also able to determine the intersection numbers
subject to two provisions,
\Cref{conj:serre_pullback_space} and \Cref{conj:serre_pullback_divisor},
about the pullback of certain divisors.
\begin{theorem}
  \label{thm:semi_lie_n_equals_2}
  Assume \Cref{conj:serre_pullback_space} and \Cref{conj:serre_pullback_divisor}.
  Then our generalized AFL conjecture, \Cref{conj:semi_lie_spherical}, holds for $n = 2$.
\end{theorem}
The proof of \Cref{thm:semi_lie_n_equals_2} is built up gradually
throughout the entire paper, culminating in \Cref{ch:finale}.

We comment briefly on the strategy of the proof.
The proof is made possible because the intersection numbers for $n=2$
are easier to work for a few reasons.
\begin{itemize}
\ii First, one can identify $\VV_2^-$ with an $E/F$-quaternion division algebra,
equipped with a compatible Hermitian form defined via quaternion multiplication.
This makes it possible to describe $\U(\VV_2^-)$ concretely as transformations obtained
via left multiplication by an element of $E$ and right multiplication by a quaternion.
\ii Secondly, it becomes possible to describe the so-called Rapoport-Zink spaces $\RZ_2$
used in the definition of the intersection number with a Lubin-Tate space $\MM_2$.
Thus the problem of computing the intersection number
$\Int\left( (g,u), f \right) \log q$
can be reduced to calculating the intersection of certain special
Kudla-Rapoport divisors on the space $\MM_2$.

However, on the Lubin-Tate space $\MM_2$,
there is a result known as the Gross-Keating formula \cite{ref:GK}
which allows one to make this intersection number fully explicit.
One can then match the resulting equation to the formulas described in
\Cref{cor:semi_lie_derivative_single}
and verify that, under the base change \Cref{ch:satake}
and the matching condition described in \Cref{ch:rs_matching},
the two obtained formulas are identical.
\end{itemize}
The two hypotheses \Cref{conj:serre_pullback_space} and \Cref{conj:serre_pullback_divisor}
are a stipulation that the pullback of two of the divisors
behaves in the way one would expect.

\subsection{Results for $n=3$ for the group AFL}
\label{sec:results_group_AFL}
For the group AFL, we were able to fully compute the orbital integral as well.
The result is too involved to state in the introduction,
but we give the following summary.
\begin{theorem}
  \label{thm:summary}
  Let $\gamma \in S_3(F)\rs$.
  Assume that $\gamma$ matches an element in $\U(\VV_3^-)\rs$.
  Then the weighted orbital integral $\Orb(\gamma, \phi, s)$ takes the form
  \[ \sum_k P_k(q) (-q^s)^k \]
  for some polynomials $P_k(q) \in \ZZ[q]$, where
  \begin{itemize}
    \ii the summation variable $k$ is a some contiguous range of integers,

    \ii the polynomials $P_k \in \ZZ[q]$ are nonzero and satisfy the property
    that every coefficient of $P_k$ besides possibly the leading coefficient is $1$;

    \ii both $\deg P_k$ and leading coefficient of $P_k$ are the integer parts
    of piecewise linear functions in $k$ with slopes in $\{0, \pm\half, \pm 1\}$.
  \end{itemize}
  The range of the summation, and the aforementioned piecewise linear function(s),
  can be written explicitly in terms of a particular representative
  in the orbit of $\gamma$.
\end{theorem}
For a full statement, see
\Cref{thm:full_orbital_ell_odd,thm:full_orbital_ell_even,thm:full_orbital_ell_neg}.
The calculation corresponds directly to the earlier results
\cite[Lemma 7.1.1 and Proposition 7.3.2]{ref:AFLspherical}
which were the case $n = 2$ of the inhomogeneous group version of the AFL
(note what \cite{ref:AFLspherical} calls $n$ is $n+1$ in our notation).
The methods, which are local in nature,
are rather similar to those employed in \cite{ref:AFL},
which can be thought of as the case $r = 0$.

\begin{remark}
  Interestingly, the formula \Cref{thm:semi_lie_formula}
  for $\guv \in (S_2(F) \times V'_2(F))\rs$
  actually fits the same template as \Cref{thm:summary},
  although the semi-Lie formula is more pleasant.
  We do not have a good explanation why the shapes of the orbital integrals
  end up being so similar.
\end{remark}

We were also able to determine the
relevant base changes in \Cref{ch:satake}.
However, we did not complete the comparison on the geometric side in this situation.
Thus we do not claim a proof of $n = 3$ of \Cref{conj:inhomog},
although we imagine such a proof could be completed
once a method for determining the intersection numbers explicitly is devised.
On the other hand, the orbital data is enough to prove the following result.
\begin{theorem}
  \label{thm:no_kernel_group}
  There is no nontrivial $\phi \in \HH(S_3(F))$ such that
   \[ \left. \pdv{}{s} \right\rvert_{s=0} \Orb \left( \gamma, \phi, s \right) = 0 \]
   holds for every $\gamma \in S_3(F)\rs$ pairing to an element of $\UU(\VV_3^-)\rs$.
\end{theorem}

\section{Roadmap}
The rest of the paper is organized as follows.

\begin{itemize}
  \ii The paper begins with some general background information.
  \begin{itemize}
    \ii In \Cref{ch:background} we provide some preliminary background
    on the spaces appearing in the overall paper and the Hecke modules that will be used.
    \ii Further background is stated in \Cref{ch:rs_matching},
    where we describe the matching of regular semisimple elements.
    \ii In \Cref{ch:satake} we provide reminders on the Satake transform.
    We also derive concrete formulas for base change when $n = 3$
    (in comparison, the analogous results for $n=2$ are
    \cite[Lemma 7.1.1]{ref:AFLspherical}),
    but these formulas are not used again later on.
  \end{itemize}

  \ii We then proceed to introduce the orbital integrals.
  \begin{itemize}
    \ii In \Cref{ch:orbital0} we introduce the weighted orbital integral
    for the group version of the AFL for full spherical Hecke algebra,
    and state the derivative explicitly for $n = 3$ in terms of a representative.
    In \Cref{ch:orbital1,ch:orbital2} we show the calculation of this formula,
    and give the full version of \Cref{thm:summary}.

    \ii In \Cref{ch:orbitalFJ0} we introduce the weighted orbital integral
    for the semi-Lie version of the AFL for full spherical Hecke algebra.
    The analogous calculation is in \Cref{ch:orbitalFJ1,ch:orbitalFJ2},
    which is used to prove \Cref{thm:semi_lie_formula} and its corollaries.
  \end{itemize}

  \ii Having completely computed the orbital integrals in these cases,
  we take a side trip in \Cref{ch:ker} to prove the ``large kernel'' results.
  We establish \Cref{conj:kernel_semi_lie} for $n = 2$
  by proving the explicit results
  \Cref{thm:semi_lie_ker_trivial} and \Cref{thm:semi_lie_finite_codim_full}.
  We also prove the existing conjecture for the group AFL for $n=3$.

  \ii We then turn our attention to the other parts of the two versions of the AFL.
  In addition to stating the relevant definitions,
  the subsequent chapters aim to prove \Cref{thm:semi_lie_n_equals_2}.
  \begin{itemize}
    \ii In \Cref{ch:transf} we define the transfer factors $\omega \in \{\pm1\}$.
    \ii In \Cref{ch:geo}, we describe the Rapoport-Zink spaces
    that the geometric side is based on, and define the intersection numbers
    on both sides of the AFL.
    \ii In \Cref{ch:jiao}, we specialize to the situation $n = 2$
    for the intersection numbers in the semi-Lie AFL only.
    The Rapoport-Zink spaces become replaced with Lubin-Tate ones,
    and we introduce the Gross-Keating formula
    that will be our primary tool for the calculation.
    \ii In \Cref{ch:finale} we tie everything together and establish
    \Cref{thm:semi_lie_n_equals_2}.
  \end{itemize}
\end{itemize}

An approximate dependency chart between the chapters is also given in Figure~\ref{fig:depchart}.

\begin{figure}[ht]
  \begin{center}
    \begin{tikzcd}
      \text{\ref{ch:background}. Background} \ar[rd, bend left = 10] \ar[dddd, bend left = 80] && \\
      \text{\ref{ch:satake}. Base change} \ar[rd, bend left = 10] \ar[rdddd, bend left]
      & \text{\ref{ch:rs_matching}. Matching} \ar[ld] \ar[rd] \ar[l] \ar[r]
      &  \text{\ref{ch:transf}. Transfer} \ar[ldddd, bend right] \\
      \text{\ref{ch:orbital0}. Group derivative} \ar[r] \ar[d]
        & \text{\ref{ch:ker}. Large kernel} &
        \text{\ref{ch:orbitalFJ0}. Semi-Lie derivative} \ar[l] \ar[d] \ar[lddd, bend right = 20]  \\
      \text{\ref{ch:orbital1}+\ref{ch:orbital2}. Group full orbital} &&
      \text{\ref{ch:orbitalFJ1}+\ref{ch:orbitalFJ2}. Semi-Lie full orbital} \\
      \text{\ref{ch:geo}. Int numbers} \ar[d] && \\
      \text{\ref{ch:jiao}. Gross-Keating} \ar[r] & \text{\ref{ch:finale}. Prove \Cref{thm:semi_lie_n_equals_2}}
    \end{tikzcd}
  \end{center}
  \caption{Dependency chart of the chapters in this paper,
    arranged to loosely resemble the Batman logo.}
  \label{fig:depchart}
\end{figure}


\chapter{General background}
\label{ch:background}

\section{Notation}
We provide a glossary of notation that will be used in this paper.
As mentioned in the introduction, $p > 2$ is a prime,
$F$ is a finite extension of $\QQ_p$,
and $E/F$ is an unramified quadratic field extension.

\begin{itemize}
  \ii For any $a \in E$, we let $\bar a$ denote the image of $a$
  under the nontrivial automorphism of $\Gal(E/F)$.
  (Hence $a = \bar a$ exactly when $a \in F$.)
  \ii Fix $\eps \in \OO_F^\times$ such that $E = F[\sqrt{\eps}]$.
  \ii Denote by $\varpi$ a uniformizer of $\OO_F$, such that $\bar \varpi = \varpi$.
  \ii Let $q \coloneqq |\OO_F/\varpi|$ be the residue characteristic.
  (Hence $|\OO_E / \varpi| = q^2$.)
  \ii Let $v$ be the associated valuation for $\varpi$.
  \ii Let $\eta$ be the quadratic character attached to $E/F$ by class field theory,
  so that $\eta(x) = -1^{v(x)}$.
  \ii $\VV_n^-$ denotes a split $E/F$-Hermitian space of dimension $n$ (unique up to isomorphism).
  \ii Let $\beta$ denote the $n \times n$ antidiagonal matrix
  \[ \beta \coloneqq \begin{bmatrix} && 1 \\ & \iddots \\ 1 \end{bmatrix} \]
  and pick the basis of $\VV_n^-$ such that the Hermitian form on $\VV_n^-$ is given by
  \[ \VV_n^- \times \VV_n^- \to E \qquad (x,y) \mapsto x^\ast \beta y. \]
  \ii Set
  \[ \U(\VV_n^-) = \{ g \in \GL_n(\OO_E) \mid g^\ast \beta g = \beta\} \]
  the unitary group over $\VV_n^-$.
  Note that $\beta$ is \emph{antidiagonal}, in contrast to the convention $\beta = \id_n$
  that is often used for unitary matrices with entries in $\CC$.
  The natural hyperspecial maximal compact subgroup
  is \[ \U(\VV_n^-) \cap \GL_n(\OO_E). \]
  \ii Let $K' = \GL_n(\OO_E)$ denote the hyperspecial maximal compact subgroup of $G' \coloneqq \GL_n(E)$.
  \ii Let $\VV_n^+$ denote the non-split $E/F$-Hermitian space of dimension $n$
  (unique up to isomorphism), and $\U(\VV_n^+)$ the corresponding unitary group.
  This space will be realized in \Cref{ch:geo}.
\end{itemize}

\section{The spaces $S_n(F)$ and $S_n(F) \times V'_n$}
For the analytic side of the two AFL conjectures we investigate,
the following two spaces will be used as inputs to our weighted orbital integrals.
\begin{definition}
  [{\cite[(4.10)]{ref:highdim2024}}]
  We define the symmetric space
  \[ S_n(F) \coloneqq \left\{ g \in \GL_n(E) \mid g \bar g = \id_n \right\}. \]
  It has a natural left action of $\GL_n(E)$ by
  \begin{align*}
    \GL_n(E) \times S_n(F) &\to S_n(F) \\
    g \cdot \gamma &\coloneqq g \gamma \bar g^{-1}.
  \end{align*}
\end{definition}

\begin{definition}
  [{\cite[(4.11)]{ref:highdim2024}}]
  We set
  \[ V'_n(F) \coloneqq F^n \times (F^n)^\vee \]
  where $-^\vee$ denotes the $F$-dual space, i.e., $(F^n)^\vee = \Hom_F(F^n, F)$.
  Then we may also consider the augmented space
  \[ S_n(F) \times V'_n(F) \]
  If we identify $F^n$ with column vectors of length $n-1$ and $(F^n)^\vee$
  with row vectors of length $n$ then we have a left action of $\GL_n(F)$ by
  \begin{align*}
    \GL_n(F) \times (S_n(F) \times V'_n)(F)
    &\to S_n(F) \times V'_n(F) \\
    h \cdot (\gamma, \uu, \vv^\top)
    &\coloneqq (h \gamma h^{-1}, h\uu, \vv^\top h^{-1}).
  \end{align*}
  Note that according to the embedding
  \begin{align*}
    S_n(F) \times V'_n(F)
    &\hookrightarrow \GL_{n+1}(E) \\
    (\gamma, \uu, \vv^\top)
    &\mapsto \begin{bmatrix} \gamma & \uu \\ \vv^\top & 0 \end{bmatrix}
  \end{align*}
  we can consider elements of $S_n(F) \times V'_n(F)$ as elements of $\GL_{n+1}(E)$ too.
  In that case the action of $h \in \GL_{n+1}(F)$
  coincides with $h \cdot g \mapsto hg\bar{h}^{-1}$ as well.
\end{definition}

\begin{definition}
  For brevity, let
   \[ K'_S \coloneqq S_n(F) \cap \GL_n(\OO_F). \]
\end{definition}

\section{Definition of Hecke algebra}
We reminder the reader the definition of a Hecke algebra.
For this subsection, $G$ will denote \emph{any}
unimodular locally compact topological group,
and $K$ any closed subgroup of $G$.

\begin{definition}
  The \emph{Hecke algebra}
  \[ \HH(G, K) \coloneqq \QQ[K \backslash G \slash K] \]
  is defined as the space of compactly supported $K$-bi-invariant
  locally constant functions on $G$.

  Given two such functions $f_1$ and $f_2$ in $\HH(G,K)$,
  one can consider define the convolution
  \[ (f_1 \ast f_2)(g) \coloneqq \int_G f_1(g\inv x) f_2(x) \; \odif x \]
  which makes $\HH(G, K)$ into a $\QQ$-algebra,
  whose identity element is $\mathbf{1}_K$.
\end{definition}

In other sources, this may be denoted $\HH_K(G)$ or even just $\HH_K$.
(So what is written in $\HH_{K'^\flat \otimes K'}$ in other places
will be written as $\HH(G'^\flat \otimes G', K'^\flat \otimes K')$ here).

In the case where $G$ is a reductive Lie group and
$K$ is the maximal compact subgroup
(or more generally whenever $(G,K)$ is a Gelfand pair),
this Hecke algebra is actually commutative.

\section{The specific Hecke algebras $\HH(\GL_n(E))$ and $\HH(\U(\VV_n^-))$ and the module $\HH(S_n(F))$}
For our purposes, we define shorthands for two specific Hecke algebras
that will come up consistently:
\begin{align*}
  \HH(\GL_n(E)) &\coloneqq \HH(\GL_n(E), \GL_n(\OO_E)) \\
  \HH(\U(\VV_n^-)) &\coloneqq \HH(\U(\VV_n^-), \U(\VV_n^-) \cap \GL_n(\OO_E))
\end{align*}
Note that $\GL_n(\OO_E)$ and $\U(\VV_n^-) \cap \GL_n(\OO_E)$
are the natural hyperspecial maximal compact subgroups of $\GL_n(E)$ and $\U(\VV_n^-$),
respectively.

Now the symmetric space $S_n(F)$ is not a group,
so it does not make sense to define the same thing here.
Nevertheless, we introduce
\[ \HH(S_n(F)) \coloneqq \mathcal C_{\mathrm{c}}^{\infty}(S_n(F))^{K'} \]
as the set of smooth compactly supported functions on $S_n(F)$
which are invariant under the action of $K' \subseteq G'$;
this is an $\HH(\GL_n(E))$-module,
where the action of $f \in \HH(\GL_n(E))$ on $\phi \in \HH(S_n(F))$ is given by
\[ (f \cdot \phi)(\gamma) \coloneqq \int_G f(g) \phi(g \cdot \gamma) \odif g \]
for $\gamma \in S_n(F)$.
This does \textbf{not} have a multiplication structure at the moment, \emph{a priori}.

Throughout this paper, to be consistent with the notation, we denote
\begin{itemize}
  \ii elements of $\HH(\U(\VV_n^-))$ using $f$ or $f_i$ or similar
    (i.e.\ lowercase Roman letters);
  \ii elements of $\HH(\GL_n(E))$ by $f'$ or $f'_i$ or similar
    (i.e.\ lowercase Roman letters with apostrophes);
  \ii elements of $\HH(S_n(F))$ by $\phi$ or $\phi'_i$
    (i.e.\ lowercase Greek letters).
\end{itemize}

%Similarly, consider $S_n(F) \times V_n$.
%We set
%\[ \HH(S_n(F) \times V_n', {?}) \coloneqq \mathcal C_{\mathrm{c}}^{\infty}(S_n(F) \times V_n')^{K'} \]
%as the set of smooth compactly supported functions on $S_n(F) \times V_n'$
%which are invariant under the action of ${?} \subseteq G'$.
%\todo{This should be $\GL_n(\OO_F)$? Does it need another name?}
%This is an \dots

\section{Arches}
We introduce one more piece of notation for a common shape that our answers will take.

\begin{definition}
  Suppose $\{a_0, a_0 + 1, \dots, a_1\}$ is an interval of integers for some $a_0 \le a_1$,
  and consider two more integers $w_1$ and $w_2$ such that $w_1 + w_2 \le \frac{a_1-a_0}{2}$.
  Then we can define a piecewise linear function
  \[ \Arch_{[a_0, a_1]}(w_1, w_2) \colon \{a_0, a_0+1, \dots, a_1\} \to \ZZ_{\ge 0} \]
  according to the following definition:
  \[
    k \mapsto
    \begin{cases}
      k - a_0 & \text{if }a_0 \le k \le a_0 + w_1 \\
      w_1 + \left\lfloor \frac{k-(a_0+w_1)}{2} \right\rfloor & \text{if } a_0 + w_1 \le k \le a_0 + w_1 + w_2 \\
      w_1 + \left\lfloor \frac{w_2}{2} \right\rfloor & \text{if } a_0 + w_1 + w_2 \le k \le a_1 - (w_1 + w_2)\\
      w_1 + \left\lfloor \frac{(a_1-w_1) - k}{2} \right\rfloor & \text{if } a_1 - (w_1 + w_2) \le k \le a_1 - w_1 \\
      a_1 - k & \text{if }a_1 - w_1 \le k \le a_1.
    \end{cases}
  \]
\end{definition}
The nomenclature is meant to be indicative of the shape of the graph,
which looks a little bit like an arch.
It is a function symmetric around $\frac{a_0+a_1}{2}$ defined piecewise.
The function grows linearly with slope $1$ at the far left for $w_1$ steps,
then changes to slope $1/2$ for $w_2$ steps (rounding down),
before stabilizing, then doing the symmetric descent on the right half.
\begin{figure}[ht]
  \begin{center}
  \begin{asy}
    size(12cm);
    draw((-1,0)--(20,0));
    draw((0,0)--(3,3)--(7,5)--(12,5)--(16,3)--(19,0), lightred);
    draw((3,3)--(3,0), grey);
    draw((7,5)--(7,0), grey);
    draw((12,5)--(12,0), grey);
    draw((16,3)--(16,0), grey);
    real eps = 0.3;
    void brack(string s, real x0, real x1) {
      draw((x0+0.1,-eps)--(x0+0.1,-2*eps)--(x1-0.1,-2*eps)--(x1-0.1,-eps), blue);
      label(s, ((x0+x1)/2, -2*eps), dir(-90), blue);
    }
    brack("$w_1 = 3$", 0, 3);
    brack("$w_2 = 4$", 3, 7);
    brack("$w_2 = 4$", 12, 16);
    brack("$w_1 = 3$", 16, 19);

    dotfactor *= 1.5;
    dot("$(0,0)$", (0,0), dir(225));
    dot((1,1), red);
    dot((2,2), red);
    dot("$(3,3)$", (3,3), dir(135));
    dot((4,3), red);
    dot((5,4), red);
    dot((6,4), red);
    dot("$(7,5)$", (7,5), dir(90));
    dot((8,5), red);
    dot((9,5), red);
    dot((10,5), red);
    dot((11,5), red);
    dot("$(12,5)$", (12,5), dir(90));
    dot((13,4), red);
    dot((14,4), red);
    dot((15,3), red);
    dot("$(16,3)$", (16,3), dir(45));
    dot((17,2), red);
    dot((18,1), red);
    dot("$(19,0)$", (19,0), dir(315));

    label(rotate(45)*"Slope $+1$", (1.5,1.5), dir(135), lightred);
    label(rotate(26.57)*"Slope $+\frac12$", (5,4), dir(125), lightred);
    label(rotate(-26.57)*"Slope $-\frac12$", (14,4), dir(55), lightred);
    label(rotate(-45)*"Slope $-1$", (17.5,1.5), dir(45), lightred);
    label("Slope $0$", (9.5,5), dir(90), lightred);
  \end{asy}
  \end{center}
  \caption{A plot of $\Arch_{[0,19]}(3,4)$.}
  \label{fig:arch}
\end{figure}

\chapter{Regular semi-simplicity and matching}
\label{ch:rs_matching}

\section{Regular semi-simple elements}
We first recall the notion of regularity
that first appeared in \cite[\S6]{ref:multoneconj}.

\begin{definition}
  \label{def:rs}
  Consider a $n \times n$ matrix
  \[ \begin{bmatrix} A & \uu \\ \vv^\top & d \end{bmatrix} \in \GL_n(E) \]
  in $\GL_n(E)$, where $A$ is an $(n-1) \times (n-1)$ matrix.
  Then we say this matrix is \emph{regular semi-simple} if
  \[ \left< \uu, A\uu, \dots, A^{n-2}\uu \right> \]
  and \[ \left< \vv^\top, \vv^\top A, \dots, \vv^\top A^{n-2} \right> \]
  are each a basis of $E^{n-1}$.
  Equivalently, the matrix
  \[ \left[ \vv^\top A^{i+j-2} \uu \right]_{i,j=1}^{n-1} \]
  should be nonsingular.
\end{definition}

\begin{remark}
  In \cite[Theorem 6.1]{ref:multoneconj}, this definition is shown to be equivalent to
  requiring that, under the action of conjugation by $\GL_{n-1}(E)$:
  \begin{itemize}
  \ii the matrix has trivial stabilizer; and
  \ii the $\GL_{n-1}(\ol E)$-orbit is a Zariski-closed subset of $\GL_n(\ol E)$.
  \end{itemize}
  Here $\ol E$ is as usual an algebraic closure of $E$.
\end{remark}

\begin{remark}
  [{\cite[Proposition 6.2]{ref:multoneconj}}]
  It turns out we can detect whether two regular semisimple elements
  \[
    \begin{bmatrix} A_1 & \uu_1 \\ \vv_1^\top & d_1 \end{bmatrix},
    \begin{bmatrix} A_2 & \uu_2 \\ \vv_2^\top & d_2 \end{bmatrix}
    \in \GL_{n}(E)
  \]
  are conjugate by an element of $\GL_{n-1}(E)$.
  This happens if and only if the following conditions all hold:
  \begin{itemize}
    \ii The matrices $A_1$ and $A_2$ have the same characteristic polynomial;
    \ii We have $\vv_1^\top A_1^i \uu_1 = \vv_2^\top A_2^i \uu_2$
    for every $i = 0, 1, \dots, n-2$; and
    \ii We have $d_1 = d_2$.
  \end{itemize}
  Thus, this gives a set of invariants that completely classify the orbits
  under the action of $\GL_{n-1}(E)$.

  Put another way, the invariants of
  \[ \begin{bmatrix} A & \uu \\ \vv^\top & d \end{bmatrix} \in \GL_n(E) \]
  are the (monic) characteristic polynomial of $A$
  (which has $n-1$ coefficients besides the leading coefficient),
  the values of $\vv^\top A^i \uu$ for $i = 0, \dots, n-2$
  and the number $d$ --- for a total of $2n-1$ numbers.
  \label{rem:invariants}
\end{remark}

We can now speak of regular-simplicity in each of the four
particular cases relevant to this paper.
\begin{definition}
  In the group version of the AFL:
  \begin{itemize}
    \ii We say $\gamma \in S_n(F)$ is regular semisimple
    if its image under the inclusion $S_n(F) \subseteq \GL_n(E)$ is regular semisimple.
    We write $\gamma \in S_n(F)\rs$.

    \ii For $g \in \U(\VV_n^\pm)$,
    we say $g$ is regular semisimple
    if its image under the inclusion $\U(\VV_n^\pm) \subseteq \GL_n(E)$ is regular semisimple.
    We write $g \in \U(\VV_n^\pm)\rs$.
  \end{itemize}
  In the semi-Lie version of the AFL:
  \begin{itemize}
    \ii We say $(\gamma, \uu, \vv^\top) \in S_n(F) \times V_n'$
    is regular semisimple if its image under the embedding
    \begin{equation}
      \begin{aligned}
        S_n(F) \times V_n' &\hookrightarrow \GL_{n+1}(E) \\
        (\gamma, \uu, \vv^\top) &\mapsto \begin{bmatrix} \gamma & \uu \\ \vv^\top & 0 \end{bmatrix}
      \end{aligned}
      \label{eq:embed_FJ_analytic}
    \end{equation}
    is regular semisimple.
    In other words, we require that
    both of the sets
    $\left( \uu, \gamma \uu \dots, \gamma^{n-1}\uu \right)$
    and
    $\left( \vv^\top, \vv^\top \gamma, \dots, \vv^\top \gamma^{n-1} \right)$
    are bases of $E^n$.
    In this case we write $(\gamma, \uu, \vv^\top) \in (S_n(F) \times V_n')\rs$.

    \ii For $(g,u) \in \U(\VV_n^\pm) \times \VV_n^\pm$ we say $(g, u)$
    is regular semisimple if its image under the embedding
    \begin{equation}
      \begin{aligned}
        \U(\VV_n^\pm) \times \VV_n^\pm &\hookrightarrow \GL_{n+1}(E) \\
        (g, u) &\mapsto \begin{bmatrix} g & u \\ u^\ast & 0 \end{bmatrix}
      \end{aligned}
      \label{eq:embed_FJ_geometric}
    \end{equation}
    is regular semisimple; here $u^\ast$ is the conjugate transpose.
    This is equivalent to the set $\left(  u, gu, \dots, g^{n-1}u \right)$
    being linearly independent (i.e.\ form a basis of $\VV_n^\pm$);
    in this case the independence of $\left( u^\ast, u^\ast g, \dots, u^\ast g^{n-1} \right)$
    is redundant, so it's enough to verify one condition.
    We write $(g,u) \in (\U(\VV_n^\pm) \times \VV_n^\pm)\rs$.
  \end{itemize}
  \label{def:regular}
\end{definition}

\section{Matching in the group version of the inhomogeneous AFL}
We now describe the matching condition used in the group version of AFL.
\begin{definition}
  We say $\gamma \in S_n(F)\rs$ matches the element $g \in \U(\VV_n^\pm)\rs$
  if $g$ is conjugate to $\gamma$ by an element of $\GL_{n-1}(E)$.
  By \Cref{rem:invariants}, this is an assertion that
  the invariants for $\gamma$ and $g$ coincide.
  \label{def:matching_inhomog}
\end{definition}
In that case, we have the following result.
\begin{proposition}
  [{\cite[Lemma 2.3]{ref:AFL}}; see also {\cite[(3.3.2)]{ref:AFLspherical}}]
  \label{prop:valuation_delta_matching_group}
  This definition of matching gives
  a bijection of regular semisimple orbits
  \[ [S_n(F)]\rs \xrightarrow{\sim} [\U(\VV_n^+)]\rs \amalg [\U(\VV_n^-)]\rs. \]
  Moreover, we can detect whether $\gamma \in S_n(F)\rs$ matches an orbit of
  $\U(\VV_n^+)\rs$ or $\U(\VV_n^-)\rs$ as follows.
  Suppose we write $\gamma$ in the format of \Cref{def:rs} and consider
  \[ \Delta \coloneqq \det \left[ \vv^\top A^{i+j-2} \uu \right]_{i,j=1}^{n-1} \neq 0. \]
  Then
  \begin{itemize}
    \ii $\gamma$ matches an orbit in $\U(\VV_n^+)\rs$ if $v(\Delta)$ is even;
    \ii $\gamma$ matches an orbit in $\U(\VV_n^-)\rs$ if $v(\Delta)$ is odd.
  \end{itemize}
\end{proposition}
In this paper, \Cref{conj:inhomog}
requires that $\gamma$ should match an element of $\U(\VV_n^-)\rs$
and consequently we will usually only be interested in the latter case.

\section{Matching in the semi-Lie version of the AFL}
For the semi-Lie version matching is defined analogously:
\begin{definition}
  [{\cite[\S1.3]{ref:liuFJ}}]
  We say $(\gamma, \uu, \vv^\top) \in (S_n(F) \times V_n')\rs$
  matches the element $(g, u) \in (\U(\VV_n^\pm) \times \VV_n^\pm)\rs$ if
  their images under the embeddings \eqref{eq:embed_FJ_analytic}
  and \eqref{eq:embed_FJ_geometric} are conjugate by an element of $\GL_n(E)$.
  Unwrapping this with \Cref{rem:invariants},
  an equivalent definition is to require both of the following conditions:
  \begin{itemize}
    \ii As elements of $\GL_n(E)$,
    both $g$ and $\gamma$ have the same characteristic polynomial.
    \ii We have $\vv^\top \gamma^i \uu = \left< g^i u, u \right>$ for all $0 \le i \le n-1$,
    where $\left< -,- \right>$ is the Hermitian form on $\VV_n^\pm$.
  \end{itemize}
  \label{def:matching_semi_lie}
\end{definition}
We have the following analogous criteria for matching.
\begin{proposition}
  [\cite{ref:liuFJ}]
  \label{prop:valuation_delta_matching_semilie}
  This definition of matching gives a bijection of regular semisimple orbits
  \[ [S_n(F) \times V_n']\rs \xrightarrow{\sim} [\U(\VV_n^+) \times \VV_n^+]\rs \amalg [\U(\VV_n^-) \times \VV_n^-]\rs. \]
  Moreover, we can detect whether $\guv \in S_n(F)\rs$ matches an orbit of
  $(\U(\VV_n^+) \times \VV_n^+)\rs$ or $(\U(\VV_n^-) \times \VV_n^-)\rs$ as follows:
  consider the determinant
  \[ \Delta \coloneqq \det \left[ \vv^\top \gamma^{i+j-2} \uu \right]_{i,j=1}^n \neq 0. \]
  Then
  \begin{itemize}
    \ii $\gamma$ matches an orbit in $(\U(\VV_n^+) \times \VV_n^+)\rs$ if $v(\Delta)$ is even;
    \ii $\gamma$ matches an orbit in $(\U(\VV_n^-) \times \VV_n^-)\rs$ if $v(\Delta)$ is odd.
  \end{itemize}
\end{proposition}
In this paper, \Cref{conj:semi_lie_spherical}
requires that $\guv$ should match an element of $(\U(\VV_n^-) \times \VV_n^-)\rs$
and consequently we will usually only be interested in the latter case.

\chapter{Base change}
\label{ch:satake}

The goal of this section is to make completely explicit the base change
\[ \BC_{S_n}^{\eta^{n-1}} \colon \HH(S_n(F)) \to \HH(\U(\VV_n^-)) \]
in the special case $n = 3$ (in which case $\BC_{S_n}^{\eta^{n-1}} = \BC_{S_n}$ as $\eta^2 = 1$).
Note that the case $n = 2$ was already done in \cite[\S7.1]{ref:AFLspherical}.
When it is not more difficult, some of the results will be stated for all $n$,
rather than $n = 3$ specifically.

Throughout this section we have the following additional notation.
\begin{definition}
  Denote by $\rproj \colon \GL_n(E) \surjto S_n(F)$ the projection defined by
  \[ \rproj(g) \coloneqq g \bar{g}\inv. \]
\end{definition}
We also let $\Sym(n)$ denotes the symmetric group in $n$ variables with order $n!$
(since $S_n(F) \subseteq \GL_n(E)$ is already reserved for the symmetric space).

\section{Background on the Satake transformation in transformation}
We recall a general form of the Satake transformation, which will be used later.

For this subsection, $G$ will denote an arbitrary connected reductive group
over some non-Archimedean local field $F$.
We will not distinguish between $G$ and $G(F)$ when there is no confusion.

To simplify things, we will assume $G$ is unramified;
but we do \emph{not} assume $G$ is split.
Introduce the following notation:
\begin{itemize}
  \ii Let $K$ be a hyperspecial maximal compact subgroup of $G$
  (it exists because $G$ is unramified).
  \ii Let $A$ denote a maximal $F$-split torus in $G$.
  All the maximal $F$-split tori in $G$ are conjugate; let $A$ denote one of them.
  \ii Let $M$ be the centralizer of $A$; this is itself a maximal torus in $G$.
  \ii Let $\prescript{\circ}{} M \coloneqq M(F) \cap K$
  be the maximal compact subgroup of $M$.
  \ii Let $P$ denote a minimal $F$-parabolic containing $A$.
  \ii Let $\delta$ denotes the modulus character of $P$.
  It can be describes as follows.
  Let $\varpi$ denote a uniformizer for $F$ and $q$ the residue characteristic.
  Then if $\rho$ is the Weyl vector and $\mu$ is a positive cocharacter, then
  \[ \delta(\mu(\varpi)) = q^{- \left< \mu, \rho\right>}. \]
  \ii Let $N$ denote the unipotent radical of $P$.
  \ii Let $W$ be the relative Weyl group for the pair $(G,A)$,
  which acts on $\HH(M, \prescript{\circ}{} M)$.
\end{itemize}
We can now state the Satake isomorphism.
\begin{definition}
  The \emph{Satake transform} is a canonical isomorphism of Hecke algebras
  \[ \Sat \colon \HH(G, K) \to \HH(M, \prescript{\circ}{} M)^W \]
  which is given by defining
  \[ (\Sat(f))(t) \coloneqq \delta(t)^\half \int_N f(nt) \odif n  \]
  for each $t \in M$.
\end{definition}
We are going to apply this momentarily in two situations:
once when $G$ is the general linear group (which is split),
and once when $G$ is a unitary group.


\section{The Satake transformation for $\HH(\GL_n(E))$ and $\HH(\U(\VV_n^-))$}
To take the Satake transform of $\HH(\U(\VV_n^-))$, we define the following abbreviations.
\begin{itemize}
  \ii Let $T$ denote the split diagonal torus of $\GL_n$.
  \ii Let
  \[ N' \coloneqq \left\{ \begin{bmatrix}
      1 & \ast & \dots & \ast \\
        & 1 & \dots & \ast \\
        &   & \ddots & \vdots \\
        &   &   & 1 \end{bmatrix}\right\} \subseteq \GL_n(E) \]
  denote the unipotent upper-triangular matrices.
\end{itemize}
Similarly for $\HH(\U(\VV_n^-))$:
\begin{itemize}
  \ii Set $m \coloneqq \left\lfloor n/2 \right\rfloor$ for brevity.
  \ii Let
  \[ A \coloneqq \left\{
    \diag(x_1, \dots, x_m, 1_{n-2m}, x_m\inv, \dots, x_1\inv) \right\} \]
  so that $A(F)$ is a maximal $F$-split torus of $\U(\VV_n^-)$.
  \ii Let $N \coloneqq N' \cap G$ denote the unipotent upper triangular matrices
  which are also unitary.
  \ii For brevity, let $W_m \coloneqq (\ZZ/2\ZZ)^m \rtimes \Sym(m)$
  be the relative Weyl group of $(G,A)$.
\end{itemize}

We can now introduce the Satake transform for our two
\emph{bona fide} Hecke algebras, using the data in Table~\ref{tab:satakestuff}.

\begin{table}[ht]
  \centering
  \begin{tabular}{lll}
    \toprule
    Group & $G' = \GL_n(E)$ & $G = \U(\VV_n^-)$ \\ \midrule
    Local field & $E$ & $F$ \\\hline
    Hyperspecial compact & $K' = \GL_n(\OO_E)$ & $K = G \cap \GL_n(\OO_E)$ \\\hline
    Max'l split torus & $T(E)$ & $A(F)$ \\\hline
    Centralizer of split torus & $T(\OO_E)$ & $A(\OO_F)$ \\\hline
    Parabolic (Borel) & Upper tri in $G'$ & Upper tri in $G$ \\\hline
    Unipotent rad.\ of parabolic & $N'$ (unipot.\ upper tri) & $N$ (unipot.\ upper tri) \\\hline
    Relative Weyl group & $\Sym(n)$ & $W_m = (\ZZ/2\ZZ)^m \rtimes \Sym(m)$ \\
    %Cocharacter group & $\ZZ^{\oplus n}$ & $\ZZ^{\oplus m}$?? \\
    %Weyl vector & $\left< \frac{n-1}{2}, \frac{n-3}{2}, \dots, -\frac{n-1}{2} \right>$
    %            & $\left< \frac{m-1}{2}, \frac{m-3}{2}, \dots, -\frac{m-1}{2} \right>$??
    \bottomrule
  \end{tabular}
  \caption{Data needed to run the Satake transformation.}
  \label{tab:satakestuff}
\end{table}

Hence, the Satake transformations obtained can be viewed as
\begin{align*}
  \Sat &\colon \HH(\GL_n(E)) \xrightarrow{\sim} \QQ[T(E) / T(\OO_E)]^{\Sym(n)} \\
  \Sat &\colon \HH(\U(\VV_n^-))\xrightarrow{\sim} \QQ[A(F) / A(\OO_F)]^{W_m}
\end{align*}
(In both cases, the modular character $\delta^{1/2}$ gives rational values,
so it is okay to work over $\QQ$.)

To make this further concrete, we remark that the cocharacter groups
involved are free abelian groups with known bases.
This identification lets us rewrite the right-hand sides above as concrete polynomials.
Specifically, we identify
\[ \QQ[T(E) / T(\OO_E)]^{\Sym(n)}
  \xrightarrow{\sim} \QQ[X_1^\pm, \dots, X_n^\pm]^{\Sym(n)} \]
by identifying $X_i$ with the
cocharacter corresponding to injection into the $i$\ts{th} factor.
Similarly, we identify
\[ \QQ[A(F) / A(\OO_F)]^{W_m}
  \xrightarrow{\sim} \QQ[Y_1^{\pm}, \dots, Y_m^{\pm}]^{W_m} \]
by identifying $Y_i + Y_i^{-1}$
with the cocharacter corresponding to
\[ x \mapsto \diag(1, \dots, x, \dots, x\inv, \dots, 1) \]
where $x$ is in the $i$\ts{th} position and $x\inv$ is in the $(n-i)$\ts{th} position,
and all other positions are $1$.
Here $\QQ[Y_1^{\pm}, \dots, Y_m^{\pm}]^{W_m}$
denotes the ring of symmetric polynomials in $Y_i + Y_i^{-1}$.

So, henceforth, we will consider
\begin{align*}
  \Sat &\colon \HH(\GL_n(E)) \xrightarrow{\sim} \QQ[X_1^\pm, \dots, X_n^\pm]^{\Sym(n)} \\
  \Sat &\colon \HH(\U(\VV_n^-)) \xrightarrow{\sim} \QQ[Y_1^{\pm}, \dots, Y_m^{\pm}]^{W_m}.
\end{align*}

\section{Relation of Satake transformation to base change}
Let
\[ \BC \colon \HH(\GL_n(E)) \to \HH(\U(\VV_n^-)) \]
denote the stable base change morphism from $\GL_n(E)$ to the unitary group $\U$.
The relevance of the Satake transformation is that
(see e.g.\ \cite[Proposition 3.4]{ref:leslie})
it gives a way to make this $\BC$ completely explicit:
we have a commutative diagram
\begin{center}
\begin{tikzcd}
  \HH(\GL_n(E))  \ar[r, "\sim"', "\Sat"] \ar[d, "\BC"]
    & \QQ[X_1^\pm, \dots, X_n^\pm]^{\Sym(n)} \ar[d, "\BC"] \\
  \HH(\U(\VV_n^-)) \ar[r, "\sim"', "\Sat"]
    & \QQ[Y_1^\pm, \dots, Y_m^\pm]^{W_m}
\end{tikzcd}
\end{center}
Here the right arrow is also denoted $\BC$ following \cite{ref:AFLspherical}
(although it is denoted $\nu$ in \cite{ref:leslie}).
This gives a way in which we can concretely calculate the map $\BC$
in some situations.

\section{The map $\BC^{\eta^{n-1}}_{S_n}$}
Before we can define the map $\BC^{\eta^{n-1}}_{S_n}$
we need one more piece of notation.
Our prior map $\rproj \colon \GL_n(E) \surjto S_n(F)$ induces a map
\begin{align*}
  \rproj_\ast \colon \HH(\GL_n(E)) &\to \HH(S_n(F)) \\
  \rproj_\ast(f')\left( g\bar{g}\inv \right) &= \int_{\GL_n(F)} f'(gh) \odif h
\end{align*}
by integration on the fibers.
A similar twisted version by $\eta$
\begin{align*}
  \rproj_\ast^\eta \colon \HH(\GL_n(E)) &\to \HH(S_n(F)) \\
  \rproj_\ast^\eta(f')\left( g\bar{g}\inv \right) &= \int_{\GL_n(F)} f'(gh) \eta(gh) \odif h
\end{align*}
is defined analogously,
where as before $\eta(g) = (-1)^{v(\det g)}$ in a slight abuse of notation.

Then Leslie \cite{ref:leslie} shows the following result.
\begin{theorem}
  [{\cite[Theorem 3.2 and Proposition 3.4]{ref:leslie}}]
  Both maps $\rproj_\ast$ and $\rproj_\ast^\eta$ induce isomorphisms
  \begin{align*}
    \BC_{S_n} \colon \HH(S_n(F)) &\xrightarrow{\sim} \HH(\GL_n(E)) \\
    \BC^{\eta^{n-1}}_{S_n} \colon \HH(S_n(F)) &\xrightarrow{\sim} \HH(\GL_n(E))
  \end{align*}
  such that
  \begin{align*}
    \BC &= \BC_{S_n} \circ \rproj_\ast \\
    \BC &= \BC^{\eta^{n-1}}_{S_n} \circ \rproj_\ast^{\eta^{n-1}}.
  \end{align*}
\end{theorem}
We take these isomorphisms promised by this theorem
as the definition of $\BC_{S_n}$ and $\BC^{\eta^{n-1}}_{S_n}$ in our conjectures
(noting when $n$ is odd they coincide, as $\eta^{n-1} = 1$).

When combined with the Satake information we have, we get the following diagram.
\begin{center}
\begin{tikzcd}
  \HH(\GL_n(E)) \ar[dd, "\rproj_\ast^{\eta^{n-1}}"', bend right = 60] \ar[r, "\sim"', "\Sat"] \ar[d, "\BC"]
    & \QQ[X_1^\pm, \dots, X_n^\pm]^{\Sym(n)} \ar[d, "\BC"] \\
  \HH(\U(\VV_n^-)) \ar[r, "\sim"', "\Sat"]
    & \QQ[Y_1^\pm, \dots, Y_m^\pm]^{W_m} \\
    \HH(S_n(F)) \ar[u, "\sim", "\BC_{S_n}^{\eta^{n-1}}"']
\end{tikzcd}
\end{center}

\section{Calculation of $\BC_{S_3}$}
Specialize now to $n=3$ (thus $m = \left\lfloor n/2 \right\rfloor = 1$).
Our goal is to make completely explicit the arrow $\BC_{S_n}$
in the diagram above.
The completed result is \Cref{prop:BC_S3}.

\subsection{Overview}
Throughout this subsection, we use the shorthand
\[ \varpi^{(n_1, n_2, n_3)} \coloneqq \diag(\varpi^{n_1}, \varpi^{n_2}, \varpi^{n_3}). \]
As a $\QQ$-module, the spaces $\HH(\U(\VV_n^-))$ and $\HH(S_n(F))$
have a canonical basis of indicator functions indexed by $\ZZ$:
\begin{itemize}
  \ii $\HH(S_n(F))$ has $\QQ$-module basis $\mathbf{1}_{K'_{S,j}}$ for $j \ge 0$.
  \ii $\HH(\U(\VV_n^-))$ has a $\QQ$-module basis given by the indicator functions
  \[ \mathbf{1}_{\varpi^{-r} \Mat(\OO_E) \cap \U(\VV_n^-)} \]
  for $r \ge 0$.
\end{itemize}
On the other hand, the natural $\QQ$-module basis for $\HH(\GL_n(E))$, namely
\[ \mathbf{1}_{K'\varpi^{(n_1, n_2, n_3)}K'} \]
is given by triples of integers $n_1 \ge n_2 \ge n_3 \ge 0$, and is much larger.
So explicit calculations for the $\rproj_\ast$ or the Satake transforms viewed in
$\CC[X_1, X_2, X_3]^{\Sym(n)}$ is nontrivial if one works with the entire basis.

Hence the overall strategy, to reduce the amount of work we have to do,
is to focus on only the $\ZZ$-indexed elements
\[
  \mathbf{1}_{\Mat_3(\OO_E), v\circ\det=r}
  = \sum_{\substack{n_1 \ge n_2 \ge n_3 \\ n_1 + n_2 + n_3 = r}}
  \mathbf{1}_{K'\varpi^{(n_1, n_2, n_3)}K'} \in \HH(\GL_n(E)).
\]
This aggregated indicator function is easier to compute,
because given an explicit matrix it is somewhat easier
to evaluate \[ \mathbf{1}_{\Mat_3(\OO_E), v\circ\det=r} \]
at it (one only needs to check it has $\OO_E$ entries
and that the determinant has valuation $r$,
rather than determining the exact coset $K'\varpi^{(n_1, n_2, n_3)}K'$).

\subsection{Satake transform of the determinant characteristic function on the top arrow}
This is the easiest calculation, and we do it for all $n$ rather than just $n = 3$.
\begin{proposition}
  For every integer $r \ge 0$, we have
  \[ \Sat(\mathbf{1}_{\Mat_n(\OO_E), v\circ\det=r})
    = q^{(n-1)r} \sum_{e_1 \dots + e_n = r} X_1^{e_1} \dots X_n^{e_n}. \]
\end{proposition}
\begin{proof}
  We evaluate the coefficient $X_1^{e_1} \dots X_n^{e_n}$.
  Choose a cocharacter $\mu$,
  and suppose $\mu(\varpi) = \varpi^{(e_1, \dots, e_n)}$ with $n_1 \ge n_2 \ge n_3$.
  Let $q_E = q^2$ be the residue characteristic of $E$.
  Take the upper triangular matrices as our Borel subgroup as usual,
  so the unipotent radical of this Borel subgroup
  are the unipotent upper triangulars $N'$ which we describe as
  \[ N' \coloneqq \left\{
      \begin{bmatrix}
      1 & y_{12} & y_{13} & \dots & y_{1n} \\
        & 1 & y_{23} & \dots & y_{2n} \\
        &   & 1 & \dots & y_{3n} \\
        &   &   & \ddots & \vdots  \\
        &   &   &   & 1
      \end{bmatrix}
    \mid y_{12}, \dots, y_{(n-1)n}\in E \right\} \]
  and with additive Haar measure is $\odif{y_{12}, y_{23} \dotso, y_{(n-1)n}}$.
  Recall also the Weyl vector for $\GL_n(E)$ is just
  \[ \rho_{\GL_n(E)} = \left< \frac{n-1}{2}, \frac{n-3}{2}, \dots, -\frac{n-1}{2} \right>. \]
  Compute
  \begin{align*}
    &\Sat(\mathbf{1}_{\Mat_n(\OO_E), v\circ\det=r})(\mu(\varpi)) \\
    &= \delta(\mu(\varpi))^\half \int_{n' \in N'}
      \mathbf{1}_{\Mat_n(\OO_E, v\circ \det = r)} (\mu(\varpi) n') \odif{n'} \\
    &= q_E^{-\left< \mu, \rho\right>}
    \underbrace{\int_{y_{12} \in E} \int_{y_{13} \in E} \dotso \int_{y_{(n-1)n} \in E}}_{\binom n2 \text{ integrals}} \\
    &\qquad
      \mathbf{1}_{\Mat_3(\OO_E), v \circ \det = r}
      \left( \begin{bmatrix}
        \varpi^{e_1} & \varpi^{e_1} y_{12} & \varpi^{e_1} y_{13} & \dots & \varpi^{e_1} y_{1n} \\
        & \varpi^{e_2} & \varpi^{e_2} y_{23} & \dots & \varpi^{e_2} y_{2n} \\
        &   & \varpi^{e_3} & \dots & \varpi^{e_3} y_{3n} \\
        &   &   & \ddots & \vdots  \\
        &   &   &   & \varpi^{e_n}
        \end{bmatrix} \right) \\
    &\qquad \odif{y_{12}, y_{23} \dotso, y_{(n-1)n}} \\
    &= q_E^{-\left(\frac{n-1}{2}e_1 + \frac{n-3}{2}e_2 + \dots + -\frac{n-1}{2} e_n \right)}
    \mathbf{1}_{e_1 + \dots + e_n = r}
    \underbrace{\int_{y_{12} \in E} \int_{y_{13} \in E} \dotso \int_{y_{(n-1)n} \in E}}_{\binom n2 \text{ integrals}} \\
    &\qquad \prod_{1 \le i < j \le n} \mathbf{1}_{\OO_E}(\varpi^{e_i} y_{ij}) \odif{y_{ij}} \\
    &= q_E^{-\left(\frac{n-1}{2}e_1 + \frac{n-3}{2}e_2 + \dots + -\frac{n-1}{2} e_n \right)}
    \mathbf{1}_{e_1 + \dots + e_n = r} \prod_{1 \le i < j \le n} q_E^{e_i} \\
    &= q_E^{-\left(\frac{n-1}{2}e_1 + \frac{n-3}{2}e_2 + \dots + -\frac{n-1}{2} e_n \right)}
    \mathbf{1}_{e_1 + \dots + e_n = r} \prod_{1 \le i \le n} q_E^{(n-i)e_i} \\
    &= \mathbf{1}_{e_1 + \dots + e_n = r} \prod_{1 \le i \le n}^n q_E^{\frac{n-1}{2} e_i} \\
    &= q_E^{\frac{n-1}{2} r} \mathbf{1}_{e_1 + \dots + e_n = r} \\
    &= \begin{cases}
      q^{\frac{n-1}{2} r} & \text{if } e_1 + \dots + e_n = r \\
      0 & \text{otherwise}.
    \end{cases}
  \end{align*}
  This gives the sum claimed earlier.
\end{proof}


\subsection{Satake transform of the indicator on the bottom arrow}
\begin{proposition}
  For each $r \ge 0$ we have
  \[ \Sat\left(\mathbf{1}_{\varpi^{-r} \Mat_3(\OO_E) \cap \U(\VV_3^-)}\right)
    = \sum_{i=0}^r q^{2r - \mathbf{1}_{r \equiv i \bmod 2}} Y_1^{\pm i} \]
  where we adopt the shorthand
  \[
    Y_1^{\pm i} \coloneqq
    \begin{cases}
      Y_1^i + Y_1^{-i} & i > 0 \\
      1 & i = 0 .
    \end{cases}
  \]
\end{proposition}
\begin{proof}
  We first need to describe \[ N = N' \cap \U(\VV_3^-) \] a little more carefully.
  For $n \in N'$ we have
  \[
    n^\ast \beta n
    =
    \begin{bmatrix} 1 \\ \bar{y_1} & 1 \\ \bar{y_2} & \bar{y_3} & 1 \end{bmatrix}
    \beta
    \begin{bmatrix}
      1 & y_1 & y_2 \\
        & 1 & y_3 \\
        & & 1
    \end{bmatrix}
    = \begin{bmatrix}
      & & 1 \\
      & 1 & y_3 + \bar{y_1} \\
      1 & y_1 + \bar{y_3} & y_2 + \bar{y_2} + y_3 \bar{y_3}
    \end{bmatrix}.
  \]
  So $n \in N$ if and only if the above matrix equals $\beta$, which means
  \[ 0 = y_3 + \bar{y_1} = y_2 + \bar{y_2} + y_3 \bar{y_3}. \]
  Then we can re-parametrize by $z_1, z_2, z_3 \in F$ according to
  \begin{align*}
    y_3 &= z_1 + z_2 \sqrt\eps \\
    y_2 &= -\frac{z_1^2 + z_2^2 \eps}{2} + z_3\sqrt\eps \\
    y_1 &= -z_1 + z_2\sqrt\eps.
  \end{align*}
  Back to the original task.
  For each $i \ge 0$ we can evaluate the Satake transform at the element
  $\nu(\varpi) = \diag(\varpi^i, 1, \varpi^{-i})$, for the cocharacter $\nu$
  corresponding to $Y_1^i + Y_1^{-i}$:
  \begin{align*}
    &\Sat\left( \mathbf{1}_{\varpi^{-r} \Mat_3(\OO_E) \cap \U(\VV_3^-)}\right)
      \left( \nu(\varpi)  \right) \\
    &= \delta(\nu(\varpi))^\half \int_{n \in N}
      \mathbf{1}_{\varpi^{-r} \Mat_3(\OO_E) \cap \U(\VV_3^-)}
      \left( \nu(\varpi) n' \right) \odif n \\
    &= \delta(\nu(\varpi))^\half \int_{n \in N}
      \mathbf{1}_{{\varpi^{-r}} \Mat_3(\OO_E) \cap \U(\VV_3^-)}
      \left( \begin{bmatrix} \varpi^i & \varpi^i y_1 & \varpi^i y_2 \\
               & 1 & y_3 \\
               & & \varpi^{-i} \end{bmatrix} \right) \odif n
  \end{align*}
  The matrix itself is always in $\U(\VV_3^-)$, because it's the product of two unitary matrices.
  So the indicator needs to check whether all the entries have valuation at least $-r$.
  If we switch characterization to the coordinates $z_1$, $z_2$, $z_3$ we described earlier,
  we see that the conditions are
  \begin{align*}
    i &\le r, \\
    v(z_1) &\ge -r, \\
    v(z_2) &\ge -r,\\
    v(z_3) &\ge -(r+i),\\
    v(z_1^2 + z_2^2 \eps) &\ge -(r+i).
  \end{align*}
  Assume $i \le r$ henceforth.
  The condition for $z_1$ and $z_2$ then really says
  \[ \min(v(z_1), v(z_2)) \ge -\left\lfloor \frac{r+i}{2} \right\rfloor. \]
  So the integral factors as a triple integral
  \[
    \int_{z_1 \in F}
    \int_{z_2 \in F}
    \int_{z_3 \in F}
    \mathbf{1}_{\varpi^{-\left\lfloor \frac{r+i}{2} \right\rfloor} \OO_F}(z_1)
    \mathbf{1}_{\varpi^{-\left\lfloor \frac{r+i}{2} \right\rfloor} \OO_F}(z_2)
    \mathbf{1}_{\varpi^{-(r+i)} \OO_F}(z_3)
    \odif{z_1,z_2,z_3}
  \]
  which is equal to
  \[ q^{2\left\lfloor \frac{r+i}{2} \right\rfloor+r+i}. \]
  Meanwhile, $\delta(\nu(\varpi))^{\half} = q^{-2i}$.
  In summary,
  \[
    \Sat\left( \mathbf{1}_{\varpi^{-r} \Mat_3(\OO_E) \cap \U(\VV_3^-)}\right) \left( \nu(\varpi) \right)
    =
    \begin{cases}
      q^{2\left\lfloor \frac{r+i}{2} \right\rfloor - i + r} & i \le r \\
      0 & i > r
    \end{cases}
  \]
  Finally, since
  \[ 2\left\lfloor \frac{r+i}{2} \right\rfloor - i + r
    = \begin{cases}
      2r & r+i \text{ is even} \\
      2r-1 & r+i \text{ is odd}
    \end{cases}
  \]
  we get the formula claimed.
\end{proof}

\subsection{Integration over fiber}
\begin{proposition}
  For every integer $r \ge 0$, we have
  \begin{align*}
    &\rproj_\ast(\mathbf{1}_{\Mat_3(\OO_E), v\circ\det=r}) \\
    &= \sum_{j=0}^r \left(
      \sum_{i=0}^{2(r-j)} \min \left( 1 + \left\lfloor \frac i2 \right\rfloor,
        1 + \left\lfloor \frac{2(r-j)-i}{2} \right\rfloor \right) q^i \right)
        \mathbf{1}_{K'_{S,j}}.
  \end{align*}
\end{proposition}
\begin{proof}
  The coefficient of $\mathbf{1}_{K'_{S,j}}$ will be equal to
  the evaluation of the integral at any $g$ such that $g\bar{g} \in K'_{S,j}$.
  Fixing $j \ge 0$, we are going to take the choice
  \[
    g = \begin{bmatrix}
      1 &   & \varpi^{-j} \sqrt{\eps} \\
      & 1 \\
      &   & 1
    \end{bmatrix}.
  \]
  We need to check this choice of $g$ indeed satisfies $g\bar{g}\inv \in K'_{S,j}$.
  This follows as
  \[ \bar{g} = \begin{bmatrix} 1 &   & -\varpi^{-j} \sqrt{\eps} \\ & 1 \\ &   & 1 \end{bmatrix}
    \implies \bar{g}\inv = \begin{bmatrix} 1 &   & \varpi^{-j} \sqrt{\eps} \\ & 1 \\ &   & 1 \end{bmatrix}
  \]
  and therefore
  \[
    g\bar{g}\inv = \begin{bmatrix}
      1 &   & 2\varpi^{-j} \sqrt{\eps} \\
      & 1 \\
      &   & 1
    \end{bmatrix} \in K'_{S,j}
  \]
  as needed.

  Having chosen the representative $g$, we aim to calculate the right-hand side of
  \[
    \rproj_\ast(\mathbf{1}_{\Mat_3(\OO_E), v\circ\det=r})(g\bar{g})
    = \int_{h \in \GL_3(F)} \mathbf{1}_{\Mat_3(\OO_E), v\circ\det=r}(gh) \odif h.
  \]
  We take (non-Archimedean) Iwasawa decomposition of $h \in \GL_3(F)$ to rewrite it as
  \[
    h =
    \begin{bmatrix} x_1 \\ & x_2 \\ && x_3 \end{bmatrix}
    \begin{bmatrix} 1 & y_1 & y_2 \\ & 1 & y_3 \\ & & 1 \end{bmatrix}
    k
  \]
  for $k \in \GL_3(\OO_F) \subseteq K'$, which does not affect the indicator function.
  Here $x_1, x_2, x_3 \in F^\times$ and $y_1, y_2, y_3 \in F$.
  In that case, note that
  \begin{align*}
    gh
    &=
    \begin{bmatrix}
      1 &   & \varpi^{-j}\sqrt\eps \\
      & 1 \\
      &   & 1
    \end{bmatrix}
    \begin{bmatrix} x_1 \\ & x_2 \\ && x_3 \end{bmatrix}
    \begin{bmatrix} 1 & y_1 & y_2 \\ & 1 & y_3 \\ & & 1 \end{bmatrix} k \\
    &=
    \begin{bmatrix}
      1 &   & \varpi^{-j}\sqrt\eps \\
      & 1 \\
      &   & 1
    \end{bmatrix}
    \begin{bmatrix} x_1 & x_1 y_1 & x_1 y_2 \\ & x_2 & x_2 y_3 \\ & & x_3 \end{bmatrix} k \\
    &=
    \begin{bmatrix}
      x_1 & x_1 y_1 & x_1 y_2 + x_3 \varpi^{-j} \sqrt\eps \\
      & x_2 & x_2 y_3 \\
      & & x_3
    \end{bmatrix}
    k.
  \end{align*}
  Hence, we can rewrite the $\rproj_\ast(\mathbf{1}_{\Mat_3(\OO_E), v\circ\det=r})$
  as a six-fold integral
  \begin{align*}
    &\rproj_\ast(\mathbf{1}_{\Mat_3(\OO_E), v\circ\det=r}) \\
    &= \int_{x_1 \in F^\times} \int_{x_2 \in F^\times} \int_{x_3 \in F^\times}
    \int_{y_1 \in F} \int_{y_2 \in F} \int_{y_3 \in F} \\
    &\quad \mathbf{1}_{\Mat_3(\OO_E), v\circ\det=r} \left(
    \begin{bmatrix}
      x_1 & x_1 y_1 & x_1 y_2 + x_3 \varpi^{-j} \sqrt\eps \\
      & x_2 & x_2 y_3 \\
      & & x_3
    \end{bmatrix}
    \right) \\
    &\quad \odif[{\times,\times,\times}]{x_1,x_2,x_3,y_1,y_2,y_3}.
  \end{align*}
  Apparently the indicator function only depends on the valuations,
  so accordingly we rewrite the six-fold integral as a discrete sum over the valuations
  $\alpha_i \coloneqq v(x_i)$.
  Then the conditions are that
  \begin{align*}
    &\alpha_1 \ge 0, \quad \alpha_2 \ge 0, \quad \alpha_3 \ge j \\
    &v(y_1) \ge - \alpha_1, \quad v(y_2) \ge - \alpha_1, \quad v(y_3) \ge -\alpha_2.
  \end{align*}
  We have $\Vol(\varpi^{\alpha_i} \OO_F^\times) = 1$
  and $\Vol(\varpi^{-\alpha_i} \OO_F) = q^{\alpha_i}$.
  Hence the integral can be rewritten as the discrete sum
  \begin{align*}
    \sum_{\substack{\alpha_1 + \alpha_2 + \alpha_3 = r \\ \alpha_1 \ge 0 \\ \alpha_2 \ge 0 \\ \alpha_3 \ge j}}
    q^{\alpha_1} \cdot q^{\alpha_1} \cdot q^{\alpha_2}
    &= \sum_{\substack{\alpha_1 + \alpha_2 \le r-j \\ \alpha_1 \ge 0 \\ \alpha_2 \ge 0}}
    q^{2\alpha_1+\alpha_2} \\
    &= \sum_{i=0}^{2(r-j)}
    \min \left( 1 + \left\lfloor \frac i2 \right\rfloor,
      1 + \left\lfloor \frac{2(r-j)-i}{2} \right\rfloor
    \right) q^i
  \end{align*}
  as desired.
\end{proof}

\subsection{Base change from $\HH(\U(\VV_3^-))$ to $\HH(S_3(F))$}
We first need to determine an element of $\HH(\U(\VV_n^-))$
which is in the pre-image of
\[ \mathbf{1}_{\varpi^{-r} \Mat_3(\OO_E) \cap \U(\VV_3^-)} \]
under $\BC \colon \HH(\GL_3(E)) \to \HH(\U(\VV_3^-))$.

For convenience, we define the shorthand
\[
  \HH(\GL_3(E)) \ni
  f'_r \coloneqq \begin{cases}
    \mathbf{1}_{\Mat_3(\OO_E), v \circ \det = r} & r \ge 0 \\
    0 & r < 0
  \end{cases}
\]
for every integer $r$.
We start with the following intermediate calculation.
\begin{align*}
  &\BC\left( \Sat \left( f'_r - q^2 f'_{r-1} \right) \right) \\
  &= \BC \left(
    q^{2r} \sum_{n_1+n_2+n_3=r} X_1^{n_1} X_2^{n_2} X_3^{n_3}
    - q^2 \cdot q^{2(r-1)} \sum_{n_1+n_2+n_3=(r-1)} X_1^{n_1} X_2^{n_2} X_3^{n_3} \right) \\
  &= q^{2r} \left( \sum_{n_1+n_2+n_3=r} Y_1^{n_1-n_3} - \sum_{n_1+n_2+n_3=(r-1)} Y_1^{n_1-n_3} \right) \\
  &= q^{2r} \left( \sum_{n_1+n_3=r} Y_1^{n_1-n_3} \right) \\
  &= q^{2r} \left( Y_1^{r} + Y_1^{r-2} + \dots + Y_1^{-r} \right).
  \intertext{Replacing $r$ with $r-1$ gives}
  &\BC\left( \Sat \left(f'_{r-1} - q^2 f'_{r-2} \right) \right) \\
  &= q^{2r-2} \left( Y_1^{r-1} + Y_1^{r-3} + \dots + Y_1^{-(r-1)} \right).
  \intertext{Adding the former equation to $q$ times the latter gives}
  &\BC\left( \Sat\left( f'_r + (q-q^2) f'_{r-1} - q^3 f'_{r-2} \right) \right) \\
  &= q^{2r} \left( Y_1^r + Y_1^{r-2} + \dots + Y_1^{-r} \right)
  + q^{2r-1} \left( Y_1^{r-1} + Y_1^{r-3} + \dots + Y_1^{-(r-1)} \right) \\
  &= \Sat(\mathbf{1}_{\varpi^{-r} \Mat_3(\OO_E) \cap \U(\VV_3^-)}).
\end{align*}
This shows that
\[ \BC(f'_r + (q-q^2) f'_{r-1} - q^3 f'_{r-2}) =
  \mathbf{1}_{\varpi^{-r} \Mat_3(\OO_E) \cap \U(\VV_3^-)} \]
so indeed $f'_r + (q-q^2) f'_{r-1} - q^3 f'_{r-2}$
lies in the desired pre-image of the map $\BC \colon \HH(\GL_3(E)) \to \HH(\U(\VV_3^-))$.

On the other hand, it is easy to check that
\begin{align*}
  &\rproj_\ast(f'_r -q^2 f'_{r-1}) \\
  &= \sum_{j=0}^r \Bigg[
      \sum_{i=0}^{2(r-j)} \min \left( 1 + \left\lfloor \frac i2 \right\rfloor,
      1 + \left\lfloor \frac{2(r-j)-i}{2} \right\rfloor \right) q^i \\
  &\qquad - \sum_{i=0}^{2(r-1-j)} \min \left( 1 + \left\lfloor \frac i2 \right\rfloor,
    1 + \left\lfloor \frac{2((r-1)-j)-i}{2} \right\rfloor \right) q^{i+2}
  \Bigg] \mathbf{1}_{K'_{S,j}} \\
  &= \sum_{j=0}^r \left[ 1+q+q^2+\dots+q^{r-j} \right] \mathbf{1}_{K'_{S,j}} \\
  \intertext{so}
  &\rproj_\ast(f'_r -q^2 f'_{r-1} + q \left( f'_{r-1} - q^2 f'_{r-3} \right)) \\
  &= \sum_{j=0}^r \left[ (1+q+q^2+\dots+q^{r-j})+(q+q^2+\dots+q^{r-j}) \right] \mathbf{1}_{K'_{S,j}} \\
  &= \sum_{j=0}^r \left[ 1 + 2q + 2q^2 + \dots + 2q^{r-j} \right] \mathbf{1}_{K'_{S,j}}.
\end{align*}
To summarize, the completed commutative diagram can be written in full as
\begin{center}
\begin{tikzcd}
  \begin{tabular}{c} $f'_r + (q-q^2) f'_{r-1}$ \\ $- q^3 f'_{r-2} \in \HH(\GL_3(E))$ \end{tabular}
    \ar[dd, "\rproj_\ast"', mapsto, bend right = 50]
    \ar[r, "\Sat", mapsto] \ar[d, "\BC", mapsto]
    & \dots \in \QQ[X_1^\pm, X_2^\pm, X_3^\pm]^{\Sym(3)} \ar[d, "\BC", mapsto] \\
  \begin{tabular}{c} $\mathbf{1}_{\varpi^{-r} \Mat_3(\OO_E) \cap \U(\VV_3^-)}$ \\ $\in \HH(\U(\VV_3^-))$ \end{tabular}
    \ar[r, "\Sat", mapsto]
    & \begin{tabular}{l}
      $q^{2r} \left( Y_1^{\pm r} + \dotsb + Y_1^{\mp r} \right)$ \\
      $+ q^{2r-1} \left( Y_1^{\pm(r-1)} + \dotsb + Y_1^{\mp(r-1)} \right)$ \\
      $\in \QQ[Y_1^\pm]^{W_1}$
      \end{tabular} \\
  \begin{tabular}{c}
    $\sum_{j=0}^r \big[ 1 + 2q + 2q^2$ \\
    $+ \dots + 2q^{r-j} \big] \mathbf{1}_{K'_{S,j}}$ \\
    $\in \HH(S_3(F))$
  \end{tabular} \ar[u, "\sim", "\BC_{S_3}"', mapsto]
\end{tikzcd}
\end{center}

Thus, we arrive at the following:
\begin{proposition}
  \label{prop:BC_S3}
  For $n = 3$, we have
  \begin{align*}
    \BC_{S_3} \left( \sum_{j=0}^r \left[ 1 + 2q + 2q^2 + \dots + 2q^{r-j} \right]
    \mathbf{1}_{K'_{S,j}} \right)
    &= \mathbf{1}_{\varpi^{-r} \Mat_3(\OO_E) \cap \U(\VV_3^-)} \\
    \BC_{S_3} \left( \mathbf{1}_{K'_{S,r}}
    + \sum_{j=0}^{r-1} 2q^{r-j} \mathbf{1}_{K'_{S,j}} \right)
    &= \mathbf{1}_{K\varpi^{(r,0,-r)}K}
  \end{align*}
  for every integer $r \ge 0$.
\end{proposition}
\begin{proof}
  The first equation is the one we just proved.
  The second one follows by noting that
  \[
    \mathbf{1}_{K\varpi^{(r,0,-r)}K}
    = \mathbf{1}_{\varpi^{-r} \Mat_3(\OO_E) \cap \U(\VV_3^-)}
    - \mathbf{1}_{\varpi^{-(r-1)} \Mat_3(\OO_E) \cap \U(\VV_3^-)}
  \]
  so one merely subtracts the left-hand sides evaluated at $r$ and $r-1$ for $r \ge 1$
  to get
  \begin{align*}
    &\phantom= \sum_{j=0}^r \left[ 1 + 2q + 2q^2 + \dots + 2q^{r-j} \right] \mathbf{1}_{K'_{S,j}}
    - \sum_{j=0}^{r-1} \left[ 1 + 2q + 2q^2 + \dots + 2q^{(r-1)-j} \right] \mathbf{1}_{K'_{S,j}} \\
    &= \mathbf{1}_{K'_{S,r}} +
      \sum_{j=0}^{r-1} \left[ 1 + 2q + 2q^2 + \dots + 2q^{r-j} \right] \mathbf{1}_{K'_{S,j}}
    - \sum_{j=0}^{r-1} \left[ 1 + 2q + 2q^2 + \dots + 2q^{(r-1)-j} \right] \mathbf{1}_{K'_{S,j}} \\
    &= \mathbf{1}_{K'_{S,r}} + \sum_{j=0}^{r-1} \left[ 2q^{r-j}\mathbf{1}_{K'_{S,j}} \right].
  \end{align*}
  as claimed.
\end{proof}


\chapter{Synopsis of the weighted orbital integral $\Orb(\gamma, \phi, s)$ for $\gamma \in S_3(F)\rs$ and $\phi \in \HH(S_3(F))$}
\label{ch:orbital0}

This section defines the weighted orbital integral
and describes the parameters which we will use to express our answer.

\section{Initial definition of the weighted orbital integral for general $S_n(F)$}
Let $H = \GL_{n-1}(F)$.
Then $H$ has a natural embedding into $\GL_n(E)$ by
\[ h \mapsto \begin{pmatrix} h & 0 \\ 0 & 1 \end{pmatrix} \]
which endows it with an action $S_n(F)$.
Then our weighted orbital integral is defined as follows.
\begin{definition}
  [{\cite[Equation (3.2.3)]{ref:AFLspherical}}]
  \label{def:orbital0}
  For brevity let $\eta(h) \coloneqq \eta(\det h)$ for $h \in H$.
  For $\gamma \in S_n(F)$, $\phi \in \HH(S_n(F))$, and $s \in \CC$,
  we define the \emph{weighted orbital integral} by
  \[ \Orb(\gamma, \phi, s) \coloneqq
    \int_{h \in H} \phi(h\inv \gamma h) \eta(h)
    \left\lvert \det(h) \right\rvert_F^{-s} \odif h. \]
\end{definition}

We remark that this weighted orbital integral is related to
a normal orbital integral on the unitary side
by the so-called relative fundamental lemma.
Specifically, for $g \in \U(\VV_n^+)$ and $f \in \HH(\U(\VV_n^+))$,
we define the normal orbital integral by
\[ \Orb^{\U(\VV_n^+)}(g, f) \coloneqq \int_{\U(\VV_n^+)} f(x^{-1}gx) \odif x. \]
Then the following result is true.
\begin{theorem}
  [Relative fundamental lemma; {\cite[Theorem 1.1]{ref:leslie}}]
  \label{thm:rel_fundamental_lemma}
  Let $\phi \in \HH(S_n(F))$ and $\gamma \in S_n(F)\rs$.
  If $\gamma$ matches an element of $\U(\VV_n^-)\rs$, then
  \[ \omega(\gamma) \Orb(\phi, \gamma, 0) = 0. \]
  If $\gamma$ instead matches an element $g \in \U(\VV_n^+)\rs$, then
  \[ \omega(\gamma) \Orb(\phi, \gamma, 0)
    = \Orb^{\U(\VV_n^+)}(g, \BC^{\eta^{n-1}}_{S_n}(\phi)) \]
  where the transfer factor $\omega$ is defined in \Cref{ch:geo}.
\end{theorem}

\section{Basis for the indicator functions in $\HH(S_3(F))$}
\label{ch:orbital0_hecke_basis}
From now on assume $n = 3$.
We have the symmetric space
\[ S_3(F) \coloneqq \left\{ g \in \GL_3(E) \mid g \bar{g} = \id_3 \right\}. \]
which has a left action under $\GL_3(E)$ by $g \cdot s \mapsto gs\bar{g}\inv$.

Then $S_3(F)$ admits the following decomposition, which we will use:
\begin{lemma}
  [Cartan decomposition of $S_3(F)$]
  For each integer $r \ge 0$ let
  \[ K'_{S,r} \coloneqq \GL_3(\OO_E) \cdot \begin{pmatrix} 0 & 0 & \varpi^r \\ 0 & 1 & 0 \\ \varpi^{-r} & 0 & 0 \end{pmatrix} \]
  denote the orbit of
  $\begin{pmatrix} 0 & 0 & \varpi^r \\ 0 & 1 & 0 \\ \varpi^{-r} & 0 & 0 \end{pmatrix}$
  under the left action of $\GL_3(\OO_E)$.
  Then we have a decomposition
  \[ S_3(F) = \coprod_{r \geq 0} K'_{S,r}. \]
\end{lemma}
\begin{proof}
  \todo{reference}
\end{proof}

The $r=0$ case will be given a special shorthand,
and can be expressed in a few equivalent ways:
\begin{align*}
  K'_S
  &\coloneqq K'_{S,0} \\
  &= \GL_3(\OO_E) \cdot \begin{pmatrix} & & 1 \\ & 1 \\ 1 \end{pmatrix} \\
  &= \GL_3(\OO_E) \cdot \id_3 = S_3(F) \cap \GL_3(\OO_E).
\end{align*}
One can equivalently define $K'_{S,r}$ to be the part of $S_3(F)$
for which the most negative valuation among the nine entries is $-r$.

For $r \geq 0$, define
\[ K'_{S, \le r} \coloneqq S_3(F) \cap \varpi^{-r} \GL_3(\OO_E). \]
We can re-parametrize the problem according to the following.
\begin{corollary}
  We have a decomposition
  \[ K'_{S, \le r} = K'_{S,0} \sqcup K'_{S,1} \sqcup \dots \sqcup K'_{S,r}. \]
\end{corollary}
Then an integral over each $K'_{S, \le r}$ lets us extract the integrals over $K'_{S,r}$.
\begin{corollary}
  For $r \ge 0$, the indicator functions $\mathbf{1}_{K'_{S, \le r}}$
  form a basis of $\HH(S_3(F))$.
\end{corollary}

Then, our goal is to compute for
\begin{equation}
  \pdv{}{s}\Orb(\gamma, \mathbf{1}_{K'_{S, \le r}}, s)
  \label{eq:orbital_goal}
\end{equation}
at $s=0$ for any $r > 0$ as well.
Note that the $r = 0$ case is already done in \cite{ref:AFL}.

\section{Parametrization of $\gamma$}
Again, assume $n = 3$.
Further assume $\gamma \in S_3(F)\rs$ is regular semisimple.
We identify some parameters for the orbit of $\gamma$
that we can use for our explicit calculations.

\subsection{Rewriting the weighted orbital integral as a double integral over $E$
  via the group $H' \cong \GL_2(F)$}
Our weighted orbital integral is at present a quadruple integral over $F$,
owing to $H = \GL_{2}(F)$ being a four-dimensional $F$-vector space.

It will be more economical to work with the weighted orbital integral as a double integral
with two coefficients in $E$, in the following sense.
Define
\[ H' \coloneqq
  \left\{ \begin{pmatrix} t_1 & t_2 \\ \bar t_2 & \bar t_1 \end{pmatrix}
    \mid t_1, t_2 \in E \right\}
\]
which is indeed a four-dimensional $F$-algebra.
As before $H' \hookrightarrow \U(\VV_n^+)$ according to the same embedding
$\GL_2(E) \hookrightarrow \GL_(3)$
and so $H'$ also acts on $S_n(E)$ by conjugation.

As an $F$-algebra, we have an isomorphism (see \cite[\S4.1]{ref:AFL})
\begin{align*}
  \iota_2 \colon H = \GL_2(F)
  &\xrightarrow{\cong} H' \\
  \begin{pmatrix} a_{11} & a_{12} \\ a_{21} & a_{22} \end{pmatrix}
  &\mapsto \begin{pmatrix} t_1 & t_2 \\ \bar t_2 & \bar t_1 \end{pmatrix} \\
  t_1 &= \half\left( a_{11} + a_{22} + \frac{a_{12}}{\sqrt{\eps}} + a_{21} \sqrt{\eps} \right) \\
  t_2 &= \half\left( a_{11} - a_{22} + \frac{a_{12}}{\sqrt{\eps}} - a_{21} \sqrt{\eps} \right).
\end{align*}
Under this isomorphism, we have
\[ h \gamma h^{-1} = \iota_2(h) \gamma \overline{\iota_2(h)^{-1}}. \]

This allows us to rewrite the weighted orbital integral over $H'$ instead.
If we write $h' = \overline{\iota_2(h)^{-1}}$,
then the following integral formula is obtained.
\begin{proposition}
  [{\cite[\S4.2]{ref:AFL}}]
  \label{prop:orbital_over_H_prime}
  For brevity let $\eta(h') \coloneqq \eta(\det h')$ for $h' \in H'$.
  For $\gamma \in S_3(F)$, $\phi \in \HH(S_3(F))$, and $s \in \CC$,
  the weighted orbital integral can instead be written as
  \[ \Orb(\gamma, \phi, s) =
    \int_{h' \in H'} \phi(\bar{h'}\inv \gamma h') \eta(h')
    \left\lvert \det(h') \right\rvert_F^{s} \odif{h'} \]
  where
  \[ \odif{h'} = \kappa \cdot \frac{\odif t_1 \odif t_2}
    {\left\lvert t_1 \bar t_1 - t_2 \bar t_2 \right\rvert_F^2} \]
  for the constant
  \[ \kappa \coloneqq \frac{1}{(1-q\inv)(1-q^{-2})}. \]
\end{proposition}

\subsection{Identifying a representative in the $H'$-orbit}
Evidently the weighted orbital integral $\Orb(\gamma, \phi, s)$ in \Cref{prop:orbital_over_H_prime}
only depends on the $H'$-orbit of $\gamma$.
So it makes sense to pick a canonical representative for the $H'$-orbit to compute
the weighted orbital integral in terms of.

Since we assumed $\gamma \in S_3(F)\rs$ is regular semisimple,
we can invoke \cite[Proposition 4.1]{ref:AFL}
to assume $\gamma$ is a representative (under the orbit of $H'$) specifically of the form
\[ \gamma(a,b,d) =
  \begin{pmatrix}
    a & 0 & 0 \\
    b & - \bar d & 1 \\
    c & 1 - d \bar d & d
  \end{pmatrix}
  \in S_3(F)\rs; \quad \text{where $c = -a \bar b + b d$} \]
over all $a \in E^1$, $b \in E$, $d \in E$ for which $(1-d\bar d)^2 - c \bar c \neq 0$.
In other words, the representatives described here cover all the regular orbits in $S_3(F)\rs$.

\subsection{Simplification due to the matching of non-quasi-split unitary group}
In this calculation, we restrict attention to the case where our regular $\gamma$
matches an element in the non-quasi-split unitary group $\U(\VV_n^-)\rs$
(rather than $\U(\VV_n^+)\rs$).
As we described in \Cref{prop:valuation_delta_matching_group},
this is controlled by the parity of the invariant
\[ v\left( (1-d\bar d)^2 - c \bar c\right) \]
being odd.
Hence, we only have to consider this case:
\begin{assume}
  \label{assume:u_odd}
  We will assume that
  \[ v\left( (1-d\bar d)^2 - c \bar c\right) \equiv 1 \pmod 2. \]
\end{assume}
This is the same assumption made in \cite[Equation (4.3)]{ref:AFL}.

We will mostly be interested in the case where $v(b) = v(d) = 0$.
In fact, few other cases even occur at all given \Cref{assume:u_odd};
we will see momentarily that either $v(b) = v(d) < 0$,
or one of $\{v(b), v(d)\}$ is zero and the other is nonnegative.

\section{Parameters used in the calculation of the weighted orbital integral}
\label{sec:param_orbital0}
The goal of the next two sections is to evaluate \eqref{eq:orbital_goal}
(and hence its derivative at $s = 0$) in terms of the parameters
\[ a \in E^1, \qquad b, d \in E, \qquad r \ge 0. \]
To simplify the notation in what follows,
it will be convenient to define several quantities that reappear frequently.
From \Cref{assume:u_odd}, we may define
\begin{equation}
  \delta \coloneqq v(1-d \bar d) = v(c) \neq -\infty.
  \label{eq:delta}
\end{equation}
Following \cite[Equation (4.3)]{ref:AFL} we will also define
\begin{equation}
  u \coloneqq \frac{\bar c}{1-d \bar d} \in \OO_E^\times
  \label{eq:u}
\end{equation}
so that $\nu(1-u \bar u) \equiv 1 \pmod 2$ and
\begin{equation}
  b = -au - \bar{d} \bar{u}.
  \label{eq:b}
\end{equation}
Note that this gives us the following repeatedly used identity
\begin{equation}
  b^2-4a\bar d = (au-\bar d \bar u)^2 - 4a\bar d(1-u\bar u).
  \label{eq:dos}
\end{equation}
Finally, define
\begin{equation}
  \ell \coloneqq v(b^2 - 4 a \ol d).
  \label{eq:ell}
\end{equation}
We will also define one additional parameter useful when $\ell$ is even
(but as we will see, redundant for odd $\ell$):
\begin{equation}
  \lambda \coloneqq v(1-u \bar u) \equiv 1 \pmod 2.
  \label{eq:lambda}
\end{equation}

Just as many pairs $(v(b), v(d))$ do not occur (given \Cref{assume:u_odd})
and $v(b) = v(d) = 0$ is the main case of interest,
the parameters $(\delta, \ell, \lambda)$ satisfy some additional relations.
We will now describe them.
\begin{proposition}
  \label{prop:parameter_constraints}
  Exactly one of the following situations is true:
  \begin{itemize}
    \ii $v(b) = v(d) = 0$, $\ell \ge 1$ is odd, $\ell < 2 \delta$, and $\lambda = \ell$.
    \ii $v(b) = v(d) = 0$, $\ell \ge 0$ is even, $\ell \le 2 \delta$, and $\lambda > \ell$ is odd.
    \ii $v(b) = 0$, $v(d) > 0$, $\ell = \delta = 0$, and $\lambda > 0$ is odd.
    \ii $v(b) > 0$, $v(d) = 0$, $\ell = 0$, $\delta \ge 0$, and $\lambda > 0$ is odd.
    \ii $v(b) = v(d) < 0$, $\ell = \delta = 2v(d) < 0$, and $\lambda > 0$ is odd.
  \end{itemize}
  See \Cref{tab:parameter_constraints}.
  Moreover, whenever $\ell$ is even,
  the quantity $b^2 - 4 a \bar d$ is a square of some element in $E$.
\end{proposition}

\begin{table}[ht]
  \centering
  \begin{tabular}{cccc}
    \toprule
    & $v(b) = 0$ & $v(b) > 0$ & $v(b) < 0$ \\
    \midrule
    $v(d) = 0$ & $0 \le \ell \le 2 \delta$ & $\ell = 0$, $\delta \ge 0$ & never \\
    $v(d) > 0$ & $\ell = \delta = 0$ & never & never \\
    $v(d) < 0$ & never & never & $v(b) = v(d) = \frac{\ell}{2} = \frac{\delta}{2} < 0$ \\
    \bottomrule
  \end{tabular}
  \caption{A table showing the five cases in \Cref{prop:parameter_constraints}.}
  \label{tab:parameter_constraints}
\end{table}


Before proving the proposition in full we first prove the following lemmas.
\begin{lemma}
  \label{lem:au_minus_du}
  Assume $v(d) \ge 0$.
  For odd $\ell$, we have
  \[ 2 v(au - \bar d \bar u) > \ell = \lambda \]
  while for even $\ell$ we instead have
  \[ 2 v(au - \bar d \bar u) = \ell < \lambda. \]
\end{lemma}
\begin{proof}
  If $v(d) = 0$, this follows from \eqref{eq:dos} directly,
  since $v\left( (au - \bar d \bar u)^2 \right)$ is even, and hence
  can never equal $v(4a \bar d (1- u \bar u)) = \lambda \equiv 1 \pmod 2$.

  Meanwhile, if $v(d) > 0$, then from \eqref{eq:b}
  it follows $v(b) = 0$, and hence $\ell = 0$.
  And $v(au - \bar d \bar u) = 0$ in this case as well.
  Since $\lambda$ is a positive odd integer, the lemma is proved.
\end{proof}

\begin{lemma}
  If $v(b) = v(d) = 0$ and $\ell \ge 0$, then $\ell \le 2 \delta$.
  \label{lem:ell_le_2_delta}
\end{lemma}
\begin{proof}
  If $\ell = 0$ there is nothing to prove so assume $\ell > 0$.
  Let us write
  \begin{equation}
    \OO_E^\times \ni \frac{a d u}{\bar u} = x + y \sqrt{\eps} \qquad x,y \in F
    \label{eq:ell_delta_adu_xy}
  \end{equation}
  which has norm
  \begin{equation}
    \OO_F^\times \ni x^2+y^2\eps
    = \frac{a d u}{\bar u} \cdot \frac{\bar a \bar d \bar u}{u} = d \bar d.
    \label{eq:ell_delta_adu_norm}
  \end{equation}

  Now, according to \Cref{lem:au_minus_du} we have that
  \[ 0 < \ell \le 2v(au - \bar d \bar u)
    = 2v\left( \frac{a d u}{\bar u} - d \bar d \right)
    = 2v\left(  (x - d \bar d) + y \sqrt{\eps}. \right) \]
  and since $d \bar d \in F$, it follows that
  \begin{align}
    v\left( x - d \bar d \right) &> \frac{\ell}{2} \label{eq:ell_delta_x_minus_d_bar_d} \\
    v(y) &> \frac{\ell}{2}. \label{eq:ell_delta_y}
  \end{align}
  In particular, \eqref{eq:ell_delta_y} implies $v(y) > 0$ which has two consequences:
  \begin{itemize}
    \ii From \eqref{eq:ell_delta_adu_xy} we get $v(x) = 0$.
    \ii From $v(y^2) > 0$ and \eqref{eq:ell_delta_adu_norm}
    we conclude \[ v(x^2 - d \bar d) = v(y^2) > \ell. \]
  \end{itemize}
  Putting \eqref{eq:ell_delta_x_minus_d_bar_d} together with the previous two bullets,
  \[ \frac{\ell}{2} \le v\left( x^2 - d \bar d - x(x - d \bar d) \right)
    = v(x) + v(1-d \bar d) = 0 + \delta \]
  and this proves $\ell \le 2\delta$.
\end{proof}

Now we can prove \Cref{prop:parameter_constraints}.
\begin{proof}[Proof of \Cref{prop:parameter_constraints}]
  It's clear the five bullets above are disjoint.
  \begin{itemize}
  \ii First assume $\ell$ is odd.
  We assert in this case we have $v(b) = v(d) = 0$.
  Indeed if $v(d) \neq 0$, then $b = -au-\bar d\bar u$ is a unit,
  and hence so is $b^2 - 4 a \bar d$, causing $\ell = 0$, contradiction.
  And if $d$ is a unit, $\ell \neq 0$ means $v(b) = 0$ too.
  In particular, $\ell > 0$.
  The rest of the claims follow by \Cref{lem:au_minus_du} and \Cref{lem:ell_le_2_delta}.
  \end{itemize}
  For the rest of the proof we only consider even $\ell$.
  Because $b = -au - \bar d \bar u$, it cannot be the case that $v(b) > 0$ and $v(d) > 0$;
  moreover if either $v(b) < 0$ or $v(d) < 0$,
  then in fact $v(b) = v(d)$.
  We consider each of the four possibilities.
  \begin{itemize}
  \ii Suppose and $v(b) = v(d) = 0$.
  Then \Cref{lem:au_minus_du} and \Cref{lem:ell_le_2_delta} imply the results.

  \ii If $v(b) = 0$ and $v(d) > 0$, then $b^2 - 4a \bar d$ is a unit
  and $1 - d \bar d$ are both units, ergo $\ell = \delta = 0$.

  \ii If $v(b) > 0$ and $v(d) = 0$ then $b^2 - 4 a \bar d$ is a unit,
  and there is nothing left to prove.

  \ii Finally suppose $v(b) = v(d) < 0$.
  Then $v(b^2) < v(4 a \bar d) < 0$, so indeed $\ell = 2 v(d) < 0$.
  And $v(1 - d \bar d) = 2v(d) < 0$ as well.
  \end{itemize}
  We now verify the last assertion that $b^2 - 4 a \bar d$
  is a square whenever $\ell$ is even.
  The proof in all cases uses \eqref{eq:dos} to show $b^2 - 4 a \bar d$
  is equal to $\varpi^\ell$ times a quadratic residue in $\OO_E^\times$.
  Indeed we need only verify that $v(4a \bar d(1 - u \bar u)) = v(d) + \lambda$
  has larger valuation than $v\left( (a u - \bar d \bar u)^2 \right) = \ell$.
  In the case $v(b) = 0$ this follows from $\lambda > \ell$.
  Whereas if $v(d) > 0$ we have $\ell = 0$
  and if $v(b) = v(d) < 0$ then $\ell = -2v(d)$;
  so in all these cases the claim is obvious too. \qedhere
\end{proof}

In the case where $\ell$ is odd (and hence $\ell \ge 1$ and $v(b) = v(d) = 0$),
we get \eqref{eq:dos} implying $\lambda = \ell$
and thus $\lambda$ will never be used --- the weighted orbital will be computed
as a function of $\ell$ and $\delta$ (and $r$).

However for even $\ell$ these numbers are never equal and our weighted orbital
integral will be stated in terms of $\ell$, $\delta$, and $\lambda$ (and $r$).
We just saw that in these situations $b^2 - 4 a \bar d$ is a square;
moving forward, we need to fix the choice of the square root $\tau$.
We do so as follows.
\begin{definition}
  [Fixing the choice of $\tau$]
  Assuming $\ell$ is even, using \eqref{eq:dos} in the form
  \[ (au-\bar d \bar u)^2 - \tau^2 = 4a\bar d(1- u\bar u) \]
  we agree now to fix the choice of the square root of $\tau$ such that
  \begin{equation}
    \begin{aligned}
      v(au-\bar d \bar u + \tau) &= \lambda + v(d) - \half \ell > 0, \\
      v(au-\bar d \bar u - \tau) &= \half \ell.
    \end{aligned}
    \label{eq:tau_choice}
  \end{equation}
\end{definition}
Here $\lambda + v(d) - \half \ell > 0$ is obvious when $v(d) \ge 0$
(since $\lambda = \ell > 0$ for odd $\ell$ and otherwise $\lambda > \ell$),
and for $v(d) < 0$ we have $v(d) = \half \ell$ anyway.

\begin{lemma}
  With this choice of $\tau$, we have
  \begin{equation}
    \begin{aligned}
      v\left( 4 - \Norm(b+\tau) \right) &= \lambda + \delta - \ell \\
      v\left( 4 - \Norm(b-\tau) \right) &= \delta.
    \end{aligned}
    \label{eq:four_norm_choice}
  \end{equation}
\end{lemma}
\begin{proof}
  We consider several cases.
  \begin{itemize}
  \ii If $v(b) = v(d) = 0$ then from \eqref{eq:dos} we have
  \[ \ell = 2v(\tau) = 2v(au - \bar d \bar u) < \lambda \]
  and thus \cite[Lemma 4.7]{ref:AFL} applies to give \eqref{eq:four_norm_choice}, verabtim.

  \ii Now suppose $v(d) > 0$ but still $v(b) = v(\tau) = 0$.
  We begin with the observation that
  \begin{equation}
    4 a \bar d = b^2 - \tau^2 = (b + \tau)(b - \tau)
    \label{eq:tau_which_which}
  \end{equation}
  and so $\{v(b+\tau), v(b-\tau)\} = \{0, v(d)\}$.
  We need to determine which is which.
  However, note that we may write
  \[ au - \bar d \bar u - \tau = -(b + \tau) - 2 \bar d \bar u. \]
  Since $v(au - \bar d \bar u - \tau) = 0$ and $v(d) > 0$,
  it follows we must have $v(b+\tau) = 0$.
  And thus $v(b-\tau) = v(d)$.
  Hence $v(4 - \Norm(b-\tau)) = 0 = \delta$,
  and we have obtained the bottom equation of \eqref{eq:four_norm_choice}.

  It remains to show that $v(4-\Norm(b+\tau)) = \lambda$ to complete the proof.
  We quote \cite[Lemma 4.6]{ref:AFL} which states more generally that
  \begin{align*}
    2\delta + \lambda
    &= v(4-\Norm(b+\tau)) + v(4-\Norm(b-\tau)) \\
    &\qquad+ v(16 + 16 d \bar d - 8 b \bar b + 8 \tau \bar\tau).
  \end{align*}
  In our case $\delta = 0$, $v(4-\Norm(b-\tau)) = 0$.
  Moreover since $v(\tau - b) > 0$ we get $v(\tau\bar\tau - b \bar b) > 0$.
  (Indeed, if $\tau = x_\tau + \eps y_\tau$ and $b = x_b + \eps y_b$,
  then $\tau\bar\tau - b \bar b = (x_\tau^2 - x_b^2 + \eps(y_\tau^2 - y_b^2))
  + \eps(x_\tau y_\tau - y_\tau y_b)$.
  Since $x_\tau \equiv x_b \pmod{\varpi}$ and $y_\tau \equiv y_b \pmod{\varpi}$,
  the conclusion is immediate.)
  Hence the final term on the right-hand side is $0$ too.

  \ii Consider $v(b) > 0$.
  As mentioned on \cite[p.\ 242]{ref:AFL}, the identity \eqref{eq:four_norm_choice}
  is still true in this situation too.

  \ii Finally assume $v(\tau) = v(b) = v(d) < 0$.
  Again from \eqref{eq:tau_which_which}
  we know $\{v(b+\tau), v(b-\tau)\} = \{0, v(d)\}$ and need to determine which is which.
  This time we write
  \[ au - \bar d \bar u + \tau = -(b - \tau) - 2 \bar d \bar u. \]
  Since $v(au - \bar d \bar u + \tau) = \lambda > 0$,
  but $v(2 \bar d \bar u) = v(d) < 0$,
  it follows we must have $v(b-\tau) < 0$,
  so in fact $v(b-\tau) = v(d)$ and $v(b+\tau) = 0$.
  So, in this case, we get $v(4 - \Norm(b-\tau)) = 2v(b) = \delta$,
  which is the bottom equation of \eqref{eq:four_norm_choice}.

  Then it remains to show that $v(4-\Norm(b+\tau)) = \lambda$,
  which is done in the same way as $v(d) > 0$ earlier.
  \qedhere
  \end{itemize}
\end{proof}

\section{Statement of the differentiated weighted orbital integral}
We can now state the answer.
\begin{theorem}
  Let \[ \gamma(a,b,d) = \begin{pmatrix}
    a & 0 & 0 \\
    b & - \bar d & 1 \\
    c & 1 - d \bar d & d
  \end{pmatrix} \in S_3(F)\rs \]
  be paired with an element of $\U(\VV_3^-)\rs$.
\end{theorem}


We dedicate \Cref{ch:orbital1,ch:orbital2} to the calculation
of the full orbital integral and its derivative at $s = 0$ given here.


\chapter{Synopsis of the weighted orbital integral
  $\Orb((\gamma, \uu, \vv^\top), \phi \otimes \oneV, s)$
  for $(\gamma, \uu, \vv^\top) \in (S_2(F) \times V'_2(F))\rs$
  and $\phi \in \HH(S_2(F))$}
\label{ch:orbitalFJ0}

Throughout this section, $H = \GL_n(F)$ (rather than $H = \GL_{n-1}(F)$)
and $K' = \GL_n(\OO_F)$.
For the concrete calculation, we are mostly interested in the case $n = 2$.

\section{Definition}
We will not work in the generality of a function on all of $S_n(F) \times V_n'(F)$.
Instead, for our conjecture, it will be enough to define the weighted orbital integral
in the case where our function is of the form
\[ \phi \otimes \oneV \]
where $\phi \in \HH(S_n(F))$ is the left component, and
the right component is the indicator function defined in the obvious way:
\begin{align*}
  \mathbf{1}_{\OO_F^n \times (\OO_F^n)^\vee} \colon V'_n(F) &\to \{0,1\} \\
  (\uu, \vv^\top) &\mapsto
  \begin{cases}
    1 & \uu \text{ and } \vv^\top \text{ have } \OO_F \text{-entries} \\
    0 & \text{otherwise}.
  \end{cases}
\end{align*}

Then, unsurprisingly from the definition of our action as
\[ h \cdot \guv = (h\gamma h\inv, h\uu, \vv^\top h\inv) \]
we analogously define the weighted orbital integral as follows.
\begin{definition}
  [{\cite[\S1.3]{ref:liuFJ}}]
  \label{def:orbitalFJ}
  For brevity let $\eta(h) \coloneqq \eta(\det h)$ for $h \in H$.
  For $\guv \in S_n(F) \times V'_n(F)$,
  $\phi \in \HH(S_n(F))$, and $s \in \CC$,
  we define the weighted orbital integral by
  \begin{align*}
    & \Orb((\gamma, \uu, \vv^\top), \phi \otimes \oneV, s) \\
    &\coloneqq \int_{h \in H} \phi(h\inv \gamma h) \oneV(h \uu, \vv^\top h^{-1})
    \eta(h) \left\lvert \det(h) \right\rvert_F^{-s} \odif h.
  \end{align*}
\end{definition}

As before it seems this weighted orbital integral should be related to an ordinary one.
To define it, fix a self-dual lattice $\Lambda_n$ in $\VV_n^+$ of full rank.
First, if $(g,u) \in \U(\VV_n^+) \times \VV_n^+$ and $f \in \HH(\U(\VV_n^+))$,
then we define an orbital integral for $\U(\VV_n^+) \times \VV_n^+$ by
\begin{equation}
  \Orb^{\U(\VV_n^+) \times \VV_n^+}\left( (g,u), f \otimes \mathbf{1}_{\Lambda_n} \right)
  \coloneqq \int_{\U(\VV_n^+)} f(x\inv g x) \mathbf{1}_{\Lambda_n}(x^{-1} u) \odif x.
  \label{eq:unweighted_orbital_semi_lie}
\end{equation}
Then in the spirit of \cite[Conjecture 1.9]{ref:liuFJ}
and \Cref{thm:rel_fundamental_lemma}, we propose the following.
\begin{conjecture}
  \label{conj:rel_fundamental_lemma_semilie}
  Let $\phi \in \HH(S_n(F))$ and $\guv \in (S_n(F) \times V_n')\rs$.
  If $\guv$ matches an element of $(\U(\VV_n^-) \times \VV_n^-)\rs$, then
  \[ \omega\guv \Orb(\phi \otimes \oneV, \guv, 0) = 0. \]
  If $\guv$ instead matches an element $g \in (\U(\VV_n^+) \times \VV_n^+)\rs$, then
  \[ \omega\guv \Orb(\phi \otimes \oneV, \guv, 0)
    = \Orb^{\U(\VV_n^+) \times \VV_n^+}((g,u), \BC^{\eta^{n-1}}_{S_n}(\phi) \otimes \mathbf{1}_{\Lambda_n}) \]
  where the transfer factor $\omega$ is defined in \Cref{ch:geo}.
\end{conjecture}
Wei Zhang suggests that this conjecture can be proven by similar means
to \Cref{thm:rel_fundamental_lemma},
but since it is not necessary for this paper we do not pursue this proof here.

\section{Basis for the indicator functions in $\HH(S_2(F))$}
\label{sec:hecke_basis_FJ}
From now on assume $n = 2$.
This section is almost an exact analog of \Cref{ch:orbital0_hecke_basis},
so we will be slightly terser.
Again set
\[ S_2(F) \coloneqq \left\{ g \in \GL_2(E) \mid g \bar{g} = \id_2 \right\}. \]
We again have a Cartan decomposition indexed by a single integer $r \ge 0$:
\begin{lemma}
  [Cartan decomposition of $S_2(F)$]
  For each integer $r \ge 0$ let
  \[ K'_{S,r} \coloneqq \GL_2(\OO_E) \cdot
    \begin{pmatrix} 0 & \varpi^r \\ \varpi^{-r} & 0 \end{pmatrix} \]
  denote the orbit of
  $\begin{pmatrix} 0 & \varpi^r \\ \varpi^{-r} & 0 \end{pmatrix}$
  under the left action of $\GL_2(\OO_E)$.
  Then we have a decomposition
  \[ S_2(F) = \coprod_{r \geq 0} K'_{S,r}. \]
\end{lemma}
\begin{proof}
  \todo{reference}
\end{proof}

Like last time, $K'_{S,r}$ is the part of $S_2(F)$
for which the most negative valuation among the nine entries is $-r$.
And as before we abbreviate the $r = 0$ term specifically:
\begin{align*}
  K'_S
  &\coloneqq K'_{S,0} \\
  &= \GL_2(\OO_E) \cdot \begin{pmatrix} & 1 \\ 1 \end{pmatrix} \\
  &= \GL_2(\OO_E) \cdot \id_2 = S_2(F) \cap \GL_2(\OO_E).
\end{align*}

Repeating the definition
\[ K'_{S, \le r} \coloneqq S_2(F) \cap \varpi^{-r} \GL_2(\OO_E)
  = K'_{S,0} \sqcup K'_{S,1} \sqcup \dots \sqcup K'_{S,r} \]
we get a basis of indicator functions for the Hecke algebra $\HH(S_2(F))$:
\begin{corollary}
  For $r \ge 0$, the indicator functions $\mathbf{1}_{K'_{S, \le r}}$
  form a basis of $\HH(S_2(F))$.
\end{corollary}

\section{Parametrization of $\gamma$}
From now on assume $n = 2$,
and that $(\gamma, \uu, \vv^\top) \in (S_2(F) \times V'_2)\rs$ is regular.

\subsection{Identifying an orbit representative}
The weighted orbital integral depends only on the $H$-orbit of $(\gamma, \uu, \vv^\top)$.
Consequently, we may assume without loss of generality
(via multiplication by a suitable change-of-basis $h \in H = \GL_2(F)$) that
\[ \uu = \begin{pmatrix} 0 \\ 1 \end{pmatrix}, \qquad
  \vv^\top = \begin{pmatrix} 0 & e \end{pmatrix} \qquad e \in F. \]
(We know $\uu$ is not the zero vector from the regular condition
applied on $(\gamma, \uu, \vv^\top)$.)

Meanwhile, we will let
$\gamma = \begin{pmatrix} a & b \\ c & d \end{pmatrix} \in \GL_2(F)$
for $a,b,c,d \in F$.
Then, viewed as an element of $\GL_3(F)$ via the embedding we described earlier, we have
\[
  (\gamma, \uu, \vv^\top)
  \mapsto \begin{pmatrix}
    a & b & 0 \\
    c & d & 1 \\
    0 & e & 0
  \end{pmatrix} \in \GL_3(F).
\]
Thus, our definition of regular requires that
$\begin{pmatrix} 0 \\ 1 \end{pmatrix}$
is linearly independent from $\begin{pmatrix} b \\ d \end{pmatrix}$
and
$\begin{pmatrix} 0 & e \end{pmatrix}$
is linearly independent from $\begin{pmatrix} c & d \end{pmatrix}$.
This is just saying that $b$, $c$, $e$ are all nonzero.
We also know that $\gamma \in S_2(F)$, which gives us relations on $a$, $b$, $c$, $d$
(the same as \cite[Equation (7.3.2)]{ref:AFLspherical}); we have
\[
  \begin{pmatrix} 1 & 0 \\ 0 & 1 \end{pmatrix}
  = \begin{pmatrix} a & b \\ c & d \end{pmatrix} \begin{pmatrix} \bar a & \bar b \\ \bar c & \bar d \end{pmatrix}
  \implies
  \begin{aligned}
    \bar b c = b \bar c &= 1 - a \bar a = 1 - d \bar d \\
    \text{and } d &= - \bar a c / \bar c = -\bar a b / \bar b.
  \end{aligned}
\]

\subsection{Simplification due to the matching of non-split unitary group}
Like before, we focus on the case where regular $(\gamma, \uu, \vv^\top)$
matches an element in the non-split unitary group.
As we described in \Cref{prop:valuation_delta_matching_semilie},
this is controlled by the parity of $v(\Delta)$, where
\[ \Delta = \det \left[ \vv^\top \gamma^{i+j} \uu \right]_{0 \le i,j \le n-1}. \]
When $n=2$, for the representatives we described before,
we have
\[ \left[ \vv^\top \gamma^{i+j} \uu \right]_{0 \le i,j \le n-1}
  = \begin{pmatrix} e & de \\ de & bce + d^2e \end{pmatrix} \]
so
\[ \Delta = bce^2 = \frac{b}{\bar b}(1-a \bar a) e^2 . \]
Hence, $v(\Delta)$ is odd if and only if $v(1-a \bar a)$ is odd.
Thus, we restrict attention to the following situation:
\begin{assume}
  \label{assume:a_odd}
  We will assume that
  \[ v(1-a \bar a) \equiv 1 \pmod 2. \]
\end{assume}
In particular, $a$ must be a unit.
And since $d = -\bar a c / \bar c$, it follows $d$ is a unit.
In other words, \Cref{assume:a_odd} gives the direct corollary
\[ v(a) = v(d) = 0. \]

\section{Parameters used in the calculation of the weighted orbital integral}
The situation is simpler than \Cref{sec:param_orbital0}
and we will state our derivative in terms of the five integers
$r$, $v(b)$, $v(c)$, $v(e)$ and $v(d-a)$.
From \Cref{assume:a_odd}, we actually get that
\begin{assume}
  \label{assume:FJ}
  We have that
  \begin{itemize}
    \ii $v(b) + v(c)$ is an odd positive integer;
    \ii $v(d-a) \ge 0$.
  \end{itemize}
\end{assume}
These are the only constraints between these five numbers we will consider
(together with $r \ge 0$).
However, we mention that we will only be interested in the case when $v(e) \ge 0$
since in the case $v(e) < 0$ we will shortly see that
$\Orb(\guv, \phi \otimes \oneV, s) = 0$ identically in $s$ in that situation.

\section{Statement of the differentiated weighted orbital integral}
We can now state the following result:
\begin{theorem}
  \label{thm:semi_lie_derivative_single}
  Let the representative
  \[
    \guv = \left( \begin{pmatrix} a & b \\ c & d \end{pmatrix},
      \begin{pmatrix} 0 \\ 1 \end{pmatrix},
      \begin{pmatrix} 0 & e \end{pmatrix} \right)
    \in (S_2(F) \times V_2'(F))\rs.
  \]
  be paired with an element of $(\U(\VV_2) \times \VV_2)\rs$
  (in particular, \Cref{assume:FJ} holds).
  Let $r \ge 0$ and define
  \begin{align*}
    N &\coloneqq \min\left( v(e), \frac{v(b)+v(c)-1}{2} + r, v(d-a) + r \right) \\
    \varkappa &\coloneqq v(e) - (v(d-a)+r)
  \end{align*}
  Then
  \begin{align*}
    \frac{(-1)^{v(c)+r}}{\log q}
    &\left. \pdv{}{s} \right\rvert_{s=0}
    \Orb(\guv, \mathbf{1}_{K'_{S, \le r}} \otimes \oneV, s) \\
    &= \sum_{j=0}^N \left( \frac{2v(e)+v(b)+v(c)+1}{2} + r - 2j \right) \cdot q^j \\
    & - q^{v(d-a)+r} \cdot
    \begin{cases}
      \frac{\varkappa}{2} & \text{if }\varkappa \equiv 0 \pmod 2 \\
      \left( v(e)+\frac{v(b)+v(c)}{2}-2v(d-a)-r \right) - \frac{\varkappa}{2}
      & \text{if }\varkappa \equiv 1 \pmod 2 \\
    \end{cases}
  \end{align*}
  where the second term is only present when $\varkappa \ge 0$ and $v(b)+v(c)>2v(d-a)$.
\end{theorem}

The formula looks somewhat simpler if one merges the contribution of
$\mathbf{1}_{K'_{S, \le r}}$ and $\mathbf{1}_{K'_{S, \le (r-1)}}$
(and in \Cref{ch:finale} we will see that
$\mathbf{1}_{K'_{S, \le r}} + \mathbf{1}_{K'_{S, \le (r-1)}}$
come up naturally again).
\begin{corollary}
  \label{cor:semi_lie_combo}
  Retain the notation of \Cref{thm:semi_lie_derivative_single} and assume $r > 0$.
  Define
  \begin{align*}
    C &\coloneqq
    \begin{cases}
      \frac{\varkappa-1}{2}
        & \text{if } \varkappa > 0 \text{ is odd}
          \text{ and } v(b) + v(c) > 2v(d-a)  \\
      \frac{\varkappa+v(b)+v(c)-2v(d-a)-1}{2}
        & \text{if } \varkappa \ge 0 \text{ is even}
          \text{ and } v(b) + v(c) > 2v(d-a)  \\
      v(e) - N
        & \text{if } v(e) \ge \frac{v(b)+v(c)-1}{2} + r
        \text{ and } 2v(d-a) > v(b) + v(c) \\
      0 & \text{otherwise}
    \end{cases} \\
    C' &\coloneqq
    \begin{cases}
      C + 1 & \text{if } \varkappa \ge 0 \text{ and } v(b)+v(c) > 2v(d-a) \\
      0 & \text{otherwise}.
    \end{cases}
  \end{align*}
  Then
  \begin{align*}
    \frac{(-1)^{v(c)+r}}{\log q} &\left. \pdv{}{s} \right\rvert_{s=0}
    \Orb(\guv, \left(\mathbf{1}_{K'_{S, \le r}} + \mathbf{1}_{K'_{S, \le (r-1)}}\right) \otimes \oneV, s) \\
    &= (q^N + q^{N-1} + \dots + 1) + C q^N + C' q^{N-1}
  \end{align*}
\end{corollary}

We dedicate \Cref{ch:orbitalFJ1,ch:orbitalFJ2} to the calculation
of the full orbital integral and the above formulas.
We have the following examples to showcase \Cref{cor:semi_lie_combo}.

\begin{example}
  When $v(e) < 0$, the entire expression is zero
  (indeed in that case the orbital integral is identically zero).
\end{example}

\begin{example}
  When $r=5$, $v(b) = -20$, $v(c) = 37$, $v(e) = 35$ and $v(d-a) > \frac{v(b)+v(c)}{2} = 8.5$
  the derivative in \Cref{cor:semi_lie_combo} equals
  \[ \log q \cdot (23q^{13} + q^{12} + q^{11} + q^{10} + q^9 + \dots + q + 1). \]
\end{example}

\begin{example}
  When $r = 6$, $v(b) = 10$, $v(c) = 5$, $v(e) = 7$, $v(d-a) > v(e)-r = 1$,
  the derivative in \Cref{cor:semi_lie_combo} equals
  \[ -\log q \cdot (q^7 + q^6 + q^5 + \dots + q + 1). \]
\end{example}

\begin{example}
  When $r = 8$, $v(b) = -101$, $v(c) = 1000$, $v(e) = 29$, $v(d-a) = 11$,
  the derivative in \Cref{cor:semi_lie_combo} equals
  \[ \log q \cdot (444 q^{19} + 445q^{18} + q^{17} + q^{16} + q^{15} + \dots + q + 1). \]
\end{example}

\section{Support of the weighted orbital integral for $S_2(F) \times V_2'(F)$}
\label{ch:orbitalFJ1}

We assume $\guv$ is as in \Cref{lem:semi_lie_params} throughout this chapter.

\subsection{Iwasawa decomposition}
The overall method is to take the Iwasawa decomposition in $KAN$ form:
\begin{lemma}
  [Iwasawa decomposition]
  Every element in $h \in \GL_2(F)$ may be parametrized as
  \[ h = k \begin{pmatrix} x_1 & 0 \\ 0 & x_2 \end{pmatrix}
    \begin{pmatrix} 1 & y \\ 0 & 1 \end{pmatrix} \]
  where $k \in K' = \GL_2(\OO_F)$, $x_1, x_2 \in \OO_F^\times$ and $y \in \OO_F$.
\end{lemma}
Because the orbits are invariant under conjugation by $K'$,
the parameter $k$ can be discarded.
The Haar measure in these coordinates
\[ \left\lvert \frac{x_1}{x_2} \right\rvert \odif[\times] x_1 \odif[\times] x_2 \odif y \]
where we take multiplicative Haar measure on $F^\times$
(normalized so that $\OO_F^\times$ has volume $1$)
and additive Haar measure on $F$ (so $\OO_F$ has volume $1$).

\subsection{Action of upper triangular matrices on $(\gamma, \uu, \vv^\top)$}
We now compute the action of an arbitrary
\[ h = \begin{pmatrix} x_1 & 0 \\ 0 & x_2 \end{pmatrix}
  \begin{pmatrix} 1 & y \\ 0 & 1 \end{pmatrix} \]
on $(\gamma, \uu, \vv^\top)$.
The main term is given by
\begin{align*}
  h \gamma h^{-1}
  &=
  \begin{pmatrix} x_1 & 0 \\ 0 & x_2 \end{pmatrix}
  \begin{pmatrix} 1 & y \\ 0 & 1 \end{pmatrix}
  \begin{pmatrix} a & b \\ c & d \end{pmatrix}
  \begin{pmatrix} 1 & -y \\ 0 & 1 \end{pmatrix}
  \begin{pmatrix} x_1^{-1} & 0 \\ 0 & x_2^{-1} \end{pmatrix} \\
  &=
  \begin{pmatrix} x_1 & 0 \\ 0 & x_2 \end{pmatrix}
  \begin{pmatrix} cy + a & -cy^2+(d-a)y+b \\ c & -cy+d \end{pmatrix}
  \begin{pmatrix} x_1^{-1} & 0 \\ 0 & x_2^{-1} \end{pmatrix} \\
  &=
  \begin{pmatrix} cy + a & \frac{x_1}{x_2} \cdot \left( -cy^2+(d-a)y+b \right) \\
    \frac{x_2}{x_1} \cdot c & -cy+d \end{pmatrix}
\end{align*}
Meanwhile, we have
\begin{align*}
  h \uu &=
    \begin{pmatrix} x_1 & 0 \\ 0 & x_2 \end{pmatrix}
    \begin{pmatrix} 1 & y \\ 0 & 1 \end{pmatrix}
    \begin{pmatrix} 0 \\ 1 \end{pmatrix}
    = \begin{pmatrix} x_1 y \\ x_2 \end{pmatrix} \\
  \vv^\top h^{-1} &=
    \begin{pmatrix} 0 & e \end{pmatrix}
    \begin{pmatrix} 1 & -y \\ 0 & 1 \end{pmatrix}
    \begin{pmatrix} x_1^{-1} & 0 \\ 0 & x_2^{-1} \end{pmatrix}
    = \begin{pmatrix} 0 & \frac{e}{x_2} \end{pmatrix}.
\end{align*}

\subsection{Description of support}
From now on we fix the notation
\begin{align*}
  n_1 &\coloneqq v(x_1) \\
  n_2 &\coloneqq v(x_2).
\end{align*}
Note that although $n_2 \ge 0$, the value of $n_1$ will often be non-positive.
In fact $n_1$ is not particularly simple to work with and we will
prefer to introduce the notation
\begin{equation}
  m \coloneqq n_2 + v(c) + r - n_1
  \label{eq:def_semi_lie_m}
\end{equation}
instead to use as a summation variable.
This is chosen so that $\frac{x_2}{x_1} \cdot c \in \varpi^{-r} \OO_F \iff m \ge 0$.
Note that it follows we have
\begin{equation}
  n_1 + n_2 = 2n_2 - m + v(c) + r
  \label{eq:n1_plus_n2_semi_lie}
\end{equation}


\subsubsection{Collating the linear constraints}
For a given $r \ge 0$, we find that $h$ contributes to the integral exactly
if $h\uu$ and $\vv^\top h\inv$ have $\OO_F$-entries,
and all the entries of $h \gamma h\inv$ are in $\varpi^{-r}\OO_F$.
The former condition is just saying that
\begin{align*}
  v(y) &\ge -n_1, \\
  0 &\le n_2 \le v(e).
\end{align*}
Now we consider the entries of $h\gamma h\inv$.
First, because $a$ and $d$ are units by \Cref{assume:a_odd},
and $r \ge 0$, it follows that
\begin{align*}
  cy + a, -cy + d \in \varpi^{-r}\OO_F
  &\iff cy \in \varpi^{-r}\OO_F \\
  &\iff v(y) \ge -v(c) - r.
\end{align*}
Moreover,
\[ \frac{x_2}{x_1} \cdot c \in \varpi^{-r} \OO_F
  \iff n_2 + v(c) -n_1 \ge - r \iff m \ge 0. \]
In summary, up until now we have the following requirements imposed:
\begin{equation}
  \begin{aligned}
  0 &\le n_2 \le v(e) \\
  0 &\le m \\
  v(y) &\ge \max(-n_1, -v(c) - r) \\
  % &= \max(m - n_2 - v(c) - r, - v(c) - r) \\
  &= \max(m-n_2, 0) - v(c) - r.
  \end{aligned}
  \label{eq:linear_constraints}
\end{equation}

\subsubsection{The quadratic constraint}
As for the quadratic constraint, we seek $y$ such that
\begin{align*}
  \phantom\iff \frac{x_1}{x_2} \cdot (-cy^2+(d-a)y+b) &\in \varpi^{-r} \OO_F \\
  \iff v\left(-y^2+ \frac{d-a}{c} y + \frac bc \right) &\ge n_2 - n_1 - v(c) - r \\
  &= m - 2v(c) - 2r.
\end{align*}

As before, we complete the square:
\[
  -y^2+ \frac{d-a}{c} y + \frac bc
  = -\left( y - \frac{d-a}{2c} \right)^2 + \frac bc + \frac{(d-a)^2}{4c^2}.
\]
Because $b \bar c = 1 - a \bar a$ has odd valuation,
it follows that $\frac b c = \frac{1-a \bar a}{c\bar c}$ has odd valuation to.
On the other hand, $\frac{(d-a)^2}{4c^2}$ has even valuation.

This motivates us to introduce the following parameter:
\begin{definition}
  [$\theta$]
  We define
  \[ \theta \coloneqq \min \left( v(b)+v(c), 2v(d-a) \right) \ge 0. \]
\end{definition}
Note that $v(b) + v(c)$ is odd,
so $\theta$ takes the odd value if $v(b)+v(c) < 2v(d-a)$ and the even value otherwise.
This definition ensures that
\[ v \left( \frac bc + \frac{(d-a)^2}{4c^2} \right) = \theta - 2v(c). \]

\subsection{Cases based on $\theta$}
Henceforth we consider two cases based on $\theta$.
\ifthesis
We number these Case 5 and Case 6 to prevent confusion
with the cases introduced in \Cref{ch:orbital1}.
\fi
\begin{description}
  \item[Case 5]
  Let's assume first that
  \[ \theta - 2v(c) \ge m - 2v(c) - 2r \iff m \le \theta + 2r. \]
  Then the only additional condition on $y$ is that
  \[
    v\left( y - \frac{d-a}{2c} \right)
    \ge \left\lceil \frac{m}{2} \right\rceil - v(c) - r.
  \]
  We refer to this as \textbf{Case 5}.

  \item[Case 6\ts+ / Case 6\ts-]
  Otherwise assume that
  \[ \theta - 2v(c) < m - 2v(c) - 2r \iff m > \theta + 2r. \]
  Then in order for $y$ to satisfy the constraint,
  we would need to be in a situation where $2v(y - \frac{d-a}{2c}) = \theta - 2v(c)$.
  So this case could only arise at all when $\theta$ is even, that is
  \[ 0 \le 2v(d-a) = \theta < v(b) + v(c) \]
  (note that $v(d-a) \ge 0$ because $a$ and $d$ are units).
  As the quantity $\frac bc + \frac{(d-a)^2}{4c^2}$ must be a perfect square,
  we denote it by $\tau^2$, with
  \[ v(\tau) = \frac{\theta}{2} - v(c). \]
  This gives us the factorization
  \[ \frac bc = \tau^2 - \frac{(d-a)^2}{4c^2}
    = \left( \tau - \frac{d-a}{2c} \right) \left( \tau + \frac{d-a}{2c} \right). \]
  The left-hand side has odd valuation $v(b) - v(c)$,
  so the two factors on the right have unequal valuations
  and hence exactly one of them has valuation the same as $v(\frac{d-a}{2c}) = v(\tau)$.
  Hence, we agree to fix the choice of the square root $\tau$ so that
  \begin{align*}
    v\left( \tau + \frac{d-a}{2c} \right) &= v(b) - v(c) - v(\tau) = v(b) - \frac{\theta}{2} \\
    v\left( \tau - \frac{d-a}{2c} \right) &= v(\tau) = \frac{\theta}{2} - v(c)
  \end{align*}
  and in particular
  $v\left( \tau + \frac{d-a}{2c} \right) > v\left( \tau - \frac{d-a}{2c} \right)$.

  In any case, the constraint on $y$ is that
  \begin{align*}
    v\left( y - \left( \frac{d-a}{2c} \pm \tau \right) \right)
      &\ge \left( m - 2v(c) - 2r \right) - v(\tau) \\
      &= \left( m - 2v(c) - 2r \right) - \left( \frac{\theta}{2} - v(c) \right) \\
      &= m - \frac{\theta}{2} - v(c) - 2r \\
    v\left( y - \left( \frac{d-a}{2c} \mp \tau \right) \right) &= v(\tau)
      = \frac{\theta}{2} - v(c).
  \end{align*}
  By assumption, the second equation is true
  whenever the first inequality is and we may disregard it.
  \textbf{Case 6\ts+} refers to the situation where the $\pm$ sign is $+$
  and \textbf{Case 6\ts-} refers to the situation where the $\mp$ sign is $-$.
  And these cases must be disjoint because the right-hand sides above are unequal.
\end{description}

\subsubsection{Analysis of Case 5}
The triple $(x_1, x_2, y) \in \OO_F^\times \times \OO_F^\times \times \OO_F$
contributes to the weighted orbital integral in Case 5 exactly if the following identities hold:
\begin{align*}
  0 &\le n_2 \le v(e) \\
  0 &\le m \le \theta + 2r \\
  v(y) &\ge \max(m-n_2,0) - v(c) - r \\
  v\left( y - \frac{d-a}{2c} \right) &\ge \left\lceil \frac{m}{2} \right\rceil - v(c) - r.
\end{align*}
However, from the definitions we already know that
\[ v\left( \frac{d-a}{2c} - 0 \right)
  \ge \frac{\theta - 2v(c)}{2} \ge \frac{m}{2} - v(c) - r \]
so the disks in the last two conditions have nonempty intersection.
Hence the earlier \Cref{lem:no_mastercard} applies to tell us that
the locus of valid $y$ is a single disk whose volume in $\OO_F$ is given by
\[ q^{-\max\left( m-n_2, \left\lceil m/2 \right\rceil, 0 \right) + v(c) + r}
  = q^{-\max\left( m-n_2, \left\lceil m/2 \right\rceil \right) + v(c) + r}. \]
The volume contribution for and $x_1 \in \OO_F^\times$ and $x_2 \in \OO_F^\times$
is also $1$, because $v(x_1)$ and $v(x_2)$ are fixed.
Hence the overall volume of the support in $H$ for this pair $(m, n_2)$ is given by
\begin{align*}
  \left\lvert \frac{x_1}{x_2} \right\rvert q^{n_2 - n_1} \Vol(\{ y \mid \dots \})
  &= q^{n_2 - n_1 -\max\left( m-n_2, \left\lceil m/2 \right\rceil \right) + v(c) + r} \\
  &= q^{m - \max\left( m-n_2, \left\lceil m/2 \right\rceil \right)}.
\end{align*}
And again, this case is summed over
\[ 0 \le n_2 \le v(e), \qquad 0 \le m \le \theta + 2r. \]

\subsubsection{Analysis of Case 6\ts+ and Case 6\ts-}
Again, this case could only occur if $\theta$ is even.
The triple $(x_1, x_2, y) \in \OO_F^\times \times \OO_F^\times \times \OO_F$
contributes to the weighted orbital integral in Case 6\ts+ and Case 6\ts-
exactly if the following identities hold:
\begin{align*}
  0 &\le n_2 \le v(e) \\
  \theta + 2r &< m \\
  v(y) &\ge \max(m-n_2,0) - v(c) - r \\
  v\left( y - \left( \frac{d-a}{2c} \pm \tau \right) \right) &\ge m - \frac{\theta}{2} - v(c) - 2r.
\end{align*}
The last two inequalities specify disks.
So in each case, via \Cref{lem:no_mastercard}
we get a nonzero contribution if and only if the distance between the centers
$0$ and $\frac{d-a}{2c} \pm \tau$ has valuation at least
that of the smaller of the two right-hand sides, that is
\begin{align*}
  v\left( \frac{d-a}{2c} \pm \tau \right)
  &\ge \min\left( \max(m-n_2,0) - v(c) - r, m - \frac{\theta}{2} - v(c) - 2r \right) \\
  &= \min\left( \max(m-n_2,0), m - \frac{\theta}{2} - r \right) - v(c) - r.
\end{align*}
Hence the upper bound on $m$ is given by two different requirements,
depending on which of the two values of
$v\left( \frac{d-a}{2c} \pm \tau \right) + v(c) + r$ is given by the case:
\begin{itemize}
  \ii In Case 6\ts+, we need at least one of the inequalities
  \[
    \begin{cases}
    \max(m-n_2, 0) \le v(b) - \frac{\theta}{2} + v(c) + r, \\
    m \le v(b) + v(c) + 2r
    \end{cases}
  \]
  to hold.
  Now the inequality $0 \le v(b) - \frac{\theta}{2} + v(c) + r$ is always true,
  as $\theta < v(b) + v(c)$, so we can disregard it.
  Therefore this can be rewritten as just
  \[ m \le \max\left( r, n_2 - \frac{\theta}{2} \right) + v(b) + v(c) + r. \]

  \ii In Case 6\ts-, we need at least one of the inequalities
  \[
    \begin{cases}
      \max(m-n_2, 0) \le \frac{\theta}{2} + r \\
      m \le \theta + 2r
    \end{cases}
  \]
  to hold.
  But $m \le \theta+2r$ is always false and $0 \le \frac{\theta}{2} + r$ is always true,
  so this simplifies to
  \[ m \le n_2 + \frac{\theta}{2} + r. \]
\end{itemize}
Assuming $m$ lies in the valid range so that the locus of valid $y$ is nonempty,
it follows that the volume is given exactly by
\[ q^{-\max(m-n_2, m - \frac{\theta}{2} - r, 0) - v(c) - r}
  = q^{-\max(m-n_2, m - \frac{\theta}{2} - r) - v(c) - r}. \]
Hence the overall volume of the support in $H$ for this pair $(m, n_2)$ is given by
\begin{align*}
  \left\lvert \frac{x_1}{x_2} \right\rvert q^{n_2 - n_1} \Vol(\{ y \mid \dots \})
  &= q^{n_2 - n_1 - \max\left( m-n_2, m - \frac{\theta}{2} - r \right) + v(c) + r} \\
  &= q^{m - \max\left( m-n_2, m - \frac{\theta}{2} - r \right)} \\
  &= q^{\min\left( n_2, \frac{\theta}{2} + r \right)}.
\end{align*}
And this sum is over two ranges of $m$
(although the ranges obviously overlap, they set of $y$ they cover is disjoint):
\begin{align*}
  \theta + 2r & < m \le \max\left( r, n_2 - \frac{\theta}{2} \right) + v(b) + v(c) + r \\
  \theta + 2r & < m \le n_2 + \frac{\theta}{2} + r.
\end{align*}
Note the second range could be empty if $n_2$ is small enough,
but the first range is always nonempty.

\chapter{Evaluation of the weighted orbital integral for $S_2(F) \times V'_2(F)$}
\label{ch:orbitalFJ2}

We now aggregate the supports we found in the previous section together with the
definition of the weighted orbital integral to extract the desired formulas.

Recall that the weighted orbital integral was defined as
\begin{align*}
  & \Orb(\guv, \phi \otimes \oneV, s) \\
  &\coloneqq
  \int_{h \in H} \phi(h\inv \gamma h)
  \oneV(h \uu, \vv^\top h^{-1})
  \eta(h) \left\lvert \det(h) \right\rvert_F^{-s} \odif h
\end{align*}
and that after taking Iwasawa decomposition as
\[ h = k \begin{bmatrix} x_1 & 0 \\ 0 & x_2 \end{bmatrix}
  \begin{bmatrix} 1 & y \\ 0 & 1 \end{bmatrix} \]
we broke the sum based on $n_1 = v(x_1)$ and $n_2 = v(x_2)$.
So the contribution to the weighted orbital integral looks like
For $h$ as above, we know that
\begin{align*}
  \eta(h) &= (-1)^{n_1 + n_2} \\
  \left\lvert \det(h) \right\rvert^{-s}_F &= (q^s)^{n_1 + n_2}.
\end{align*}
Applying \eqref{eq:n1_plus_n2_semi_lie} we find that
\[
  \eta(h)
  \left\lvert \det(h) \right\rvert^{-s}_F
  = (q^s)^{2n_ - m + v(c) + r}.
\]

\section{The contribution for Case 5}
We assume $\theta + 2r \ge 0$, because otherwise the entire sum is empty.
Hence, the total contribution for \textbf{Case 5} is
\begin{align*}
  I^{\text{5}}
  &\coloneqq \sum_{n_2 = 0}^{v(e)} \sum_{m = 0}^{\theta + 2r}
  q^{m - \max\left( m-n_2, \left\lceil m/2 \right\rceil \right)}
  (-q^s)^{2n_2 - m + v(c) + r} \\
  &\coloneqq \sum_{n_2 = 0}^{v(e)} \sum_{m = 0}^{\theta + 2r}
  q^{\min\left( n_2, \left\lfloor m/2 \right\rfloor \right)}
  (-q^s)^{2n_2 - m + v(c) + r}.
\end{align*}
We'll change the summation variable to
\[ k \coloneqq 2n_2 - m + v(c) + r
\iff m = 2n_2 - k + v(c) + r \]
Then
\begin{align*}
  I^{\text{5}}
  &\coloneqq \sum_{n_2 = 0}^{v(e)}
  \sum_{k = 2n_2 - \theta + v(c) - r}^{2n_2 + v(c) + r}
  q^{\min\left( n_2, n_2 + \left\lfloor \frac{v(c)+r-k}{2} \right\rfloor \right)} (-q^s)^{k} \\
  &= \sum_{n_2 = 0}^{v(e)}
  \sum_{k = 2n_2 - \theta + v(c) - r}^{2n_2 + v(c) + r}
  q^{n_2 - \max\left( 0, \left\lceil \frac{k-(v(c)+r)}{2} \right\rceil \right)} (-q^s)^{k}.
\end{align*}
We then interchange the order of summation so that $k$ is outside.
Then $k$ runs from the lowest value of $k = - \theta + v(c) - r$
to the largest value $k = 2v(e) + v(c) + r$ over all choices of $n_2$.
Since
\[ 2n_2 - \theta + v(c) - r \le k \le 2 n_2 + v(c) + r \]
then in addition to $0 \le n_2 \le v(e)$ we also need
\[ \frac{k-v(c)-r}{2} \le n_2 \le \frac{k + \theta - v(c) + r}{2}. \]
In other words, we obtain
\begin{align*}
  I^{\text{5}}
  &= \sum_{k = - \theta + v(c) - r}^{2v(e) + v(c) + r}
  (-1)^k (q^s)^k \sum_{n_2 = \max\left(0, \left\lceil \frac{k - v(c) - r}{2} \right\rceil \right)}
  ^{\min\left(v(e), \left\lfloor \frac{k + \theta - v(c) + r}{2} \right\rfloor\right)}
  q^{n_2 - \max\left( 0, \left\lceil \frac{k-(v(c)+r)}{2} \right\rceil \right)} \\
  &= \sum_{k = - \theta + v(c) - r}^{2v(e) + v(c) + r}
  (-1)^k (q^s)^k
  \left( q^{\min\left( v(e), \left\lfloor \frac{k+\theta-v(c)+r}{2} \right\rfloor \right) - \max\left( 0, \left\lceil \frac{k-v(c)-r}{2} \right\rceil \right)} + \dots + q^0 \right).
\end{align*}
Here, we retain the convention from \Cref{ch:orbital2} that ellipses of the form
\[ q^i + \dots + q^{i'} \]
will denote the expression $q^i + q^{i-1} + \dots + q^{i'}$
(i.e.\ within any ellipses, the exponents are understood to decrease by $1$,
and the sums are always nonempty, meaning $i \ge i'$).

To simplify the exponent, write
\begin{equation}
  \begin{aligned}
    &\min\left( v(e), \left\lfloor \tfrac{k+\theta-v(c)+r}{2} \right\rfloor \right)
    - \max\left( 0, \left\lceil \tfrac{k-v(c)-r}{2} \right\rceil \right) \\
    &= \min\left( v(e), \left\lfloor \tfrac{k+\theta-v(c)+r}{2} \right\rfloor \right)
    + \min\left( 0, \left\lfloor \tfrac{v(c)+r-k}{2} \right\rfloor \right) \\
    &= \min\left( \left\lfloor \tfrac{k+\theta-v(c)+r}{2} \right\rfloor,
      v(e) + \left\lfloor \tfrac{v(c)+r-k}{2} \right\rfloor,
      v(e),
      \left\lfloor \tfrac{k+\theta-v(c)+r}{2} \right\rfloor
      + \left\lfloor \tfrac{v(c)+r-k}{2} \right\rfloor \right).
  \end{aligned}
  \label{eq:case_5_exponent}
\end{equation}
This already completes \Cref{thm:semi_lie_formula}
in the situation when $\theta$ is odd since
\textbf{Case 6\ts+} and \textbf{Case 6\ts-} do not appear at all.
However, let's turn to the remaining cases first.

\section{The contribution for Case 6\ts+ and Case 6\ts-}
Herein we assume $\theta = 2v(d-a) > v(b) + v(c)$ is even,
and in particular $\theta \ge 0$.
We get a contribution of
\begin{align*}
  I^{\text{6+}}
  &\coloneqq \sum_{n_2 = 0}^{v(e)}
  \sum_{m = \theta + 2r + 1}
  ^{\max\left( r, n_2 - \frac{\theta}{2} \right) + v(b) + v(c) + r}
    q^{\min\left( n_2, \frac{\theta}{2} + r \right)}
    (-q^s)^{2n_2 - m + v(c) + r} \\
  I^{\text{6-}}
  &\coloneqq \sum_{n_2 = 0}^{v(e)}
  \sum_{m = \theta + 2r + 1}^{n_2 + \frac{\theta}{2} + r}
    q^{\min\left( n_2, \frac{\theta}{2} + r \right)}
    (-q^s)^{2n_2 - m + v(c) + r}.
\end{align*}
We will split $I^{\text{6+}}$ into two parts:
\begin{align*}
  I^{\text{6+}}
  &= \sum_{n_2 = 0}^{\frac{\theta}{2} + r}
  \sum_{m = \theta + 2r + 1}^{v(b) + v(c) + 2r}
    q^{n_2} (-q^s)^{2n_2 - m + v(c) + r} \\
  &+ q^{\frac{\theta}{2} + r} \sum_{n_2 = \frac{\theta}{2} + r + 1}^{v(e)}
  \sum_{m = \theta + 2r + 1}^{n_2 - \frac{\theta}{2} + v(b) + v(c) + r}
    (-q^s)^{2n_2 - m + v(c) + r}.
\end{align*}
Note that the second sum is nonempty only when $v(e) > \frac{\theta}{2} + r$.
So we consider cases on this in what follows.

\subsection{Sub-case where $v(e) \le \frac{\theta}{2} + r$}
First, suppose $v(e) \le \frac{\theta}{2} + r$.
Then the contribution of \textbf{Case 6\ts{-}} is void,
since the inner sum of $I^{\text{6-}}$ contributes only when $n_2 > \frac{\theta}{2} + r$.
We only need to consider
\begin{align*}
  I^{\text{6+}}
  &= \sum_{n_2=0}^{v(e)} \sum_{m=\theta+2r+1}^{v(b)+v(c)+2r}
    q^{n_2} (-q^s)^{2n_2-m+v(c)+r} \\
  &= \sum_{n_2=0}^{v(e)} \sum_{k=2n_2-v(b)-r}^{2n_2-\theta+v(c)-r-1}
    q^{n_2} (-q^s)^k.
\end{align*}
Swapping the summation order so that $k$ is outside,
the sum runs from the lowest value $k = -v(b) - r$
up to the highest value $k = 2v(e) - \theta + v(c) - r - 1$,
subject to $0 \le n_2 \le v(e)$ and
\begin{align*}
  2n_2 - v(b) - r &\le k \le 2n_2 - \theta + v(c) - r - 1 \\
  \iff \left\lceil \frac{k + \theta - v(c) + r + 1}{2} \right\rceil
  &\le n_2 \le \left\lfloor \frac{k + v(b) + r}{2} \right\rfloor.
\end{align*}
Thus,
\[ I^{\text{6+}}
  = \sum_{k = -v(b) - r}^{2v(e) - \theta + v(c) - r -1}
    \sum_{n_2 = \max(0, \left\lceil \frac{k + \theta - v(c) + r + 1}{2} \right\rceil)}
    ^{\min(v(e), \left\lfloor \frac{k + v(b) + r}{2} \right\rfloor)}
    q^{n_2} (-q^s)^k.
\]

\subsection{Sub-case where $v(e) > \frac{\theta}{2} + r$}
We start on $I^{\text{6-}}$; note if $n_2 \le \frac{\theta}{2} + r$
then the inner sum of $I^{\text{6-}}$ has empty range anyway.
Consequently, we can simply write
\begin{align*}
  I^{\text{6-}}
  &= q^{\frac{\theta}{2} + r} \sum_{n_2 = \frac{\theta}{2} + r + 1}^{v(e)}
  \sum_{m = \theta + 2r + 1}^{n_2 + \frac{\theta}{2} + r} (-q^s)^{2n_2 - m + v(c) + r}
\end{align*}
which in particular is nonempty.
In that case, simplifying the inner sum gives
\[
  I^{\text{6-}}
  = q^{\frac{\theta}{2} + r} \sum_{n_2 = \frac{\theta}{2} + r + 1}^{v(e)}
  \left(
    (-q^s)^{2n_2 - \theta + v(c) - r - 1}
    + \dots
    + (-q^s)^{n_2 - \frac{\theta}{2} + v(c)}
  \right).
\]
We collect the coefficient of $(-q^s)^k$ for each $k$.
The lowest value of $k$ which appears is $k = v(c) + r + 1$;
the highest one is $k = 2v(e) - \theta + v(c) - r - 1$.
For these $k$,
the coefficient is the number of integers $n_2$ such that
\[ \frac{\theta}{2} + r + 1 \le n_2 \le v(e) \]
and
\begin{align*}
  n_2 - \frac{\theta}{2} + v(c) &\le k \le 2n_2 - \theta - r - 1 + v(c) \\
  \iff \frac{k + \theta - v(c) + r + 1}{2} &\le n_2 \le k + \frac{\theta}{2} - v(c).
\end{align*}
Note we already have $\frac{k + \theta - v(c) + r + 1}{2} \ge \frac{\theta}{2}+r+1$
for $k$ in the desired range.
Hence we have
\begin{align*}
  I^{\text{6-}}
  &=
  q^{\frac{\theta}{2} + r}
  \sum_{k = v(c) + r + 1}^{2v(e) - \theta + v(c) - r - 1}
  \bigg( 1 + \min\left( v(e), k + \frac{\theta}{2} - v(c) \right) \\
    &\hspace{16ex} - \max\left( \frac{\theta}{2} + r + 1,
      \left\lceil \frac{k + \theta - v(c) + r + 1}{2} \right\rceil \right) \bigg) (-q^s)^k \\
  &=
  q^{\frac{\theta}{2} + r}
  \sum_{k = v(c) + r + 1}^{2v(e) - \theta + v(c) - r - 1}
  \left( 1 + \min\left( v(e), k + \frac{\theta}{2} - v(c) \right)
    - \left\lceil \frac{k + \theta - v(c) + r + 1}{2} \right\rceil
  \right) (-q^s)^k.
\end{align*}

The second double sum of $I^{\text{6+}}$ is again nonempty
since $v(e) > \frac{\theta}{2} + r$.
So we compute it
in a similar way to $I^{\text{6-}}$ by putting
\begin{align*}
  &q^{\frac{\theta}{2} + r} \sum_{n_2 = \frac{\theta}{2} + r + 1}^{v(e)}
  \sum_{m = \theta + 2r + 1}^{n_2 - \frac{\theta}{2} + v(b) + v(c) + r}
    (-q^s)^{2n_2 - m + v(c) + r} \\
  &= q^{\frac{\theta}{2} + r} \sum_{n_2 = \frac{\theta}{2} + r + 1}^{v(e)}
  \left(
    (-q^s)^{2n_2 - \theta + v(c) - r - 1}
    + \dots
    + (-q^s)^{n_2 + \frac{\theta}{2} - v(b)}
  \right).
\end{align*}
Again we calculate the coefficient of $(-q^s)^k$.
The values of $k$ run from the lowest value $k = \theta - v(b) + r + 1$
and end at the highest value $k = 2v(e) - \theta + v(c) - r - 1$.
In this range we need $\frac{\theta}{2} + r + 1 \le n_2 \le v(e)$ and
\begin{align*}
  n_2 + \frac{\theta}{2} - v(b) &\le k \le 2n_2 - \theta + v(c) - r - 1 \\
  \iff \frac{k + \theta - v(c) + r + 1}{2} &\le n_2 \le k - \frac{\theta}{2} + v(b).
\end{align*}
The double sum therefore becomes
\begin{align*}
  &\phantom=
  q^{\frac{\theta}{2} + r}
  \sum_{k = \theta - v(b) + r + 1}^{2v(e) - \theta + v(c) - r - 1}
  \bigg( 1 + \min\left( v(e), k - \frac{\theta}{2} + v(b) \right) \\
    &\hspace{16ex} - \max\left( \frac{\theta}{2} + r + 1,
      \left\lceil \frac{k + \theta - v(c) + r + 1}{2} \right\rceil \right) \bigg) (-q^s)^k
\end{align*}
It is natural to split this sum into $k \le v(c) + r$ and $k > v(c) + r$.
In the former case, we have both $k - \frac{\theta}{2} + v(b) \le v(e)$
and $\frac{\theta}{2} + r + 1 \ge \left\lceil \frac{k + \theta - v(c) + r + 1}{2} \right\rceil$;
in the latter case we have just
$\frac{\theta}{2} + r + 1 \le \left\lceil \frac{k + \theta - v(c) + r + 1}{2} \right\rceil$
instead.
Hence, the double sum simplifies further to
\begin{align*}
  &= q^{\frac{\theta}{2} + r}
  \sum_{k = \theta - v(b) + r + 1}^{v(c) + r}
  \bigg( 1 + \left( k - \frac{\theta}{2} + v(b) \right)
    - \left( \frac{\theta}{2} + r + 1 \right) \bigg) (-q^s)^k \\
  &+ q^{\frac{\theta}{2} + r}
  \sum_{k = v(c) + r + 1}^{2v(e) - \theta + v(c) - r - 1}
  \bigg( 1 + \min\left( v(e), k - \frac{\theta}{2} + v(b) \right)
    - \left\lceil \frac{k + \theta - v(c) + r + 1}{2} \right\rceil \bigg) (-q^s)^k \\
  &= q^{\frac{\theta}{2} + r}
  \sum_{k = \theta - v(b) + r + 1}^{v(c) + r}
  \left( k - \theta + v(b) - r \right) (-q^s)^k \\
  &+ q^{\frac{\theta}{2} + r}
  \sum_{k = v(c) + r + 1}^{2v(e) - \theta + v(c) - r - 1}
  \bigg( 1 + \min\left( v(e), k - \frac{\theta}{2} + v(b) \right)
    - \left\lceil \frac{k + \theta - v(c) + r + 1}{2} \right\rceil \bigg) (-q^s)^k.
\end{align*}
Meanwhile, the first sum within $I^{\text{6+}}$ can be computed as
\begin{align*}
  \sum_{n_2 = 0}^{\frac{\theta}{2} + r}
  \sum_{m = \theta + 2r + 1}^{v(b) + v(c) + 2r}
    q^{n_2} (-q^s)^{2n_2 - m + v(c) + r}
  &= \sum_{n_2 = 0}^{\frac{\theta}{2} + r} q^{n_2}
    \sum_{m = \theta + 2r + 1}^{v(b) + v(c) + 2r}
      (-q^s)^{2n_2 - m + v(c) + r} \\
  &= \sum_{n_2 = 0}^{\frac{\theta}{2} + r} q^{n_2}
    \sum_{k = 2n_2 - v(b) - r}^{2n_2 - \theta + v(c) - r - 1} (-q^s)^k.
\end{align*}
We now interchange the summation so that $k$ is outside,
running from the lowest value $k = -v(b) - r$
to the highest value $k = v(c) + r - 1$.
From
\[ 2n_2 - v(b) - r \le k \le 2n_2 - \theta + v(c) - r - 1 \]
we require that $0 \le n_2 \le \frac{\theta}{2} + r$ and
\[ \frac{k + \theta - v(c) + r + 1}{2} \le n_2 \le \frac{k + r + v(b)}{2}. \]
In other words, we get
\[
  \sum_{k = - v(b) - r}^{v(c) + r - 1}
  (-q^s)^k
  \sum_{n_2 = \max\left(0, \left\lceil \frac{k + \theta - v(c) + r + 1}{2} \right\rceil \right)}
  ^{\min\left( \frac{\theta}{2} + r, \left\lfloor \frac{v(b) + r + k}{2} \right\rfloor \right) } q^{n_2}.
\]
Hence the total contribution from \textbf{Case 6} can be written as
\begin{align*}
  I^{\text{6+}} + I^{\text{6-}}
  &=
  \sum_{k = - v(b) - r}^{v(c) + r - 1} (-q^s)^k \left(
    q^{\min\left( \frac{\theta}{2} + r, \left\lfloor \frac{v(b) + r + k}{2} \right\rfloor \right)}
    + \dots
    + q^{\max\left(0, \left\lceil \frac{k + \theta - v(c) + r + 1}{2} \right\rceil \right)}
    \right) \\
  &+ q^{\frac{\theta}{2} + r}
  \sum_{k = \theta - v(b) + r + 1}^{v(c) + r}
  \left( k - \theta + v(b) - r \right) (-q^s)^k \\
  &+ q^{\frac{\theta}{2} + r}
  \sum_{k = v(c) + r + 1}^{2v(e) - \theta + v(c) - r - 1}
  \bigg( 2 +
    \min\left( v(e), k - \frac{\theta}{2} + v(b) \right)
    + \min\left( v(e), k + \frac{\theta}{2} - v(c) \right) \\
    &\hspace{16ex} - 2\left\lceil \frac{k + \theta - v(c) + r + 1}{2} \right\rceil \bigg) (-q^s)^k.
\end{align*}
We'd like to further simplify the coefficient of $q^{\frac{\theta}{2}+r}$ as follows.
First, we may as well write
\begin{align*}
  2 - 2\left\lceil \frac{k + \theta - v(c) + r + 1}{2} \right\rceil
  &= 2 - \left( (k + \theta - v(c) + r + 1)
  + \mathbf{1}_{k + \theta + v(c) + r \equiv 1 \pmod 2} \right) \\
  &= \mathbf{1}_{k + \theta + v(c) + r \equiv 0 \pmod 2}
  + v(c) - \theta - k - r.
\end{align*}
Set aside the indicator function
$\mathbf{1}_{k + \theta + v(c) + r \equiv 0 \pmod 2}$ momentarily;
we will merge it in a moment.
To consolidate the minimum's in the third double sum,
note that we have
\[ v(c) + r + 1 \le v(e) + \frac{\theta}{2} - v(b)
< v(e) + v(c) - \frac{\theta}{2} \le 2v(e) - \theta + v(c) - r - 1. \]
Hence, based on the value of $k$, we get the following coefficients:
\begin{itemize}
  \ii If $v(c) + r + 1 \le k \le v(e) + \frac{\theta}{2} - v(b)$, we get
  \begin{align*}
    \left( v(c) - \theta - k - r \right)
    &+ \left( k - \frac{\theta}{2} + v(b) \right)
    + \left( k + \frac{\theta}{2} - v(c) \right)  \\
    &= k - \theta + v(b) - r.
  \end{align*}

  \ii If $v(e) + \frac{\theta}{2} - v(b) \le k \le v(e) + v(c) - \frac{\theta}{2}$, we get
  \begin{align*}
    \left( v(c) - \theta - k - r \right)
    &+ v(e)
    + \left( k + \frac{\theta}{2} - v(c) \right)  \\
    &= v(e) - \frac{\theta}{2} - r.
  \end{align*}

  \ii If $v(e) + v(c) - \frac{\theta}{2} \le k \le 2v(e) - \theta + v(c) - r$, we get
  \begin{align*}
    \left( v(c) - \theta - k - r \right)
    &+ v(e) + v(e) \\
    &= 2v(e) + v(c) - \theta - r.
  \end{align*}
\end{itemize}
Noting the expression in the first bullet also matches the coefficient of $(-q^s)^k$
for $\theta - v(b) + r + 1 \le k \le v(c) + r$, we can now write
\begin{align*}
  I^{\text{6+}} + I^{\text{6-}}
  &=
  \sum_{k = - v(b) - r}^{v(c) + r - 1} (-q^s)^k \left(
    q^{\min\left( \frac{\theta}{2} + r, \left\lfloor \frac{v(b) + r + k}{2} \right\rfloor \right)}
    + \dots
    + q^{\max\left(0, \left\lceil \frac{k + \theta - v(c) + r + 1}{2} \right\rceil \right)}
    \right) \\
  &+ q^{\frac{\theta}{2} + r}
  \sum_{k = \theta - v(b) + r + 1}^{2v(e) - \theta + v(c) - r - 1} \cc_\guv(k)(-q^s)^k \\
  &+ q^{\frac{\theta}{2} + r}
  \sum_{k = v(c) + r + 1}^{2v(e) - \theta + v(c) - r - 1}
  \mathbf{1}_{k + \theta + v(c) + r \equiv 0 \pmod 2} (-q^s)^k
\end{align*}
as the overall contribution from \textbf{Case 6}, where
\[ \cc_\guv(k) \coloneqq \min \left( k - \theta + v(b) - r,
  v(e) - \frac{\theta}{2} - r,
  2v(e) + v(c) - \theta - r \right). \]

\section{Completed formula for the overall orbital integral for $S_2 \times V'_2(F)$}
We now put together all the previous calculations to give the following formula.
\begin{theorem}
  \label{thm:semi_lie_formula}
  Let the representative
  \[
    \guv = \left( \begin{bmatrix} a & b \\ c & d \end{bmatrix},
      \begin{bmatrix} 0 \\ 1 \end{bmatrix},
      \begin{bmatrix} 0 & e \end{bmatrix} \right)
    \in (S_2(F) \times V_2'(F))\rs.
  \]
  be paired with an element of $(\U(\VV_2) \times \VV_2)\rs$.
  If $v(e) < 0$ or $v(b) + v(c) < -2r$, the entire orbital integral is $0$.
  Otherwise define
  \[ \nn_\guv(k) \coloneqq \min\left( \left\lfloor \tfrac{k + (v(b)+r)}{2} \right\rfloor,
    \left\lfloor \tfrac{(2v(e)+v(c)+r)-k}{2} \right\rfloor, N \right) \]
  where
  \[ N \coloneqq \min \left(
      v(e), \left\lfloor \tfrac{v(b)+v(c)}{2} \right\rfloor + r,
      v(d-a) + r \right). \]
  Also, if $v(d-a) < v(e) - r$ and $v(b) + v(c) > 2v(d-a)$, then additionally define
  \begin{align*}
    \cc_\guv(k) &= \min\big( k - (2v(d-a)-v(b)+r), \\
      &\qquad (2v(e)+v(c)-2v(d-a)-r)-k, v(e)-v(d-a)-r \big).
  \end{align*}
  Otherwise define $\cc_\guv(k) = 0$.
  Then we have
  \begin{align*}
    &\phantom= \Orb(\guv, \mathbf{1}_{K'_{S, \le r}} \otimes \oneV, s) \\
    &= \sum_{k = -(v(b)+r)}^{2v(e)+v(c)+r} (-1)^k
    \left( 1 + q + q^2 + \dots + q^{\nn_\guv(k)} \right) (q^s)^k \\
    &+ \sum_{k = 2v(d-a)-v(b)+r}^{2v(e)+v(c)-2v(d-a)-r} (-1)^k \cc_\guv(k) q^{v(d-a) + r} (q^s)^k.
  \end{align*}
\end{theorem}

For reference, we provide \Cref{fig:semi_lie_sketch} sketching the shapes
of $\nn_\guv$ and $\cc_\guv$, which may be easier to think about.

\begin{figure}
  \centering
  \begin{asy}
    usepackage("amsmath");
    size(16cm);
    draw((0,0)--(6,3)--(19,3)--(25,0), lightred);
    draw((-1,0)--(26,0), grey);
    draw((-1.4,0)--(-1.8,0)--(-1.8,3)--(-1.4,3), brown);
    label("$N$", (-1.8,1.5), dir(180), brown);
    draw((26.4,0)--(26.8,0)--(26.8,3)--(26.4,3), brown);
    label("$N$", (26.8,1.5), dir(0), brown);

    label("$2N$", (3,3.8), dir(90), brown);
    label("$2N$", (22,3.8), dir(90), brown);
    draw((0,3.4)--(0,3.8)--(6,3.8)--(6,3.4), brown);
    draw((19,3.4)--(19,3.8)--(25,3.8)--(25,3.5), brown);

    draw((0,0)--(0,-3), dotted, Margins);
    draw((6,3)--(6,-1.5), dotted, Margins);
    draw((19,3)--(19,-1.5), dotted, Margins);
    draw((25,0)--(25,-3), dotted, Margins);

    for (int i=0; i<=25; ++i) {
      real y = min(floor(i/2), floor((25-i)/2), 3);
      dot((i, y), red);
    }

    label("$k=\boxed{-v(b)+r}$", (0,-3), dir(-90));
    label("$k=\boxed{-v(b)+r+2N}$", (6,-1.5), dir(-90));
    label("$k=\boxed{2v(e)+v(c)+r-2N}$", (19,-1.5), dir(-90));
    label("$k=\boxed{2v(e)+v(c)+r}$", (25,-3), dir(-90));
    label("$\mathbf{n}_{(\gamma, \mathbf{u}, \mathbf{v}^\top)}$", (12.5,3), dir(90), red);

    real h = 11; // offset downwards
    draw((6,-h)--(11,5-h)--(14,5-h)--(19,-h), lightblue);
    draw((5,-h)--(20,-h), grey);
    for (int i=6; i<=19; ++i) {
      real y = min(i-6, 19-i, 5);
      dot((i, y - h), blue);
    }
    label("$\mathbf{c}_{(\gamma, \mathbf{u}, \mathbf{v}^\top)}$, only if $N = v(d-a)+r$", (12.5,5-h), dir(90), blue);

    draw((6,-h)--(6,-3.5), dotted, Margins);
    draw((19,-h)--(19,-3.5), dotted, Margins);
    draw((11,5-h)--(11,-2.5-h), dotted, Margins);
    draw((14,5-h)--(14,-2.5-h), dotted, Margins);
    label("$k=\boxed{-v(b)+v(e)+v(d-a)+2r}$", (11,-2.5-h), dir(230));
    label("$k=\boxed{v(c)+v(e)-v(d-a)-2r}$", (14,-2.5-h), dir(310));

    draw((6,-h-0.4)--(6,-h-0.8)--(10.8,-h-0.8)--(10.8,-h-0.4), purple);
    draw((14.2,-h-0.4)--(14.2,-h-0.8)--(19,-h-0.8)--(19,-h-0.4), purple);
    label("$v(e)-v(d-a)-r$", (7, -h-0.8), dir(-90), purple);
    label("$v(e)-v(d-a)-r$", (18, -h-0.8), dir(-90), purple);
    draw((4.6,-h)--(4.2,-h)--(4.2,5-h)--(4.6,5-h), purple);
    draw((20.4,-h)--(20.8,-h)--(20.8,5-h)--(20.4,5-h), purple);
    label((4.2,2.5-h), "$v(e)-v(d-a)-r$", dir(180), purple);
    label((20.8,2.5-h), "$v(e)-v(d-a)-r$", dir(0), purple);
  \end{asy}
  \caption{Sketch of the functions in \Cref{thm:semi_lie_formula}.
    The boxed numbers indicate values of $k$.}
  \label{fig:semi_lie_sketch}
\end{figure}


\begin{proof}
  First, suppose $\theta = v(b) + v(c) < 2v(d-a)$ is odd.
  Then
  \[ \nn_\guv(k) \coloneqq \min\left( \left\lfloor \tfrac{k + (v(b)+r)}{2} \right\rfloor,
      \left\lfloor \tfrac{(2v(e)+v(c)+r)-k}{2} \right\rfloor,
      v(e), \left\lfloor \tfrac{v(b)+v(c)}{2} \right\rfloor + r \right) \]
  and $\cc_\guv$ terms do not appear.
  Now, in that case, the exponent \eqref{eq:case_5_exponent} can be simplified, because
  \begin{align*}
    \left\lfloor \frac{k+\theta-v(c)+r}{2} \right\rfloor
    &= \left\lfloor \frac{k+v(b)+r}{2} \right\rfloor \\
    v(e) + \left\lfloor \frac{v(c)+r-k}{2} \right\rfloor
    &= \left\lfloor \frac{(2v(e) + v(c) + r) - k}{2} \right\rfloor \\
    \left\lfloor \frac{k+\theta-v(c)+r}{2} \right\rfloor
    + \left\lfloor \frac{v(c)+r-k}{2} \right\rfloor
    &= \frac{k+\theta-v(c)+r}{2} + \frac{v(c)+r-k}{2} - \half \\
    &= \frac{\theta-1}{2} + r \\
    &= \left\lfloor \frac{v(b)+v(c)}{2} \right\rfloor + r < v(d-a) + r.
  \end{align*}
  Hence, $\nn_\guv$ coincides with the exponent in \eqref{eq:case_5_exponent}.
  So the result is true in this case.

  Now assume instead $\theta = 2v(d-a) < v(b) + v(c)$ is even.
  Notice that
  \[ \left\lfloor \frac{k+\theta-v(c)+r}{2} \right\rfloor
    + \left\lfloor \frac{v(c)+r-k}{2} \right\rfloor
    = \frac{\theta}{2} + r - \mathbf{1}_{k + v(c) + r \equiv 0 \pmod 2}. \]

  First assume that $v(e) \le \frac{\theta}{2} + r$ and
  consider \eqref{eq:case_5_exponent}.
  For the range of values of $k$ in $I^{\text{6+}}$, that is
  \[ -v(b) - r \le k \le 2v(e)-\theta+v(c)-r-1 \]
  we have the first term of \eqref{eq:case_5_exponent} is smallest, as
  \begin{align*}
    \left\lfloor \frac{k + \theta - v(c) + r}{2} \right\rfloor &< v(e) \\
    &\le v(e) + \left\lfloor \frac{v(b)+v(c)}{2} \right\rfloor
      \le v(e) + \left\lfloor \frac{v(c)+r-k}{2} \right\rfloor \\
    \left\lfloor \frac{k + \theta - v(c) + r}{2} \right\rfloor &\le v(e)-1
      \le \frac{\theta}{2} + r -1.
  \end{align*}
  So the contributions from \textbf{Case 5} and \textbf{Case 6}
  fit together to give
  \begin{align*}
    \left( q^{\left\lfloor \frac{k+\theta-v(c)+r}{2} \right\rfloor}
      + \dots + q^0 \right)
    &+
    \left(
    q^{\min\left( v(e), \left\lfloor \frac{v(b) + r + k}{2} \right\rfloor \right)}
    + \dots
    + q^{\max\left(0, \left\lceil \frac{k + \theta - v(c) + r + 1}{2} \right\rceil \right)}
    \right) \\
    &=
    q^{\min(v(e), \left\lfloor \frac{v(b) + r + k}{2} \right\rfloor)} + \dots + q^0
  \end{align*}
  which thus matches the formula for $\nn_\guv(k)$.

  Now suppose instead $v(e) > \frac{\theta}{2} + r$.
  First, a similar analysis gives that the first part of $I^{\text{6+}}$ fits
  together with $I^{\text{5}}$ again.
  Indeed if
  \[ -v(b) - r \le k \le v(c) + r - 1 \]
  then in \eqref{eq:case_5_exponent} we get the first exponent again, and hence
  we have
  \begin{align*}
    \left( q^{\left\lfloor \frac{k+\theta-v(c)+r}{2} \right\rfloor}
      + \dots + q^0 \right)
    &+
    \left(
    q^{\min\left( \frac{\theta}{2} + r, \left\lfloor \frac{v(b) + r + k}{2} \right\rfloor \right)}
    + \dots
    + q^{\max\left(0, \left\lceil \frac{k + \theta - v(c) + r + 1}{2} \right\rceil \right)}
    \right) \\
    &=
    q^{\min(\frac{\theta}{2}+r, \left\lfloor \frac{v(b) + r + k}{2} \right\rfloor)} + \dots + q^0
  \end{align*}
  which matches the claimed formula for $\nn_\guv$ in this range.

  The remaining contribution from \textbf{Case 6\ts{+} and Case 6\ts{-}} is
  \begin{align*}
  &+ q^{\frac{\theta}{2} + r}
  \sum_{k = \theta - v(b) + r + 1}^{2v(e) - \theta + v(c) - r - 1} \cc_\guv(k)(-q^s)^k \\
  &+ q^{\frac{\theta}{2} + r}
  \sum_{k = v(c) + r + 1}^{2v(e) - \theta + v(c) - r - 1}
  \mathbf{1}_{k + \theta + v(c) + r \equiv 0 \pmod 2} (-q^s)^k
  \end{align*}

  The first sum matches the claimed coefficient $\cc_\guv$
  (except the summation in the theorem statement includes
  endpoints at $k = \theta - v(b) + r$
  and $k = 2v(e) - \theta + v(c) - r$,
  but $\cc_\guv(k) = 0$ at these two endpoints,
  so there is no change).

  Meanwhile the second sum accounts for the discrepancy between
  the final term of \eqref{eq:case_5_exponent} and the formula for $\nn_\guv$.
  That is, the range of $k$ for which \eqref{eq:case_5_exponent}
  achieves the last minimum is exactly
  $v(c) + r + 1 \le k \le 2v(e) - \theta + v(c) - r - 1$;
  only in those cases does \eqref{eq:case_5_exponent}
  differs from $\nn_\guv$ by exactly
  $\mathbf{1}_{k + \theta + v(c) + r \equiv 0 \pmod 2}$.
  This final step shows the claimed formulas coincide.
\end{proof}

\begin{example}
  When $v(e) = 0$ the expression is particularly simple.
  The assumption $v(d-a) \ge v(e)-r$ is automatically true, and
  $\nn_{\guv}$ is identically zero, so
  \[ \Orb(\guv, \mathbf{1}_{K'_{S, \le r}} \otimes \oneV, s)
    = \sum_{k=-(v(b)+r)}^{v(c)+r} (-q^s)^k. \]
\end{example}

\begin{example}
  Suppose $r = 14$, $v(b) = -5$, $v(c) = 100$, $v(e) = 3$.
  We have $v(d-a) \ge 0 > -11 = v(e) - r$.
  Hence the above formula reads
  \begin{align*}
    \Orb(\guv, \mathbf{1}_{K'_{S, \le 14}} \otimes \oneV, s)
    &= -q^{-9s} \\
    &+ q^{-8s} \\
    &- (q+1) \cdot q^{-7s} \\
    &+ (q+1) \cdot q^{-6s} \\
    &- (q^2+q+1) \cdot q^{-5s} \\
    &+ (q^2+q+1) \cdot q^{-4s} \\
    &- (q^3+q^2+q+1) \cdot q^{-3s} \\
    &+ (q^3+q^2+q+1) \cdot q^{-2s} \\
    &- (q^3+q^2+q+1) \cdot q^{-s} \\
    &+ (q^3+q^2+q+1) \cdot q^{0} \\
    &- (q^3+q^2+q+1) \cdot q^{s} \\
    &+ (q^3+q^2+q+1) \cdot q^{2s} \\
    &\vdotswithin= \\
    &- (q^3+q^2+q+1) \cdot q^{113s} \\
    &+ (q^3+q^2+q+1) \cdot q^{114s} \\
    &- (q^2+q+1) \cdot q^{115s} \\
    &+ (q^2+q+1) \cdot q^{116s} \\
    &- (q+1) \cdot q^{117s} \\
    &+ (q+1) \cdot q^{118s} \\
    &- q^{119s} \\
    &+ q^{120s}.
  \end{align*}
\end{example}
\begin{example}
  Suppose $r = 2$, $v(b) = -5$, $v(c) = 100$, $v(e) = 20$, $v(d-a) = 1$.
  Then we have
  \begin{align*}
    \Orb(\guv, \mathbf{1}_{K'_{S, \le 2}} \otimes \oneV, s)
    &= -q^{3s} \\
    &+ q^{4s} \\
    &- (q+1) \cdot q^{5s} \\
    &+ (q+1) \cdot q^{6s} \\
    &- (q^2+q+1) \cdot q^{7s} \\
    &+ (q^2+q+1) \cdot q^{8s} \\
    &- (q^3+q^2+q+1) \cdot q^{9s} \\
    &+ (2q^3+q^2+q+1) \cdot q^{10s} \\
    &- (3q^3+q^2+q+1) \cdot q^{9s} \\
    &+ (4q^3+q^2+q+1) \cdot q^{10s} \\
    &- (5q^3+q^2+q+1) \cdot q^{9s} \\
    &+ (6q^3+q^2+q+1) \cdot q^{10s} \\
    &\vdotswithin= \\
    &- (17q^3+q^2+q+1) \cdot q^{25s} \\
    &+ (18q^3+q^2+q+1) \cdot q^{26s} \\
    &- (18q^3+q^2+q+1) \cdot q^{27s} \\
    &+ (18q^3+q^2+q+1) \cdot q^{28s} \\
    &\vdotswithin= \\
    &- (18q^3+q^2+q+1) \cdot q^{117s} \\
    &+ (18q^3+q^2+q+1) \cdot q^{118s} \\
    &- (18q^3+q^2+q+1) \cdot q^{119s} \\
    &+ (17q^3+q^2+q+1) \cdot q^{120s} \\
    &- (16q^3+q^2+q+1) \cdot q^{121s} \\
    &+ (15q^3+q^2+q+1) \cdot q^{122s} \\
    &\vdotswithin= \\
    &+ (3q^3+q^2+q+1) \cdot q^{134s} \\
    &- (2q^3+q^2+q+1) \cdot q^{135s} \\
    &+ (q^3+q^2+q+1) \cdot q^{136s} \\
    &- (q^2+q+1) \cdot q^{137s} \\
    &+ (q^2+q+1) \cdot q^{138s} \\
    &- (q+1) \cdot q^{139s} \\
    &+ (q+1) \cdot q^{140s} \\
    &- q^{141s} \\
    &+ q^{142s}.
  \end{align*}
\end{example}


\chapter{Large and small kernels}
\label{ch:ker}

In this chapter:
\begin{itemize}
  \ii For the orbital integral on $S_3(F)$,
  we use \todo{ref} to prove \Cref{thm:large_kernel_group_full} which implies \Cref{thm:large_kernel_group}.
  \ii For the orbital integral on $S_2(F) \times V'_2(F)$,
  we use \Cref{cor:semi_lie_combo} to prove
  both \Cref{thm:semi_lie_ker_trivial} and \Cref{thm:semi_lie_ker_huge}.
\end{itemize}

\section{In the group AFL, the kernel still has finite codimension for $n=3$}
The main result is the following:
\begin{theorem}
  \label{thm:large_kernel_group_full}
  For any $r \ge 3$ we always have
  \[ \partial \Orb\left(\gamma, \mathbf{1}_{K'_{S, r}} + 2q \mathbf{1}_{K'_{S, r-1}}
    + q^2 \mathbf{1}_{K'_{S, r-2}}\right) = (2q+2) \log q. \]
\end{theorem}

\subsection{Proof of \Cref{thm:large_kernel_group}}
\Cref{thm:large_kernel_group_full} already implies the codimension of the kernel is at most $3$.
In order to show it is exactly $3$,\todo{prove it}.

We now turn to showing that the kernel is not contained in any maximal ideal.
This requires us to invoke the explicit results from \Cref{ch:satake}.
Consider the composed isomorphisms
\[ \HH(S_3(F)) \xrightarrow{\BC_S} \HH(\U(\VV_3^+)) \xrightarrow{\Sat} \QQ[Y + Y^{-1}]. \]
Let $K =  \GL_n(\OO_E) \cap \U(\VV_3^+)$ and note for any $r \ge 2$ we have
\begin{align*}
  \mathbf{1}_{K\varpi^{(r,0,-r)}K}
  &\xmapsto{\BC_S^{-1}}
  \mathbf{1}_{K'_{S, r}} + 2q \mathbf{1}_{K'_{S, r-1}} + 2q^2 \mathbf{1}_{K'_{S, r-2}} + 2q^3 \mathbf{1}_{K'_{S, r-3}} + \dots \\
  \implies
  \mathbf{1}_{K\varpi^{(r,0,-r)}K} - q^2 \mathbf{1}_{K\varpi^{(r-2,0,-(r-2))}K}
  &\xmapsto{\BC_S^{-1}}
  \mathbf{1}_{K'_{S, r}} + 2q \mathbf{1}_{K'_{S, r-1}} + q^2 \mathbf{1}_{K'_{S, r-2}}.
\end{align*}
Hence for $r \ge 3$ if we define
\begin{align*}
  P_r(Y) &\coloneqq \Sat\left(\BC_S(\mathbf{1}_{K'_{S, r}}
    + 2q \mathbf{1}_{K'_{S, r-1}} + q^2 \mathbf{1}_{K'_{S, r-2}})\right) \\
  &= \Sat\left( \mathbf{1}_{K\varpi^{(r,0,-r)}K} - q^2 \mathbf{1}_{K\varpi^{(r-2,0,-(r-2))}K} \right) \\
  &= \Sat\Big(
    \mathbf{1}_{\varpi^{-r} \Mat_3(\OO_E) \cap \U(\VV_3^+)}
    - \mathbf{1}_{\varpi^{-(r-1)} \Mat_3(\OO_E) \cap \U(\VV_3^+)} \\
    &\qquad- q^2 \mathbf{1}_{\varpi^{-(r-2)} \Mat_3(\OO_E) \cap \U(\VV_3^+)}
    + q^2 \mathbf{1}_{\varpi^{-(r-3)} \Mat_3(\OO_E) \cap \U(\VV_3^+)}
    \Big) \\
  &= \left( q^{2r} \sum_{j=-r}^r Y^j + q^{2r-1} \sum_{j=-(r-1)}^{r-1} Y^j \right)
  - \left( q^{2r-2} \sum_{j=-(r-1)}^{r-1} Y^j + q^{2r-3} \sum_{j=-(r-2)}^{r-2} Y^j \right) \\
  &\qquad- \left( q^{2r-2} \sum_{j=-(r-2)}^{r-2} Y^j + q^{2r-3} \sum_{j=-(r-3)}^{r-3} Y^j \right)
  + \left( q^{2r-4} \sum_{j=-(r-3)}^{r-3} Y^j + q^{2r-5} \sum_{j=-(r-4)}^{r-4} Y^j \right)
\end{align*}
then it follows that $P_r(Y) - P_3(Y)$ is contained in the kernel for any $r \ge 3$.

To show this kernel generates the entire ring, it would be sufficient to prove
there is no $Y \in \CC^\times$ such that $P_3(Y) = P_4(Y) = P_5(Y) = P_6(Y) = \dotsb$.
However, using the explicit formula for $P_r(Y)$ above,
a direct calculation gives the following two identities:
\begin{align*}
  P_5(Y) - \frac{q^2}{Y} P_4(Y) - q^2 Y P_4(Y) - \frac{q^2}{Y} P_3(Y) &= -q^5 \\
  P_6(Y) - \frac{q^2}{Y} P_5(Y) - q^2 Y P_5(Y) - \frac{q^2}{Y} P_4(Y) &= 0.
\end{align*}
So such a common root $Y$ cannot exist.

\section{In the semi-Lie case, the kernel is trivial if we allow $v(e)$ to vary, for every fixed choice of $\gamma$}
We prove \Cref{thm:semi_lie_ker_trivial} in this section.
We treat $\gamma$ as fixed, and let
\[ \theta \coloneqq \min\left( v(b)+v(c), 2v(d-a) \right) \ge 0 \]
as we did in \Cref{ch:orbitalFJ1}.

\begin{lemma}
  \label{lem:semi_lie_ker_full_rank}
  Fix $\gamma$. Let $N \ge 0$ be a nonnegative integer.
  We define an $(N+ \left\lfloor \frac{\theta}{2} \right\rfloor + 2) \times (N+1)$ matrix $M$ as follows:
  for $0 \le i \le N+ \left\lfloor \frac{\theta}{2} \right\rfloor + 1$ and $0 \le r \le N$,
  the $i$\ts{th} row and $r$\ts{th} column takes the value
  \[
    M_{i,r} \coloneqq \frac{(-1)^r}{\log q} \partial \Orb\left(
      \left(\gamma, \begin{pmatrix} 0 \\ 1 \end{pmatrix},
      \begin{pmatrix} 0 & \varpi^i \end{pmatrix} \right),
      \varphi\right).
  \]
  Then $M$ has full rank.
\end{lemma}
The basic strategy of the proof will be to perform some sequences of row operations.
Specifically, we introduce the following definition.
\begin{itemize}
  \ii For each $i = N + \left\lfloor \frac{\theta}{2} \right\rfloor,\dots,0$ in that order,
  subtract the $i$\ts{th} row of $M$
  from the $(i+1)$\ts{th} row of $M$.
  Denote the new matrix as $M'$.

  \ii For each $i = N + \left\lfloor \frac{\theta}{2} \right\rfloor - 1,\dots,0$ in that order,
  subtract the $i$\ts{th} row of $M'$
  from the $(i+2)$\ts{nd} row of $M'$.
  Denote the new matrix as $M''$.
\end{itemize}
Then the basic premise is to show that $M''$ has an upper triangular submatrix.
This is easier to see with some illustrations, which we give below.

\begin{example}
  For example, for $N = 4$ and $v(b)+v(c)=1$, $v(d-a) = 0$ (hence $\theta = 0$), we have
  \[
    M = \begin{pmatrix}
    1 & 2 & 3 & 4 & 5 \\
    1 & q + 3 & 2q + 4 & 3q + 5 & 4q + 6 \\
    2 & q + 4 & q^{2} + 3q + 5 & 2q^{2} + 4q + 6 & 3q^{2} + 5q + 7 \\
    2 & 2q + 5 & q^{2} + 4q + 6 & q^{3} + 3q^{2} + 5q + 7 & 2q^{3} + 4q^{2} + 6q + 8 \\
    3 & 2q + 6 & 2q^{2} + 5q + 7 & q^{3} + 4q^{2} + 6q + 8 & q^{4} + 3q^{3} + 5q^{2} + 7q + 9 \\
    3 & 3q + 7 & 2q^{2} + 6q + 8 & 2q^{3} + 5q^{2} + 7q + 9 & q^{4} + 4q^{3} + 6q^{2} + 8q + 10
    \end{pmatrix}
  \]
  hence
  \[ M'= \begin{pmatrix}
    1 & 2 & 3 & 4 & 5 \\
    0 & q + 1 & 2q + 1 & 3q + 1 & 4q + 1 \\
    1 & 1 & q^{2} + q + 1 & 2q^{2} + q + 1 & 3q^{2} + q + 1 \\
    0 & q + 1 & q + 1 & q^{3} + q^{2} + q + 1 & 2q^{3} + q^{2} + q + 1 \\
    1 & 1 & q^{2} + q + 1 & q^{2} + q + 1 & q^{4} + q^{3} + q^{2} + q + 1 \\
    0 & q + 1 & q + 1 & q^{3} + q^{2} + q + 1 & q^{3} + q^{2} + q + 1
    \end{pmatrix}
  \]
  and finally
  \[ M'' = \begin{pmatrix}
    1 & 2 & 3 & 4 & 5 \\
    0 & q + 1 & 2q + 1 & 3q + 1 & 4q + 1 \\
    0 & -1 & q^{2} + q - 2 & 2q^{2} + q - 3 & 3q^{2} + q - 4 \\
    0 & 0 & -q & q^{3} + q^{2} - 2q & 2q^{3} + q^{2} - 3q \\
    0 & 0 & 0 & -q^{2} & q^{4} + q^{3} - 2q^{2} \\
    0 & 0 & 0 & 0 & -q^{3}
    \end{pmatrix}. \]
\end{example}

\begin{example}
  For example, for $N = 4$, $v(b)+v(c)=17$, $v(d-a) = 2$ (hence $\theta=4$), we have
  \[
    M = \begin{pmatrix}
    9 & 10 & 11 & \dots \\
    8q + 10 & 9q + 11 & 10q + 12 & \dots \\
    7q^{2} + 9q + 11 & 8q^{2} + 10q + 12 & 9q^{2} + 11q + 13 & \dots \\
    q^{2} + 10q + 12 & 7q^{3} + 9q^{2} + 11q + 13 & 8q^{3} + 10q^{2} + 12q + 14 & \dots \\
    8q^{2} + 11q + 13 & q^{3} + 10q^{2} + 12q + 14 & 7q^{4} + 9q^{3} + 11q^{2} + 13q + 15 & \dots \\
    2q^{2} + 12q + 14 & 8q^{3} + 11q^{2} + 13q + 15 & q^{4} + 10q^{3} + 12q^{2} + 14q + 16 & \dots \\
    9q^{2} + 13q + 15 & 2q^{3} + 12q^{2} + 14q + 16 & 8q^{4} + 11q^{3} + 13q^{2} + 15q + 17 & \dots \\
    3q^{2} + 14q + 16 & 9q^{3} + 13q^{2} + 15q + 17 & 2q^{4} + 12q^{3} + 14q^{2} + 16q + 18 & \dots
    \end{pmatrix}
  \]
  hence
  \[ M' = \begin{pmatrix}
    9 & 10 & 11 & 12 & \dots \\
    8q + 1 & 9q + 1 & 10q + 1 & 11q + 1 & \dots \\
    7q^{2} + q + 1 & 8q^{2} + q + 1 & 9q^{2} + q + 1 & 10q^{2} + q + 1 & \dots \\
    -6q^{2} + q + 1 & 7q^{3} + q^{2} + q + 1 & 8q^{3} + q^{2} + q + 1 & 9q^{3} + q^{2} + q + 1 & \dots \\
    7q^{2} + q + 1 & -6q^{3} + q^{2} + q + 1 & 7q^{4} + q^{3} + q^{2} + q + 1 & 8q^{4} + \dots + 1 & \dots \\
    -6q^{2} + q + 1 & 7q^{3} + q^{2} + q + 1 & -6q^{4} + q^{3} + q^{2} + q + 1 & 7q^{5} + \dots + 1 & \dots \\
    7q^{2} + q + 1 & -6q^{3} + q^{2} + q + 1 & 7q^{4} + q^{3} + q^{2} + q + 1 & -6q^{5} + \dots + 1 & \dots \\
    -6q^{2} + q + 1 & 7q^{3} + q^{2} + q + 1 & -6q^{4} + q^{3} + q^{2} + q + 1 & 7q^{5} + \dots + 1 & \dots
    \end{pmatrix}
  \]
  and finally
  \[
    M'' = \begin{pmatrix}
    9 & 10 & 11 & 12 & 13 \\
    8q + 1 & 9q + 1 & 10q + 1 & 11q + 1 & 12q + 1 \\
    7q^{2} + q - 8 & 8q^{2} + q - 9 & 9q^{2} + q - 10 & 10q^{2} + q - 11 & 11q^{2} + q - 12 \\
    -6q^{2} - 7q & 7q^{3} + q^{2} - 8q & 8q^{3} + q^{2} - 9q & 9q^{3} + q^{2} - 10q & 10q^{3} + q^{2} - 11q \\
    0 & -6q^{3} - 7q^{2} & 7q^{4} + q^{3} - 8q^{2} & 8q^{4} + q^{3} - 9q^{2} & 9q^{4} + q^{3} - 10q^{2} \\
    0 & 0 & -6q^{4} - 7q^{3} & 7q^{5} + q^{4} - 8q^{3} & 8q^{5} + q^{4} - 9q^{3} \\
    0 & 0 & 0 & -6q^{5} - 7q^{4} & 7q^{6} + q^{5} - 8q^{4} \\
    0 & 0 & 0 & 0 & -6q^{6} - 7q^{5}
    \end{pmatrix}.
  \]
\end{example}
\begin{example}
  For example, for $N = 4$, $v(b)+v(c)=5$, $v(d-a) = 8$ (hence $\theta=5$), we have
  \[ M = \begin{pmatrix}
    3 & 4 & 5 & \dots \\
    2q + 4 & 3q + 5 & 4q + 6 & \dots \\
    q^{2} + 3q + 5 & 2q^{2} + 4q + 6 & 3q^{2} + 5q + 7 & \dots \\
    2q^{2} + 4q + 6 & q^{3} + 3q^{2} + 5q + 7 & 2q^{3} + 4q^{2} + 6q + 8 & \dots \\
    3q^{2} + 5q + 7 & 2q^{3} + 4q^{2} + 6q + 8 & q^{4} + 3q^{3} + 5q^{2} + 7q + 9 & \dots \\
    4q^{2} + 6q + 8 & 3q^{3} + 5q^{2} + 7q + 9 & 2q^{4} + 4q^{3} + 6q^{2} + 8q + 10 & \dots \\
    5q^{2} + 7q + 9 & 4q^{3} + 6q^{2} + 8q + 10 & 3q^{4} + 5q^{3} + 7q^{2} + 9q + 11 & \dots \\
    6q^{2} + 8q + 10 & 5q^{3} + 7q^{2} + 9q + 11 & 4q^{4} + 6q^{3} + 8q^{2} + 10q + 12 & \dots \\
    \end{pmatrix} \]
  hence
  \[ M' = \begin{pmatrix}
    3 & 4 & 5 & 6 & 7 \\
    2q + 1 & 3q + 1 & 4q + 1 & 5q + 1 & 6q + 1 \\
    q^{2} + q + 1 & 2q^{2} + q + 1 & 3q^{2} + q + 1 & 4q^{2} + q + 1 & 5q^{2} + q + 1 \\
    q^{2} + q + 1 & q^{3} + q^{2} + q + 1 & 2q^{3} + q^{2} + q + 1 & 3q^{3} + q^{2} + q + 1 & 4q^{3} + q^{2} + q + 1 \\
    q^{2} + q + 1 & q^{3} + q^{2} + q + 1 & q^{4} + \dots + 1 & 2q^{4} + \dots + 1 & 3q^{4} + \dots + 1 \\
    q^{2} + q + 1 & q^{3} + q^{2} + q + 1 & q^{4} + \dots + 1 & q^{5} + \dots + 1 & 2q^{5} + \dots + 1 \\
    q^{2} + q + 1 & q^{3} + q^{2} + q + 1 & q^{4} + \dots + 1 & q^{5} + \dots + 1 & q^{6} + \dots + 1 \\
    q^{2} + q + 1 & q^{3} + q^{2} + q + 1 & q^{4} + \dots + 1 & q^{5} + \dots + 1 & q^{6} + \dots + 1
    \end{pmatrix} \]
  and finally
  \[ M'' = \begin{pmatrix}
    3 & 4 & 5 & 6 & 7 \\
    2q + 1 & 3q + 1 & 4q + 1 & 5q + 1 & 6q + 1 \\
    q^{2} + q - 2 & 2q^{2} + q - 3 & 3q^{2} + q - 4 & 4q^{2} + q - 5 & 5q^{2} + q - 6 \\
    q^{2} - q & q^{3} + q^{2} - 2q & 2q^{3} + q^{2} - 3q & 3q^{3} + q^{2} - 4q & 4q^{3} + q^{2} - 5q \\
    0 & q^{3} - q^{2} & q^{4} + q^{3} - 2q^{2} & 2q^{4} + q^{3} - 3q^{2} & 3q^{4} + q^{3} - 4q^{2} \\
    0 & 0 & q^{4} - q^{3} & q^{5} + q^{4} - 2q^{3} & 2q^{5} + q^{4} - 3q^{3} \\
    0 & 0 & 0 & q^{5} - q^{4} & q^{6} + q^{5} - 2q^{4} \\
    0 & 0 & 0 & 0 & q^{6} - q^{5}
    \end{pmatrix}. \]
\end{example}

\begin{proof}
  In order to prove $M$ has full rank, it suffices to prove $M''$ has full rank.
  We now confirm the patterns shown by the example above.

  By quoting \Cref{cor:semi_lie_derivative_single} we will write
  \begin{align*}
    M_{i,r}
    &= \sum_{j=0}^{\min(i,r + \left\lfloor \frac{\theta}{2} \right\rfloor)}
      \left( i + \frac{v(b)+v(c)+1}{2} + r - 2j \right) q^j \\
    &- \mathbf{1}_{\substack{\theta \equiv 0 \bmod 2 \\ i \ge r + \theta/2}}
      \cdot q^{v(d-a)+r} \cdot \left( \frac{i-r}{2} + t_i \right)
  \end{align*}
  where
  \[
    t_i = \begin{cases}
      -\frac{v(d-a)}{2} &\text{if } i + r \equiv v(d-a) \pmod 2 \\
      \frac{v(b)+v(c)-3v(d-a)}{2} &\text{if } i + r \not\equiv v(d-a) \pmod 2 \\
    \end{cases}
  \]
  depends only on the parity of $i$.
  Hence for $i \ge 1$ we always have
  \begin{align*}
    M'_{i,r}
    &= M_{i,r} - M_{i-1,r} \\
    &=- \mathbf{1}_{\substack{\theta \equiv 0 \bmod 2 \\ i \ge r + \theta/2}}
      \cdot q^{v(d-a)+r} \cdot \left( \frac{i-r}{2} + t_i \right) \\
    &+ \mathbf{1}_{\substack{\theta \equiv 0 \bmod 2 \\ i-1 \ge r + \theta/2}}
      \cdot q^{v(d-a)+r} \cdot \left( \frac{i-1-r}{2} + t_{i-1} \right) \\
    &+ \mathbf{1}_{i \le r + \left\lfloor \frac{\theta}{2} \right\rfloor} \cdot
      \left( \frac{v(b)+v(c)+1}{2} + r - i \right) q^i \\
    &+ \sum_{j=0}^{\min(i-1,r + \left\lfloor \frac{\theta}{2} \right\rfloor)} q^j.
  \end{align*}
  From this we can make the following deductions
  on \[ M''_{i,r} = M'_{i,r} - M'_{i-2,r} \]
  by cancelling most of the terms.
  \begin{itemize}
    \ii If $i \ge r + \left\lfloor \frac{\theta}{2} \right\rfloor + 3$
    then $M''_{i,r} = 0$ is clear.
    \ii If $i = r + \left\lfloor \frac{\theta}{2} \right\rfloor + 2$
    we contend that $M''_{i,r} = 0$ too.
    \begin{itemize}
      \ii When $\theta = v(b) + v(c)$ is odd, the surviving terms are
      \[
        - \left( \frac{v(b)+v(c)+1}{2} + r - (i-2) \right) q^{ i-2}
        + q^{r + \left\lfloor \frac{\theta}{2} \right\rfloor}
      \]
      Substituting in $i = r + \frac{v(b)+v(c)-1}{2} + 2$ gives zero, as needed.

      \ii When $\theta = 2v(d-a)$ is even, the surviving terms are
      \begin{align*}
        &-q^{v(d-a)+r} \cdot \left( \frac{i-r}{2} + t_{i} \right)
          + q^{v(d-a)+r} \cdot \left( \frac{(i-2)-r}{2} + t_{i-2} \right) \\
        &\qquad+ q^{v(d-a)+r} \cdot \left( \frac{i-1-r}{2} + t_{i-1} \right) \\
        &\qquad- \left( \frac{v(b)+v(c)+1}{2} + r - (i-2) \right) q^{i-2} \\
        &\qquad+ q^{r + \left\lfloor \frac{\theta}{2} \right\rfloor} \\
        &= q^{v(d-a)+r} \cdot \left( -1 + \frac{i-1-r}{2} + t_{i-1} - \frac{v(b)+v(c)+1}{2} - r  + i - 2 +1 \right) \\
        &= q^{v(d-a)+r} \cdot \left( -\frac{3}{2}r + t_{i-1} - \frac{v(b)+v(c)}{2} + \frac32 i \right)
      \end{align*}
      Substituting in $i = r + v(d-a) + 2$ (and since $t_i = t_{i-2}$),
      we also get exactly $0$,
      since $t_{i-1} = \frac{v(b)+v(c)-3v(d-a)}{2}$.
    \end{itemize}

    \ii If $i = r + \left\lfloor \frac{\theta}{2} \right\rfloor + 1$,
    we consider again cases on the parity of $\theta$.
    \begin{itemize}
      \ii When $\theta = v(b)+v(c)$ is odd, the surviving terms are
      \[
        -\left( \frac{v(b)+v(c)+1}{2} + r - (i-2) \right) q^{i-2} \\
        + q^{r+\left\lfloor \frac{\theta}{2} \right\rfloor} + q^{r+\left\lfloor \frac{\theta}{2} \right\rfloor-1} \\
        = q^{r+\left\lfloor \frac{\theta}{2} \right\rfloor} - q^{r+\left\lfloor \frac{\theta}{2} \right\rfloor - 1}.
      \]

      \ii When $\theta = 2v(d-a)$ is even, the surviving terms are
      \begin{align*}
        &-q^{v(d-a)+r} \cdot \left( \frac{i-r}{2} + t_{i} \right) \\
        &\qquad+ q^{v(d-a)+r} \cdot \left( \frac{i-1-r}{2} + t_{i-1} \right) \\
        &\qquad- \left( \frac{v(b)+v(c)+1}{2} + r - (i-2) \right) q^{i-2} \\
        &\qquad+ q^{r + \left\lfloor \frac{\theta}{2} \right\rfloor}
        + q^{r + \left\lfloor \frac{\theta}{2} \right\rfloor - 1} \\
        &= \left( t_{i-1}-t_i+\half  \right) q^{r + \left\lfloor \frac{\theta}{2} \right\rfloor} \\
        &\qquad- \left( \frac{v(b)+v(c)+1}{2} + r - (i-2) - 1 \right) q^{r + \left\lfloor \frac{\theta}{2} \right\rfloor - 1} \\
        &= \left( -\frac{v(d-a)}{2} - \frac{v(b)+v(c)-3v(d-a)}{2} + \frac12 \right) q^{r + \left\lfloor \frac{\theta}{2} \right\rfloor} \\
        &\qquad- \left( \frac{v(b)+v(c)+1}{2} + r - ((r+v(d-a)+1)-2) - 1 \right) q^{r + \left\lfloor \frac{\theta}{2} \right\rfloor - 1} \\
        &= - \frac{v(b)+v(c)-1-2v(d-a)}{2} q^{r + \left\lfloor \frac{\theta}{2} \right\rfloor} \\
        &\qquad- \frac{v(b)+v(c)+1-2v(d-a)}{2} q^{r + \left\lfloor \frac{\theta}{2} \right\rfloor - 1}.
      \end{align*}
      In the edge case where $r = 0$ and $\theta \le 1$,
      it can be checked the same formula still holds with the last term omitted.
    \end{itemize}
  \end{itemize}
  Hence we have found a diagonal of $M''$ below with all entries are zero,
  and on which all entries are nonzero except possibly $M''_{1,0} = 0$
  in the case where $r = 0$ and $\theta \le 1$.

  Assuming $r + \left\lfloor \frac{\theta}{2} \right\rfloor \ge 2$,
  suppose we take the rows from $i = r + \left\lfloor \frac{\theta}{2} \right\rfloor + 1$
  up to $i = r + \left\lfloor \frac{\theta}{2} \right\rfloor + N + 1$.
  Then the resulting matrix is upper triangular.
  The determinant is the product of the diagonal entries;
  up to multiplication by sign and a power of $q$, it equals
  and the determinant is equal to
  \[
    \begin{cases}
    (q-1)^{N+1} & \text{if } \theta \equiv 1 \pmod 2 \\
    \left( \frac{v(b)+v(c)-1-2v(d-a)}{2} q + \frac{v(b)+v(c)+1-2v(d-a)}{2} \right)^{N+1} & \text{if } \theta \equiv 0 \pmod 2
    \end{cases}
  \]
  which is manifestly nonzero for any odd prime power $q$.

  In the situation where $r + \left\lfloor \frac{\theta}{2} \right\rfloor = 1$,
  we use the same rows except that we replace the row for $i=1$
  with the row for $i=0$, which has leftmost entry $M_{0,0} = \frac{v(b)+v(c)+1}{2} > 0$.
  Hence the same proof still shows that the determinant is nonzero.
\end{proof}

\subsection{Proof of \Cref{thm:semi_lie_ker_trivial}}
Suppose we are given some function
\[ \phi = \sum_{r=0}^N (-1)^r c_r \mathbf{1}_{K'_{S, \le r}} \in \HH(S_2(F)). \]
Letting $M$ be the matrix given in \Cref{lem:semi_lie_ker_full_rank},
we are supposed to have
\[ M \begin{pmatrix} c_0 \\ c_1 \\ \vdots \\ c_N \end{pmatrix} = \mathbf{0}. \]
Since $M$ has full rank, it follows that $c_0 = \dots = c_N = 0$.

\section{In the semi-Lie case, the kernel has finite codimension for fixed $v(e)$}
We start with the following lemma.
\begin{lemma}
  \label{lem:semi_lie_large_r}
  For $r \ge v(e) + 2$, we have
  \[
    \partial \Orb \left( \guv,
        \mathbf{1}_{K'_{S, \le r}} + 2 \mathbf{1}_{K'_{S, \le (r-1)}} + \mathbf{1}_{K'_{S, \le (r-2)}}
      \right) = 0.
  \]
\end{lemma}
\begin{proof}
  This follows directly from \Cref{cor:semi_lie_combo} which gives
  \begin{align*}
    \sum_{j=0}^{v(e)} q^j
    &= \frac{(-1)^r}{\log q} \partial
    \Orb \left( \guv,
      \left(\mathbf{1}_{K'_{S, \le r}} + \mathbf{1}_{K'_{S, \le (r-1)}}\right)
      \right) \\
    &= \frac{(-1)^{r-1}}{\log q} \partial
    \Orb \left( \guv,
      \left(\mathbf{1}_{K'_{S, \le (r-1)}} + \mathbf{1}_{K'_{S, \le (r-2)}}\right)
      \right). \qedhere
  \end{align*}
\end{proof}

\subsection{Proof of \Cref{thm:semi_lie_ker_huge}}
The first part of \Cref{thm:semi_lie_ker_huge} now follows directly from
\Cref{lem:semi_lie_large_r}.

It remains to show the kernel is not contained in any maximal ideal.
Again, treat $v(e)$ as fixed.
Consider the composed isomorphism from \Cref{ch:satake}
given by
\[ \HH(S_2(F)) \xrightarrow{\BC^\eta_S} \HH(\U(\VV_2^+)) \xrightarrow{\Sat} \QQ[Y + Y^{-1}]. \]
By combining \cite[Equation (7.1.9)]{ref:AFLspherical}
(which is also \Cref{lem:finale_base_change} later)
and \cite[Equation (7.1.4)]{ref:AFLspherical}
we find that
\begin{align*}
  \Sat\left(\BC^\eta_S\left(\mathbf{1}_{K', \le r} + \mathbf{1}_{K', \le (r-1)}\right)\right)
  &= (-1)^r \Sat(\mathbf{1}_{\varpi^{-r} \Mat_2(\OO_E) \cap \VV_n^+}) \\
  &= (-1)^r \left( q^r \sum_{j=-r}^r Y^j - q^{r-1} \sum_{j=-(r-1)}^{r-1} Y^j \right).
\end{align*}
Hence, if we define the polynomial
\begin{align*}
  P_r(Y) &\coloneqq (-1)^r \Sat\left(\BC^\eta_S\left(
    \mathbf{1}_{K', \le r} + 2\mathbf{1}_{K', \le (r-1)}
    + \mathbf{1}_{K', \le (r-2)} \right)\right) \\
  &= \left( q^r \sum_{j=-r}^r Y^j - q^{r-1} \sum_{j=-(r-1)}^{r-1} Y^j \right)
    - \left( q^{r-1} \sum_{j=-(r-1)}^{r-1} Y^j - q^{r-2} \sum_{j=-(r-2)}^{r-2} Y^j \right) \\
  &= q^r \sum_{j=-r}^r Y^j - 2q^{r-1} \sum_{j=-(r-1)}^{r-1} Y^j + q^{r-2} \sum_{j=-(r-2)}^{r-2} Y^j
\end{align*}
then for any $r \ge v(e)+2$,
all the polynomials $P_r(Y) - P_{v(e)+2}(Y)$ lie in the kernel.

We now prove there is no choice of a single $Y \in \CC^\times$ for which
$P_r(Y)$ is eventually constant, which would complete the proof.
Indeed, if we write
\[
  P_r(Y) - q P_{r-1}(Y)
  = q^r(Y^r + Y^{-r}) - 2q^{r-1} (Y^{r-1} + Y^{-(r-1)}) + q^{r-2}(Y^{r-2} + Y^{-(r-2)})
\]
then
\begin{align*}
  (P_r(Y) - q P_{r-1}(Y)) - (P_{r-1}(Y) - q P_{r-2}(Y))
  &= q^3(Y^r + Y^{-r}) - 3q^2(Y^{r-1} + Y^{-(r-1)}) \\
  &\qquad + 3q(Y^{r-2} + Y^{-(r-2)}) - (Y^{r-3} + Y^{-(r-3)}).
\end{align*}
So we are done after we prove the following standalone lemma.
\begin{lemma}
  There is no $Y \in \CC^\times$ such that
  \[ q^3(Y^r + Y^{-r}) - 3q^2(Y^{r-1} + Y^{-(r-1)})
    + 3q(Y^{r-2} + Y^{-(r-2)}) - (Y^{r-3} + Y^{-(r-3)}) = 0 \]
  holds for all sufficiently large integers $r$.
\end{lemma}
\begin{proof}
  Assume for contradiction such a $Y \in \CC^\times$ existed.
  Let $Y^{\pm k} \coloneqq Y^k + Y^{-k}$ for brevity for every integer $k \ge 1$.
  By writing the recursion relations
  \begin{align*}
    Y^{\pm (r-1)} &= Y^{\pm 1} \cdot Y^{\pm (r-2)} - Y^{\pm (r-3)} \\
    Y^{\pm r} &= Y^{\pm 1} \cdot Y^{\pm (r-1)} - Y^{\pm (r-2)} \\
    &= ((Y^{\pm 1})^2 - 1) \cdot Y^{\pm (r-2)} - Y^{\pm 1} \cdot Y^{\pm (r-3)}
  \end{align*}
  we can deduce that
  \begin{align*}
    0 &= q^3 \cdot Y^{\pm r} - 3q^2 \cdot Y^{\pm (r-1)} + 3q \cdot Y^{\pm (r-2)} - Y^{\pm (r-3)} \\
    &= \left[ q^3 \cdot \left( (Y^{\pm 1})^2 - 1 \right)
      - 3q^2 \cdot Y^{\pm 1} + 3q \right] \cdot Y^{\pm (r-2)}
    - \left[ q^3 \cdot Y^{\pm 1} - 3q^2 + 1 \right] \cdot Y^{\pm (r-3)}.
  \end{align*}

  Now, in general there is no complex number $Y \in \CC^\times$ such that
  $Y^r + Y^{-r} = 0$ for two consecutive values of $r$.
  Hence, if either bracketed coefficient is zero, then so must be the other one.
  However, in that case, we would conclude that
  $Y^{\pm 1} = \frac{3q^2-1}{q^3}$ from the second bracketed coefficient, meaning
  \[ 0 = q^3 \cdot \left( \left( \frac{3q^2-1}{q^3} \right)^2 - 1 \right)
    - 3q^2 \cdot \frac{3q^2-1}{q^3} + 3q = \frac{(q^2-1)^3}{q^3} \]
  which is a contradiction, because $q > 1$.

  Hence neither bracketed coefficient can be zero,
  from which we conclude that there is some nonzero constant $c$ such that
  \[ Y^{\pm (r-2)} = c \cdot Y^{\pm(r-3)} \neq 0 \]
  holds for all large $r$.
  But then
  \[ c \cdot Y^{\pm(r-3)}
    = Y^{\pm 1} \cdot Y^{\pm (r-3)} - Y^{\pm (r-4)}
    = Y^{\pm 1} \cdot Y^{\pm (r-3)} - \frac{Y^{\pm (r-3)}}{c} \]
  and hence $c = Y^{\pm 1} - \frac 1c$.
  So either $c = Y$ or $c = \frac 1Y$.
  Then from $Y^{\pm (r-2)} = c \cdot Y^{\pm(r-3)}$ we derive that $Y = \pm 1$.

  But substituting $Y = 1$ in the original equation would imply $(q-1)^3 = 0$
  while $Y = -1$ would imply $(q+1)^3 = 0$ neither of which is possible.
  This contradiction completes the proof of the lemma.
\end{proof}

\section{A sequence of test functions almost lying in the kernel in the semi-Lie case}
We make one additional remark on the kernel that unifies both of the preceding two sections.
For this section, we define the following indicator function for $r \ge 3$:
\[ \phi_r \coloneqq \mathbf{1}_{K', \le r} + \mathbf{1}_{K', \le (r-1)}
   - q^2 (\mathbf{1}_{K', \le (r-2)} + \mathbf{1}_{K', \le (r-3)}). \]
We give the following theorem which can be thought of as a simultaneously
refined version of both \Cref{lem:semi_lie_ker_full_rank} and
\Cref{lem:semi_lie_large_r}.
Roughly, it says that we can define a sequence of test functions
\[ \phi_r + (q+1)\phi_{r-1} + q\phi_{r-2} \qquad r \ge 5 \]
such that for any fixed $\guv$, there are at most three values of $r$
for which the orbital integral does not vanish.
\begin{theorem}
  \label{thm:semi_lie_finite_codim_full}
  Suppose $\guv \in (S_2(F) \times V_2'(F))\rs$
  pairs with an element of $(\U(\VV_2^-) \times \VV_2^-)\rs$.
  Then
  \[ \partial \Orb \left( \guv, \phi_r + (q+1)\phi_{r-1} + q\phi_{r-2} \right) = 0 \]
  holds for all $r \ge 5$ with at most three exceptions,
  namely those $r$ with
  \[ v(e) - \min\left(\frac{v(b)+v(c)-1}{2}, v(d-a)\right) + 2
    \le r \le v(e) - \min\left(\frac{v(b)+v(c)-1}{2}, v(d-a)\right) + 4. \]
\end{theorem}
\begin{proof}
  As always $\theta \coloneqq \min\left( v(b)+v(c), 2v(d-a) \right)$
  as in \Cref{ch:orbitalFJ1}.
  Consider \Cref{cor:semi_lie_combo}, and let $N$, $C$, $C'$ be as in the statement.
  Let $N^{\flat\flat}$, $C^{\flat\flat}$, $(C')^{\flat\flat}$
  be the changes to those constants when one replaces $r$ by $r-2$.
  Then one can record the changes to these parameters explicitly, see \Cref{tab:semi_lie_change_by_two}.

  \begin{table}
    \begin{tabular}{lll}
      \toprule
      Assumptions & Parameters for $r$ & Parameters for $r-2$ \\
      \toprule
      $r \ge v(e) - \left\lfloor \tfrac{\theta}{2} \right\rfloor + 3$
        & $\begin{aligned} N &= v(e) \\ C &= C' = 0 \end{aligned}$
        & $\begin{aligned} N^{\flat\flat} &= v(e) \\ C^{\flat\flat} &= (C')^{\flat\flat} = 0 \end{aligned}$ \\
      \midrule
      $\begin{aligned} r &= v(e) - \left\lfloor \tfrac{\theta}{2} \right\rfloor + 2 \\ \theta &= 2v(d-a) \\ &\text{(exceptional)} \end{aligned}$
        & $\begin{aligned} N &= v(e) \\ C &= C' = 0 \end{aligned}$
        & $\begin{aligned} N^{\flat\flat} &= (r-2) + \left\lfloor \tfrac{\theta}{2} \right\rfloor = v(e) \\
          C^{\flat\flat} &= \tfrac{-1 + (v(b)+v(c)-2v(d-a))}{2} \\
          (C')^{\flat\flat} &= C^{\flat\flat}+1 \end{aligned}$ \\
      \midrule
      $\begin{aligned} r &= v(e) - \left\lfloor \tfrac{\theta}{2} \right\rfloor + 1 \\ \theta &= 2v(d-a) \end{aligned}$
        & $\begin{aligned} N &= v(e) \\ C &= C' = 0 \end{aligned}$
        & $\begin{aligned} N^{\flat\flat} &= (r-2) + \left\lfloor \tfrac{\theta}{2} \right\rfloor =v(e)-1 \\
          C^{\flat\flat} &= 0 \\
          (C')^{\flat\flat} &= 1 \end{aligned}$ \\
      \midrule
      $\begin{aligned} r &\le v(e) - \left\lfloor \tfrac{\theta}{2} \right\rfloor \\ \theta &= 2v(d-a) \\ \varkappa &\equiv 0 \pmod 2 \end{aligned}$
        & $\begin{aligned} N &= r + \left\lfloor \tfrac{\theta}{2} \right\rfloor \\
          C &= \tfrac{\varkappa-1 + (v(b)+v(c)-2v(d-a))}{2} \\
          C' &= C+1 \end{aligned}$
        & $\begin{aligned} N^{\flat\flat} &= (r-2) + \left\lfloor \tfrac{\theta}{2} \right\rfloor \\
          C^{\flat\flat} &= \tfrac{(\varkappa-2)-1 + (v(b)+v(c)-2v(d-a))}{2} \\
          (C')^{\flat\flat} &= C^{\flat\flat}+1 \end{aligned}$ \\
      \midrule
      $\begin{aligned} r &\le v(e) - \left\lfloor \tfrac{\theta}{2} \right\rfloor \\ \theta &= 2v(d-a) \\ \varkappa &\equiv 1 \pmod 2 \end{aligned}$
        & $\begin{aligned} N &= r + \left\lfloor \tfrac{\theta}{2} \right\rfloor \\
          C &= \tfrac{\varkappa-1}{2} \\
          C' &= C+1 \end{aligned}$
        & $\begin{aligned} N^{\flat\flat} &= (r-2) + \left\lfloor \tfrac{\theta}{2} \right\rfloor \\
          C^{\flat\flat} &= \tfrac{(\varkappa-2)-1}{2} \\
          (C')^{\flat\flat} &= C^{\flat\flat}+1 \end{aligned}$ \\
      \midrule
      $\begin{aligned} r &\le v(e) - \left\lfloor \tfrac{\theta}{2} \right\rfloor \\ \theta &= v(b)+v(c) \end{aligned}$
        & $\begin{aligned} N &= r + \left\lfloor \tfrac{\theta}{2} \right\rfloor \\
          C &= v(e) - \left\lfloor \tfrac{\theta}{2} \right\rfloor + r \\
          C' = 0 \end{aligned}$
        & $\begin{aligned} N^{\flat\flat} &= (r-2) + \left\lfloor \tfrac{\theta}{2} \right\rfloor \\
          C^{\flat\flat} &= v(e) - \left\lfloor \tfrac{\theta}{2} \right\rfloor + (r-2) \\
          (C')^{\flat\flat} &= 0 \end{aligned}$ \\
      \bottomrule
    \end{tabular}
    \caption{Comparison of $N$ to $N^{\flat\flat}$, etc.,
      needed to carry out the proof of \Cref{thm:semi_lie_finite_codim_full}.
      Note the exceptional case $r = v(e) - \left\lfloor \frac{\theta}{2} \right\rfloor + 2$
      differs from all the others because $C-C^{\flat\flat}$ can be large.}
    \label{tab:semi_lie_change_by_two}
  \end{table}
  Note in particular except for the single exceptional value
  $r = v(e) - \left\lfloor \frac{\theta}{2} \right\rfloor + 2$
  we should always have $(N^{\flat\flat}+2)-N \in \{0,1,2\}$,
  $C - C^{\flat\flat} \in \{0,1,2\}$, $(C') - (C')^{\flat\flat} \in \{0,1\}$.
  More explicitly, we have the following result from \Cref{tab:semi_lie_change_by_two}
  for every $r \ge 3$:
  \begin{align*}
    &\frac{(-1)^{r+v(c)}}{\log q}
    \partial \Orb \left( \guv, \phi_r \right) \\
    &=
    \begin{cases}
      -q^{r+\left\lfloor \frac{\theta}{2} \right\rfloor} -q^{r+\left\lfloor \frac{\theta}{2} \right\rfloor - 1} + q + 1
        & \text{if } r \le v(e) - \left\lfloor \frac{\theta}{2} \right\rfloor + 1 \text{ and } \theta = v(b)+v(c) \\
      -2q^{r+\left\lfloor \frac{\theta}{2} \right\rfloor} + q + 1 & \text{if } r \le v(e) - \left\lfloor \frac{\theta}{2} \right\rfloor + 1 \text{ and } \theta = 2v(d-a) \\
      -q^{v(e)+2} - q^{v(e)+1} + q + 1  & \text{if } r \ge v(e) - \left\lfloor \frac{\theta}{2} \right\rfloor + 3.
    \end{cases}
  \end{align*}
  It follows that for $r \ge v(e) - \left\lfloor \frac{\theta}{2} \right\rfloor + 4$ we have
  \begin{equation}
    \frac{(-1)^{r+v(c)}}{\log q}
    \partial \Orb \left( \guv, \phi_r + \phi_{r-1} \right) = 0
    \label{eq:semi_lie_finite_codim_large_r}
  \end{equation}
  while when $r \le v(e) - \left\lfloor \frac{\theta}{2} \right\rfloor + 1$ we have
  \begin{equation}
    \frac{(-1)^{r+v(c)}}{\log q}
    \partial \Orb \left( \guv, \phi_r + q\phi_{r-1} \right)
    = 1-q^2.
    \label{eq:semi_lie_finite_codim_small_r}
  \end{equation}
  Then \Cref{thm:semi_lie_finite_codim_full} follows directly
  from \eqref{eq:semi_lie_finite_codim_large_r} and \eqref{eq:semi_lie_finite_codim_small_r}.
\end{proof}

\chapter{Transfer factors}
\label{ch:transf}

In this short chapter we document the definitions of the transfer
factors appearing in \Cref{conj:inhomog} and \Cref{conj:semi_lie_spherical}.

\section{Transfer factor for the inhomogeneous group AFL}
The definition of the transfer factor in \Cref{conj:inhomog} is given below:
\begin{definition}
  [{\cite[Equation 2.7]{ref:ZhiyuParahoric}}]
  Choose any $\gamma \in S_n(F)\rs$.
  Let $\mathbf{e} = \begin{pmatrix} 0 & \dots & 0 & 1 \end{pmatrix}^\top \in F^n$
  be a column vector.
  Then the transfer factor $\omega(\gamma)$ is defined by
  \[ \omega(\gamma) \coloneqq
    \eta\left( \det\left( \left( \gamma^i \mathbf{e} \right)_{i=0}^{n-1} \right) \right).
  \]
\end{definition}

For $n=3$, because we gave our answer in terms of a representative
of an $H'$-orbit, it is not trivial to state the transfer factor
in terms of the $a$, $b$, $d$ in \Cref{lem:parameter_constraints}.
However, we don't need this transfer factor anyway in what follows.

\section{Transfer factor for the semi-Lie AFL}
\begin{definition}
  [{\cite[Equation 2.2]{ref:ZhiyuParahoric}}]
  Choose any $\guv \in (S_n(F) \times V_n'(F))\rs$.
  Then the transfer factor $\omega(\gamma)$ is defined by
  \[ \omega\guv \coloneqq
    \eta\left( \det\left( \left( \gamma^i \uu \right)_{i=0}^{n-1} \right) \right).
  \]
\end{definition}

In particular, for $n = 2$, the transfer factor on our representative
\[ \guv = \left( \begin{pmatrix} a & b \\ c & d \end{pmatrix},
  \begin{pmatrix} 0 \\ 1 \end{pmatrix}, \begin{pmatrix} 0 & e \end{pmatrix} \right) \]
used in \Cref{ch:orbitalFJ0} is given by
\begin{equation}
    \omega\guv
    = \eta\left( \det
      \left( \gamma^0 \uu, \gamma^1 \uu \right)
    \right)
    = \eta\left( \det
      \begin{pmatrix} 0 & b \\ 1 & d \end{pmatrix}
    \right)
    = (-1)^{v(b)} = (-1)^{v(c)+1}.
  \label{eq:semi_lie_transfer}
\end{equation}


\chapter{The geometric side}
\label{ch:geo}

\section{Rapoport-Zink spaces}
We briefly recall the theory of Rapoport-Zink spaces.
This follows the exposition in \cite[\S4.1]{ref:survey}.

Let $\breve F$ denote the completion of a maximal unramified extension of $F$,
and let $\FF$ denote the residue field of $\OO_{\breve F}$.
Suppose $S$ is a $\Spf \OO_{\breve F}$-scheme.
Then we can consider triples $(X, \iota, \lambda)$ consisting of the following data.
\begin{itemize}
  \ii $X$ is a formal $\varpi$-divisible $n$-dimensional $\OO_F$-module over $S$
  whose relative height is $2n$.

  \ii $\iota \colon \OO_E \to \End(X)$ is an action of $\OO_E$
  such that the induced action of $\OO_F$ on $\Lie X$
  is via the structure morphism $\OO_F \to \SO_S$.

  We require that $\iota$ satisfies the Kottwitz condition of signature $(n-1,1)$,
  meaning that for all $a \in \OO_E$,
  the characteristic polynomial of $\iota(a)$ on $\Lie X$
  is exactly \[ (T-a)^{n-1} (T-\bar a) \in \SO_S[T]. \]

  \ii $\lambda \colon X \to X^\vee$ is a principal $\OO_F$-relative polarization.

  We require that the Rosati involution of $\lambda$
  induces the map $a \mapsto \bar a$ on $\OO_F$
  (i.e.\ the nontrivial automorphism of $\Gal(E/F)$).
\end{itemize}
The triple is called supersingular if $X$ is a supersingular strict $\OO_F$-module.

For each $n \ge 1$, over $\FF$
we choose a supersingular triple $(\XX_n, \iota_{\XX_n}, \lambda_{\XX_n})$;
it's unique up to $\OO_E$-linear quasi-isogeny compatible with the polarization,
and refer to it as the \emph{framing object}.
We can now define the Rapoport-Zink space:
\begin{definition}
  For each $n \ge 1$, we let $\RZ_n$ denote the
  functor over $\Spf \OO_{\breve F}$ defined as follows.
  Let $S$ be an $\Spf \OO_{\breve F}$ scheme, and let
  $\ol S \coloneqq S \times_{\Spf \OO_{\breve F}} \Spec \FF$.
  For every $\Spf \OO_{\breve F}$ scheme, we let $\RZ_n(S)$
  be the set of isomorphism classes of quadruples
  \[ (X, \iota, \lambda, \rho) \]
  where $(X, \iota, \lambda)$ is one of the triples as we described, and
  \[ \rho \colon X \times_S \ol S \to \XX_n \times_{\Spec \FF} \ol S \]
  is a \emph{framing}, meaning it is an height zero $\OO_F$-linear quasi-isogeny
  and satisfies
  \[ \rho^\ast((\lambda_{\XX_n})_{\ol S}) = \lambda_{\ol S}. \]
\end{definition}
Then $\RZ_n$ is formally smooth over $\OO_{\breve F}$ of relative dimension $n-1$.

Henceforth, we also make the following abbreviation.
\begin{definition}
  For integers $m$ and $n$,
  \[ \RZ_{m,n} \coloneq \RZ_{m} \times_{\Spf \OO_{\breve F}} \RZ_n. \]
\end{definition}

\section{A realization of the non-split Hermitian space $\VV_n$ of dimension $n$}
For the following definition (and later on), we need a variation of $\RZ_1$:
\begin{definition}
  Let $(\EE, \iota_\EE, \lambda_\EE)$ be the unique triple over $\FF$
  whose Rosati involution has signature $(1,0)$
  (note this is different from $\RZ_1$ where the signature is $(0,1)$ instead).
\end{definition}

At the same time, we can define the following Hermitian space.
\begin{definition}
  For each $n \ge 1$, let
  \[ \VV_n \coloneqq \Hom_{\OO_E}^\circ (\EE, \XX_n) \]
  which we call the space of special homomorphisms.
  When endowed with the form
  \[ \left< x,y \right> = \lambda_{\EE}^{-1} \circ y^\vee \circ \lambda_{\XX_n} \circ x
    \in \End_{E}^\circ(\EE) \simeq E \]
  it becomes an $n$-dimensional $E/F$-Hermitian space.
  \label{def:VV_n}
\end{definition}
\begin{proposition}
  Up to isomorphism, $\VV_n$ is the unique $n$-dimensional
  nondegenerate non-split $E/F$-Hermitian space.
\end{proposition}
\begin{proof}
  \todo{ref}
\end{proof}

\section{Intersection numbers for the group version of AFL for the full spherical Hecke algebra}
Here we reproduce the definition of the intersection number used in \Cref{conj:inhomog}.

Compared to the formulation of the group version and semi-Lie version of the AFL,
the intersection number requires the introduction of a
\emph{Hecke operator} $\TT_{\varphi}$ for an element
\[ \varphi \in \HH(G^\flat \times G, K^\flat \times K) \]
as introduced in \cite{ref:AFLspherical}.
This definition is too involved to reproduce here in its entirety,
we give a summary for this special cases in which we need.

First consider the given $f \in \HH(G, K)$.
The main work of the construction is to define another
derived formal scheme $\mathcal{T}_{\mathbf{1}_{K^\flat} \otimes f}$
(see \cite[\S6.1]{ref:AFLspherical}) together with two projection maps
\begin{center}
\begin{tikzcd}
  & \ar[ld] \mathcal T_{\mathbf{1}_{K^\flat} \otimes f} \ar[rd] & \\
  \RZ_{n-1,n} && \RZ_{n-1,n}.
\end{tikzcd}
\end{center}
This definition is carried out in \cite[\S5]{ref:AFLspherical},
by defining it first for so-called \emph{atomic elements} of the spherical Hecke algebra,
which form basis elements of a certain presentation of this Hecke algebra
as the unitary group for a polynomial algebra;
we refer the reader to \emph{loc.\ cit.}~for the full details.

Now, take the natural closed embedding
\[ \RZ_{n-1} \to \RZ_n \]
and let
\[ \Delta \colon \RZ_{n-1} \hookrightarrow \RZ_{n-1,n} \]
be the associated graph morphism.
Then we denote by $\iota \colon \Delta(\RZ_n) \hookrightarrow \RZ_{n-1,n}$
the inclusion mapping of $\Delta$.
Once this is done, consider then the diagram
\begin{center}
\begin{tikzcd}
  & \pi_1^\ast(\Delta_{\RZ_{n-1,n}}) \ar[ld] \ar[r] \ar[rd]
    & \ar["\pi_1" near end, ld] \mathcal T_{\mathbf{1}_{K^\flat} \otimes f}
      \ar["\pi_2" near end, rd] \\
  \Delta_{\RZ_{n-1,n}} \ar[r, "\iota"', hook] & \RZ_{n-1,n}
  & (\pi_2)_\ast(\pi_1^\ast(\Delta_{\RZ_{n-1,n}})) \ar[r] & \RZ_{n-1,n}.
\end{tikzcd}
\end{center}
That is, one takes the pullback of
$\Delta_{\RZ_{n-1,n}} \xhookrightarrow{\iota} \RZ_{n-1,n}$
along the projection
\[ \RZ_{n-1,n} \xleftarrow{\pi_1} \mathcal{T}_{\mathbf{1}_{K^\flat} \otimes f} \]
and then takes the pushforward along the other projection
\[ \mathcal{T}_{\mathbf{1}_{K^\flat} \otimes f} \xrightarrow{\pi_2} \RZ_{n-1,n}. \]
\begin{definition}
  Set
  \[
    \TT_{\mathbf{1}_{K^\flat} \otimes f} (\Delta_{\RZ_{n-1,n}})
    \coloneqq (\pi_2)_\ast(\pi_1^\ast(\Delta_{\RZ_{n-1,n}})).
  \]
\end{definition}
This is the part of the intersection number depending on $f$
(or rather $\TT_{\mathbf{1}_{K^\flat} \otimes f}$).
As for our $g \in \U(\VV_n)\rs$,
consider the translation $(1,g) \cdot \Delta_{\RZ_{n-1,n}}$.
The intersection number is then defined as by taking the intersection
of these two objects using the derived tensor product $\jiao$ of the structure sheaves.
\begin{definition}
  [{\cite[Equation (6.1.1)]{ref:AFLspherical}}]
  We define the intersection number in \Cref{conj:inhomog} by
  \[
    \Int((1,g), \mathbf{1}_{K^\flat} \otimes f)
    \coloneqq \chi_{\RZ_{n-1,n}} \left(
      \SO_{\TT_{\mathbf{1}_{K^\flat} \otimes f} (\Delta_{\RZ_{n-1}})}
      \jiao_{\SO_{\RZ_{n-1,n}}} \SO_{(1,g) \cdot \Delta_{\RZ_{n-1}}} \right).
  \]
  \label{def:intersection_number_inhomog}
\end{definition}
Here $\chi$ denotes the Euler-Poincar\'{e} characteristic,
meaning that if $X$ is a formal scheme over $\Spf \OO_{\breve F}$
then given a finite complex $\mathcal{F}$ of $\SO_X$-modules we set
\[ \chi_X(\mathcal{F}) = \sum_i \sum_j (-1)^{i+j}
  \operatorname*{len}_{\OO_{\breve F}} H^j(X, H_i(\mathcal F)) \]
provided all the lengths are finite.

\section{Intersection numbers for the semi-Lie version of AFL for the full spherical Hecke algebra}
Now we continue to define an intersection number needed for the proposed
\Cref{conj:semi_lie_spherical} from earlier.
The definition mirrors the one given in the last section.
Here we reproduce the definition of the intersection number used in \Cref{conj:inhomog}.

We work here with $\RZ_{n,n}$ rather than $\RZ_{n-1,n}$.
The change is that we need to incorporate the new $u \in \VV_n$ that was not present before.
In order to do this one considers a certain relative Cartier divisor $\mathcal Z(u)$
on $\RZ_n$ for each nonzero $u \in \VV_n$.
This divisor was defined by Kudala and Rapport in \cite{ref:KR}
and accordingly we call it a \emph{KR-divisor} following \cite[\S4.3]{ref:survey}.
The definition is given as follows.
\begin{definition}
  Recall that $(\EE, \iota_\EE, \lambda_\EE)$ is the unique triple over $\FF$
  whose Rosati involution has signature $(1,0)$.
  Then the formal $\OO_F$-module has a unique lifting called its \emph{canonical lifting},
  which we denote by the triple $(\mathcal{E}, \iota_{\mathcal{E}}, \lambda_{\mathcal E})$.
  Then the KR-divisor $\mathcal Z(u)$ is the locus where the quasi-homomorphism
  $\EE \to \XX_n$ lifts to a homomorphism from $\mathcal{E}$ to the universal object over $\RZ_n$.
\end{definition}
Colloquially, the KR-divisor $\mathcal Z(u)$ can be thought of as a
moduli space parametrized by diagrams of the form
\begin{center}
\begin{tikzcd}
  \mathcal{E}\ar[r] \ar[d, dash] & \mathcal{X}_n \ar[d, dash] \\
  \EE \ar[r, "u"] & \XX_n.
\end{tikzcd}
\end{center}
Note also by the definition that $g \mathcal Z(u) = \mathcal Z (gu)$.

The main change is then that we can consider $\Delta_{\mathcal Z(u)}$ as the image of
\[ \mathcal Z(u) \hookrightarrow \RZ_n \xrightarrow{\Delta} \RZ_{n,n} \]
where $\Delta \colon \RZ_n \to \RZ_{n,n}$ now denotes the diagonal map.
If one defines an appropriate space $\mathcal T_{\mathbf{1} \otimes f}$ for $f \in \HH(G, K)$ together with
\begin{center}
\begin{tikzcd}
  & \ar[ld] \mathcal T_{\mathbf{1} \otimes f} \ar[rd] & \\
  \RZ_{n,n} && \RZ_{n,n}
\end{tikzcd}
\end{center}
then one can then repeat the diagram from before:
\begin{center}
\begin{tikzcd}
  & \pi_1^\ast(\Delta_{\mathcal Z(u)}) \ar[ld] \ar[r] \ar[rd]
    & \ar["\pi_1" near end, ld] \mathcal T_{f}
      \ar["\pi_2" near end, rd] \\
  \Delta_{\mathcal Z(u)} \ar[r, "\iota"', hook] & \RZ_{n,n}
  & (\pi_2)_\ast(\pi_1^\ast(\Delta_{\mathcal Z(u)})) \ar[r] & \RZ_{n,n}.
\end{tikzcd}
\end{center}
In other words, we again take a pullback followed by a pushforward
but this time of $\Delta_{\mathcal Z(u)} \hookrightarrow \RZ_{n,n}$.
This lets us write an analogous definition:
\begin{definition}
  Set
  \[
    \TT_{\mathbf{1}_K \otimes f} (\Delta_{\mathcal Z(u)})
    \coloneqq (\pi_2)_\ast(\pi_1^\ast(\Delta_{\mathcal Z(u)})).
  \]
\end{definition}
Meanwhile to replace $(1,g) \Delta_{\RZ_{n,n-1}}$, we let
\[ \Gamma_g \subseteq \RZ_{n,n} \]
denote the graph of the automorphism of $\RZ_n$ induced by $g$.
This finally allows us to write a definition of the intersection number in the semi-Lie case:
\begin{definition}
  We define the intersection number in \Cref{conj:semi_lie_spherical} as
  \[
    \Int((g,u), f)
    \coloneqq \chi_{\RZ_{n,n}} \left(
      \SO_{\TT_{\mathbf{1} \otimes f}(\Delta_{\ZD(u)})}
      \jiao_{\SO_{\RZ_{n,n}}} \SO_{\Gamma_g} \right).
  \]
  \label{def:intersection_number_semi_lie_spherical}
\end{definition}

\section{Transfer factors}
\todo{This entire section needs to be written}

\section{An analogy between the geometric and analytic sides}
With the intersection number now defined for \Cref{conj:semi_lie_spherical},
we provide some philosophical discussion about the connection.
All of this is for philosophical cheerleading only,
and is not meant to formally assert any definitions or results.
But it may help in motivating the formulation of the conjecture.

For simplicity we only focus on the semi-Lie AFL originally proposed by Liu to start;
which is the special case of \Cref{conj:semi_lie_spherical}
when $f = \mathbf{1}_K$ and $\phi = \mathbf{1}_{K'}$.

\paragraph{The geometric side}
On the geometric side, $\RZ_n$ is the RZ-space acted on by $\U(\VV_n)$,
and hence $\U(\VV_n) \times \U(\VV_n)$ acts on $\RZ_{n,n}$.
Roughly speaking, we are considering the two morphisms
\[ \RZ_n \xrightarrow{\Delta} \RZ_{n,n} \xleftarrow{\Gamma_g} \RZ_n \]
with $\Delta$ being thought of as the diagonal morphism
and $\Gamma_g$ as the graph under multiplication by $g \in \U(\VV_n)\rs$.

Hence loosely speaking, the intersection $\Int\left( (g,u), \mathbf{1}_K \right)$
can be thought of as the intersection of three images in $\RZ_{n,n}$:
\begin{itemize}
  \ii A ``diagonal'' object $\Delta$;
  \ii A ``graph'' object $\Gamma$;
  \ii A third object $\mathcal Z(u)$, the KR-divisor, parametrized to diagrams
  \begin{center}
  \begin{tikzcd}
    \mathcal{E}\ar[r] \ar[d, dash] & \mathcal{X}_n \ar[d, dash] \\
    \EE \ar[r, "u"] & \XX_n.
  \end{tikzcd}
  \end{center}
\end{itemize}
The derived tensor product $\jiao$ and the derived tensor products
are used together with some formalism to make this intersection idea precise.
The intersection of the ``diagonal'' and ``graph'' is the \emph{fixed point locus},
and in fact could be formally defined as the intersection
\[ \Gamma_g \cap \Delta_{\RZ_n} \]
viewed as a closed formal subscheme of $\RZ_{n,n}$;
see \cite[equation (4.6)]{ref:survey}.

\paragraph{The analytic side}
On the other hand, consider the analytic side.
We will try to explain how the weighted orbital integral in \Cref{def:orbitalFJ}
can be thought of as some weighted intersection of analogous objects.

Recall we set $G = \U(V_0)$ and $K = G \cap \GL_n(\OO_E)$
the hyperspecial maximal compact subgroup.
In that case the quotient $G/K$ can be identified as
\[ G/K \simeq \left\{ \Lambda \subseteq V_0 \mid \Lambda^\vee = \Lambda \right\} \]
that is, the set of self-dual lattices $\Lambda$ of full rank,
which thus has a natural action of $G$.
Henceforth we denote elements of $G/K$ by $h$,
and fix one particular such lattice $\Lambda_0$, acted on by $\OO_E$.
Hence $G/K$ can be thought of as
\[ G/K \simeq \left\{ h \Lambda_0 \mid h \in G/K \right\}. \]

Recall from \eqref{eq:unweighted_orbital_semi_lie} we have an orbital integral
that looks something like
\[ \int_{h \in G/K} \mathbf{1}_K(h^{-1} g h) \mathbf{1}_{\Lambda_0}(h \cdot u) \odif h \]
where $u \in V_0$, and $\Lambda_0$ is a fixed particular lattice in $G/K$.
See for example the ``relative fundamental lemma''
stated as \cite[Conjecture 1.9]{ref:liuFJ},
for a precise equivalence between the stated weighted orbital integral and ours.

Like before, we can consider two maps
\[ G/K \xrightarrow{\Delta} G/K \times G/K \xleftarrow{\Gamma_g} \times G/K \]
which are the diagonal morphism and the graph of the action of $g$.
Hence the intersection are those cosets $hK$ for which
\[ hK = ghK \iff h^{-1} g h \in K. \]
Hence the indicator function $\mathbf{1}_K(h^{-1} g h)$
plays the analog of the fixed point locus in the geometric side.

Meanwhile, the term $h \cdot u$ plays a role analogous
to the KR-divisor on the geometric side, giving the third intersection object.
We have
\[ h \cdot u \in \Lambda_0 \iff u \in h^{-1} \Lambda_0 \]
and so the object corresponding to the KR-divisor $\mathcal Z(u)$
is the subset in $G/K$ of those lattices containing $u$, that is
\[ \left\{ \Lambda' \mid \Lambda' \ni u \right\}. \]
The analog to the earlier diagram that we described for $\mathcal Z(u)$ is then
\begin{align*}
  \OO_E &\to \Lambda' \\
  1 &\mapsto u.
\end{align*}

Up until now this whole section is written for $f = \mathbf{1}_K$ and $\phi = \mathbf{1}_{K'}$.
In the general situation,
if one replaces $\mathbf{1}_K$ in the above integral by a general $f$,
then this corresponds to changing the analog of the fixed point locus;
the idea of \cite{ref:AFLspherical} is that
this should correspond to replacing $\Delta_{\mathcal Z(u)}$
with $\TT_f(\Delta_{\mathcal Z(u)})$ on the geometric side.

\section{Intersection numbers for $\Int((g,u), f)$ for $n = 2$}
\label{ch:jiao}
This chapter is dedicated to computing intersection numbers
for the semi-Lie version of AFL in the special case $n = 2$.

\subsection{Background on quaternion division algebra}
Through this section we let $\DD$ be a quaternion division algebra over $F$,
with a fixed maximal order $\OO_\DD$.
We will make $\DD$ explicit in the following way for our calculations to follow.

\subsubsection{Structure as a noncommutative algebra}
As $F$-vector spaces we will write
\[ \DD = E \oplus E \Pi \]
where $\Pi$ is selected so that $\Pi^2 = \varpi$.
We endow $\DD$ with a noncommutative multiplication according to
\[ \Pi t = \bar t \Pi \qquad \text{ for all } t \in E \]
where $\ol{t}$ is the image of $t \in E$ under the nontrivial element of $\Gal(E/F)$.

\subsubsection{Conjugation of elements of $\DD$}
In general, suppose $x \in \DD$ is any element
decomposed as $x = a + b \Pi$ for $a,b \in E$.
Then we denote by $\bar x \in \DD$ the conjugate in $\DD$ defined by
\[ \bar x \coloneqq \bar a - b \Pi \]
where, again, $\bar a$ is the image of $a \in E$ under the nontrivial element of $\Gal(E/F)$.
It is an anti-involution, meaning that $\ol{\bar x} = x$ and $\ol{xy} = \bar y \bar x$.

(Notice that we have a slight abuse of notation here in that we have
used the same notation to denote both conjugation under the Galois action of $\Gal(E/F)$
as well as the conjugation in $\DD$.
However, there is no ambiguity resulting because when $E$ is viewed as a subset of $\DD$,
the two symbols denote the same element of $E$:
that is we have
\[ \ol{a + 0 \Pi} = \bar{a} + 0 \Pi \]
in any event.
In other words, the restriction of the quaternion conjugation to $E$
coincides with the nontrivial element of $\Gal(E/F)$,
so we do not need to introduce a separate notation for it.)

This allows us to define the reduced norm and trace in $\DD$.
The reduced trace is given by
\[ \tr x \coloneqq x + \bar x = \Tr{E/F}(a) = 2x_0 \in F. \]
We may thus define
\[ \DT \coloneqq \left\{ u \in \DD \mid \tr u = 0 \right\} \]
which has codimension $1$ inside $\DD$ (i.e.\ is three-dimensional as an $F$-vector space).
Since $\tr(a+b\Pi) = \Tr_{E/F}(a)$, we could also write
\[ \DT = \{ a+b\Pi \mid a,b \in E \text{ and } \Tr_{E/F}(a) = 0 \}. \]

The reduced norm is similarly defined by
\begin{align*}
  \Nm x &= x \bar x = (a + b \Pi)(\bar a - b \Pi) \\
  &= a \bar a + b \Pi \bar a - a b \Pi - b \Pi b \Pi \\
  &= a \bar a - b \bar b \varpi \\
  &= \Norm_{E/F}(a) - \Norm_{E/F}(b) \varpi \in F.
\end{align*}

As an $F$-vector space, $\DD$ has a basis given by
$\{1, \sqrt{\eps}, \Pi, \sqrt{\eps}\Pi\}$, that is
\[ \DD = F \oplus F\sqrt\eps \oplus F \Pi \oplus F\sqrt\eps\Pi. \]
It will be convenient to introduce the following notation:
\begin{definition}
  [$x_0$ and $x_-$]
  For $x \in \DD$, we introduce the notation $x_0$ and $x_-$ to mean
  \begin{itemize}
    \ii $x_0$ is the projection into the first component $F$; and
    \ii $x_- = x - x_0$ is the projection into
    $\DT = F\sqrt\eps \oplus F \Pi \oplus F\sqrt\eps\Pi$.
  \end{itemize}
\end{definition}
In particular, the formula for conjugation then reads as the simpler
\[ \bar x = x_0 - x_-. \]

\subsubsection{Hermitian structure}
We view $\DD$ as an $E/F$-Hermitian space under left multiplication by $E$;
that is, for $a,b,t \in E$ we consider
\[ t \cdot (a + b \Pi) = a t + b t \Pi. \]
as the action of $E$ on $\DD$.
Then we equip $\DD$ with a $E/F$-Hermitian form
$\left\langle \bullet, \bullet \right\rangle \colon \DD \times \DD \to E$
defined by
\[ \left\langle x,y \right\rangle = \half \Tr_{\DD/E}(x \bar y) \]
i.e.\ the projection of $x \bar y \in \DD = E \oplus E\Pi$ onto the first component.
In particular, note that
\[ \left\langle x,x \right\rangle = x \bar{x} = \Nm x \]
or equivalently
\[ \left\langle a+b\Pi, a+b\Pi \right\rangle = a \bar a - b \bar b \varpi. \]

\subsubsection{Identification of $\VV_n^-$ with $\DD$}
We continue using the notation $(\EE, \iota_\EE, \lambda_\EE)$ as the triple over $\FF$
whose Rosati involution has signature $(1,0)$.
Moreover, we will take the identification
\[ \End(\EE) \simeq \OO_\DD \]
see \cite[Remark 2.5]{ref:KR},
and hence the corresponding identification
\[ \VV_n^- \simeq \DD. \]

\subsection{The invariants for the orbit of $(g,u)$}
\label{sec:g_u_invariants}

We specialize to the situation where $u \in \OO_\DD$ and $g \in \U(\VV_n^-)$.

\subsubsection{Coordinates for $g$}
To impose coordinates on $g$, we appeal to the following fact.
\begin{lemma}
  [Description of $\U(\VV_2^-)$]
  Every unitary map in $\U(\VV_2^-)$ can be described in the form
  \[ x \mapsto \lambda\inv x (\alpha + \beta \Pi) \]
  for some quaternion $\alpha + \beta \Pi \in \DD^\times$
  and an element $\lambda \in E^\times$ such that
  \[ \Nm(\alpha + \beta \Pi) = \Norm_{E/F}(\lambda). \]

  Moreover, such a description is unique up to multiplication by elements of $F$.
  In other words,
  \[ \U(\VV_2^-) \simeq (E^\times \times \DD^\times)^\circ / \Delta(F^\times) \]
  where $(\DD^\times \times E^\times)^\circ$
  denotes those pairs $(\lambda, \alpha + \beta \Pi)$
  with $\Norm_{E/F}(\lambda) = \Nm(\alpha + \beta \Pi)$,
  and $\Delta(F^\times)$ is the diagonal embedding of $F^\times$.
\end{lemma}
\begin{remark}
  In this paper we will not have a need to compose multiple such unitary maps.
  However, if we did, then our notation above swaps the multiplication order.
  That is, if we have $g_1, g_2 \in \U(\VV_2^-)$ represented by pairs
  $g_1 \leftrightarrow (\lambda_1, \alpha_1 + \beta \Pi_1)$
  and $g_2 \leftrightarrow (\lambda_2, \alpha_2 + \beta \Pi_2)$
  under the isomorphism above, then
  \[ g_1 \circ g_2 \leftrightarrow (\lambda_2 \lambda_1, (\alpha_2 + \beta_2 \Pi)(\alpha_1 + \beta \Pi)). \]
\end{remark}

Note that in the definition for $g$, if $v(\lambda) \neq 0$
then we can factor out powers of $\varpi = \bar\varpi$ from $\lambda$
and put them into $\alpha$ and $\beta$ instead.
Hence, by relabeling $\alpha$ and $\beta$, we may assume without loss of generality that:
\begin{assume}
  We assume WLOG that $v(\lambda) = 0$ (and hence $v(\alpha)=0$, $v(\beta) \ge 0$).
\end{assume}
This frees us from having to deal with $v(\lambda)$ offsets in subsequent calculation.

We encode $g$ as a matrix on $\DD$ now (with the obvious $E$-basis $1$, $\Pi$,
again viewing $\DD$ as a left-$E$ module)
so we can compute its determinant and trace.
We have
\begin{align*}
  g(1) &= \lambda\inv \cdot 1 \cdot (\alpha + \beta \Pi)
    = \lambda\inv \alpha + \lambda\inv \beta \Pi \\
  g(\Pi) &= \lambda\inv \cdot \Pi (\alpha + \beta \Pi)
    = \lambda\inv \bar\beta \varpi + \lambda\inv \bar\alpha \Pi.
\end{align*}
Hence, written as a matrix with respect to the obvious basis $\{1, \Pi\}$ we have
\[ g = \lambda\inv \begin{pmatrix}
  \alpha & \bar\beta \varpi \\
  \beta &  \bar\alpha
  \end{pmatrix}. \]

\subsubsection{Coordinates for $u$}
We also impose coordinates for $u$ according to
\[ u = s + t \Pi \in \OO_\DD \qquad s, t \in E. \]
To make the calculation that follows less complicated,
we are going to make the following assumption on $u$.
\begin{assume}
  We assume WLOG that either $u \in E$ or $u \in E \Pi$.
  That is, either $s = 0$ or $t = 0$.
  \label{assume:st_zero}
\end{assume}
This assumption can be made without loss of generality because
the invariants and the intersection only depend
on the $\SU(2)$-orbit of the pair $(g,u)$,
and any element $u \in \DD^\times$ can be mapped under an element of $\SU(2)$
into a pair for which $u \in E$ or $u \in E \Pi$.

In order for $(g,u)$ to be regular semisimple we require that
\begin{align*}
  u &= s + t \Pi \\
  g(u) &= \lambda\inv (s + t \Pi)(\alpha + \beta \Pi) \\
  &= \lambda\inv \left( (s \alpha + t \bar\beta \varpi) + (s \beta + t \bar\alpha)\Pi \right)
\end{align*}
are linearly independent, meaning
\begin{align*}
  0 &\neq
  \det \begin{pmatrix}
    s & s \alpha + t \bar \beta \varpi \\
    t & s \beta + t \bar \alpha
  \end{pmatrix}
  = st(\bar\alpha - \alpha) + \beta s^2 - \bar\beta t^2 \varpi \\
  &= \beta s^2 - \bar\beta t^2 \varpi \\
  &= \begin{cases}
    -\bar\beta t^2 \varpi & \text{if } s = 0 \\
    \beta s^2 & \text{if } t = 0.
  \end{cases}
\end{align*}
Hence, we have a requirement that $\beta \neq 0$
and $s$ and $t$ are not both zero (we require $st = 0$ from \Cref{assume:st_zero}).
\begin{remark}[$\alpha \neq 0$]
  Note that necessarily $\alpha$ is nonzero as well.
  This follows from the requirement that
  $\alpha \bar \alpha - \beta \bar \beta \varpi = \lambda \bar\lambda$;
  if $\alpha = 0$ we would get a left-hand side with odd valuation
  but a right-hand side with even valuation.
\end{remark}

\subsubsection{The invariants of the matching}
At this point we can state:
\begin{lemma}
  \label{lem:g_u_invariants}
  Under the coordinates we just described,
  the four corresponding invariants of \Cref{def:matching_semi_lie} are:
  \begin{align*}
    \Tr g &= \lambda^{-1} (\alpha + \bar \alpha) \\
    \det g &= \lambda^{-2} (\alpha \bar \alpha - \beta \bar\beta \varpi) \\
    \left\langle u,u \right\rangle &= s \bar s - t \bar t \varpi \\
    \left\langle g(u), u \right\rangle &= \begin{cases}
        \lambda\inv \bar\alpha \Nm u & \text{if } s = 0 \\
        \lambda\inv \alpha \Nm u & \text{if } t = 0.
      \end{cases}
  \end{align*}
\end{lemma}
\begin{proof}
  The first three are immediate.
  The last one follows by computing
  \begin{align*}
    \left\langle g(u), u \right\rangle
    &= \left\langle \lambda\inv(s + t \Pi)(\alpha + \beta \Pi), s + t \Pi \right\rangle \\
    &= \lambda\inv \left\langle
      (s \alpha + t \bar \beta \varpi) + (t \bar \alpha + s \beta) \Pi,
      s + t \Pi \right\rangle \\
    &= \lambda\inv \cdot \half \Tr_{\DD/E} \left[
      ((s \alpha + t \bar \beta \varpi) + (t \bar \alpha + s \beta) \Pi)
      (\bar s - t \Pi) \right] \\
    &= \lambda\inv \left( s \bar s \alpha - t \bar t \bar \alpha \varpi \right)
  \end{align*}
  and recalling that $s t = 0$.
\end{proof}

\subsection{A basis for $\HH(\U(\VV_2^+))$}
\label{sec:hecke_unitary_basis}

As before $K = \GL_2(\OO_E) \cap \U(\VV_2^+)$denotes the maximal hyperspecial compact subgroup of $\U(\VV_2^+)$.
For each $r \ge 0$, we define
\[ \fr \coloneqq \mathbf{1}_{\varpi^{-r} \Mat_2(\OO_E) \cap \U(\VV_2^+)} \in \HH(\U(\VV_2^+)). \]
For convenience $\fr = 0$ for $r < 0$.
We also set
\[  \mathbf{1}_{K, r} \coloneqq \fr - \mathbf{1}_{K, \le (r-1)} \]
which is the indicator function for the coset
\[ K \begin{pmatrix} 0 & \varpi^r \\ \varpi^{-r} & 0 \end{pmatrix} K. \]
Note when $r = 0$, $\mathbf{1}_{K, 0} = \mathbf{1}_K = \mathbf{1}_{K, \le 0}$.

Analogous to \Cref{sec:hecke_basis_FJ} we then have the following result.
\begin{proposition}[$\fr$ basis]
  The functions $\fr$ (for $r \ge 0$) form a basis of $\HH(\U(\VV_2^+))$.
  (Similarly, so do $\mathbf{1}_{K, r}$ for $r \ge 0$.)
\end{proposition}
\begin{proof}
  This follows from the fact that
  \[ \U(\VV_2^+) = \coprod_{r \ge 0}
    K \begin{pmatrix} 0 & \varpi^r \\ \varpi^{-r} & 0 \end{pmatrix} K. \]
  See the comment in \cite[Equation (7.1.5)]{ref:AFLspherical}.
\end{proof}

The base change for this basis is given later in \Cref{lem:finale_base_change}.

\subsection{Background on special divisors for $n = 2$}
\subsubsection{The Rapoport-Zink space $\RZ_2$ and $\TTr$}
Recall $\RZ_2$ from \Cref{ch:geo}.
With the Hecke operator $\TT$ from \cite{ref:AFLspherical}
(see \Cref{ch:geo} for discussion)
we introduce $\TTr = \TT_{\mathbf{1}_K \otimes \mathbf{1}_{K, \le r}}(\Delta_{\RZ_n})$
so that we have the diagram
\begin{center}
\begin{tikzcd}
  & \ar[ld] \TTr \ar[rd] & \\
  \RZ_{2} && \RZ_{2}
\end{tikzcd}
\end{center}

\subsubsection{The Lubin-Tate space $\MM_2$}
We introduce the notation $\MM_2$ for the \emph{Lubin-Tate space} for $n = 2$.
It is defined almost in the same way as $\RZ_2$ except that we replace
$\XX_2$ with $\EE$ now.

\begin{definition}
  [Lubin-Tate space]
  We let $\MM_2$ denote the functor over $\Spf \OO_{\breve F}$ defined as follows.
  Let $S$ be an $\Spf \OO_{\breve F}$ scheme, and let
  $\ol S \coloneqq S \times_{\Spf \OO_{\breve F}} \Spec \FF$.
  For every $\Spf \OO_{\breve F}$ scheme, we let $\RZ_n(S)$
  be the set of isomorphism classes of quadruples
  \[ (Y, \iota, \lambda, \rho) \]
  where $(Y, \iota, \lambda)$ is one of the triples as we described, and
  \[ \rho \colon Y \times_S \ol S \to \EE \times_{\Spec \FF} \ol S \]
  is a \emph{framing}, meaning it is a height zero $\OO_F$-linear quasi-isogeny
  and satisfies
  \[ \rho^\ast((\lambda_{\EE})_{\ol S}) = \lambda_{\ol S}. \]
\end{definition}

\begin{proposition}
  [{\cite[Example 5.5.6]{ref:AFLspherical}}]
  The Serre tensor construction produces an identification
  \[ \Serre \colon \RZ_2 \xrightarrow{\sim} \MM_2. \]
  By abuse of notation we will also use the same symbol for the map
  \[ \Serre \colon \RZ_{2,2} \xrightarrow{\sim} \MM_2 \times \MM_2. \]
\end{proposition}

Recall we have an action of $\U(\VV_2^-)$ (actually $\PU(\VV_2^-)$) on $\RZ_2$.
We describe now the corresponding action on $\MM_2$.
We have an isomorphism of short exact sequences
\begin{center}
\begin{tikzcd}
  1 \ar[r] & \OO_E^\times / \OO_F \ar[r] & (\OO_\DD^\times \times \OO_E^\times)^\circ / \Delta(\OO_F^\times) \ar[r] & \OO_\DD^\times / \OO_F^\times \ar[r] & 1 \\
  1 \ar[r] & \U(1) \ar[r] \ar[u, equals] & \U(\VV_2^-) \ar[r] \ar[u, equals] & \PU(\VV_2^-) \ar[r] \ar[u, equals] & 1.
\end{tikzcd}
\end{center}
The image of $\alpha + \beta \Pi \in \OO_\DD^\times / \OO_F^\times$
is then $(\lambda, \alpha + \beta \Pi) \in \PU(2)$
for any choice of $\lambda$ with $\lambda \bar \lambda = \Nm (\alpha + \beta \Pi)$.

\subsubsection{The divisor $\ZO4$ on $\MM_2 \times \MM_2$}
Now we define the special orthogonal divisor $\ZO4(u)$
on $\MM_2 \times \MM_2$ as follows.
\begin{definition}
  [$\ZO4(u)$]
  Let $u \in \OO_\DD$.
  Then we define the divisor $\ZO4(u)$ to be the pairs $(Y, Y') \in \MM_2 \times \MM_2$
  such that there exists $\varphi \colon Y \to Y'$ with the following property.
  Let $S$ be an $\Spf \OO_{\breve F}$ scheme and consider the map on special fiber
  \[ Y \times_S \ol S \xrightarrow{\varphi \times_S \ol S} Y' \times_S \ol S. \]
  Also, from $Y \in \MM_2$ and $Y' \in \MM_2$ we have the data of framings
  $\rho \colon Y \times_S \ol S \to \EE \times_{\Spec \FF} \ol S$
  and $\rho' \colon Y' \times_S \ol S \to \EE \times_{\Spec \FF} \ol S$.
  Moreover, $u$ gives a map
  \[
    \EE \times_{\Spec \FF} \ol S
    \xrightarrow{u \times _{\Spec \FF} \ol S}
    \EE \times_{\Spec \FF} \ol S.
  \]
  Then we require the following diagram to commute:
  \begin{center}
  \begin{tikzcd}
    Y \times_S \ol S \ar[rr, "\varphi \times_S \ol S"] \ar[d, "\rho"] && Y' \times_S \ol S\ar[d, "\rho'"] \\
    \EE \times_{\Spec \FF} \ol S \ar[rr, "u \times_{\Spec \FF} \ol S"] && \EE \times_{\Spec \FF} \ol S.
  \end{tikzcd}
  \end{center}
\end{definition}

\ifthesis
We propose that the Serre tensor construction identifies
the space $\TTr$ we previously described with the following $\ZO4$ divisor:
\begin{conjecture}
  [$\TTr \simeq \ZO4(\varpi^r)$]
  \label{hypo:serre_pullback_space}
  The Serre tensor construction gives an isomorphism
  \[ \TTr \simeq \ZO4(\varpi^r) \]
  such that we get an analogous diagram
  \begin{center}
  \begin{tikzcd}
    & \ar[ld] \ZO4(\varpi^r) \ar[rd] & \\
    \MM_{2} && \MM_{2}.
  \end{tikzcd}
  \end{center}
\end{conjecture}
We assume this conjecture henceforth.
(In fact, for $n = 2$ we could even go as far as to take this as a definition of $\TTr$,
but then the generalization \Cref{conj:semi_lie_spherical} would be less obvious,
since a definition like this would not easily extend to $n > 2$.)
\fi
\ifpaper
We need the following additional hypothesis:
\begin{hypothesis}
  [$\TTr \simeq \ZO4(\varpi^r)$]
  \label{hypo:serre_pullback_space}
  The Serre tensor construction gives an isomorphism
  \[ \TTr \simeq \ZO4(\varpi^r) \]
  such that we get an analogous diagram
  \begin{center}
  \begin{tikzcd}
    & \ar[ld] \ZO4(\varpi^r) \ar[rd] & \\
    \MM_{2} && \MM_{2}.
  \end{tikzcd}
  \end{center}
\end{hypothesis}
This hypothesis is shown in the in-preparation
\cite{ref:CLZ} by Ryan C.~Chen, Weixiao Lu, and Wei Zhang.
Even without this hypothesis,
for $n = 2$ we could even go as far as to take this as a definition of $\TTr$,
but then the generalization \Cref{conj:semi_lie_spherical} would be less obvious,
since a definition like this would not easily extend to $n > 2$.
\fi

\subsubsection{The divisor $\ZO3$ on $\MM_2$}
Turning to $\MM_2 \times \MM_2$, we will henceforth always identify $\MM_2$
with its image under the diagonal map
\begin{center}
\begin{tikzcd}
  \ZO4(1) \ar[r, "\sim"] & \MM_2 \ar[r, hook, "\Delta_{\MM_2}"] & \MM_2 \times \MM_2.
\end{tikzcd}
\end{center}

\begin{definition}
  [$\ZO3(u)$]
  Suppose now $u \in \DT$.
  Then we define the divisor $\ZO3(u)$ to be those $X \in \MM_2$
  for which we have a diagram
  \begin{center}
  \begin{tikzcd}
    X \ar[r, "\varphi"] \ar[d, dash] & X \ar[d, dash] \\
    \EE \ar[r, "u"] & \EE.
  \end{tikzcd}
  \end{center}
  Note that basically by definition, for $u \in \OO_\DD$ and $\tr u = 0$ we have
  \[ \ZO3(u) \simeq \ZO4(u) \cap \ZO4(1) \]
  when we identify $\MM_2$ with its image in $\MM_2 \times \MM_2$.
\end{definition}

\subsection{Comparison of the unitary and orthogonal special divisors}
We now relate $\ZD(u)$ to $\ZO3(u)$ through our
isomorphism $\RZ_{2,2} \xrightarrow{\sim} \MM_2 \times \MM_2$.
Recall that we have the notation
\[ \ZD(u)^\circ \coloneqq \ZD(u) - \ZD\left( \frac{u}{\varpi} \right). \]
Define $\ZO4(u)^\circ$ and $\ZO3(u)^\circ$ similarly.

\begin{lemma}
  [$\ZO3(\bar u \sqrt{\eps} u)^\circ$, {\cite{ref:CLZ}}]
  \label{lem:serre_pullback_divisor}
  Let $u \in \VV_n^-$, and consider it as an element $u \in \DD$.
  Then pullback along the Serre tensor construction identifies
  \[ \Serre^\ast \ZD(u)^\circ \simeq \ZO3(\bar u \sqrt{\eps} u)^\circ. \]
\end{lemma}
\begin{proof}
  This is shown in the in-preparation
  \cite{ref:CLZ} by Ryan C.~Chen, Weixiao Lu, and Wei Zhang.
  (Although \cite{ref:CLZ} is technically written for $F = \QQ_p$,
  that restriction is for other parts of the paper that don't affect this lemma.)
\end{proof}

\subsection{The intersection number as a triple product}
We return to the intersection number
\[ \Int((g,u), \fr) = \chi_{\RZ_{n,n}} \left(
      \Sheaf_{\TT_{\mathbf{1} \otimes \fr}(\Delta_{\ZD(u)})}
      \jiao_{\Sheaf_{\RZ_{n,n}}} \Sheaf_{\Gamma_g} \right) \]
which we will rewrite more succinctly using angle brackets as
\[ \Int((g,u), \fr) = \Big\langle
  \mathbb{T}_{\mathbf{1}_K \otimes \mathbf{1}_{K, \le r}}
  \Delta_{\ZD(u)}, \Gamma_g \Big\rangle_{\RZ_{2,2}} \]
in analogy to \cite[\S6.1]{ref:AFLspherical}.
(Note that $\Delta$ here is the diagonal map $\RZ_2 \to \RZ_{2,2}$.)
For this calculation, it would be sufficient to split
\[ \ZD(u) = \sum_{i \ge 0} \ZD(u/\varpi^i)^\circ. \]
Accordingly, let us introduce the notation
\begin{align*}
  \Int^\circ((g,u), \fr)
  &\coloneqq \Int((g,u), \fr) - \Int\left( \left( g, \frac{u}{\varpi} \right), \fr \right) \\
  &= \Big\langle \mathbb{T}_{\mathbf{1}_K \otimes \mathbf{1}_{K, \le r}}
    \Delta_{\ZD(u)^\circ}, \Gamma_g \Big\rangle_{\RZ_{2,2}}.
\end{align*}
From \Cref{lem:serre_pullback_divisor} we get
\begin{align*}
  \Int^\circ((g,u), \fr)
  &= \left< \ZO4(g \cdot \varpi^r), \; \Delta(\Serre^\ast(\ZD(u))^\circ) \right>_{\MM_2 \times \MM_2} \\
  &= \left< \ZO4(g \cdot \varpi^r), \; \ZO3(\bar u \sqrt{\eps} u)^\circ \right>_{\MM_2 \times \MM_2} \\
  &= \left< \ZO4(g \cdot \varpi^r), \; \ZO4(1)^\circ, \; \ZO4(\bar u \sqrt{\eps} u)^\circ \right>_{\MM_2 \times \MM_2} \\
  &= \left< \ZO4(g \cdot \varpi^r), \; \ZO4(1), \; \ZO4(\bar u \sqrt{\eps} u) \right>_{\MM_2 \times \MM_2} \\
  &\qquad- \left< \ZO4(g \cdot \varpi^r), \; \ZO4(1), \; \ZO4\left( \frac{\bar u \sqrt{\eps} u}{\varpi} \right) \right>_{\MM_2 \times \MM_2}.
\end{align*}
In that case we have
\begin{equation}
  \Int((g,u), \fr)
  = \sum_{i \ge 0} \Int^\circ\left( \left( g, \frac{u}{\varpi^i} \right), \fr \right).
  \label{eq:int_to_int_circ}
\end{equation}

\subsection{The formula of Gross-Keating}
\label{sec:GK}
In what follows, we let
\[ \left\langle x, y \right\rangle _0
  = \frac{\left\langle x,y \right\rangle + \ol{\left\langle x,y \right\rangle}}{2} \]
denote half the $E/F$-trace of $\left\langle x,y \right\rangle \in E$.
Let $\ODT \coloneqq \OO_\DD \cap \DT$.

\begin{proposition}
  [Gross-Keating]
  \label{prop:GK}
  Let $x,y \in \ODT$ and let
  \begin{align*}
    n_1 &= \min\left( v(\left\langle x,x \right\rangle _0), v(\left\langle x,y \right\rangle _0), v(\left\langle y,y \right\rangle _0) \right) \\
    n_2 &= v\left( \left\langle x,x \right\rangle _0 \left\langle y,y \right\rangle _0 - \left\langle x,y \right\rangle^2_0 \right) - n_1
  \end{align*}
  so that $0 \le n_1 \le n_2$.
  Then if $n_1$ is odd, we have
  \[
    \left< \ZO4(1), \; \ZO4(x), \; \ZO4(y) \right>_{\MM_2 \times \MM_2}
    = \sum_{j=0}^{\frac{n_1-1}{2}} (n_1+n_2-4j) q^j
  \]
  while if $n_1$ is even we instead have
  \[
    \left< \ZO4(1), \; \ZO4(x), \; \ZO4(y) \right>_{\MM_2 \times \MM_2}
    = \frac{n_2-n_1+1}{2} q^{n_1/2} + \sum_{j=0}^{n_1/2-1} (n_1+n_2-4j) q^j.
  \]
\end{proposition}
\begin{proof}
  This is a rewriting of \cite[Proposition 14.6]{ref:Kudla1997}
  which is itself a special case of \cite[Proposition 5.4]{ref:GK}.
\end{proof}

We now compute all the quantities needed to invoke the Gross-Keating formula.
We start by writing
\begin{align*}
  \bar u \sqrt{\eps} u
  &= (\ol s - t \Pi) \sqrt{\eps} (s + t \Pi) \\
  &= (\ol s - t \Pi) (s \sqrt{\eps} + t \sqrt{\eps} \Pi) \\
  &= s \ol s \sqrt{\eps} - t \ol{s\sqrt{\eps}} \Pi + \ol s t \sqrt{\eps} \Pi - t \ol{t \sqrt{\eps}} \varpi \\
  &= (s \ol s + t \ol t \varpi) \sqrt{\eps} + 2 \ol s t \sqrt{\eps} \Pi.
\end{align*}
We now invoke \Cref{assume:st_zero} to simplify this to just
\[ \bar u \sqrt{\eps} u = (s \ol s + t \ol t \varpi) \sqrt{\eps}. \]
This assumption will also let us write
\[ (s \ol s + t \ol t \varpi)^2 = (s \ol s - t \ol t \varpi)^2 = (\Nm u)^2. \]

Next we consider
\[ g \cdot \varpi^r = \varpi^r(\alpha + \beta \Pi). \]
(Here the action of $g$ is the one on $\MM_2$,
which is why we write $g \cdot \varpi^r$ rather than $g(\varpi^r$).)
From now on, let's write
\[ \alpha = \alpha_0 + \alpha_1 \sqrt{\eps} \]
for $\alpha_0, \alpha_1 \in F$.
Then we use the notation
\begin{align*}
  x &\coloneqq \bar u \sqrt{\eps} u = (s \bar s + t \bar t \varpi) \sqrt{\eps} \in \ODT \\
  y &\coloneqq (g \cdot \varpi^r)_- = \varpi^r(\alpha_1\sqrt{\eps} + \beta \Pi) \in \ODT.
\end{align*}
Then we can compute
\begin{align*}
  \left\langle x,x \right\rangle _0
  &= \Nm x \\
  &= \Norm_{E/F}((s \bar s + t \bar t \varpi)\sqrt{\eps}) \\
  &= -\eps(s \bar s + t \bar t \varpi)^2 = -\eps (\Nm u)^2 \\
  % --------------------------------------------
  \left\langle y,y \right\rangle _0
  &= \Nm \left( \varpi^r(\alpha_1\sqrt{\eps} + \beta \Pi) \right)  \\
  &= \varpi^{2r} (-\alpha_1^2\eps - \beta\bar\beta \varpi) \\
  % --------------------------------------------
  \left\langle x,y \right\rangle _0 &= (\bar x y)_0 \\
  &= \left[
    -(s \bar s + t \bar t \varpi) \sqrt{\eps}
    \cdot
    \varpi^{r} (\alpha_1\sqrt{\eps} + \beta \Pi)
  \right]_0 \\
  &= - \varpi^r \alpha_1 \eps (s \bar s + t \bar t \varpi).
\end{align*}
This lets us compute the determinant
\begin{align*}
  \left\langle x,x \right\rangle _0 \left\langle y,y \right\rangle _0 - \left\langle x,y \right\rangle^2_0
  &= - \eps (\Nm u)^2\cdot \varpi^{2r}
    (-\alpha_1^2 \eps - \beta \bar\beta \varpi)
  - (\varpi^r \alpha_1 \eps (s \bar s + t \bar t \varpi))^2 \\
  &= \eps (\Nm u)^2\cdot \varpi^{2r} \cdot
    (\varpi \beta \bar\beta).
\end{align*}
Hence we arrive at an exact formula for
\[ \left< \ZO4(1), \; \ZO4(x), \; \ZO4(y) \right>_{\MM_2 \times \MM_2} \]
in terms of the valuations of the above formulas,
which we will explicate in the next section after matching $(g,u)$
to the corresponding element in $(S_2(F) \times V_2'(F))\rs$.

\chapter{Proof of \Cref{thm:semi_lie_n_equals_2}}
\label{ch:finale}
We now put together all the results from the previous chapters to
prove \Cref{thm:semi_lie_n_equals_2}.

\section{Matching $\guv$ and $(g,u)$, and the invariants for the matching}
\subsection{The invariants for the orbit of $\guv$}
From \Cref{ch:orbitalFJ0}, recall that we considered
\[
  (\gamma, \uu, \vv^\top)
  =
  \left( \begin{bmatrix} a & b \\ c & d \end{bmatrix}
    \begin{bmatrix} 0 \\ 1 \end{bmatrix}, \begin{bmatrix} 0 & e \end{bmatrix} \right)
  \in (S_2(F) \times V'_2(F))\rs
\]
subject to the conditions
\begin{align*}
  \bar b c = b \bar c &= 1 - a \bar a = 1 - d \bar d \\
  \text{and } d &= - \bar a c / \bar c = -\bar a b / \bar b.
\end{align*}
It will again be enough to consider the situation in which $\det \gamma = 1$.
The invariants in this case as described in \Cref{def:matching_semi_lie} are:
\begin{itemize}
  \ii $\Tr \gamma = a + d$
  \ii $\det \gamma = ad - bc$
  \ii $\vv^\top \uu = e$
  \ii $\vv^\top \gamma \uu = \begin{bmatrix} 0 & e \end{bmatrix}
  \begin{bmatrix} a & b \\ c & d \end{bmatrix} \begin{bmatrix} 0 \\ 1 \end{bmatrix} = de$.
\end{itemize}
Note that the parameters $b$ and $c$ are absent; but we have
\[ v(b) + v(c) = v(\det \gamma - a d). \]

\subsection{Matching}
We now take these results and line them up with \Cref{sec:g_u_invariants}.

\begin{lemma}
  \label{lem:finale_match}
  Suppose
  \[ g = \lambda\inv \begin{bmatrix}
    \alpha & \bar\beta \varpi \\
    \beta &  \bar\alpha
    \end{bmatrix} \]
  and $u = s + t \Pi$ as above with $s t = 0$.
  Suppose $(g,u)$ matches with
  \[ (\gamma, \uu, \vv^\top) = \left( \begin{bmatrix} a & b \\ c & d \end{bmatrix},
    \begin{bmatrix} 0 \\ 1 \end{bmatrix}, \begin{bmatrix} 0 & e \end{bmatrix} \right)
    \in (S_2(F) \times V'_2(F))\rs. \]
  Then we have
  \begin{align*}
    a &= \begin{cases}
      \lambda\inv \bar\alpha & \text{if } s = 0 \\
      \lambda\inv \alpha & \text{if } t = 0 \\
    \end{cases} \\
    d &= \begin{cases}
      \lambda\inv \alpha & \text{if } s = 0 \\
      \lambda\inv \bar\alpha & \text{if } t = 0 \\
    \end{cases} \\
    bc &= \beta \bar\beta \varpi \lambda^{-2} \\
    e &= \Nm u.
  \end{align*}
  Moreover, we also have the identity
  \[
    v(d - a)
    =
  \]
\end{lemma}
\begin{proof}
  Setting the invariants above equal we obtain
  \begin{align*}
    a + d &= \lambda\inv (\alpha + \bar\alpha) \\
    \det \gamma = ad - bc &= \det g \\
    e &= \Nm u \\
    de &= \begin{cases}
      \lambda\inv \bar\alpha \Nm u & \text{if } s = 0 \\
      \lambda\inv \alpha \Nm u & \text{if } t = 0.
    \end{cases}
  \end{align*}
  So the equations for $e$, $d$ and $a$ are immediate.
  In both cases we get $ad = \alpha \bar \alpha \lambda^{-2}$
  and hence
  \[ \beta \bar \beta \varpi \lambda^{-2}
    = -(\det g - \alpha \bar \alpha \lambda^{-2})
    = -(\det \gamma - ad) = bc. \qedhere \]
\end{proof}

\section{Translation of the Gross-Keating data to the orbital side}
We combine the results of \Cref{sec:GK} with \Cref{lem:finale_match}.
Retaining the notation
\begin{align*}
  x &\coloneqq \bar u \sqrt{\eps} u = (s \bar s + t \bar t \varpi) \sqrt{\eps} \in \ODT \\
  y &\coloneqq (g(\varpi^r))_- = (\varsigma\sqrt\eps + \lambda\inv \beta \Pi) \in \ODT
\end{align*}
from \Cref{sec:GK}, we obtain the following:
\begin{align*}
  v(\left\langle x,x \right\rangle _0)
    &= 2 v(\Nm u) \\
    &= 2v(e) \\
  v\left( \left\langle x,y \right\rangle _0 \right)
    &= r + v(\varsigma) + v(\Nm u) \\
    &= r + v(d-a) + v(e) \\
  v(
    \left\langle x,x \right\rangle _0 \left\langle y,y \right\rangle _0
    - \left\langle x,y \right\rangle^2_0
  )
    &= 2r + 2 v(\Nm u) + v(\det g - \alpha \bar \alpha \lambda^{-2}) \\
    &= 2r + 2v(e) + v(b) + v(c).
\end{align*}
Notice that the last valuation is odd.
Therefore we can always extract $v(\left\langle y,y \right\rangle _0)$ by writing
\begin{align*}
  v\left(\left\langle x,x \right\rangle _0\right) + v\left(\left\langle y,y \right\rangle _0\right)
  &= \min \left( v\left( \left\langle x,x \right\rangle _0 \left\langle y,y \right\rangle _0 - \left\langle x,y \right\rangle^2_0 \right),
    v(\left\langle x,y \right\rangle^2_0) \right) \\
  \implies  v\left(\left\langle y,y \right\rangle _0\right)
  &= \min(2r + 2v(e) + v(b) + v(c), 2r + 2v(d-a) + 2v(e)) - 2v(e) \\
  &= 2r + \min(v(b) + v(c), 2v(d-a)).
\end{align*}
Hence, we have
\begin{align*}
  &\phantom= \min\left(
    v(\left\langle x,x \right\rangle _0),
    v(\left\langle x,y \right\rangle _0),
    v(\left\langle y,y \right\rangle _0),
  \right) \\
  &= \min\left( 2v(e), r+v(d-a)+v(e), 2r+\min(v(b)+v(c), 2v(d-a)) \right) \\
  &= \min\left( 2v(e), r+v(d-a)+v(e), 2r+v(b)+v(c), 2r+2v(d-a) \right) \\
  &= \min\left( 2v(e), v(b)+v(c)+2r, 2v(d-a)+2r \right)
\end{align*}
where we can drop $r+v(d-a)+v(e)$ from the minimum because it equals
$\frac{2v(e) + (2v(d-a)+2r)}{2}$.

Now recall the right-hand side of \Cref{prop:GK}, that is
\[
  \begin{cases}
    \sum_{j=0}^{\frac{n_1-1}{2}} (n_1+n_2-4j) \cdot q^j & \text{if } n_1 \equiv 1 \pmod 2 \\
    \frac{n_2-n_1+1}{2} q^{n_1/2} + \sum_{j=0}^{n_1/2-1} (n_1+n_2-4j) \cdot q^j & \text{if } n_1 \equiv 0 \pmod 2.
  \end{cases}
\]
for $0 \le n_1 \le n_2$.
Then apply \Cref{prop:GK} to obtain that
\[
  \left< \ZO4(1), \;
    \ZO4(\bar u \sqrt{\eps} u), \;
    \ZO4((g(\varpi^r))_-) \right>_{\MM_2 \times \MM_2}
\]
is equal to the above formula applied at
\begin{align*}
  n_1 &\coloneqq \min(2v(e), v(b)+v(c)+2r, 2v(d-a)+2r) \\
  n_2 &\coloneqq 2v(e) + v(b) + v(c) + 2r - n_1.
\end{align*}
Note that
\[ n_1 + n_2 = 2v(e) + v(b) + v(c) + 2r. \]
For brevity, we henceforth introduce the symbol $\GK$ for the sum above;
in other words, \Cref{conj:serre_pullback_space} and \Cref{conj:serre_pullback_divisor}
together amount to asserting that
\[ \left< \ZO4(1), \; \ZO4(\bar u \sqrt{\eps} u), \; \ZO4((g(\varpi^r))_-) \right>_{\MM_2 \times \MM_2}
    = \GK(r, v(b), v(c), v(e), v(d-a)). \]
In that case we also have
\[
  \left< \ZO4(1), \;
    \ZO4\left(\frac{\bar u \sqrt{\eps} u}{\varpi}\right), \;
    \ZO4((g(\varpi^r))_-) \right>_{\MM_2 \times \MM_2}
    = \GK(r, v(b), v(c), v(e)-1, v(d-a)). \]
Subtracting the two gives
\begin{equation}
  \begin{aligned}
    \Int^\circ((g,u), \fr) &=
      \left< \ZO4(1), \;
        \ZO4(\bar u \sqrt{\eps} u)^\circ, \;
        \ZO4((g(\varpi^r))_-) \right>_{\MM_2 \times \MM_2} \\
    &= \GK(r, v(b), v(c), v(e), v(d-a)) \\
    &\qquad- \GK(r, v(b), v(c), v(e)-1, v(d-a)).
  \end{aligned}
  \label{eq:int_subtract}
\end{equation}

\section{Base change}
\label{sec:finale_base_change}
The base change for $n=2$ was already calculated in \cite{ref:AFLspherical}
and we simply recall the result here.

As in \Cref{ch:jiao}, we define
\begin{align*}
  \fr &\coloneqq \mathbf{1}_{\varpi^{-r} \Mat_2(\OO_E) \cap \U(\VV_n^+)} \in \HH(\U(\VV_n^+)) \\
  \mathbf{1}_{K, r} &\coloneqq \fr - \mathbf{1}_{K, \le (r-1)} \\
  &= \mathbf{1}_{\varpi^{-r} \Mat_2(\OO_E) \cap \U(\VV_n^+)} \in \HH(\U(\VV_n^+)) \\
\end{align*}

\begin{lemma}
  [{\cite[Lemma 7.1.1]{ref:AFLspherical}}]
  \label{lem:finale_base_change}
  For $n = 2$ and $r \ge 1$ we have
  \begin{align*}
    \BC_S^{\eta}(
      \mathbf{1}_{K'_{S, \le r}}
      + \mathbf{1}_{K'_{S, \le (r-1)}}
      )
      &= (-1)^r \mathbf{1}_{K, r} \\
      &= (-1)^r(\mathbf{1}_{K, \le r} - \mathbf{1}_{K, \le (r-1)}).
  \end{align*}
\end{lemma}
\begin{proof}
  This follows directly from \cite[equation (7.1.9)]{ref:AFLspherical}.
\end{proof}

\section{Transfer factor}
For $n = 2$, the transfer factor is
\begin{align*}
  \omega\guv
  &= \eta\left( \det
    \left[ \gamma^0 \uu, \gamma^1 \uu \right]
  \right) \\
  &= \eta\left( \det
    \begin{bmatrix} 0 & 1 \\ b & c \end{bmatrix}
  \right) \\
  &= (-1)^{v(b)} = - (-1)^{v(c)}.
\end{align*}

\section{Comparison to orbital formula}
In what follows, we introduce the shorthand
\[ \partial\Orb(r, v(b), v(c), v(e), v(d-a))
  \coloneqq
  \left. \pdv{}{s} \right\rvert_{s=0}
  \Orb(\guv, \mathbf{1}_{K'_{S, \le r}} \otimes \oneV, s). \]
The main claim is that the following formula holds:
\begin{theorem}
  \label{thm:miracle}
  We have
  \begin{align*}
    &\frac{(-1)^{r+v(c)}}{\log q}\Big(\partial\Orb(r, v(b), v(c), v(e), v(d-a))
      + \partial\Orb(r, v(b), v(c), v(e)-1, v(d-a))\Big) \\
    &= \GK(r, v(b), v(c), v(e), v(d-a)).
  \end{align*}
\end{theorem}

We continue to use the notation $N$ and $\varkappa$ from \Cref{ch:orbitalFJ2} defined by
\begin{align*}
  N &\coloneqq \min \left( v(e),
    \tfrac{v(b)+v(c)-1}{2} + r, v(d-a) + r \right) \\
  \varkappa &\coloneqq v(e) - v(d-a) - r \ge 0
\end{align*}
and prove \Cref{thm:miracle} by exhausting the
cases based on which value of $N$ is smallest.

\subsection{Proof of \Cref{thm:miracle}
  when $v(e) \le \tfrac{v(b)+v(c)-1}{2} + r$
  and $v(e) \le v(d-a) + r$}
In the Gross-Keating formula we have simply
\begin{align*}
  n_1 &= 2v(e) \\
  n_2 &= v(b) + v(c) + 2r.
\end{align*}
Hence, we have
\begin{align*}
  \GK(r, v(b), v(c), v(e), v(d-a))
  &= \frac{-2v(e)+v(b)+v(c)+2r+1}{2} q^{v(e)} \\
  &\qquad+ \sum_{j=0}^{v(e)-1} (2v(e)+v(b)+v(c)+2r-4j) \cdot q^j.
\end{align*}
On the orbital side, we refer to \Cref{thm:semi_lie_derivative_single}
and compare it to the single instance of Gross-Keating above.
The exponent of $j$ runs up to $v(e)$ in one case and $v(e)-1$ in the second;
that is we need
\begin{align*}
  &\phantom= \sum_{j=0}^{v(e)} \left( \frac{2v(e)+v(b)+v(c)+1}{2} + r - 2j \right) \cdot q^j \\
  &\qquad+ \sum_{j=0}^{v(e)-1} \left( \frac{2(v(e)-1)+v(b)+v(c)+1}{2} + r - 2j \right) \cdot q^j \\
  &= \frac{-2v(e)+v(b)+v(c)+2r+1}{2} q^{v(e)}
  + \sum_{j=0}^{v(e)-1} (2v(e)+v(b)+v(c)+2r-4j) \cdot q^j
\end{align*}
which is obvious.

\subsection{Proof of \Cref{thm:miracle}
  when $\tfrac{v(b)+v(c)-1}{2} + r < v(e)$
  and $v(b)+v(c) < 2v(d-a)$}
\subsubsection{Case where $r > 0$}
Set $N = \frac{v(b)+v(c)-1}{2} + r$.
In the Gross-Keating formula we have simply
\begin{align*}
  n_1 &= 2N+1 \\
  n_2 &= 2v(e).
\end{align*}
Hence
\[ \GK(r, v(b), v(c), v(e), v(d-a))
  = \sum_{j=0}^N (2v(e)+v(b)+v(c)+2r-4j) \cdot q^j. \]
We compare this to \Cref{thm:semi_lie_derivative_single}; we check
\begin{align*}
  &\phantom= \sum_{j=0}^{N} \left( \frac{2v(e)+v(b)+v(c)+1}{2} + r - 2j \right) \cdot q^j \\
  &\qquad+ \sum_{j=0}^{N} \left( \frac{2(v(e)-1)+v(b)+v(c)+1}{2} + r - 2j \right) \cdot q^j \\
  &= \sum_{j=0}^N (2v(e)+v(b)+v(c)+2r-4j) \cdot q^j
\end{align*}
which is clear.

\subsection{Proof of \Cref{thm:miracle}
  when $v(d-a)+r < v(e)$ and $2v(d-a) < v(b)+v(c)$}
Hence $N = v(d-a) + r$ and $\varkappa \coloneqq v(e) - (v(d-a)+r) > 0$.
In the Gross-Keating side formula, we now have
\begin{align*}
  n_1 &\coloneqq 2v(d-a) + 2r = 2N \\
  n_2 &= 2v(e) + v(b) + v(c) - 2v(d-a).
\end{align*}
Hence
\begin{align*}
  \GK(r, v(b), v(c), v(e), v(d-a))
  &= \frac{2v(e)+v(b)+v(c)-4v(d-a)-2r+1}{2} q^{N} \\
    &\qquad+ \sum_{j=0}^{N-1} (2v(e)+v(b)+v(c)+2r-4j) \cdot q^j.
\end{align*}
This time, the relevant combination of \Cref{thm:semi_lie_derivative_single} is
\begin{align*}
  &\phantom= \sum_{j=0}^N q^j
  \cdot \left( \frac{2v(e)+v(b)+v(c)+1}{2} + r - 2j \right) \\
  &\qquad + q^{N} \cdot
  \begin{cases}
    -\frac{\varkappa}{2} & \text{if }\varkappa \equiv 0 \pmod 2 \\
    \frac{\varkappa}{2} - \left( v(e)+\frac{v(b)+v(c)}{2}-2v(d-a)-r \right)
    & \text{if }\varkappa \equiv 1 \pmod 2
  \end{cases} \\
  &\qquad + \sum_{j=0}^N q^j
  \cdot \left( \frac{2(v(e)-1)+v(b)+v(c)+1}{2} + r - 2j \right) \\
  &\qquad + q^{N} \cdot
  \begin{cases}
    -\frac{\varkappa-1}{2} & \text{if }\varkappa-1 \equiv 0 \pmod 2 \\
    \frac{\varkappa-1}{2} - \left( (v(e)-1)+\frac{v(b)+v(c)}{2}-2v(d-a)-r \right)
    & \text{if }\varkappa-1 \equiv 1 \pmod 2 \\
  \end{cases} \\
  &= \sum_{j=0}^N q^j
  \cdot \left( 2v(e)+v(b)+v(c) + 2r - 4j \right) \cdot q^j \\
  &\qquad + q^{N} \cdot
  \begin{cases}
    -\frac{\varkappa}{2} + \frac{\varkappa-1}{2} - \left( (v(e)-1) +\frac{v(b)+v(c)}{2}-2v(d-a)-r \right)
    & \text{if }\varkappa \equiv 0 \pmod 2 \\
    -\frac{\varkappa-1}{2} + \frac{\varkappa}{2} - \left( v(e)+\frac{v(b)+v(c)}{2}-2v(d-a)-r \right)
    & \text{if }\varkappa \equiv 1 \pmod 2 \\
  \end{cases} \\
  &= \sum_{j=0}^N q^j
  \cdot \left( 2v(e)+v(b)+v(c) + 2r - 4j \right) \cdot q^j \\
  &\qquad + q^{N} \cdot
    \left( \half - \left( v(e)+\frac{v(b)+v(c)}{2}-2v(d-a)-r \right) \right) \\
  &= \sum_{j=0}^{N-1} q^j
  \cdot \left( 2v(e)+v(b)+v(c) + 2r - 4j \right) \cdot q^j \\
  &\qquad + q^{N} \cdot
    \left( \left( 2v(e)+v(b)+v(c) + 2r - 4N \right) +
    \half - \left( v(e)+\frac{v(b)+v(c)}{2}-2v(d-a)-r \right) \right).
\end{align*}
The coefficient of $q^N$ is given by
\begin{align*}
  &\phantom= \left( 2v(e)+v(b)+v(c) + 2r - 4(v(d-a)-r) \right) +
  \half - \left( v(e)+\frac{v(b)+v(c)}{2}-2v(d-a)-r \right) \\
  &= \frac{2v(e)+v(b)+v(c)-4v(d-a)-2r+1}{2}
\end{align*}
which matches the one from Gross-Keating.
Hence \Cref{thm:miracle} is completely proved.

\section{Conclusion (proof of \Cref{thm:semi_lie_n_equals_2})}
From \Cref{thm:miracle} we have
\begin{align*}
  \GK(r, v(b), v(c), v(e), v(d-a))
  &= \frac{(-1)^{r+v(c)}}{\log q} \Big(
      \partial \Orb(r, v(b), v(c), v(e), v(d-a)) \\
      &\qquad + \partial \Orb(r, v(b), v(c), v(e)-1, v(d-a))
    \Big) \\
  \GK(r, v(b), v(c), v(e)-1, v(d-a))
  &= \frac{(-1)^{r+v(c)}}{\log q} \Big(
      \partial \Orb(r, v(b), v(c), v(e)-1, v(d-a)) \\
      &\qquad + \partial \Orb(r, v(b), v(c), v(e)-2, v(d-a))
    \Big)
\end{align*}
so subtraction (and recalling \eqref{eq:int_subtract}) gives
\begin{equation}
  \begin{aligned}
    \Int^\circ((g,u), \fr)
    &= \GK(r, v(b), v(c), v(e), v(d-a)) \\
    &\qquad- \GK(r, v(b), v(c), v(e)-1, v(d-a)) \\
    &= \frac{(-1)^{r+v(c)}}{\log q} \Big(
        \partial \Orb(r, v(b), v(c), v(e), v(d-a)) \\
        &\qquad- \partial \Orb(r, v(b), v(c), v(e)-2, v(d-a))
      \Big).
  \end{aligned}
  \label{eq:descent_by_two}
\end{equation}
We now show that \eqref{eq:descent_by_two} implies \Cref{thm:semi_lie_n_equals_2}.
Because $r = 0$ is known already, it suffices to verify for $r > 0$.

Suppose we sum \eqref{eq:descent_by_two} with $u$ replaced by
$u/\varpi^i$ for $i = 0, 1, \dots$.
The left-hand side equals $\Int((g,u), \fr)$ by \eqref{eq:int_to_int_circ}.
On the right-hand side this has the effect of decreasing $v(e)$ by $2$ since $e = \Nm u$.
Hence the sum of the right-hand sides telescopes and gives us the identity
\begin{equation}
  \Int((g,u), \mathbf{1}_{K, \le r})
  = \frac{(-1)^{v(c)+r}}{\log q}\left. \pdv{}{s} \right\rvert_{s=0}
    \Orb(\guv, \mathbf{1}_{K'_{S, \le r}} \otimes \oneV, s).
  \label{eq:match_penultimate}
\end{equation}
Subtracting the same equation from itself with $r$ replaced by $r-1$ gives
\begin{align*}
  &\Int((g,u), \mathbf{1}_{K, \le r} - \mathbf{1}_{K, \le (r-1)}) \\
  &= \frac{(-1)^{v(c)+r}}{\log q}\left. \pdv{}{s} \right\rvert_{s=0}
  \Orb(\guv, (\mathbf{1}_{K'_{S, \le r}} + \mathbf{1}_{K'_{S, \le (r-1)}}) \otimes \oneV, s).
\end{align*}
which, since $(-1)^{v(c)} = -\omega\guv$, becomes
\begin{equation}
  \begin{aligned}
    &\Int((g,u), (-1)^r(\mathbf{1}_{K, \le r} - \mathbf{1}_{K, \le (r-1)})) \\
    &= \frac{-\omega\guv}{\log q}\left. \pdv{}{s} \right\rvert_{s=0}
    \Orb(\guv, (\mathbf{1}_{K'_{S, \le r}} + \mathbf{1}_{K'_{S, \le (r-1)}}) \otimes \oneV, s).
  \end{aligned}
  \label{eq:match_final}
\end{equation}
But \Cref{lem:finale_base_change} says that
$(-1)^r(\mathbf{1}_{K, \le r} - \mathbf{1}_{K, \le (r-1)}) \in \HH(\U(\VV_n^+))$
matches $\mathbf{1}_{K'_{S, \le r}} + \mathbf{1}_{K'_{S, \le (r-1)}} \in \HH(S_2(F))$
for any $r \ge 0$.
And hence from the $r = 0$ case we can inductively conclude
\Cref{thm:semi_lie_n_equals_2} for $r > 0$, completing the proof.

\section{A particularly simple formula for a certain intersection number}
We mention in particular that the value of
\[ \Int^\circ((g,u), 1_{K,r})
  = \Big\langle \mathbb{T}_{\mathbf{1}_K \otimes \mathbf{1}_{K, r}}
    \Delta_{\ZD(u)^\circ}, \Gamma_g \Big\rangle_{\RZ_{2,2}} \]
(note the change from $\fr$ to $\mathbf{1}_{K, \le r}$ here)
has a particularly simple formula that seems worth mentioning.
We phrase this entirely based on the quantities in the geometric side
to keep in self-contained.

\begin{theorem}
  Let $r \ge 1$ and $v(\Nm u) > 0$ for $u \in \VV_2^-$, and let
  \[ g =
    \begin{bmatrix} \alpha & \beta \varpi \\ \bar\beta & \bar \alpha \end{bmatrix}
    \lambda^{-1} \in \U(\VV_2^-). \]
  Then
  \[ \Big\langle \mathbb{T}_{\mathbf{1}_K \otimes \mathbf{1}_{K, r}}
    \Delta_{\ZD(u)^\circ}, \Gamma_g \Big\rangle_{\RZ_{2,2}} \]
  is equal to
  \[
    \begin{cases}
      (C+1) q^{N} + (C+2) q^{N-1}
        & \text{if } v(\Nm u)-r = v(\alpha - \bar\alpha) - v(\lambda) \le v(\beta) - v(\lambda) \\
      2q^N & \text{if } v(\beta) - v(\lambda) + r < \min(v(\Nm u), v(\alpha - \bar\alpha) - v(\lambda) + r) \\
      q^N + q^{N-1} & \text{otherwise}
    \end{cases}
  \]
  where
  \[ N = \min(v(\Nm u), v(\beta) - v(\lambda) + r, v(\alpha-\bar\alpha) - v(\lambda) + r) \]
  we write
  \[ C = v(\beta) - v(\alpha - \bar\alpha) \ge 0 \]
  in the first case.
\end{theorem}

\begin{proof}
  Recall that
  \begin{align*}
    \GK(r, v(b), v(c), v(e), v(d-a))
    &= \frac{(-1)^{r+v(c)}}{\log q} \Big(
      \partial \Orb(r, v(b), v(c), v(e), v(d-a)) \\
      &\qquad+ \partial \Orb(r, v(b), v(c), v(e)-1, v(d-a))
      \Big) \\
    \GK(r-1, v(b), v(c), v(e), v(d-a))
    &= \frac{(-1)^{r-1+v(c)}}{\log q} \Big(
      \partial \Orb(r-1, v(b), v(c), v(e), v(d-a)) \\
      &\qquad- \partial \Orb(r-1, v(b), v(c), v(e)-1, v(d-a))
      \Big)
  \end{align*}
  when we subtract we obtain that
  \begin{align*}
    \Int^\circ((g,u), \mathbf{1}_{K, r})
    &= \frac{(-1)^{r+v(c)}}{\log q} \Big(
      \partial \Orb(r, v(b), v(c), v(e), v(d-a)) \\
      &\qquad+ \partial \Orb(r-1, v(b), v(c), v(e), v(d-a)) \\
      &\qquad- \partial \Orb(r, v(b), v(c), v(e)-2, v(d-a)) \\
      &\qquad+ \partial \Orb(r-1, v(b), v(c), v(e)-2, v(d-a))
    \Big).
  \end{align*}
  Gathering the first two terms lets us apply the simpler formula \Cref{cor:semi_lie_combo} twice;
  doing so gives
  \begin{align*}
    \Int^\circ((g,u), \mathbf{1}_{K, r})
    &= \left( (q^N + q^{N-1} + \dots + 1) + C q^N + C' q^{N-1} \right) \\
    &- \left( (q^{N^\flat} + q^{N^\flat-1} + \dots + 1) + C^\flat q^{N^\flat} + (C')^\flat q^{N^\flat-1} \right)
  \end{align*}
  where $N$, $C$, $C'$ are is in \Cref{cor:semi_lie_combo},
  and $N^\flat$, $C^\flat$, $(C')^\flat$ are the same quantities
  with $v(e)$ replaced by $v(e)-2$.
  Let $\varkappa$ and $\varkappa^\flat = \varkappa - 2$ be also as in \Cref{cor:semi_lie_combo}.

  We consider cases now.
  \begin{itemize}
    \ii Suppose first $v(e) \le \frac{v(b)+v(c)-1}{2}+r$ and $v(e) < v(d-a)+r$.
    Then $N = v(e)$ and $N^\flat = v(e) - 2$ and $C = C' = (C^\flat) = (C')^\flat = 0$,
    Hence in this case we have
    \[ \Int^\circ((g,u), \mathbf{1}_{K, r}) = q^{N} + q^{N-1}. \]

    \ii Next suppose $2v(d-a) > v(b) + v(c)$ and consider cases on $v(e)$.
    We only need to consider $v(e) > \frac{v(b)+v(c)-1}{2} + r$.
    \begin{itemize}
      \ii If $v(e) = \frac{v(b)+v(c)-1}{2} + r + 1$
      then we have
      \[ N = \frac{v(b)+v(c)-1}{2} + r, \qquad N^\flat = \frac{v(b)+v(c)-1}{2} + r - 1 \]
      and
      $C = 1$, $C^\flat = 0$, and $C' = (C')^\flat = 0$.
      Consequently we get
      \[ \Int^\circ((g,u), \mathbf{1}_{K, r}) = 2 q^N. \]

      \ii Once $v(e) \ge \frac{v(b)+v(c)-1}{2} + r + 2$
      we always have $N = N^\flat = \frac{v(b)+v(c)-1}{2} + r$,
      \[ C - C^\flat = (v(e)-N)-((v(e)-2)-N) = 2 \]
      and $C' = (C')^\flat = 0$.
      Hence in this case we have
      \[ \Int^\circ((g,u), \mathbf{1}_{K, r}) = 2 q^N \]
      as well.
    \end{itemize}

    \ii Finally suppose $2v(d-a) < v(b) + v(c)$ and consider cases on $v(e)$.
    We only need to consider $v(e) \ge v(d-a) + r + 1$.
    \begin{itemize}
      \ii If $v(e) = v(d-a) + r$,
      then \[ N = v(d-a) + r, \qquad N^\flat = v(d-a) + r - 2. \]
      In this case $\varkappa = 0$ (and $\varkappa^\flat=-2$).
      So $C^\flat = (C')^\flat = 0$ but we have larger terms
      \begin{align*}
        C &= \frac{v(b) + v(c)- 2v(d-a) - 1}{2} \\
        C' &= \frac{v(b) + v(c)- 2v(d-a) + 1}{2}.
      \end{align*}
      Hence, we get an exception case
      \begin{align*}
        \Int^\circ((g,u), \mathbf{1}_{K, r})
        &= \frac{v(b) + v(c) - 2v(d-a) + 1}{2} q^{N} \\
        &\qquad+ \frac{v(b) + v(c) - 2v(d-a) + 3}{2} q^{N-1}
      \end{align*}

      \ii If $v(e) = v(d-a) + r + 1$,
      then we have
      \[ N = v(d-a) + r, \qquad N^\flat = v(d-a) + r - 1. \]
      In this case $\varkappa = 1$ (and $\varkappa^\flat=-1$) so we have
      $C = 0$, $C' = 1$, $C^\flat = 0 = (C')^\flat = 0$.
      Consequently we get
      \[ \Int^\circ((g,u), \mathbf{1}_{K, r}) = q^{N} + q^{N-1}. \]

      \ii Once $v(e) \ge v(d-a) + r + 2$,
      we always have $N = N^\flat = v(d-a) + r$ and
      \[ C - C^\flat = (C') - (C'^\flat) = 1 \]
      regardless of the parity of $\varkappa$.
      Hence in this case we also get
      \[ \Int^\circ((g,u), \mathbf{1}_{K, r}) = q^{N} + q^{N-1}. \]
    \end{itemize}
  \end{itemize}
  Hence, in summary we get that
  \begin{equation}
  \begin{aligned}
    &\Int^\circ((g,u), \mathbf{1}_{K, r}) \\
    &= \begin{cases}
      (C+1) q^{N} + (C+2) q^{N-1}
        & \text{if } v(e)-r = v(d-a) \le \frac{v(b)+v(c)-1}{2} \\
      2q^N & \text{if } \frac{v(b)+v(c)-1}{2} + r < \min(v(e), v(d-a)+r) \\
      q^N + q^{N-1} & \text{otherwise}.
    \end{cases}
  \end{aligned}
  \label{eq:int_circle_orbital_param}
  \end{equation}
  where
  \[ C = \frac{v(b) + v(c)- 2v(d-a) - 1}{2} \ge 0 \]
  in the first case.
  Then \eqref{eq:int_circle_orbital_param} translates via
  \Cref{lem:finale_match} into the desired claim.
\end{proof}


\printbibliography[title=References]

\end{document}
