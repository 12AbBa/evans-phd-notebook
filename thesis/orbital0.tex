\chapter{Synopsis of the orbital integral $\Orb(\gamma, \phi, s)$ for $\gamma \in S_3(F)\rs$ and $\phi \in \HH(S_3(F), K')$}
\label{ch:orbital0}

This section defines the orbital integral
and describes the parameters which we will use to express our answer.

\section{Initial definition of the orbital integral for general $S_n(F)$}
Let $H = \GL_{n-1}(F)$.
Then $H$ has a natural embedding into $G = \GL_n(E)$ by
\[ h \mapsto \begin{bmatrix} h & 0 \\ 0 & 1 \end{bmatrix} \]
which endows it with an action $S_n(E)$.
Then our orbital integral is defined as follows.
\begin{definition}
  For brevity let $\eta(h) \coloneqq \eta(\det h)$ for $h \in H$.
  For $\gamma \in S_n(F)$, $\phi \in \HH(S_n(F), K')$, and $s \in \CC$,
  we define the orbital integral by
  \[ \Orb(\gamma, \phi, s) \coloneqq
    \int_{h \in H} \phi(h\inv \gamma h) \eta(h)
    \left\lvert \det(h) \right\rvert_F^{-s} \odif h. \]
  \label{def:orbital0}
\end{definition}

\section{Basis for the indicator functions in $\HH(S_3(F), K')$}
\label{ch:orbital0_hecke_basis}
From now on assume $n = 3$.
We have the symmetric space
\[ S_3(F) \coloneqq \left\{ g \in \GL_3(E) \mid g \bar{g} = \id_3 \right\}. \]
which has a left action under $\GL_3(E)$ by $g \cdot s \mapsto gs\bar{g}\inv$.

Then $S_3(F)$ admits the following decomposition, which we will use:
\begin{lemma}
  [Cartan decomposition of $S_3(F)$]
  For each integer $r \ge 0$ let
  \[ K'_{S,r} \coloneqq \GL_3(\OO_E) \cdot \begin{bmatrix} 0 & 0 & \varpi^r \\ 0 & 1 & 0 \\ \varpi^{-r} & 0 & 0 \end{bmatrix} \]
  denote the orbit of
  $\begin{bmatrix} 0 & 0 & \varpi^r \\ 0 & 1 & 0 \\ \varpi^{-r} & 0 & 0 \end{bmatrix}$
  under the left action of $\GL_3(\OO_E)$.
  Then we have a decomposition
  \[ S_3(F) = \coprod_{r \geq 0} K'_{S,r}. \]
\end{lemma}
The $r=0$ case will be given a special shorthand,
and can be expressed in a few equivalent ways:
\begin{align*}
  K'_S
  &\coloneqq K'_{S,0} \\
  &= \GL_3(\OO_E) \cdot \begin{bmatrix} & & 1 \\ & 1 \\ 1 \end{bmatrix} \\
  &= \GL_3(\OO_E) \cdot \id_3 = S_3(F) \cap \GL_3(\OO_E).
\end{align*}
One can equivalently define $K'_{S,r}$ to be the part of $S_3(F)$
for which the most negative valuation among the nine entries is $-r$.

For $r \geq 0$, define
\[ K'_{S, \le r} \coloneqq S_3(F) \cap \varpi^{-r} \GL_3(\OO_E). \]
We can re-parametrize the problem according to the following.
\begin{proposition}
  \[ K'_{S, \le r} = K'_{S,0} \sqcup K'_{S,1} \sqcup \dots \sqcup K'_{S,r}. \]
\end{proposition}
Then an integral over each $K'_{S, \le r}$ lets us extract the integrals over $K'_{S,r}$.
\begin{proposition}
  For $r \ge 0$, the indicator functions $\mathbf{1}_{K'_{S, \le r}}$
  form a basis of $\HH(S_3(F), K')$.
\end{proposition}
\todo{this is like an $\HH(G', K')$-basis I think? double check this}

Then, our goal is to compute for
\begin{equation}
  \pdv{}{s}\Orb(\gamma, \mathbf{1}_{K'_{S, \le r}}, s)
  \label{eq:orbital_goal}
\end{equation}
at $s=0$ for any $r > 0$ as well.
Note that the $r = 0$ case is already done in \cite{ref:AFL}.
\todo{comment some basis thing}

\section{Parametrization of $\gamma$}
Again, assume $n = 3$.
Further assume $\gamma \in S_3(F)\rs$ is regular semisimple.
We identify some parameters for the orbit of $\gamma$
that we can use for our explicit calculations.

\subsection{Rewriting the orbital integral as a double integral over $E$
  via the group $H' \cong \GL_2(F)$}
Our orbital integral is at present a quadruple integral over $F$,
owing to $H = \GL_{2}(F)$ being a four-dimensional $F$-vector space.

It will be more economical to work with the orbital integral as a double integral
with two coefficients in $E$, in the following sense.
Define
\[ H' \coloneqq
  \left\{ \begin{bmatrix} t_1 & t_2 \\ \bar t_2 & \bar t_1 \end{bmatrix}
    \mid t_1, t_2 \in E \right\}
\]
which is indeed a four-dimensional $F$-algebra.
As before $H' \hookrightarrow G$ according to the same embedding
$\GL_2(E) \hookrightarrow \GL_(3)$
and so $H'$ also acts on $S_n(E)$ by conjugation.

As an $F$-algebra, we have an isomorphism (see \cite[\S4.1]{ref:AFL})
\begin{align*}
  \iota_2 \colon H = \GL_2(F)
  &\xrightarrow{\cong} H' \\
  \begin{bmatrix} a_{11} & a_{12} \\ a_{21} & a_{22} \end{bmatrix}
  &\mapsto \begin{bmatrix} t_1 & t_2 \\ \bar t_2 & \bar t_1 \end{bmatrix} \\
  t_1 &= \half\left( a_{11} + a_{22} + \frac{a_{12}}{\sqrt{\eps}} + a_{21} \sqrt{\eps} \right) \\
  t_2 &= \half\left( a_{11} - a_{22} + \frac{a_{12}}{\sqrt{\eps}} - a_{21} \sqrt{\eps} \right).
\end{align*}
Under this isomorphism, we have
\[ h \gamma h^{-1} = \iota_2(h) \gamma \overline{\iota_2(h)^{-1}}. \]
\todo{maybe i should actually check this to make sure I'm not crazy.}

This allows us to rewrite the orbital integral over $H'$ instead.
If we write $h' = \overline{\iota_2(h)^{-1}}$,
then the following integral formula is obtained.
\begin{proposition}
  [{\cite[\S4.2]{ref:AFL}}]
  \label{prop:orbital_over_H_prime}
  For brevity let $\eta(h') \coloneqq \eta(\det h')$ for $h' \in H'$.
  For $\gamma \in S_3(F)$, $\phi \in \HH(S_3(F), K')$, and $s \in \CC$,
  the orbital integral can instead be written as
  \[ \Orb(\gamma, \phi, s) =
    \int_{h' \in H'} \phi(\bar{h'}\inv \gamma h') \eta(h')
    \left\lvert \det(h') \right\rvert_F^{s} \odif{h'} \]
  where
  \[ \odif{h'} = \kappa \cdot \frac{\odif t_1 \odif t_2}
    {\left\lvert t_1 \bar t_1 - t_2 \bar t_2 \right\rvert_F^2} \]
  for the constant
  \[ \kappa \coloneqq \frac{1}{(1-q\inv)(1-q^{-2})}. \]
\end{proposition}

\subsection{Identifying a representative in the $H'$-orbit}
Evidently the orbital integral $\Orb(\gamma, \phi, s)$ in \Cref{prop:orbital_over_H_prime}
only depends on the $H'$-orbit of $\gamma$.
So it makes sense to pick a canonical representative for the $H'$-orbit to compute
the orbital integral in terms of.

Since we assumed $\gamma \in S_3(F)\rs$ is regular semisimple,
we can invoke \cite[Proposition 4.1]{ref:AFL}
to assume $\gamma$ is a representative specifically of the form
\[ \gamma(a,b,d) =
  \begin{bmatrix}
    a & 0 & 0 \\
    b & - \bar d & 1 \\
    c & 1 - d \bar d & d
  \end{bmatrix}
  \in S_3(F)\rs; \quad \text{where $c = -a \bar b + b d$} \]
over all $a \in E^1$, $b \in E$, $d \in E$ for which $(1-d\bar d)^2 - c \bar c \neq 0$.
In other words, the representatives described here cover all the regular orbits in $S_3(F)\rs$.

\subsection{Simplification due to the matching of non-quasi-split unitary group}
In this calculation, we restrict attention to the case where our regular $\gamma$
matches an element in the non-quasi-split unitary group.
\todo{I need to ask Wei exactly what's up here}
This is controlled by the parity of the invariant
\[ v\left( (1-d\bar d)^2 - c \bar c\right) \]
being odd.
Hence, we only have to consider this case:
\begin{assume}
  We will assume that
  \[ v\left( (1-d\bar d)^2 - c \bar c\right) \equiv 1 \pmod 2. \]
  \label{assume:u_odd}
\end{assume}

We will also see quite early on in our calculation
that the orbital integral vanishes if $v(d) < -r$.
Hence, we will always assume:
\begin{assume}
  $v(d) \geq -r$.
  \label{assume:vd_ge_minus_r}
\end{assume}

We will mostly be interested in the case where $v(b) = v(d) = 0$.
In fact, few other cases even occur at all given \Cref{assume:u_odd};
we will see momentarily that either $v(b) = v(d) \in \{-1, -2, \dots, -r\}$,
or one of $\{v(b), v(d)\}$ is zero and the other is nonnegative.

\section{Quantities to state the answer in terms of}
As we described earlier, our goal is to evaluate \eqref{eq:orbital_goal}
in terms of the parameters
\[ a \in E^1, \qquad b, d \in E, \qquad r \ge 0. \]
To simplify the notation in what follows,
it will be convenient to define several quantities that reappear frequently.
From \Cref{assume:u_odd}, we may define
\begin{equation}
  \delta \coloneqq v(1-d \bar d) = v(c) \neq -\infty.
  \label{eq:delta}
\end{equation}
Following \cite{ref:AFL} we will also define
\begin{equation}
  u \coloneqq \frac{\bar c}{1-d \bar d} \in \OO_E^\times
  \label{eq:u}
\end{equation}
so that $\nu(1-u \bar u) \equiv 1 \pmod 2$ and
\begin{equation}
  b = -au - \bar{d} \bar{u}.
  \label{eq:b}
\end{equation}
Note that this gives us the following repeatedly used identity
\begin{equation}
  b^2-4a\bar d = (au-\bar d \bar u)^2 - 4a\bar d(1-u\bar u).
  \label{eq:dos}
\end{equation}
Finally, define
\begin{equation}
  \ell \coloneqq v(b^2 - 4 a \ol d).
  \label{eq:ell}
\end{equation}
We will also define one additional parameter useful when $\ell$ is even
(but as we will see, redundant for odd $\ell$):
\begin{equation}
  \lambda \coloneqq v(1-u \bar u) \equiv 1 \pmod 2.
  \label{eq:lambda}
\end{equation}

Just as many pairs $(v(b), v(d))$ do not occur (given \Cref{assume:u_odd})
and $v(b) = v(d) = 0$ is the main case of interest,
the parameters $(\delta, \ell, \lambda)$ satisfy some additional relations.
We will now describe them.
\begin{proposition}
  \label{prop:parameter_constraints}
  Exactly one of the following situations is true:
  \begin{enumerate}[(a)]
    \ii $v(b) = v(d) = 0$, $\ell \ge 1$ is odd, $\ell < 2 \delta$, and $\lambda = \ell$.
    \ii $v(b) = v(d) = 0$, $\ell \ge 0$ is even, $\ell \le 2 \delta$, and $\lambda > \ell$ is odd.
    \ii $v(b) = 0$, $v(d) > 0$, $\ell = \delta = 0$, and ???
    \ii $v(b) > 0$, $v(d) = 0$, $\ell = 0$, $\delta \ge 0$, and ???
    \ii $v(b) = v(d) \in \{-1, \dots, -r\}$, $\ell = \delta = 2v(d) < 0$, and ???
  \end{enumerate}
  See \Cref{tab:parameter_constraints}.
  Moreover, whenever $\ell$ is even,
  the quantity $b^2 - 4 a \bar d$ is a square of some element in $E$.
\end{proposition}
\begin{proof}
  First assume $\ell$ is odd. We assert in this case we have
  \begin{equation}
    v(b) = v(d) = 0.
    \label{eq:odd_b_d_zero}
  \end{equation}
  Indeed if $v(d) \neq 0$, then $b = -au-\bar d\bar u$ is a unit,
  and hence so is $b^2 - 4 a \bar d$, causing $\ell = 0$, contradiction.
  And if $d$ is a unit, $\ell \neq 0$ means $v(b) = 0$ too.
  \todo{prove the result of (a)}

  For the rest of the proof we only consider even $\ell$.
  Because $b = -au - \bar d \bar u$, it cannot be the case that $v(b) > 0$ and $v(d) > 0$;
  moreover if either $v(b) < 0$ or $v(d) < 0$,
  then in fact $v(b) = v(d)$.
  \todo{wrap this up}

  Hence

  We now verify the last assertion that $b^2 - 4 a \bar d$
  is a square whenever $\ell$ is even.
  The proof in all cases uses \eqref{eq:dos} to show $b^2 - 4 a \bar d$
  is equal to $\varpi^\ell$ times a quadratic residue in $\OO_E^\times$.
  Indeed we need only verify that $v(4a \bar d(1 - u \bar u)) = v(d) + \lambda$
  has larger valuation than $v\left( (a u - \bar d \bar u)^2 \right) = \ell$.
  \begin{itemize}
    \ii In the case this follows from $\lambda > \ell$;
    \ii In the case where $v(d) > 0$ we have $\ell = 0$ and this is clear;
    \ii If $v(b) = v(d) < 0$ then $\ell = -2v(d)$ and this is clear too.
  \end{itemize}

\end{proof}

\begin{table}[ht]
  TODO: TABLE GOES HERE
  \caption{A table showing the five cases in \Cref{prop:parameter_constraints}.}
  \label{tab:parameter_constraints}
\end{table}
\todo{fill in the table}

In the case where $\ell$ is odd (and hence $\ell \ge 1$ and $v(b) = v(d) = 0$),
we get \eqref{eq:dos} implying $\lambda = \ell$
and thus $\lambda$ will never be used --- the orbital will be computed
as a function of $\ell$ and $\delta$ (and $r$).
However for even $\ell$ these numbers are never equal and our orbital
integral will be stated in terms of $\ell$, $\delta$, and $\lambda$ (and $r$).

\section{Result}
We can now state the answer.
\todo{put the differentiated result here}
