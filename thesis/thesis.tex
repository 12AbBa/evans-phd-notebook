\documentclass[12pt]{amsart}
% Preamble - Evan's thesis version

%%% Load packages
\usepackage{hyperref}
\usepackage[obeyFinal,textsize=scriptsize,shadow,loadshadowlibrary]{todonotes}
\usepackage[shortlabels]{enumitem}
\usepackage[usenames,dvipsnames,svgnames]{xcolor}
\usepackage{amsmath}
\usepackage{amssymb}
\usepackage{amsthm}
\usepackage{booktabs}
\usepackage[nameinlink]{cleveref}
\usepackage{derivative}
\usepackage{graphicx}
\usepackage{mathdots}
\usepackage{mathrsfs}
\usepackage{mathtools}
\usepackage{microtype}

\usepackage{asymptote}
\usepackage{tikz-cd}
\usetikzlibrary{decorations.pathmorphing}

\allowdisplaybreaks

\usepackage[backend=biber,backref=true,style=alphabetic]{biblatex}

\newtheorem{theorem}{Theorem}[section]
\newtheorem{lemma}[theorem]{Lemma}
\newtheorem{proposition}[theorem]{Proposition}
\newtheorem{corollary}[theorem]{Corollary}

\theoremstyle{definition}
\newtheorem{assume}[theorem]{Assumption}
\newtheorem{definition}[theorem]{Definition}
\newtheorem{example}[theorem]{Example}
\newtheorem{ques}[theorem]{Question}
\newtheorem{claim}[theorem]{Claim}
\newtheorem{conjecture}[theorem]{Conjecture}
\newtheorem{remark}[theorem]{Remark}
\newtheorem{question}[theorem]{Question}

%%% Macros
\providecommand{\ol}{\overline}
\providecommand{\ul}{\underline}
\providecommand{\wt}{\widetilde}
\providecommand{\wh}{\widehat}
\providecommand{\eps}{\varepsilon}
\providecommand{\half}{\frac{1}{2}}
\providecommand{\inv}{^{-1}}
\newcommand{\dang}{\measuredangle} %% Directed angle
\providecommand{\CC}{\mathbb C}
\providecommand{\FF}{\mathbb F}
\providecommand{\NN}{\mathbb N}
\providecommand{\QQ}{\mathbb Q}
\providecommand{\RR}{\mathbb R}
\providecommand{\ZZ}{\mathbb Z}
\providecommand{\ts}{\textsuperscript}
\providecommand{\dg}{^\circ}
\providecommand{\ii}{\item}
\newcommand{\surjto}{\twoheadrightarrow}

\DeclareMathOperator*{\Arch}{ARCH}
\DeclareMathOperator{\BC}{BC}
\DeclareMathOperator{\End}{End}
\DeclareMathOperator{\Gal}{Gal}
\DeclareMathOperator{\GL}{GL}
\DeclareMathOperator{\Hom}{Hom}
\DeclareMathOperator{\Int}{Int}
\DeclareMathOperator{\Lie}{Lie}
\DeclareMathOperator{\Mat}{Mat}
\DeclareMathOperator{\Norm}{N}
\DeclareMathOperator{\Orb}{Orb}
\DeclareMathOperator{\SL}{SL}
\DeclareMathOperator{\Sat}{Sat}
\DeclareMathOperator{\Spec}{Spec}
\DeclareMathOperator{\Spf}{Spf}
\DeclareMathOperator{\Sym}{Sym}
\DeclareMathOperator{\U}{U}
\DeclareMathOperator{\Vol}{Vol}
\DeclareMathOperator{\antidiag}{antidiag}
\DeclareMathOperator{\diag}{diag}
\DeclareMathOperator{\id}{id}
\DeclareMathOperator{\rproj}{proj}

\newcommand{\cc}{\mathbf c}
\newcommand{\nn}{\mathbf n}
\newcommand{\uu}{\mathbf u}
\newcommand{\vv}{\mathbf v}
\newcommand{\HH}{\mathcal H}
\newcommand{\EE}{\mathbb E}
\newcommand{\TT}{\mathbb T}
\newcommand{\VV}{\mathbb V}
\newcommand{\XX}{\mathbb X}

\newcommand{\RZ}{\mathcal N} % for RZ space
\newcommand{\ZD}{\mathcal Z} % Cartier divisor

\newcommand{\BG}{\mathbf{G}}
\newcommand{\BK}{\mathbf{K}}
\newcommand{\BM}{\mathbf{M}}
\newcommand{\BN}{\mathbf{N}}
\newcommand{\BP}{\mathbf{P}}
\newcommand{\BS}{\mathbf{S}}

\newcommand{\guv}{{(\gamma, \uu, \vv^\top)}}
\newcommand{\oneV}{\mathbf{1}_{\OO_F^n \times (\OO_F^n)^\vee}}
\newcommand{\rs}{_{\mathrm{rs}}}

\newcommand{\OO}{O} % to avoid confusion with structure sheaf I guess
\newcommand{\SO}{\mathcal{O}} % sheaf O

\newcommand{\jiao}{\mathop{\otimes}^{\mathbf{L}}} % this is a cute macro name (交)
% found in arXiv:2402.1767v1 while trying to search for where \mathbb{L} was defined
% (answer: it's not, this entire glyph is just generally used for derived tensor products)

\addbibresource{refs.bib}

\title[Explicit formulas for orbital integrals for AFL]
  {Explicit formulas for orbital integrals related to the
  inhomogeneous version of the arithmetic fundamental lemma
  for spherical Hecke algebras}

\author{Evan Chen}
\date{\today}
\address{Department of Mathematics, Massachusetts Institute of Technology}
\email{evanchen@alum.mit.edu}

\begin{document}

\maketitle

\begin{abstract}
  As an analog to the Jacquet-Rallis fundamental lemma that appears in the
  relative trace formula approach to the Gan-Gross-Prasad conjectures,
  the arithmetic fundamental lemma was proposed by W.\ Zhang and used in an approach
  to the arithmetic Gan-Gross-Prasad conjectures.
  Both the Jacquet-Rallis fundamental lemma and arithmetic fundamental lemma
  were recently generalized to conjectural statements that hold
  for an arbitrary function in the associated spherical Hecke algebras.
  This paper produces explicit formulas for the orbital integrals that appear
  in certain cases of the arithmetic fundamental lemma for the spherical Hecke algebra,
  thus verifying the conjecture in some particular cases.
\end{abstract}

\tableofcontents
\newpage

\chapter{Introduction}
Throughout this whole paper, $p > 2$ is a prime,
$F$ is a finite extension of $\QQ_p$,
and $E/F$ is an unramified quadratic field extension.

\section{Brief history and motivation for the arithmetic fundamental lemma}
The primary motivation for this paper arises from
the study of conjectured variants of the arithmetic fundamental lemma
for spherical Hecke algebras proposed in \cite{ref:AFLspherical}.
This section briefly provides an overview of the historical context
that led to the formulation of these conjectures.
This history is also summarized in \Cref{fig:history}.

Because this subsection is meant for motivation only, in this survey we do not give
complete definitions or statements, being content to outline a brief gist.
A more detailed account can be found in \cite{ref:survey}.

\begin{figure}[ht]
  \centering
  \begin{tikzcd}
      \text{\cite{ref:waldspurger}} \ar[d] \\
      \text{\cite{ref:GP1,ref:GP2}} \ar[d] \\
    \text{GGP \cite{ref:GGP}}
      \ar[d, dotted, leftrightarrow, "\text{analog}"]
      & \ar[l, Rightarrow, "\text{used to prove}"] \text{FL \cite{ref:JR}}
        \ar[d, dotted, leftrightarrow, "\text{analog}"] \ar[r]
      & \text{\cite{ref:leslie}}
        \ar[d, dotted, leftrightarrow, "\text{analog}"] \\
    \text{Arith.\ GGP \cite{ref:GGP}}
      & \ar[l, Rightarrow, "\text{used to prove}"] \text{AFL \cite{ref:AFL}} \ar[r]
      & \text{\cite{ref:AFLspherical}} \\
    \text{\cite{ref:GZshimura}} \ar[u] \\
    \text{\cite{ref:gross_zagier}} \ar[u]
  \end{tikzcd}
  \caption{The history behind the fundamental lemma and its arithmetic counterpart.
    Unlabeled arrows denote generalizations.}
  \label{fig:history}
\end{figure}

\subsection{The GGP conjectures, and the fundamental lemma of Jacquet-Rallis}
In modern arithmetic geometry, a common theme is that there are deep connections
between geometric data with the values of related $L$-functions.

This story begins with a result of
Waldspurger \cite{ref:waldspurger} which showed a formula
relating the nonvanishing of an automorphic period integral
to the central value of the same $L$-functions.
Later, a conjecture that generalizes Waldspurger's formula
was proposed by Gross-Prasad in \cite{ref:GP1,ref:GP2}.
This was further generalized to a series of conjectures
now known as the Gan-Gross-Prasad (GGP) conjectures,
which were proposed in 2012 in \cite{ref:GGP};
they generalize the Gross-Prasad conjecture to different classical groups.
Specifically, the GGP conjecture predict the nonvanishing of a period integral
based on the values of the $L$-function of a certain cuspidal automorphic representation.

In 2011, Jacquet-Rallis \cite{ref:JR} proposed an approach to the Gross-Prasad conjectures
for unitary groups via a relative trace formula (RTF).
The idea is to compare a RTF for the general linear group to one for a unitary group.
This approach relies on a so-called \emph{fundamental lemma},
which links values of certain orbital integrals
over two reductive groups over a non-Archimedean local field.

Let's be a bit more precise about what this fundamental lemma says.
Let $\VV_n^+$ denote the split $E/F$-Hermitian space of dimension $n$ (unique up to isomorphism),
fix a unit vector $w_0$ in it,
and let $(\VV_n^+)^\flat$ be the orthogonal complement of the span of $w_0$.
Let $G'^\flat \coloneqq \GL_{n-1}(E)$, $G' \coloneqq \GL_n(E)$,
$G^\flat \coloneqq \U((\VV_n^+)^\flat)(F)$ and $G \coloneqq \U(\VV_n^+)(F)$.
For certain
\[ \gamma \in G'^\flat \times G', \qquad g \in G^\flat \times G \]
the Jacquet-Rallis fundamental lemma proposes a relation between two orbital integrals.
Specifically, it supplies a relation between
\begin{itemize}
\item the orbital integral of $\gamma$ with respect to
  the indicator function $\mathbf{1}_{K'^\flat \times K'}$
  of the natural hyperspecial compact subgroup
  \[ K'^\flat \times K' \subset G'^\flat \times G' = \GL_{n-1}(E) \times \GL_n(E); \]
  and
\item the orbital integral of $g$ with respect to
  the indicator function $\mathbf{1}_{K^\flat \times K}$
  of the natural hyperspecial compact subgroup
  \[ K^\flat \times K \subset G^\flat \times G = \U((\VV_n^+)^\flat)(F) \times \U(\VV_n^+)(F). \]
\end{itemize}
In other words, it states that
\begin{equation}
  \Orb(\gamma, \mathbf{1}_{K'^\flat \times K'}) = \omega(\gamma) \Orb(g, \mathbf{1}_{K^\flat \times K})
  \label{eq:old_FL}
\end{equation}
where $\omega(\gamma)$ is a suitable \emph{transfer factor}.
The fundamental lemma has since been proved completely;
a local proof was given by Beuzart-Plessis \cite{ref:BeuzartPlessis}
while a global proof was given for large characteristic by W.\ Zhang \cite{ref:Wei2021}.

\subsection{The arithmetic GGP conjectures, and the arithmetic fundamental lemma}
At around the same time Waldspurger's formula was published,
Gross-Zagier \cite{ref:gross_zagier} proved a formula
relating the height of Heegner points
on certain modular curves to the derivative at $s=1$ of certain $L$-functions.
The Gross-Zagier formula was then generalized over several decades,
culminating in \cite{ref:GZshimura} where the formula is established
for Shimura curves over arbitrary totally real fields.

An arithmetic analogue of the original Gan-Gross-Prasad conjectures,
which we henceforth refer to as \emph{arithmetic GGP} \cite{ref:GGP},
can then be formulated, further generalizing Gross-Zagier's formula.
Here the modular curves in Gross-Zagier
are replaced with higher dimensional Shimura varieties.
Rather than the period integrals considered previously,
one instead takes intersection numbers of cycles on some Shimura varieties.
Specifically, if one considers the Shimura variety associated to a classical group,
the arithmetic GGP conjecture predicts a relation between intersection numbers
on this Simura variety with the central derivative of automorphic $L$-functions.

By analogy to the work Jacquet-Rallis \cite{ref:JR},
the arithmetic GGP conjectures should have a corresponding
\emph{arithmetic fundamental lemma} (henceforth AFL),
which was proposed by W.\ Zhang \cite[Conjecture 2.9]{ref:AFL}.
The arithmetic fundamental lemma then relates the derivative
of the weighted orbital integral with respect to the indicator function
$\mathbf{1}_{K'^\flat \times K'} \in \HH(G'^\flat \times G, K'^\flat \times K')$, that is
\[ \left. \pdv{}{s} \right\rvert_{s=0} \Orb(\gamma, \mathbf{1}_{K'^\flat \times K'}, s) \]
for $\gamma \in G'^\flat \times G'$,
to arithmetic intersection numbers on a certain Rapoport-Zink formal moduli space.
The AFL in \cite{ref:AFL} has since been proven over $p$-adic fields for any prime $p$ in
Mihatsch-Zhang \cite{ref:MZ2021}, W.\ Zhang \cite{ref:Wei2021}, Z.\ Zhang \cite{ref:Zhiyu}.

\subsection{The semi-Lie version of the AFL proposed by Liu}
There is another different version of the AFL, proposed by Yifeng Liu in
\cite[Conjecture 1.12]{ref:liuFJ},
which is often referred to as the \emph{semi-Lie version} of the AFL.
Its statement has been shown to be equivalent to AFL,
\cite[Remark 1.13]{ref:liuFJ} (and is thus now proven).
In contrast, the original AFL proposed by Zhang in \cite[Conjecture 2.9]{ref:AFL}
is sometimes referred to as the \emph{group version}.

A more detailed account of this equivalence is described in \cite[\S1.4]{ref:liuFJ}.

\subsection{Generalizations of FL and AFL to the full spherical Hecke algebra}
Recently it was shown by Leslie \cite{ref:leslie} that in fact
\eqref{eq:old_FL} holds in greater generality where the indicator function
$\mathbf{1}_{K^\flat \times K}$ can be replaced by any element in the spherical
Hecke algebra $\varphi \in \HH(G'^\flat \times G', K'^\flat \times K')$.
In that case, $\mathbf{1}_{K'^\flat \times K}$ needs to be replaced
by the corresponding element $\varphi'$ under a certain base change homomorphism
\begin{align*}
  \HH(G'^\flat \times G', K'^\flat \times K') &\to \HH(G^\flat \times G, K^\flat \times K) \\
  \varphi' &\mapsto \varphi
\end{align*}

In that case, the identity \eqref{eq:old_FL} still hold as
\begin{equation}
  \Orb(\gamma, \varphi') = \omega(\gamma) \Orb(g, \varphi).
  \label{eq:eq:leslie_FL}
\end{equation}
To complete the analogy illustrated in \Cref{fig:history},
there should thus be a generalization of the AFL in which
$\mathbf{1}_{K'^\flat \times K}$ is replaced by any element of the Hecke algebra
$\HH(G'^\flat \times G, K'^\flat \times K)$.
This formula is proposed by \cite{ref:AFLspherical},
and is the primary focus of this paper; we discuss it in the next section.

\section{Formulation of AFL conjectures to the full spherical Hecke algebra}
\subsection{The inhomogeneous version of the arithmetic fundamental lemma for spherical Hecke algebras proposed by Li-Rapoport-Zhang}
In contrast to the vague motivational cheerleading in the previous section,
starting in this section we will give more precise statements,
even though we will necessarily need to reference definitions appearing in later sections.

Retain the notation $G' \coloneqq \GL_n(E)$, and $G \coloneqq \U(\VV_n^+)(F)$,
with $K' \subset G'$ and $K \subset G$ the natural hyperspecial compact subgroups.
Also, let $q$ denote the residue characteristic of $F$.
Moreover, define the symmetric space
\[ S_n(F) \coloneqq \left\{ g \in \GL_n(E) \mid g \bar{g} = \id_n \right\}. \]
Finally, let $\VV_n^-$ be the non-split Hermitian space of dimension $n$
(unique up to isomorphism),
and let $\VV_n^+$ be the split one (again unique up to isomorphism).
Denote by $\U(\VV_n^-)$ the corresponding unitary groups.

For concreteness, we focus on the inhomogeneous version
of the arithmetic fundamental lemma, which is \cite[Conjecture 6.2.1]{ref:AFLspherical}.
This allows us to deal with just $G'$ instead of $G'^\flat \times G'$, etc.,
so that the Hecke algebra $\HH(G'^\flat \times G', K'^\flat \times K')$
can be replaced by the simpler one $\HH(\GL_n(E)) \coloneqq \HH(G', K')$.
Similarly, $\HH(G^\flat \times G, K^\flat \times K)$
can be replaced by the simpler $\HH(\U(\VV_n^+)) \coloneqq \HH(G, K)$.

The AFL conjecture provides a bridge between a geometric left-hand side
(given by an intersection number)
and an analytic right-hand side (given by an weighted orbital integral).
Stating it requires several pieces of data.
We only mention these pieces by name here, with definitions given later:
\begin{itemize}
  \ii On the geometric side, we have an intersection number.
  It uses the following ingredients.
  \begin{itemize}
    \ii We choose a regular semisimple element $g \in \U(\VV_n^-)\rs$.
    (The notation $\U(\VV_n^-)\rs$ denotes the regular semisimple elements of $\U(\VV_n^-)$, etc.
    The notion of regular semisimple is defined in \Cref{def:regular}.)
    \ii We choose a function $f \in \HH(\GL_n(E))$ from the spherical Hecke algebra,
    defined in \Cref{ch:background}.
    \ii We define a certain \emph{intersection number} $\Int((g,u), f)$
    in \Cref{def:intersection_number_inhomog}.
    These intersection numbers take place in a Rapoport-Zink space
    described in \Cref{ch:geo}.
  \end{itemize}

  \ii On the analytic side, we have an weighted orbital integral.
  It uses the following ingredients.
  \begin{itemize}
    \ii We choose a regular semisimple element $\gamma \in S_n(F)\rs$.
    \ii We choose a test function $\phi$ which comes
    from a certain $\HH(\GL_n(E))$-module that we will denote $\HH(S_n(F))$.
    This module $\HH(S_n(F))$ is defined in \Cref{ch:background}.
    \ii The weighted orbital integral $\Orb(\gamma, \phi, s)$
    is itself defined in \Cref{def:orbital0}.
    (It is connected to an unweighted orbital integral on the unitary group
    according to \Cref{thm:rel_fundamental_lemma}.)
    \ii There is also an extra transfer factor $\omega \in \{\pm1\}$
    which we define in \Cref{ch:transf}.
  \end{itemize}

  \ii We need a way to connect the inputs between the two parts of our conjecture.
  Specifically, $f$ and $\phi$ need to be linked, and $g$ and $\gamma$ need to be linked.
  This is done as follows.
  \begin{itemize}
    \ii Once the regular semisimple element $g \in \U(\VV_n^-)\rs$ is chosen,
    we require $\gamma \in S_n(F)\rs$ to be a \emph{matching} element.
    This matching is defined in \Cref{def:matching_inhomog}.
    (Alternatively, one could imagine picking $\gamma \in S_n(F)\rs$ first
    and finding corresponding $g$;
    it turns out $\gamma$ will match an element of $\U(\VV_n^\pm)\rs$ is general,
    and the conjecture is only formulated in the case where $g \in \U(\VV_n^-)$).

    \ii Once $f \in \HH(\U(\VV_n^+))$ is chosen, we select
    \[ \phi = (\BC_{S_n}^{\eta^{n-1}})^{-1}(f) \]
    to be the image of $f$ under a \emph{base change}.
    This base change is defined and then calculated explicitly for $n = 3$ in \Cref{ch:satake}.
  \end{itemize}
\end{itemize}
With all our protagonists now having names and references,
we can now state the conjecture proposed in \cite{ref:AFLspherical}.

\begin{conjecture}
  [Inhomogeneous version of the AFL for the full spherical Hecke algebra,
  {\cite[Conjecture 6.2.1]{ref:AFLspherical}}]
  \label{conj:inhomog}
  Let $f \in \HH(\U(\VV_n^+))$ be any element of the Hecke algebra,
  and let $\phi = (\BC_{S_n}^{\eta^{n-1}})^{-1}(f) \in \HH(S_n(F))$ be its image
  under base change as defined in \Cref{ch:satake}.
  Then for matching (as defined in \Cref{def:matching_inhomog}) regular semisimple elements
  \[ g \in \U(\VV_n^-)\rs \longleftrightarrow \gamma \in S_n(F)\rs \]
  we have
  \begin{equation}
    \Int\left( (1,g), \mathbf{1}_{K^\flat} \otimes f \right) \log q
    = -\omega(\gamma) \left. \pdv{}{s} \right\rvert_{s=0} \Orb(\gamma, \phi, s)
    \label{eq:inhomog}
  \end{equation}
  where the weighted orbital integral $\Orb(\dots)$ is defined in \Cref{def:orbital0},
  the transfer factor $\omega$ is defined in \Cref{ch:transf},
  and the intersection number $\Int(\dots)$ is defined in \Cref{ch:geo}.
\end{conjecture}

At present, the (inhomogeneous) AFL is the case where $f = \mathbf{1}_K$,
and is thus proven.
Note that in the case of interest where $\gamma \in S_n(F)\rs$
matches an element of $\U(\VV_n^-)\rs$ (rather than $\U(\VV_n^+)\rs$),
the actual value of $\Orb(\gamma, \phi, s)$ at $s = 0$ is always zero
by \Cref{thm:rel_fundamental_lemma};
so the conjecture instead looks at the first derivative at $s = 0$.

The generalized conjecture is also proved in full for
$n = 2$ in \cite[Theorem 1.0.1]{ref:AFLspherical}
(in that reference, our $n$ denotes the $n+1$ in \emph{loc.\ cit.}).
The part of the calculation involving weighted orbital integral has two parts:
\begin{itemize}
  \ii The calculation makes $\BC_{S_{n}}^{\eta^{n-1}}$
  completely explicit in a natural basis for $n = 2$.
  The result is \cite[Lemma 7.1.1]{ref:AFLspherical}.

  \ii The calculation makes explicit the value of the weighted orbital integral
  \[ \Orb(\gamma, \phi, s) \]
  for any $\gamma \in S_n(F)\rs$ and $\phi \in \HH(S_n(F))$,
  in terms of invariants of $\gamma$ and a decomposition of $\phi$ in a natural basis.
  The result is \cite[Proposition 7.3.2]{ref:AFLspherical}.
\end{itemize}
Combining these two (hence obtaining the right-hand side of \eqref{eq:inhomog})
with a calculation of intersection numbers in \cite[Corollary 7.4.3]{ref:AFLspherical}
(which is the left-hand side of \eqref{eq:inhomog})
shows that \Cref{conj:inhomog} holds for $n = 2$,
cf.\ \cite[Theorem 7.5.1]{ref:AFLspherical}.

\subsection{A proposed arithmetic fundamental lemma for spherical Hecke algebras in the semi-Lie case}
The primary focus of this paper is an analogous conjecture to \Cref{conj:inhomog}
for the semi-Lie version (also called the Fourier-Jacobi case).
It serves to complete the analogy given in \Cref{tab:semi_lie_analogy}.

\begin{table}[ht]
  \centering
  \begin{tabular}{lll}
    \toprule
    Version & AFL for $\mathbf{1}$ (now proven) & Full spherical Hecke \\
    \midrule
    Group & \cite[Conjecture 2.9]{ref:AFL} & \cite[Conjecture 6.2.1]{ref:AFLspherical} \\
    Semi-Lie & \cite[Conjecture 1.12]{ref:liuFJ} & \Cref{conj:semi_lie_spherical} \\
    \bottomrule
  \end{tabular}
  \caption{Table showing the analogy between the proposed
    \Cref{conj:semi_lie_spherical} and the existing conjectures.}
  \label{tab:semi_lie_analogy}
\end{table}

In this variation, as in \cite{ref:liuFJ},
rather than matching $g \in \U(\VV_n^-)\rs$ to $\gamma \in S_n(F)\rs$,
we consider an augmented space larger than $\U(\VV_n^-)$ and $S_n(F)$.
Specifically, one considers a matching between tuples of regular semisimple elements
\[ (g, u) \in (\U(\VV_n^-) \times \VV_n^-)\rs
  \longleftrightarrow (\gamma, \uu, \vv^\top) \in (S_n(F) \times V'_n(F))\rs \]
where $V'_n(F) = F^n \times (F^n)^\vee$ (defined in \Cref{def:matching_semi_lie})
consists of pairs of column vectors and row vectors of length $n$,
and the space $\VV_n^-$ is defined in \Cref{def:VV_n_nonsplit}.
The notion of \emph{matching} is defined in \Cref{def:matching_semi_lie} as well.
Meanwhile, we still use the same test functions $f$ and $\phi$,
as we did for \cite[Conjecture 6.2.1]{ref:AFLspherical}.
Finally, we also update the definition of intersection number
to accommodate the new term $u$ in \Cref{def:intersection_number_semi_lie_spherical}.

\begin{conjecture}
  [Semi-Lie version of the AFL for the full spherical Hecke algebra]
  Let $f \in \HH(\U(\VV_n^+))$ be any element of the Hecke algebra,
  and let $\phi = (\BC_{S_n}^{\eta^{n-1}})^{-1}(f) \in \HH(S_n(F))$ be its image
  under base change defined in \Cref{ch:satake}.
  Then for matching (as defined in \Cref{def:matching_semi_lie}) regular semisimple elements
  \[ (g, u) \in (\U(\VV_n^-) \times \VV_n^-)\rs \longleftrightarrow
    \guv \in (S_n(F) \times V'_n(F))\rs \]
  we have
  \begin{equation}
    \Int\left( (g,u), f \right) \log q \\
    = -\omega\guv \left. \pdv{}{s} \right\rvert_{s=0}
    \Orb(\guv, \phi \otimes \oneV, s)
  \end{equation}
  where the orbital integral $\Orb(\dots)$ is defined in \Cref{def:orbitalFJ},
  the transfer factor is defined in \Cref{ch:transf},
  and the intersection number $\Int(\dots)$ is defined in \Cref{ch:geo}.
  \label{conj:semi_lie_spherical}
\end{conjecture}
Note that in this version the new orbital integral $\Orb(\guv, \phi \otimes \oneV, s)$
is defined similarly.
However, as far as we know, no analog of \Cref{thm:rel_fundamental_lemma}
(linking it to an orbital integral on the unitary side) appears in the literature.
Thus we record the corresponding statement as \Cref{conj:rel_fundamental_lemma_semilie}.
Like before, \Cref{conj:rel_fundamental_lemma_semilie}
predicts that $\Orb(\guv, \phi \otimes \oneV, 0) = 0$ in the case of interest,
which in this case can be checked independently.

\begin{remark}
  For $n = 1$, the Hecke algebra $\HH(S_n(F))$ is trivial
  and therefore \Cref{conj:semi_lie_spherical}
  becomes a special case of the known result \cite{ref:liuFJ}.
  Therefore $n=2$ is the first case of \Cref{conj:semi_lie_spherical} worth examining.
\end{remark}

\subsection{A proposed conjecture on the kernel of indicator functions on which the orbital has identically vanishing derivative}
\label{sec:intro_large_kernel}

In \cite{ref:AFLspherical}, it was observed that for $n = 2$
there was a rather large space of $\phi \in \HH(S_2(F))$ such that
\[ \left. \pdv{}{s} \right\rvert_{s=0}
  \Orb \left(\gamma, \phi, s \right) = 0 \]
held identically across all $\gamma \in S_2(F)$.
In fact, the space of such $\phi$ has codimension $2$
as a vector subspace of $\HH(S_2(F))$ for $n = 2$.
They thus stated a conjecture the kernel was ``large'' for general $n$
as \cite[Conjecture 1.0.2]{ref:AFLspherical}.

It is therefore natural to ask whether a similar large kernel result
could hold for the analogous orbital integral in the semi-Lie case.
In fact, even for $n=2$ the behavior is somewhat different.
We propose the following conjecture, which we have proved for $n = 2$
(see \Cref{thm:semi_lie_ker_trivial} and \Cref{thm:semi_lie_ker_huge} momentarily).
\begin{conjecture}
  \label{conj:kernel_semi_lie}
  Let $n \ge 2$.
  \begin{enumerate}
  \item[(a)]
  Choose any $\gamma \in S_n(F)$ which could appear in a triple $\guv$
  matching some element of $\U(\VV_n^-) \times \VV_n^-$.
  Suppose $\phi \in \HH(S_n(F))$ satisfies
  \[ \left. \pdv{}{s} \right\rvert_{s=0}
    \Orb \left(\guv, \phi \otimes \oneV, s \right) = 0 \]
  for every such $(\uu, \vv^\top) \in V'_n(F)$,
  Then in fact $\Orb \left(\guv, \phi \otimes \oneV, s \right) = 0$ for all $s \in \CC$.

  \item[(b)]
  Fix a choice of $(\uu, \vv^\top) \in V'_n(F)$.
  Consider all $\gamma \in S_n(F)\rs$ for which $\guv$
  matches with an element of $(\U(\VV_n^-) \times \VV_n^-)\rs$.
  Consider the map of vector spaces
  \begin{align*}
    \HH(S_n(F))
    &\to C^\infty( \left\{ \text{elements of } S_n(F)\rs \text{ matching as above} \right\} ) \\
      \phi &\mapsto \left( \gamma \mapsto \left. \pdv{}{s} \right\rvert_{s=0}
      \Orb \left(\guv, \phi \otimes \oneV, s \right) \right).
  \end{align*}
  Then this map has large kernel,
  in the sense it is not contained in any maximal ideal of $\HH(S_n(F))$.
  \end{enumerate}
\end{conjecture}

\section{Results}
Most of the results here are dedicated toward the semi-Lie version of the AFL,
which is the new contribution provided by this paper.
But in \Cref{sec:results_group_AFL} we mention some other results
we proved for the group version of the AFL.

\subsection{Formulas for the orbital side of the semi-Lie AFL conjecture for $n=2$}
\label{sec:semi_lie_2_intro_params}
The main case of interest in this thesis is the new conjectured AFL
for the spherical Hecke algebra in the semi-Lie situation in the
specific case $n = 2$ where one can provide evidence for the conjecture.
On the orbital side, the various ingredients can be described concretely
in the following way:
\begin{itemize}
  \ii The Hecke algebra $\HH(S_2(F))$ has a natural basis of
  indicator functions $\mathbf{1}_{K', \le r}$ for each $r \ge 0$;
  see \Cref{ch:orbitalFJ0} for a definition.

  \ii Suppose $\guv \in (S_2(F) \times V'_2(F))\rs$
  pairs with an element of $(\U(\VV_2^-) \times \VV_2^-)\rs$.
  Then under the action $\GL_2(F)$ we may assume $\guv$ is of the form
  \[
    \guv = \left( \begin{pmatrix} a & b \\ c & d \end{pmatrix},
      \begin{pmatrix} 0 \\ 1 \end{pmatrix},
      \begin{pmatrix} 0 & e \end{pmatrix} \right)
    \in (S_2(F) \times V_2'(F))\rs
  \]
  (that is, we can find an orbit representative of this form).
  The parameters $a$, $b$, $c$, $d$ need to satisfy certain dependencies
  for the matching to hold; these are described in \Cref{ch:orbitalFJ0}.
\end{itemize}

Then we were able to derive the following fully explicit formula.
See \Cref{sec:proof_semi_lie_formula} for some concrete examples and illustrations.
\begin{theorem}
  \label{thm:semi_lie_formula}
  Let $\guv$ be as above.
  If $v(e) < 0$ or $v(b) + v(c) < -2r$, the entire orbital integral is $0$.
  Otherwise define
  \[ \nn_\guv(k) \coloneqq \min\left( \left\lfloor \tfrac{k + (v(b)+r)}{2} \right\rfloor,
    \left\lfloor \tfrac{(2v(e)+v(c)+r)-k}{2} \right\rfloor, N \right) \]
  where
  \[ N \coloneqq \min \left(
      v(e), \tfrac{v(b)+v(c)-1}{2} + r,
      v(d-a) + r \right). \]
  Also, if $v(d-a) < v(e) - r$ and $v(b) + v(c) > 2v(d-a)$, then additionally define
  \begin{align*}
    \cc_\guv(k) &= \min\big( k - (2v(d-a)-v(b)+r), \\
      &\qquad (2v(e)+v(c)-2v(d-a)-r)-k, v(e)-v(d-a)-r \big).
  \end{align*}
  Otherwise define $\cc_\guv(k) = 0$.
  Then we have
  \begin{align*}
    &\phantom= \Orb(\guv, \mathbf{1}_{K'_{S, \le r}} \otimes \oneV, s) \\
    &= \sum_{k = -(v(b)+r)}^{2v(e)+v(c)+r} (-1)^k
    \left( 1 + q + q^2 + \dots + q^{\nn_\guv(k)} \right) (q^s)^k \\
    &+ \sum_{k = 2v(d-a)-v(b)+r}^{2v(e)+v(c)-2v(d-a)-r} (-1)^k \cc_\guv(k) q^{v(d-a) + r} (q^s)^k.
  \end{align*}
\end{theorem}

Differentiating this yields the following result:
\begin{corollary}
  \label{cor:semi_lie_derivative_single}
  Let $r$, $\guv$ and $N$ be as in \Cref{thm:semi_lie_formula}
  Also define $\varkappa \coloneqq v(e) - (v(d-a)+r)$.
  Then
  \begin{align*}
    \frac{(-1)^{v(c)+r}}{\log q}
    &\partial \Orb(\guv, \mathbf{1}_{K'_{S, \le r}}) \\
    &= \sum_{j=0}^N \left( \frac{2v(e)+v(b)+v(c)+1}{2} + r - 2j \right) \cdot q^j \\
    & - q^{v(d-a)+r} \cdot
    \begin{cases}
      \frac{\varkappa}{2} & \text{if }\varkappa \equiv 0 \pmod 2 \\
      \left( v(e)+\frac{v(b)+v(c)}{2}-2v(d-a)-r \right) - \frac{\varkappa}{2}
      & \text{if }\varkappa \equiv 1 \pmod 2 \\
    \end{cases}
  \end{align*}
  where the second term is only present when $\varkappa \ge 0$ and $v(b)+v(c)>2v(d-a)$.
\end{corollary}

The formula simplifies even further if one considers instead
$\mathbf{1}_{K'_{S, \le r}} + \mathbf{1}_{K'_{S, \le (r-1)}}$;
and indeed we will see that this particular combination comes up naturally
with a special role later as well.
\begin{corollary}
  \label{cor:semi_lie_combo}
  Let $r$, $\guv$, $N$, $\varkappa$ be as in \Cref{cor:semi_lie_derivative_single}.
  Assume $r \ge 1$ and define
  \begin{align*}
    C &\coloneqq
    \begin{cases}
      \frac{\varkappa-1}{2}
        & \text{if } \varkappa > 0 \text{ is odd}
          \text{ and } v(b) + v(c) > 2v(d-a)  \\
      \frac{\varkappa+v(b)+v(c)-2v(d-a)-1}{2}
        & \text{if } \varkappa \ge 0 \text{ is even}
          \text{ and } v(b) + v(c) > 2v(d-a)  \\
      v(e) - N
        & \text{if } v(e) \ge \frac{v(b)+v(c)-1}{2} + r
        \text{ and } 2v(d-a) > v(b) + v(c) \\
      0 & \text{otherwise}
    \end{cases} \\
    C' &\coloneqq
    \begin{cases}
      C + 1 & \text{if } \varkappa \ge 0 \text{ and } v(b)+v(c) > 2v(d-a) \\
      0 & \text{otherwise}.
    \end{cases}
  \end{align*}
  Then
  \begin{align*}
    \frac{(-1)^{v(c)+r}}{\log q} &
    \partial\Orb(\guv, \mathbf{1}_{K'_{S, \le r}} + \mathbf{1}_{K'_{S, \le (r-1)}}) \\
    &= (q^N + q^{N-1} + \dots + 1) + C q^N + C' q^{N-1}
  \end{align*}
\end{corollary}

\begin{example}
  We show some examples of \Cref{cor:semi_lie_combo}:
  \begin{itemize}
  \ii When $r=5$, $v(b) = -20$, $v(c) = 37$, $v(e) = 35$ and $v(d-a) > \frac{v(b)+v(c)}{2} = 8.5$
  the derivative in \Cref{cor:semi_lie_combo} equals
  \[ \log q \cdot (23q^{13} + q^{12} + q^{11} + q^{10} + q^9 + \dots + q + 1). \]
  \ii When $r = 6$, $v(b) = 10$, $v(c) = 5$, $v(e) = 7$, $v(d-a) > v(e)-r = 1$,
  the derivative in \Cref{cor:semi_lie_combo} equals
  \[ -\log q \cdot (q^7 + q^6 + q^5 + \dots + q + 1). \]
  \ii When $r = 8$, $v(b) = -101$, $v(c) = 1000$, $v(e) = 29$, $v(d-a) = 11$,
  the derivative in \Cref{cor:semi_lie_combo} equals
  \[ \log q \cdot (444 q^{19} + 445q^{18} + q^{17} + q^{16} + q^{15} + \dots + q + 1). \]
  \end{itemize}
\end{example}

\subsection{Kernel results for the semi-Lie orbital integral when $n=2$}
As we mentioned our earlier conjecture \Cref{conj:kernel_semi_lie}
is true for $n = 2$.
More precisely, we have the following two theorems.

\begin{theorem}
  \label{thm:semi_lie_ker_trivial}
  Consider any $\gamma = \begin{pmatrix} a & b \\ c & d \end{pmatrix} \in S_2(F)$
  as in \Cref{sec:semi_lie_2_intro_params}.
  Assume $v(b)+v(c)>0$ (so that the orbital integrals are not all zero.)
  Then there don't exist any nonzero functions $\phi \in \HH(S_2(F))$ such that
  \[ \left. \pdv{}{s} \right\rvert_{s=0}
    \Orb \left(
      \left( \gamma, \begin{pmatrix} 0 \\ 1 \end{pmatrix}, \begin{pmatrix} 0 & \varpi^i \end{pmatrix} \right),
      \phi \otimes \oneV, s
    \right) = 0 \]
  holds for every integer $i \ge 0$.
  Thus part (a) of \Cref{conj:kernel_semi_lie} holds for $n = 2$.
\end{theorem}

\begin{theorem}
  \label{thm:semi_lie_ker_huge}
  Fix a choice of $(\uu, \vv^\top) \in V_2'(F)$ with $\vv^\top \uu \neq 0$.
  Consider all $\gamma \in S_2(F)$ for which $\guv$
  matches with an element of $(\U(\VV_n^-) \times \VV_n^-)\rs$.
  Then the set of $\phi \in \HH(S_2(F))$ such that
  \[
    \left. \pdv{}{s} \right\rvert_{s=0}
    \Orb \left( \guv, \phi \otimes \oneV, s \right) = 0
  \]
  holds for all such $\gamma$ is a $\QQ$-vector subspace of $\HH(S_2(F))$
  whose codimension is at most $v(\uu \vv^\top) + 2$.
  Moreover, the set of such $\phi$ is not contained in any maximal ideal of $\HH(S_2(F))$.
\end{theorem}

\begin{remark}
  In each case the Hecke algebra is isomorphic as a $\QQ$-algebra to $\QQ[T]$
  for a single variable $T = Y+Y^{-1}$.
  So actually it's mildly surprising that we get a result on finite codimension.
  In general, a finite codimension vector subspace of $\QQ[T]$ could
  be contained in a maximal ideal,
  such as the codimension one subspace $T \mathbb Q[T] \subset \QQ[T]$.
  Conversely, a finite \emph{dimension} vector subspace such as
  the one-dimensional space $\QQ \subseteq \QQ[T]$ is not contained in any maximal ideal.

  Thus neither of being finite codimension and generating all of $\QQ[T]$ imply each other.
  However, the author's opinion is that being finite codimension should be more surprising,
  since even a finite-dimensional $\QQ$-vector subspace of $\QQ[T]$
  (indeed, any subspace containing $1$) could still generate the entire ring $\QQ[T]$.

  In particular, we do not currently have a reason to expect that for larger $n$
  the codimension of the kernel will still be finite in $\HH(S_n(F))$,
  i.e.\ an analogous finite-codimension conjecture
  to \Cref{conj:large_kernel_group} seems possibly too optimistic.
  Nonetheless, it might still be interesting to consider other different ways
  of formalizing the notion of ``large kernel'' in \Cref{conj:large_kernel_group}
  and \Cref{conj:kernel_semi_lie}.
\end{remark}

\subsection{The geometric side of the semi-Lie AFL conjecture for $n=2$}
On the geometric side, we were also able to determine the intersection numbers
subject to two provisions,
\Cref{conj:serre_pullback_space} and \Cref{conj:serre_pullback_divisor},
about the pullback of certain divisors.
\begin{theorem}
  \label{thm:semi_lie_n_equals_2}
  Assume \Cref{conj:serre_pullback_space} and \Cref{conj:serre_pullback_divisor}.
  Then our generalized AFL conjecture, \Cref{conj:semi_lie_spherical}, holds for $n = 2$.
\end{theorem}
The proof of \Cref{thm:semi_lie_n_equals_2} is built up gradually
throughout the entire paper, culminating in \Cref{ch:finale}.

We comment briefly on the strategy of the proof.
The proof is made possible because the intersection numbers for $n=2$
are easier to work for a few reasons.
\begin{itemize}
\ii First, one can identify $\VV_2^-$ with an $E/F$-quaternion division algebra,
equipped with a compatible Hermitian form defined via quaternion multiplication.
This makes it possible to describe $\U(\VV_2^-)$ concretely as transformations obtained
via left multiplication by an element of $E$ and right multiplication by a quaternion.
\ii Secondly, it becomes possible to describe the so-called Rapoport-Zink spaces $\RZ_2$
used in the definition of the intersection number with a Lubin-Tate space $\MM_2$.
Thus the problem of computing the intersection number
$\Int\left( (g,u), f \right) \log q$
can be reduced to calculating the intersection of certain special
Kudla-Rapoport divisors on the space $\MM_2$.

However, on the Lubin-Tate space $\MM_2$,
there is a result known as the Gross-Keating formula \cite{ref:GK}
which allows one to make this intersection number fully explicit.
One can then match the resulting equation to the formulas described in
\Cref{cor:semi_lie_derivative_single}
and verify that, under the base change \Cref{ch:satake}
and the matching condition described in \Cref{ch:rs_matching},
the two obtained formulas are identical.
\end{itemize}
The two hypotheses \Cref{conj:serre_pullback_space} and \Cref{conj:serre_pullback_divisor}
are a stipulation that the pullback of two of the divisors
behaves in the way one would expect.

\subsection{Results for $n=3$ for the group AFL}
\label{sec:results_group_AFL}
For the group AFL, we were able to fully compute the orbital integral as well.
The result is too involved to state in the introduction,
but we give the following summary.
\begin{theorem}
  \label{thm:summary}
  Let $\gamma \in S_3(F)\rs$.
  Assume that $\gamma$ matches an element in $\U(\VV_3^-)\rs$.
  Then the weighted orbital integral $\Orb(\gamma, \phi, s)$ takes the form
  \[ \sum_k P_k(q) (-q^s)^k \]
  for some polynomials $P_k(q) \in \ZZ[q]$, where
  \begin{itemize}
    \ii the summation variable $k$ is a some contiguous range of integers,

    \ii the polynomials $P_k \in \ZZ[q]$ are nonzero and satisfy the property
    that every coefficient of $P_k$ besides possibly the leading coefficient is $1$;

    \ii both $\deg P_k$ and leading coefficient of $P_k$ are the integer parts
    of piecewise linear functions in $k$ with slopes in $\{0, \pm\half, \pm 1\}$.
  \end{itemize}
  The range of the summation, and the aforementioned piecewise linear function(s),
  can be written explicitly in terms of a particular representative
  in the orbit of $\gamma$.
\end{theorem}
For a full statement, see
\Cref{thm:full_orbital_ell_odd,thm:full_orbital_ell_even,thm:full_orbital_ell_neg}.
The calculation corresponds directly to the earlier results
\cite[Lemma 7.1.1 and Proposition 7.3.2]{ref:AFLspherical}
which were the case $n = 2$ of the inhomogeneous group version of the AFL
(note what \cite{ref:AFLspherical} calls $n$ is $n+1$ in our notation).
The methods, which are local in nature,
are rather similar to those employed in \cite{ref:AFL},
which can be thought of as the case $r = 0$.

\begin{remark}
  Interestingly, the formula \Cref{thm:semi_lie_formula}
  for $\guv \in (S_2(F) \times V'_2(F))\rs$
  actually fits the same template as \Cref{thm:summary},
  although the semi-Lie formula is more pleasant.
  We do not have a good explanation why the shapes of the orbital integrals
  end up being so similar.
\end{remark}

We were also able to determine the
relevant base changes in \Cref{ch:satake}.
However, we did not complete the comparison on the geometric side in this situation.
Thus we do not claim a proof of $n = 3$ of \Cref{conj:inhomog},
although we imagine such a proof could be completed
once a method for determining the intersection numbers explicitly is devised.
On the other hand, the orbital data is enough to prove the following result.
\begin{theorem}
  \label{thm:no_kernel_group}
  There is no nontrivial $\phi \in \HH(S_3(F))$ such that
   \[ \left. \pdv{}{s} \right\rvert_{s=0} \Orb \left( \gamma, \phi, s \right) = 0 \]
   holds for every $\gamma \in S_3(F)\rs$ pairing to an element of $\UU(\VV_3^-)\rs$.
\end{theorem}

\section{Roadmap}
The rest of the paper is organized as follows.

\begin{itemize}
  \ii The paper begins with some general background information.
  \begin{itemize}
    \ii In \Cref{ch:background} we provide some preliminary background
    on the spaces appearing in the overall paper and the Hecke modules that will be used.
    \ii Further background is stated in \Cref{ch:rs_matching},
    where we describe the matching of regular semisimple elements.
    \ii In \Cref{ch:satake} we provide reminders on the Satake transform.
    We also derive concrete formulas for base change when $n = 3$
    (in comparison, the analogous results for $n=2$ are
    \cite[Lemma 7.1.1]{ref:AFLspherical}),
    but these formulas are not used again later on.
  \end{itemize}

  \ii We then proceed to introduce the orbital integrals.
  \begin{itemize}
    \ii In \Cref{ch:orbital0} we introduce the weighted orbital integral
    for the group version of the AFL for full spherical Hecke algebra,
    and state the derivative explicitly for $n = 3$ in terms of a representative.
    In \Cref{ch:orbital1,ch:orbital2} we show the calculation of this formula,
    and give the full version of \Cref{thm:summary}.

    \ii In \Cref{ch:orbitalFJ0} we introduce the weighted orbital integral
    for the semi-Lie version of the AFL for full spherical Hecke algebra.
    The analogous calculation is in \Cref{ch:orbitalFJ1,ch:orbitalFJ2},
    which is used to prove \Cref{thm:semi_lie_formula} and its corollaries.
  \end{itemize}

  \ii Having completely computed the orbital integrals in these cases,
  we take a side trip in \Cref{ch:ker} to prove the ``large kernel'' results.
  We establish \Cref{conj:kernel_semi_lie} for $n = 2$
  by proving the explicit results
  \Cref{thm:semi_lie_ker_trivial} and \Cref{thm:semi_lie_finite_codim_full}.
  We also prove the existing conjecture for the group AFL for $n=3$.

  \ii We then turn our attention to the other parts of the two versions of the AFL.
  In addition to stating the relevant definitions,
  the subsequent chapters aim to prove \Cref{thm:semi_lie_n_equals_2}.
  \begin{itemize}
    \ii In \Cref{ch:transf} we define the transfer factors $\omega \in \{\pm1\}$.
    \ii In \Cref{ch:geo}, we describe the Rapoport-Zink spaces
    that the geometric side is based on, and define the intersection numbers
    on both sides of the AFL.
    \ii In \Cref{ch:jiao}, we specialize to the situation $n = 2$
    for the intersection numbers in the semi-Lie AFL only.
    The Rapoport-Zink spaces become replaced with Lubin-Tate ones,
    and we introduce the Gross-Keating formula
    that will be our primary tool for the calculation.
    \ii In \Cref{ch:finale} we tie everything together and establish
    \Cref{thm:semi_lie_n_equals_2}.
  \end{itemize}
\end{itemize}

An approximate dependency chart between the chapters is also given in Figure~\ref{fig:depchart}.

\begin{figure}[ht]
  \begin{center}
    \begin{tikzcd}
      \text{\ref{ch:background}. Background} \ar[rd, bend left = 10] \ar[dddd, bend left = 80] && \\
      \text{\ref{ch:satake}. Base change} \ar[rd, bend left = 10] \ar[rdddd, bend left]
      & \text{\ref{ch:rs_matching}. Matching} \ar[ld] \ar[rd] \ar[l] \ar[r]
      &  \text{\ref{ch:transf}. Transfer} \ar[ldddd, bend right] \\
      \text{\ref{ch:orbital0}. Group derivative} \ar[r] \ar[d]
        & \text{\ref{ch:ker}. Large kernel} &
        \text{\ref{ch:orbitalFJ0}. Semi-Lie derivative} \ar[l] \ar[d] \ar[lddd, bend right = 20]  \\
      \text{\ref{ch:orbital1}+\ref{ch:orbital2}. Group full orbital} &&
      \text{\ref{ch:orbitalFJ1}+\ref{ch:orbitalFJ2}. Semi-Lie full orbital} \\
      \text{\ref{ch:geo}. Int numbers} \ar[d] && \\
      \text{\ref{ch:jiao}. Gross-Keating} \ar[r] & \text{\ref{ch:finale}. Prove \Cref{thm:semi_lie_n_equals_2}}
    \end{tikzcd}
  \end{center}
  \caption{Dependency chart of the chapters in this paper,
    arranged to loosely resemble the Batman logo.}
  \label{fig:depchart}
\end{figure}

\chapter{General background}
\label{ch:background}

\section{Notation}
We provide a glossary of notation that will be used in this paper.
As mentioned in the introduction, $p > 2$ is a prime,
$F$ is a finite extension of $\QQ_p$,
and $E/F$ is an unramified quadratic field extension.

\begin{itemize}
  \ii For any $a \in E$, we let $\bar a$ denote the image of $a$
  under the nontrivial automorphism of $\Gal(E/F)$.
  (Hence $a = \bar a$ exactly when $a \in F$.)
  \ii Fix $\eps \in \OO_F^\times$ such that $E = F[\sqrt{\eps}]$.
  \ii Denote by $\varpi$ a uniformizer of $\OO_F$, such that $\bar \varpi = \varpi$.
  \ii Let $q \coloneqq |\OO_F/\varpi|$ be the residue characteristic.
  \ii Let $v$ be the associated valuation for $\varpi$.
  \ii Let $\eta$ be the quadratic character attached to $E/F$ by class field theory,
  so that $\eta(x) = -1^{v(x)}$.
  \ii Set $G' \coloneqq \GL_n(E)$.
  \ii Set $K' \coloneqq \GL_n(\OO_E) \subseteq G'$ as the hyperspecial maximal compact subgroup of $G$.
  \ii $V_0$ denotes a split $E/F$-Hermitian space of dimension $n$ (unique up to isomorphism).
  \ii Let $\beta$ denote the $n \times n$ antidiagonal matrix
  \[ \beta \coloneqq \begin{bmatrix} && 1 \\ & \iddots \\ 1 \end{bmatrix} \]
  and pick the basis of $V_0$ such that the Hermitian form on $V_0$ is given by
  \[ V_0 \times V_0 \to E \qquad (x,y) \mapsto x^\ast \beta y. \]
  \ii Set
  \[ G \coloneqq \U(V_0) = \{ g \in \GL_n(\OO_E) \mid g^\ast \beta g = \beta\} \]
  the unitary group over $V_0$.
  Note that $\beta$ is \emph{antidiagonal}, in contrast to the convention $\beta = \id_n$
  that is often used for unitary matrices with entries in $\CC$.
  \ii Set
  \[ K \coloneqq G \cap \GL_n(\OO_E) \]
  as the natural hyperspecial maximal compact subgroup.
  \ii Let $\VV_n$ denote the non-split $E/F$-Hermitian space of dimension $n$
  (unique up to isomorphism), and $\U(\VV_n)$ the corresponding unitary group.
  This space will be realized in \Cref{ch:geo}.
\end{itemize}

\section{The spaces $S_n(F)$ and $S_n(F) \times V'_n$}
For the analytic side of the two AFL conjectures we investigate,
the following two spaces will be used as inputs to our orbital integrals.
\begin{definition}
  [{\cite[(4.10)]{ref:highdim2024}}]
  We define the symmetric space
  \[ S_n(F) \coloneqq \left\{ g \in \GL_n(E) \mid g \bar g = \id_n \right\}. \]
  It has a natural left action of $\GL_n(E)$ by
  \begin{align*}
    \GL_n(E) \times S_n(F) &\to S_n(F) \\
    g \cdot \gamma &\coloneqq g \gamma \bar g^{-1}.
  \end{align*}
\end{definition}

\begin{definition}
  [{\cite[(4.11)]{ref:highdim2024}}]
  We set
  \[ V'_n(F) \coloneqq F^n \times (F^n)^\vee \]
  where $-^\vee$ denotes the $F$-dual space, i.e., $(F^n)^\vee = \Hom_F(F^n, F)$.
  Then we may also consider the augmented space
  \[ S_n(F) \times V'_n(F) \]
  If we identify $F^n$ with column vectors of length $n-1$ and $(F^n)^\vee$
  with row vectors of length $n$ then we have a left action of $\GL_n(F)$ by
  \begin{align*}
    \GL_n(F) \times (S_n(F) \times V'_n)(F)
    &\to S_n(F) \times V'_n(F) \\
    h \cdot (\gamma, \uu, \vv^\top)
    &\coloneqq (h \gamma h^{-1}, h\uu, \vv^\top h^{-1}).
  \end{align*}
  Note that according to the embedding
  \begin{align*}
    S_n(F) \times V'_n(F)
    &\hookrightarrow \GL_{n+1}(E) \\
    (\gamma, \uu, \vv^\top)
    &\mapsto \begin{bmatrix} \gamma & \uu \\ \vv^\top & 0 \end{bmatrix}
  \end{align*}
  we can consider elements of $S_n(F) \times V'_n(F)$ as elements of $\GL_{n+1}(E)$ too.
  In that case the action of $h \in \GL_{n+1}(F)$
  coincides with $h \cdot g \mapsto hg\bar{h}^{-1}$ as well.
\end{definition}

\begin{definition}
  For brevity, let
   \[ K'_S \coloneqq S_n(F) \cap \GL_n(\OO_F). \]
\end{definition}

\section{Definition of Hecke algebra}
We reminder the reader the definition of a Hecke algebra.
For this subsection, $G$ will denote \emph{any}
unimodular locally compact topological group,
and $K$ any closed subgroup of $G$.

\begin{definition}
  The \emph{Hecke algebra}
  \[ \HH(G, K) \coloneqq \QQ[K \backslash G \slash K] \]
  is defined as the space of compactly supported $K$-binvariant
  locally constant functions on $G$.

  Given two such functions $f_1$ and $f_2$ in $\HH(G,K)$,
  one can consider define the convolution
  \[ (f_1 \ast f_2)(g) \coloneqq \int_G f_1(g\inv x) f_2(x) \; \odif x \]
  which makes $\HH(G, K)$ into a $\QQ$-algebra,
  whose identity element is $\mathbf{1}_K$.
\end{definition}
In the case where $G$ is a reductive Lie group and
$K$ is the maximal compact subgroup
(or more generally whenever $(G,K)$ is a Gelfand pair),
this Hecke algebra is actually commutative.

\section{The specific Hecke algebras $\HH(G',K')$ and $\HH(G,K)$
  for $G' = \GL_n(E)$ and $G = \U(V_0)$, and the module $\HH(S_n(F), K')$}
For our purposes, we define two Hecke algebras:
\begin{align*}
  \HH(G', K') &\coloneqq \HH(\GL_n(E), \GL_n(\OO_E)) \\
  \HH(G, K) &\coloneqq \HH(\U(V_0), \U(V_0) \cap \GL_n(\OO_E))
\end{align*}

Now the symmetric space $S_n(F)$ is not a group,
so it does not make sense to define the same thing here.
Nevertheless, we introduce
\[ \HH(S_n(F), K') \coloneqq \mathcal C_{\mathrm{c}}^{\infty}(S_n(F))^{K'} \]
as the set of smooth compactly supported functions on $S_n(F)$
which are invariant under the action of $K' \subseteq G'$;
this is an $\HH(G', K')$-module, where the action of $f \in \HH(G', K')$ is given by
\[ f \cdot \varphi = {?} \]
\todo{how do I set RHS}
This does \textbf{not} have a multiplication structure at the moment, \emph{a priori},
although later we will see how one could be imposed.

Throughout this paper, to be unambiguous with the notation, we denote
\begin{itemize}
  \ii elements of $\HH(G,K)$ using $f$ or $f_i$ or similar
    (i.e.\ lowercase Roman letters);
  \ii elements of $\HH(G',K')$ by $f'$ or $f'_i$ or similar
    (i.e.\ lowercase Roman letters with apostrophes);
  \ii elements of $\HH(S_n(F),K')$ by $\phi$ or $\phi'_i$
    (i.e.\ lowercase Greek letters).
\end{itemize}

%Similarly, consider $S_n(F) \times V_n$.
%We set
%\[ \HH(S_n(F) \times V_n', {?}) \coloneqq \mathcal C_{\mathrm{c}}^{\infty}(S_n(F) \times V_n')^{K'} \]
%as the set of smooth compactly supported functions on $S_n(F) \times V_n'$
%which are invariant under the action of ${?} \subseteq G'$.
%\todo{This should be $\GL_n(\OO_F)$? Does it need another name?}
%This is an \dots

\section{Arches}
We introduce one more piece of notation for a common shape that our answers will take.

\begin{definition}
  Suppose $\{a_0, a_0 + 1, \dots, a_1\}$ is an interval of integers for some $a_0 \le a_1$,
  and consider two more integers $w_1$ and $w_2$ such that $w_1 + w_2 \le \frac{a_1-a_0}{2}$.
  Then we can define a piecewise linear function
  \[ \Arch_{[a_0, a_1]}(w_1, w_2) \colon \{a_0, a_0+1, \dots, a_1\} \to \ZZ_{\ge 0} \]
  according to the following definition:
  \[
    k \mapsto
    \begin{cases}
      k - a_0 & \text{if }a_0 \le k \le a_0 + w_1 \\
      w_1 + \left\lfloor \frac{k-(a_0+w_1)}{2} \right\rfloor & \text{if } a_0 + w_1 \le k \le a_0 + w_1 + w_2 \\
      w_1 + \left\lfloor \frac{w_2}{2} \right\rfloor & \text{if } a_0 + w_1 + w_2 \le k \le a_1 - (w_1 + w_2)\\
      w_1 + \left\lfloor \frac{(a_1-w_1) - k}{2} \right\rfloor & \text{if } a_1 - (w_1 + w_2) \le k \le a_1 - w_1 \\
      a_1 - k & \text{if }a_1 - w_1 \le k \le a_1.
    \end{cases}
  \]
\end{definition}
The nomenclature is meant to be indicative of the shape of the graph,
which looks a little bit like an arch.
It is a function symmetric around $\frac{a_0+a_1}{2}$ defined piecewise.
The function grows linearly with slope $1$ at the far left for $w_1$ steps,
then changes to slope $1/2$ for $w_2$ steps (rounding down),
before stabilizing, then doing the symmetric descent on the right half.
\begin{figure}[ht]
  \begin{center}
  \begin{asy}
    size(12cm);
    draw((-1,0)--(20,0));
    draw((0,0)--(3,3)--(7,5)--(12,5)--(16,3)--(19,0), lightred);
    draw((3,3)--(3,0), grey);
    draw((7,5)--(7,0), grey);
    draw((12,5)--(12,0), grey);
    draw((16,3)--(16,0), grey);
    real eps = 0.3;
    void brack(string s, real x0, real x1) {
      draw((x0+0.1,-eps)--(x0+0.1,-2*eps)--(x1-0.1,-2*eps)--(x1-0.1,-eps), blue);
      label(s, ((x0+x1)/2, -2*eps), dir(-90), blue);
    }
    brack("$w_1 = 3$", 0, 3);
    brack("$w_2 = 4$", 3, 7);
    brack("$w_2 = 4$", 12, 16);
    brack("$w_1 = 3$", 16, 19);

    dotfactor *= 1.5;
    dot("$(0,0)$", (0,0), dir(225));
    dot((1,1), red);
    dot((2,2), red);
    dot("$(3,3)$", (3,3), dir(135));
    dot((4,3), red);
    dot((5,4), red);
    dot((6,4), red);
    dot("$(7,5)$", (7,5), dir(90));
    dot((8,5), red);
    dot((9,5), red);
    dot((10,5), red);
    dot((11,5), red);
    dot("$(12,5)$", (12,5), dir(90));
    dot((13,4), red);
    dot((14,4), red);
    dot((15,3), red);
    dot("$(16,3)$", (16,3), dir(45));
    dot((17,2), red);
    dot((18,1), red);
    dot("$(19,0)$", (19,0), dir(315));

    label(rotate(45)*"Slope $+1$", (1.5,1.5), dir(135), lightred);
    label(rotate(26.57)*"Slope $+\frac12$", (5,4), dir(125), lightred);
    label(rotate(-26.57)*"Slope $-\frac12$", (14,4), dir(55), lightred);
    label(rotate(-45)*"Slope $-1$", (17.5,1.5), dir(45), lightred);
    label("Slope $0$", (9.5,5), dir(90), lightred);
  \end{asy}
  \end{center}
  \caption{A plot of $\Arch_{[0,19]}(3,4)$.}
  \label{fig:arch}
\end{figure}

\section{Synopsis of the orbital integral $\Orb(\gamma, \phi, s)$ for $\gamma \in S_3(F)$ and $\phi \in \HH(S_3(F), K')$}
\label{sec:orbital0}

This section defines the orbital integral
and describes the parameters which we will use to express our answer.

\subsection{Initial definition of the orbital integral for general $S_n(F)$}
Let $H = \GL_{n-1}(F)$.
Then $H$ has a natural embedding into $G = \GL_n(E)$
by \[ h \mapsto \left[ \begin{smallmatrix} h & 0 \\ 0 & 1 \end{smallmatrix} \right] \]
which endows it with an action $S_n(E)$.
Then our orbital integral is defined as follows.
\begin{definition}
  For brevity let $\eta(h) \coloneqq \eta(\det h)$ for $h \in H$.
  For $\gamma \in S_n(F)$, $\phi \in \HH(S_n(F), K')$, and $s \in \CC$,
  we define the orbital integral by
  \[ \Orb(\gamma, \phi, s) \coloneqq
    \int_{h \in H} \phi(h\inv \gamma h) \eta(h)
    \left\lvert \det(h) \right\rvert_F^{-s} \odif h. \]
\end{definition}

\subsection{Reparametrizing the orbital integral as a double integral over $E$
  via the group $H' \cong \GL_2(F)$}
From now on assume $n = 3$.
Then our orbital integral at present a quadruple integral over $F$,
owing to $H = \GL_{2}(F)$ being a four-dimensional $F$-vector space.

It will be more economical to work with the orbital integral as a double integral
with two coefficients in $E$, in the following sense.
Define
\[ H' \coloneqq
  \left\{ \begin{bmatrix} t_1 & t_2 \\ \bar t_2 & \bar t_1 \end{bmatrix}
    \mid t_1, t_2 \in E \right\}
\]
which is indeed a four-dimensional $F$-algebra.
As before $H' \hookrightarrow G$ according to the same embedding
$\GL_2(E) \hookrightarrow \GL_(3)$
and so $H'$ also acts on $S_n(E)$ by conjugation.

As an $F$-algebra, we have an isomorphism (see \cite[\S4.1]{ref:AFL})
\begin{align*}
  \iota_2 \colon H = \GL_2(F)
  &\xrightarrow{\cong} H' \\
  \begin{bmatrix} a_{11} & a_{12} \\ a_{21} & a_{22} \end{bmatrix}
  &\mapsto \begin{bmatrix} t_1 & t_2 \\ \bar t_2 & \bar t_1 \end{bmatrix} \\
  t_1 &= \half\left( a_{11} + a_{22} + \frac{a_{12}}{\sqrt{\eps}} + a_{21} \sqrt{\eps} \right) \\
  t_2 &= \half\left( a_{11} - a_{22} + \frac{a_{12}}{\sqrt{\eps}} - a_{21} \sqrt{\eps} \right).
\end{align*}
Under this isomorphism, we have
\[ h \gamma h^{-1} = \iota_2(h) \gamma \overline{\iota_2(h)^{-1}}. \]
\todo{maybe i should actually check this to make sure I'm not crazy.}

This allows us to rewrite the orbital integral over $H'$ instead.
If we write $h' = \overline{\iota_2(h)^{-1}}$,
then the following integral formula is obtained.
\begin{proposition}
  [{\cite[\S4.2]{ref:AFL}}]
  \label{prop:orbital_over_H_prime}
  For brevity let $\eta(h') \coloneqq \eta(\det h')$ for $h' \in H'$.
  For $\gamma \in S_3(F)$, $\phi \in \HH(S_3(F), K')$, and $s \in \CC$,
  the orbital integral can instead be written as
  \[ \Orb(\gamma, \phi, s) =
    \int_{h' \in H'} \phi(\bar{h'}\inv \gamma h') \eta(h')
    \left\lvert \det(h') \right\rvert_F^{s} \odif{h'} \]
  where
  \[ \odif{h'} = \kappa \cdot \frac{\odif t_1 \odif t_2}
    {\left\lvert t_1 \bar t_1 - t_2 \bar t_2 \right\rvert_F^2} \]
  for the constant
  \[ \kappa \coloneqq \frac{1}{(1-q\inv)(1-q^{-2})}. \]
\end{proposition}

\subsection{Invariants for the orbital integral}
Evidently the orbital integral $\Orb(\gamma, \phi, s)$ in \Cref{prop:orbital_over_H_prime}
only depends on the $H'$-orbit of $\gamma$.
So it makes sense to pick a canonical representative for the $H'$-orbit to compute
the orbital integral in terms of.

We assume henceforth that $\gamma$ is regular.
Then by \cite[Proposition 4.1]{ref:AFL},
we can assume $\gamma$ is a representative specifically of the form
\[ \gamma(a,b,d) =
  \begin{bmatrix}
    a & 0 & 0 \\
    b & - \bar d & 1 \\
    c & 1 - d \bar d & d
  \end{bmatrix}
  \in S_3(F); \quad \text{where $c = -a \bar b + b d$} \]
over all $a \in E^1$, $b \in E$, $d \in E$ for which $(1-d\bar d)^2 - c \bar c \neq 0$,
cover all the \emph{regular} orbits, which are the ones we care about.

\subsection{Basis for the indicator functions in $\HH(S_3(F), K')$}
We have the symmetric space
\[ S_3(F) \coloneqq \left\{ g \in \GL_3(E) \mid g \bar{g} = \id_3 \right\}. \]
which has a left action under $\GL_3(E)$ by $g \cdot s \mapsto gs\bar{g}\inv$.

Then $S_3(F)$ admits the following decomposition:
\begin{lemma}
  [Cartan decomposition]
  For each integer $r \ge 0$ let
  \[ K'_{S,r} \coloneqq \GL_3(\OO_E) \cdot \begin{bmatrix} 0 & 0 & \varpi^r \\ 0 & 1 & 0 \\ \varpi^{-r} & 0 & 0 \end{bmatrix} \]
  denote the orbit of
  $\begin{bmatrix} 0 & 0 & \varpi^r \\ 0 & 1 & 0 \\ \varpi^{-r} & 0 & 0 \end{bmatrix}$
  under the left action of $\GL_3(\OO_E)$.
  Then we have a decomposition
  \[ S_3(F) = \coprod_{r \geq 0} K'_{S,r}. \]
\end{lemma}
The $r=0$ case will be given a special shorthand,
and can be expressed in a few equivalent ways:
\begin{align*}
  K'_S
  &\coloneqq K'_{S,0} \\
  &= \GL_3(\OO_E) \cdot \begin{bmatrix} & & 1 \\ & 1 \\ 1 \end{bmatrix} \\
  &= \GL_3(\OO_E) \cdot \id_3 = S_3(F) \cap \GL_3(\OO_E).
\end{align*}
One can equivalently define $K'_{S,r}$ to be the part of $S_3(F)$
for which the most negative valuation among the nine entries is $-r$.

For $r \geq 0$, define
\[ K'_{S, \le r} \coloneqq S_3(F) \cap \varpi^{-r} \GL_3(\OO_E). \]
We can re-parametrize the problem according to the following.
\begin{proposition}
  \[ K'_{S, \le r} = K'_{S,0} \sqcup K'_{S,1} \sqcup \dots \sqcup K'_{S,r}. \]
\end{proposition}
Then an integral over each $K'_{S, \le r}$ lets us extract the integrals over $K'_{S,r}$.
\begin{proposition}
  For $r \ge 0$, the indicator functions $\mathbf{1}_{K'_{S, \le r}}$
  form a basis of $\HH(S_3(F), K')$.
\end{proposition}
\todo{this is like an $\HH(G', K')$-basis I think? double check this}

Then, our goal is to compute for
\begin{equation}
  \pdv{}{s}\Orb(\gamma, \mathbf{1}_{K'_{S, \le r}}, s)
  \label{eq:orbital_goal}
\end{equation}
at $s=0$ for any $r > 0$ as well.
Note that the $r = 0$ case is already done in \cite{ref:AFL}.
\todo{comment some basis thing}

\subsection{Simplifying assumptions}
For the purposes here, we will only care about the following case:
\begin{assume}
  \[ v\left( (1-d\bar d)^2 - c \bar c\right) \equiv 1 \pmod 2 \]
  \label{assume:u_odd}
\end{assume}
\todo{I need to ask Wei exactly why we're only doing this case}

We will thus also assume:
\begin{assume}
  $v(d) \geq -r$.
\end{assume}
This is fine because if this $v(d) < -r$ then the integral will always vanish
(because the bottom-right entry of $\Gamma(\gamma, t, m)$ is no-good).

We will really mostly be interested in the case where $v(b) = v(d) = 0$.
In fact, few other cases even occur at all
given \Cref{assume:u_odd};
we will see momentarily that either
$v(b) = v(d) \in \{-1, -2, \dots, -r\}$,
or one of $\{v(b), v(d)\}$ is zero and the other is nonnegative.

\subsection{Parameters to state the answer in terms of}
As we described earlier, our goal is to evaluate \eqref{eq:orbital_goal} in terms of
\[ a \in E^1, \qquad b, d \in E, \qquad r \ge 0. \]
To simplify the notation in what follows,
it will be convenient to define several quantities that reappear frequently.
From \Cref{assume:u_odd}, we may define
\begin{equation}
  \delta \coloneqq v(1-d \bar d) = v(c) \neq -\infty.
  \label{eq:delta}
\end{equation}
Following \cite{ref:AFL} we will also define
\begin{equation}
  u \coloneqq \frac{\bar c}{1-d \bar d} \in \OO_E^\times
  \label{eq:u}
\end{equation}
so that $\nu(1-u \bar u) \equiv 1 \pmod 2$ and
\begin{equation}
  b = -au - \bar{d} \bar{u}.
  \label{eq:b}
\end{equation}
Note that this gives us the following repeatedly used identity
\begin{equation}
  b^2-4a\bar d = (au-\bar d \bar u)^2 - 4a\bar d(1-u\bar u).
  \label{eq:dos}
\end{equation}
Finally, define
\begin{equation}
  \ell \coloneqq v(b^2 - 4 a \ol d).
  \label{eq:ell}
\end{equation}
We will also define one additional parameter useful when $\ell$ is even
(but as we will see, redundant for odd $\ell$):
\begin{equation}
  \lambda \coloneqq v(1-u \bar u) \equiv 1 \pmod 2.
  \label{eq:lambda}
\end{equation}

Just as many pairs $(v(b), v(d))$ do not occur (given \Cref{assume:u_odd})
and $v(b) = v(d) = 0$ is the main case of interest,
the parameters $(\delta, \ell, \lambda)$ satisfy some additional relations.
We will now describe them.
\begin{proposition}
  Exactly one of the following situations is true.
  \begin{enumerate}[a.]
    \ii $v(b) = v(d) = 0$, $\ell \ge 1$ is odd,
      $\ell < 2 \delta$, and $\lambda = \ell$.
    \ii $v(b) = v(d) = 0$, $\ell \ge 0$ is even,
      $\ell \le 2 \delta$, and $\lambda > \ell$ is odd.
    \ii $v(b) = 0$, $v(d) > 0$, $\ell = \delta = 0$, and ???
    \ii $v(b) > 0$, $v(d) = 0$, $\ell  = 0$, $\delta \ge 0$, and ???
    \ii $v(b) = v(d) \in \{-1, \dots, -r\}$,
    $\ell = \delta = 2v(d) < 0$, and ???
  \end{enumerate}
  \label{prop:parameter_constraints}
\end{proposition}
\begin{proof}
  \todo{Write this out; import from previous if needed.
    Also figure out exactly what the constraints on $\lambda$ are.}
\end{proof}

\todo{imported this; need to rearrange}
\begin{proposition}
  Whenever $\ell$ is odd, we must have
  \begin{equation}
    v(b) = v(d) = 0.
    \label{eq:odd_b_d_zero}
  \end{equation}
\end{proposition}
\begin{proof}
  [Proof of \eqref{eq:odd_b_d_zero}]
  If $v(d) \neq 0$, then $b = -au-\bar d\bar u$ is a unit,
  and hence so is $b^2 - 4 a \bar d$, causing $\ell = 0$, contradiction.
  And if $d$ is a unit, $\ell \neq 0$ means $v(b) = 0$ too.
\end{proof}

In the case where $\ell$ is odd (and hence $\ell \ge 1$ and $v(b) = v(d) = 0$),
we get \eqref{eq:dos} implying $\lambda = \ell$
and this definition will never be used --- the orbital will be computed
as a function of $\ell$ and $\delta$ (and $r$).
However for even $\ell$ these numbers are never equal and our orbital
integral will be stated in terms of $\ell$, $\delta$, and $\lambda$ (and $r$).

\subsection{Result}
We can now state the answer.
\todo{put the result here}

\section{Setup for the orbital integral}
\label{sec:orbital1}

In this section we set up the framework of the orbital integral
based on the definitions in the previous section.
This involves rewriting the integral as an infinite discrete double sum
over two parameters $(n,m)$ that we will introduce later,
and determining the volume of the supports of the indicator function.

\subsection{Reparametrization in terms of valuations}
\subsubsection{Computation of value in indicator function}
We are integrating over $t_1 \in E$ and $t_2 \in E$.
Regarding $g \in H'$ as an element of $\GL_3$ as described before, we have
\[ g = \begin{bmatrix}
  t_1 & t_2 & 0  \\
  \bar t_2 & \bar t_1 & 0 \\
  0 & 0 & 1
  \end{bmatrix}. \]
We therefore have
\[ \bar g \inv = \begin{bmatrix}
  \frac{t_1}{t_1 \bar t_1 - t_2 \bar t_2} & \frac{-\bar t_2}{t_1 \bar t_1 - t_2 \bar t_2} & 0 \\
  \frac{-t_2}{t_1 \bar t_1 - t_2 \bar t_2} & \frac{\bar t_1}{t_1 \bar t_1 - t_2 \bar t_2} & 0 \\
  0 & 0 & 1 \end{bmatrix}. \]
Hence
\begin{align*}
  \bar g \inv \gamma g
  &=
  \begin{bmatrix}
  \frac{t_1}{t_1 \bar t_1 - t_2 \bar t_2} & \frac{-\bar t_2}{t_1 \bar t_1 - t_2 \bar t_2} & 0 \\
  \frac{-t_2}{t_1 \bar t_1 - t_2 \bar t_2} & \frac{\bar t_1}{t_1 \bar t_1 - t_2 \bar t_2} & 0 \\
  0 & 0 & 1 \end{bmatrix}
  \begin{bmatrix}
    a & 0 & 0 \\
    b & - \bar d & 1 \\
    c & 1 - d \bar d & d
  \end{bmatrix}
  \begin{bmatrix}
  t_1 & t_2 & 0  \\
  \bar t_2 & \bar t_1 & 0 \\
  0 & 0 & 1
  \end{bmatrix} \\
  &=
  \begin{bmatrix}
  \frac{t_1}{t_1 \bar t_1 - t_2 \bar t_2} & \frac{-\bar t_2}{t_1 \bar t_1 - t_2 \bar t_2} & 0 \\
  \frac{-t_2}{t_1 \bar t_1 - t_2 \bar t_2} & \frac{\bar t_1}{t_1 \bar t_1 - t_2 \bar t_2} & 0 \\
  0 & 0 & 1 \end{bmatrix}
  \begin{bmatrix}
    at_1 & at_2 & 0 \\
    bt_1 - \bar d \bar t_2 & b t_2 - \bar d \bar t_1 & 1 \\
    ct_1 + (1-d\bar d)\bar t_2 & ct_2 + (1-d \bar d) \bar t_1 & d
  \end{bmatrix}
  \\
  &=
  \begin{bmatrix}
    \dfrac{at_1^2 - bt_1 \bar t_2 + d \bar t_2^2}{t_1 \bar t_1 - t_2 \bar t_2}
    & \dfrac{at_1t_2 - bt_2 \bar t_2 + \bar d \bar t_1 \bar t_2}{t_1 \bar t_1 - t_2 \bar t_2}
    & \dfrac{-\bar t_2}{t_1 \bar t_1 - t_2 \bar t_2} \\[2ex]
    \dfrac{-at_1t_2+bt_1\bar t_1-\bar d \bar t_1 \bar t_2}{t_1 \bar t_1 - t_2 \bar t_2}
    & \dfrac{-at_2^2+b\bar t_1 t_2-d\bar t_1^2}{t_1 \bar t_1 - t_2 \bar t_2}
    & \dfrac{\bar t_1}{t_1 \bar t_1 - t_2 \bar t_2} \\[2ex]
    ct_1 + (1-d\bar d)\bar t_2 & ct_2 + (1-d \bar d) \bar t_1 & d
  \end{bmatrix}
\end{align*}
Let us define \[ t = t_2 \bar t_1 \inv \iff t_2 = t \bar t_1. \]
This lets us rewrite everything in terms of the ratio $t$ and $t_1 \in E$:
\[
  \bar g \inv \gamma g
  =
  \begin{bmatrix}
    \dfrac{t_1^2(a-b\bar t+\bar d \bar t^2)}{t_1 \bar t_1(1-t \bar t)}
    & \dfrac{t_1 \bar t_1(at-bt\bar t+\bar d \bar t)}{t_1 \bar t_1(1-t \bar t)}
    & \dfrac{t_1 \cdot (-\bar t)}{t_1 \bar t_1 (1-t \bar t)} \\[2ex]
    \dfrac{t_1\bar t_1(-at+b-\bar d \bar t)}{t_1 \bar t_1(1-t \bar t)}
    & \dfrac{\bar t_1^2(-at^2+bt-\bar d)}{t_1 \bar t_1(1-t \bar t)}
    & \dfrac{-\bar t_1}{t_1 \bar t_1(1-t \bar t)} \\[2ex]
    t_1(c + (1-d\bar d)\bar t) & \bar t_1(ct + (1-d \bar d)) & d
  \end{bmatrix}
\]
This new parametrization is better because $t_1$ only plays the role of
a scale factor on the outside, with ``interesting'' terms only involving $t$.
To make this further explicit, we write
\[ t_1 = \varpi^{-m} \epsilon \]
for $m \in \ZZ$ and $\epsilon \in \OO_E^\times$.
Then we actually have
\[
  \begin{bmatrix} \bar\epsilon \\ & \epsilon \\ & & 1 \end{bmatrix}
  \bar g \inv \gamma g
  \begin{bmatrix} \epsilon\inv \\ & \bar\epsilon\inv \\ & & 1 \end{bmatrix}
  =
  \begin{bmatrix}
    \dfrac{a-b\bar t+\bar d \bar t^2}{1-t \bar t}
    & \dfrac{at-bt\bar t+\bar d \bar t}{1-t \bar t}
    & \dfrac{-\varpi^m \bar t}{1-t \bar t} \\[2ex]
    \dfrac{-at+b-\bar d \bar t}{1-t \bar t}
    & \dfrac{-at^2+bt-\bar d}{1-t \bar t}
    & \dfrac{-\varpi^m}{1-t \bar t} \\[2ex]
    \dfrac{c + (1-d\bar d)\bar t}{\varpi^m} & \dfrac{ct + (1-d \bar d)}{\varpi^m} & d
  \end{bmatrix}
\]
For brevity, we will let $\Gamma(\gamma, t, m)$ denote the right-hand matrix.
The conjugation by
$\left[ \begin{smallmatrix} \epsilon\inv \\ & \bar\epsilon\inv \\ & & 1 \end{smallmatrix} \right]$
has no effect on any of the $K'_{S, \le r}$, so that we can simply use
\[ \mathbf{1}_{K'_{S, \le r}}(\bar g \inv \gamma g) = \mathbf{1}_{K'_{S, \le r}}(\Gamma(\gamma, t, m)) \]
in the work that follows.
For brevity, we abbreviate
\[ \mathbf{1}_{\le r}(\gamma, t, m) \coloneqq \mathbf{1}_{K'_{S, \le r}}(\Gamma(\gamma, t, m)). \]

\subsubsection{Reparametrizing the integral in terms of $t$ and $m$}
From now on, following \cite{ref:AFL} we always fix the notation
\begin{align*}
  m &= m(t_1) \coloneqq -v(t_1) \\
  n &= n(t) \coloneqq v(1-t\bar t).
\end{align*}
We need to rewrite the integral, phrased originally via $\odif g$,
in terms of the parameters $t$ (hence $n$), $m$, and $\gamma$.
We start by observing that
\[ \det g = t_1 \bar t_1 - t_2 \bar t_2 = t_1 \bar t_1 (1 - t\bar t) \]
which means that
\[ v(\det g) = -2m + n \]
ergo
\begin{align*}
  \left\lvert \det g \right\rvert_F &= q^{-v(\det g)} = q^{2m-n} \\
  \eta(g) &= (-1)^{v(\det g)} = (-1)^n.
\end{align*}
Meanwhile, from $t_2 = t \bar t_1$ we derive
\[ \odif t_2 = \left\lvert t_1 \right\rvert_E \odif t = q^{2m} \odif t. \]

Bringing this all into the orbital integral gives
\begin{align*}
  \Orb(\gamma, \mathbf{1}_{K'_{S, \le r}} s)
  &= \kappa \int_{t, t_1 \in E} \mathbf{1}_{\le r}(\gamma, t, m)
  (-1)^n \left( q^{2m-n} \right)^{s-2} \odif t_1 \cdot (q^{2m} \odif t) \\
  &= \kappa \int_{t, t_1 \in E} \mathbf{1}_{\le r}(\gamma, t, m)
  (-1)^n q^{s(2m-n)} \cdot q^{2n-2m} \odif t \odif t_1.
\end{align*}

\subsection{Description of the support of $\mathbf{1}_{\le r}$ when $n \le 0$}
\begin{claim}
  Whenever $n = 0$ (this requires $v(t) \geq 0$),
  \[
    \mathbf{1}_{\le r}(\gamma, t, m) =
    \begin{cases}
      1 & \text{if } -r \le m \le \delta+r \\
      0 & \text{otherwise.}
    \end{cases}
  \]
\end{claim}
\begin{proof}
  We have to consider the nine entries of $\Gamma(\gamma, t, m)$ in tandem.

  The upper $2 \times 2$ matrix is always in $\omega^{-r}\OO_E$,
  because $v(t) \geq 0$, $v(d) \geq -r$, $v(b) \geq -r$, and $v(a) = 0$ suffices.

  In the right column, since $v(t) \geq 0$ and $n = 0$, the condition is simply $m \ge -r$.

  In the bottom row, we need
  $v\left( c+(1-d\bar d) \bar t \right)-m \geq -r$
  and $v\left( ct +(1-d\bar d) \right)-m \geq -r$.
  If $v(t) > 0$ this is equivalent to $m-r \leq \delta$.
  In the case where $v(t) = 0$ we instead use the observation that
  \begin{equation}
    \left[ c + (1-d \bar d) \bar t \right]
    - \bar t \left[ ct + (1-d \bar d) \right] = (1-t\bar t) c
    \label{eq:ctrick}
  \end{equation}
  which forces at least one of $ct + (1-d \bar d)$ and $c + (1-d \bar d) \bar t$ to
  have valuation $\delta$. So the claim follows now.
\end{proof}

\begin{claim}
  Suppose $n = -2k < 0$, equivalently, $v(t) = -k < 0$, for some $k$.
  \[
    \mathbf{1}_{\le r}(\gamma, t, m) =
    \begin{cases}
      1 & \text{if } -r \le m+k \le \delta+r \\
      0 & \text{otherwise.}
    \end{cases}
  \]
\end{claim}
\begin{proof}
  The proof is similar to the previous claim, but simpler.

  Since $k > 0$, the fraction $\frac{t^2}{1-t \bar t}$ has positive valuation,
  so the upper $2 \times 2$ of $\Gamma(\gamma, t, m)$ is always in $\varpi^{-r}\OO_E$.
  Turning to the right column, the condition reads exactly $m+k \geq -r$.
  Finally, in the bottom row, from $v(t) > 0$ and $v(c) = \delta$
  the condition is simply $-k+\delta-m \geq -r$.
\end{proof}

\subsection{Description of the support of $\mathbf{1}_{\le r}$ when $n > 0$}
In this situation we evaluate over $n > 0$ only.
In this case $t$ is automatically a unit.

\subsubsection{Volume lemma}
The following two lemmas will be useful.

\begin{lemma}
  Let $\xi \in \OO_E^\times$, $\rho \in \ZZ$, and $n \ge \max(\rho, 1)$ an integer.
  Then
  \begin{align*}
    &\Vol\left( \left\{ x \in E \mid v(1-x \bar x) = n,
      \; v(x-\xi) \ge \rho \right\} \right) \\
    &=
    \begin{cases}
      0 & \text{if } v(1-\xi\bar\xi) < \rho \\
      q^{-n}(1-q^{-2}) & \text{if } \rho \le 0 \\
      q^{-(n+\rho)}(1-q\inv) & \text{if } v(1-\xi\bar\xi) \ge \rho \ge 1.
    \end{cases}
  \end{align*}
  \label{lem:volume}
\end{lemma}
\begin{proof}
  The case $\rho > 0$ is proved in \cite[Lemma 4.4]{ref:AFL}.

  When $\rho \le 0$, the condition $v(x - \xi) \ge \rho$ is vacuously true,
  so we just are computing
  $\Vol\left( \left\{ x \in E \mid v(1-x \bar x) = n \right\} \right)$.
  Follows from summing the previous lemma over $\rho = 1$
  and $\xi \in \OO_E / \varpi$ (there are $q_E-1 = q^2-1$ choices for $\xi$).
  \todo{This doesn't actually check out. Check margin of page 235 of notebook.}
\end{proof}

We also comment on the well-known fact that in an ultrametric space,
any two disks are either disjoint or one is contained in the other.
(In other words, the Mastercard logo cannot be drawn. See \Cref{fig:no_mastercard}.)
\begin{proposition}
  Choose $\xi_1, \xi_2 \in E$ and $\rho_1 \geq \rho_2$.
  Consider the two disks:
  \begin{align*}
    D_1 &= \left\{ x \in E \mid v(x-\xi_1) \ge \rho_1 \right\} \\
    D_2 &= \left\{ x \in E \mid v(x-\xi_2) \ge \rho_2 \right\}.
  \end{align*}
  Then, if $v(\xi_1-\xi_2) \geq \rho_2$, we have $D_1 \subseteq D_2$.
  If not, instead $D_1 \cap D_2 = \varnothing$.
  \label{prop:no_mastercard}
\end{proposition}
\begin{proof}
  Because $E$ is an ultrametric space and $\Vol(D_1) \leq \Vol(D_2)$,
  we either have $D_1 \subseteq D_2$ or $D_1 \cap D_2 = \varnothing$.
  The latter condition checks which case we are in by testing if $\xi_1 \in D_2$,
  since $\xi_1 \in D_1$.
\end{proof}
\todo{is it $q$ or $q_E$}

\begin{figure}
\begin{asy}
  defaultpen(fontsize(12pt));
  size(7cm);
  pair O1 = 0.57*dir(-30);
  pair O2 = (0,0);
  real r1 = 0.4;
  real r2 = 1;
  filldraw(circle(O2, r2), rgb(0.9, 0.9, 0.9));
  filldraw(circle(O1, r1), rgb(0.8, 0.8, 0.9));
  pair P1 = O1+r1*dir(230);
  pair P2 = O2+r2*dir( 70);
  draw(O1--P1);
  draw(O2--P2);
  dot("$\xi_1$", O1, dir(90));
  dot("$\xi_2$", O2, dir(180));
  label("$q^{-\rho_1}$", midpoint(O1--P1), dir(P1-O1)*dir(90));
  label("$q^{-\rho_2}$", midpoint(O2--P2), dir(P2-O2)*dir(90));
\end{asy}
\caption{Figure corresponding to \Cref{prop:no_mastercard}.}
\label{fig:no_mastercard}
\end{figure}

We package both of these results together
lemma that will be used repeatedly.
\begin{lemma}
  \label{lem:quadruple_ineq}
  Let $\xi_1, \xi_2 \in \OO_E^\times$ and let $\rho_1 \ge \rho_2$ be integers.
  Also let $n \ge \max(\rho_1, 1)$ be an integer.
  Then the set of points $x \in E$ satisfying all of the equations
  \begin{align*}
    v(x - \xi_1) &\ge \rho_1 \\
    v(x - \xi_2) &\ge \rho_2 \\
    v(1 - x \ol x) &= n
  \end{align*}
  has positive volume if and only if
  \[ v(1 - \xi_1 \bar{\xi_1}) \ge \rho_1, \qquad \rho_2 \le v(\xi_1 - \xi_2). \]
  In that case, the volume is equal to
  \[
    \begin{cases}
      q^{-(n+\rho_1)}(1-q\inv) & \text{if } \rho_1 \ge 1 \\
      q^{-n}(1-q^{-2}) & \text{if } \rho_1 \le 0.
    \end{cases}
  \]
\end{lemma}
In the situation where $\xi_i \notin \OO_E^\times$,
the condition $v(x-\xi_i) = \min(0, v(\xi_i))$ becomes independent of the value of $x$,
and so \Cref{lem:quadruple_ineq} becomes unnecessary
(\Cref{lem:volume} will suffice).
We will deal with this situation when it arises.

\subsubsection{Setup}
Consider the upper $2 \times 2$ matrix of $\Gamma(\gamma, t, m)$.
Using the identities
\begin{align*}
  \dfrac{a-b\bar t+\bar d \bar t^2}{1-t \bar t}
    - \bar t \cdot \dfrac{at-bt\bar t+\bar d \bar t}{1-t \bar t}
    &= a-b\bar t \in \varpi^{-r} \OO_E \\[2ex]
  \dfrac{a-b\bar t+\bar d \bar t^2}{1-t \bar t}
    + \bar t \cdot \dfrac{-at+b-\bar d \bar t}{1-t \bar t}
    &= a \in \varpi^{-r} \OO_E \\[2ex]
  \dfrac{-at+b-\bar d \bar t}{1-t \bar t}
    - \bar t \cdot \dfrac{-at^2+bt-\bar d}{1-t \bar t}
    &= -a+b \in \varpi^{-r} \OO_E,
\end{align*}
it follows that as soon as one entry is in $\varpi^{-r} \OO_E$, they all are.
Meanwhile, the requirements on the other entries amount to
\begin{align}
  m & \geq n - r \\
  v\left( c+(1-d \bar d) \bar t \right) &\geq m-r \label{eq:cddtop} \\
  v\left( ct+(1-d \bar d) \right) &\geq m-r \label{eq:cddbot}
\end{align}
According to the earlier identity \eqref{eq:ctrick},
if \eqref{eq:cddtop} is assumed true,
then \eqref{eq:cddbot} is equivalent to
\[ \delta + v(1-t \bar t) \ge m-r. \]
Meanwhile, since $v(c+(1-d \bar d) \bar t) = v(\bar c + (1-d \bar d)t)$,
\eqref{eq:cddtop} is itself equivalent to
\[ v(t+u) + \delta \geq m-r \]
by reading the definition of \eqref{eq:u}.

Finally, we use a tricky substitution
\[ (2at-b)^2 - (b^2-4a\bar d) = -4a(-at^2+bt-\bar d) \]
to rewrite $v(-at^2+bt-\bar d) \geq n-r$
as $v\left( (2at-b)^2 - (b^2-4a\bar d) \right) \geq n-r$.

In summary:
\begin{claim}
  Assume $t$ is such that $n = v(1-t \bar t) > 0$.
  Then $\mathbf{1}_{\le r}(\gamma, t, m) = 1$ if and only if
  \[ n - r \leq m \leq n + \delta + r \]
  and $t$ lies in the set specified by
  \begin{align*}
    v\left( (2at-b)^2 - (b^2-4a\bar d) \right) &\geq n-r \\
    v(t+u) &\ge m-\delta-r.
  \end{align*}
\end{claim}

\subsubsection{Rewriting the quadratic constraint on the valuation of $t$}
We now analyze the inequality
\begin{equation}
  v\left( (2at-b)^2 - (b^2-4a\bar d) \right) \geq n-r
  \label{eq:2atb_dos}
\end{equation}
and divide it into several (disjoint) possibilities.
Recalling that $\ell = b^2 - 4 a \bar d$, there are three possibilities:
\begin{itemize}
  \ii If $\ell \ge n-r$, then \eqref{eq:2atb_dos} is equvialent to
  \[ 2v(2at-b) \ge n-r
    \iff v\left( t - \frac{b}{2a} \right) \ge \left\lceil \frac{n-r}{2} \right\rceil. \]
  We will further subdivide this into two cases.
  \begin{itemize}
    \ii \textbf{Case 1} is the situation where $\left\lceil \frac{n-r}{2} \right\rceil \ge m - \delta - r$.
    \ii \textbf{Case 2} is the situation where $\left\lceil \frac{n-r}{2} \right\rceil < m - \delta - r$.
  \end{itemize}

  \ii If $\ell < n-r$, then \eqref{eq:2atb_dos} could only hold if $v(2at-b) = \frac{\ell}{2}$.
  Note that in particular, this requires $\ell$ to be even.
  If this happens, then $b^2 - 4 a \bar d$ must be a square and we denote it $\tau^2$.
  Thus, \eqref{eq:2atb_dos} then reads
  \[ v(2at-b+\tau) + v(2at-b-\tau) \ge n-r. \]
  Since we are assuming $n > \ell + r$,
  it must be the case that one of the two factors
  $v(2at-b\mp\tau)$ is equal to $v(\tau) = \ell / 2$ exactly.

  Suppose $v(2at-b+\tau) = \frac{\ell}{2}$, so we need
  \[ v\left( t - \frac{b+\tau}{2a} \right) = v(2at-b-\tau) \ge n - \frac{\ell}{2} - r. \]
  We further subdivide this into two cases:
  \begin{itemize}
    \ii \textbf{Case 3\ts+} is the situation where $n - \frac{\ell}{2} - r > m - \delta - r$.
    \ii \textbf{Case 4\ts+} is the situation where $n - \frac{\ell}{2} - r \le m - \delta - r$.
  \end{itemize}
  Replacing $\tau$ with $-\tau$ above gives us two additional cases
  which we denote \textbf{Case 3\ts-} and \textbf{Case 4\ts-}.
\end{itemize}

This gives us six cases, with each $t \in E$ satisfying at most one of them.
(If $\ell$ is odd, only \textbf{Case 1} and \textbf{Case 2} are used.)
In each case, for a given pair $(n,m)$ we are interested in the volume of $t$
such that two disk inequalities hold together with the assumption $n = v(1-t \bar t)$.

We rewrite these six cases in the format specified by \Cref{lem:quadruple_ineq},
noting that each possibility will actually split into two sub-cases
(although the lemma will only apply in the cases where the centers
$\xi_i$ are actually in $\OO_E^\times$;
this which will be true in the main case $v(b) = v(d) = 0$).
This gives the \Cref{tab:orbital_cases} below;

\begin{table}[ht]
  \centering
  \begin{tabular}{ll cccc cc}
    & Assume & $v(1-\xi_1 \bar{\xi_1})$ & $\rho_1$ and $\rho_2$ & $v(\xi_1-\xi_2)$ & $\xi_1$ & $\xi_2$ \\ \hline
    \textbf{1} & $n \le \ell+r$
        & $v(4-b\bar{b})$
        & $\left\lceil \frac{n-r}{2} \right\rceil \ge m-\delta-r$
        & $v(au - \bar d \bar u)$
        & $\frac{b}{2a}$ & $-u$ \\
    \textbf{2} & $n \le \ell+r$
        & $\lambda$
        & $m-\delta-r > \left\lceil \frac{n-r}{2} \right\rceil$
        & $v(au- \bar d \bar u)$
        & $-u$ & $\frac{b}{2a}$ \\ \hline
    \textbf{3\ts+} & $n > \ell + r$
          & $v(1-\frac{\Norm(b+\tau)}{4})$
          & $n-\frac{\ell}{2}-r > m-\delta-r$
          & $v(au-\bar d \bar u+\tau)$
          & $\frac{b+\tau}{2a}$ & $-u$ \\
    \textbf{3\ts-} & $n > \ell + r$
          & $v(1-\frac{\Norm(b-\tau)}{4})$
          & $n-\frac{\ell}{2}-r > m-\delta-r$
          & $v(au-\bar d \bar u-\tau)$
          & $\frac{b-\tau}{2a}$ & $-u$ \\
    \textbf{4\ts+} & $n > \ell + r$
          & $v(1-\frac{\Norm(b+\tau)}{4})$
          & $m-\delta-r \ge n-\frac{\ell}{2}-r$
          & $v(au-\bar d \bar u+\tau)$
          & $-u$ & $\frac{b+\tau}{2a}$ \\
    \textbf{4\ts-} & $n > \ell + r$
          & $v(1-\frac{\Norm(b+\tau)}{4})$
          & $m-\delta-r \ge n-\frac{\ell}{2}-r$
          & $v(au-\bar d \bar u-\tau)$
          & $-u$ & $\frac{b-\tau}{2a}$ \\
  \end{tabular}
  \caption{The six cases for calculating the orbital integral.}
  \label{tab:orbital_cases}
\end{table}
Note that in generating this table, we did the calculations
\begin{align*}
  v\left( u + \frac{b}{2a} \right)
  &= v\left( \frac{au-\bar d \bar u}{2a} \right)
  = v(au - \bar d \bar u) \\
  v\left( u + \frac{b \pm \tau}{2} \right)
  &= v(au - \bar d \bar u \pm \tau).
\end{align*}
to populate the entries for $v(\xi_1 - \xi_2)$,
as well as the identity
\[ 1 - \frac{b \pm \tau}{2a} \cdot \frac{\bar b \pm \bar \tau}{2\bar a}
= \frac{4 - \Norm(b \pm \tau)}{4} \]
to calculate $v(1-\xi_1\bar{\xi_1})$ entries in the latter four cases.

\subsubsection{Analysis of Case 1 and 2 assuming $n > 0$ and $v(b) = v(d) = 0$.}
We analyze Case 1 and 2 assuming $v(b) = v(d) = 0$.

Considering $n > 0$ and $n-r \le m \le n+\delta+r$ as fixed,
we compute the volume of the set of $t$
for which $n = v(1-t\bar t)$ and $\mathbf{1}_{\le r}(\gamma,t,m) = 1$.

In addition to the constraint $n \le \ell + r$,
we see that the two cases have the following additional requirements:
\begin{description}
  \ii[Case 1] If $m < \left\lceil \frac{n-r}{2} \right\rceil + \delta + r$ then we need
  \begin{align}
    v(4-b\bar b) &\ge \left\lceil \frac{n-r}{2} \right\rceil \label{eq:odd_ineq1} \\
    v(au - \bar d \bar u) &\ge m - \delta - r \label{eq:odd_ineq2}.
  \end{align}

  \ii[Case 2] If $m \geq \left\lceil \frac{n-r}{2} \right\rceil + \delta + r$ then we need
  \begin{align}
    \lambda = v(1-u \bar u) &\ge m-\delta-r \label{eq:odd_ineq3} \\
    v(au - \bar d \bar u) &\ge \left\lceil \frac{n-r}{2} \right\rceil \label{eq:odd_ineq4}.
  \end{align}
\end{description}

We will now show that some of these inequalities are redundant and can be ignored.

\begin{claim}
  \eqref{eq:odd_ineq2} and \eqref{eq:odd_ineq4} are redundant
  i.e.\ they are automatically true for $0 < n \le \ell + r$.
\end{claim}
\begin{proof}
  First, assume that $\ell$ is odd.
  Then \eqref{eq:dos} together with $v(b) = v(d) = 0$ gives
  \begin{equation}
    \lambda = \ell = v(1-u \bar u) < 2v(au - \bar d \bar u).
    \label{eq:odd_center_distance}
  \end{equation}
  On the other hand, if $\ell$ is even, then we have instead
  \begin{equation}
    \ell = 2v(au - \bar d \bar u) < v(1-u\bar u) = \lambda
    \label{eq:even_center_distance_case12}
  \end{equation}
  Thus, regardless of the parity of $\ell$, we always have
  \[ v(au - \bar d \bar u) \ge \frac{\ell}{2} \ge \frac{n-r}{2}. \qedhere \]
\end{proof}

\begin{claim}
  \eqref{eq:odd_ineq1} is redundant.
\end{claim}
\begin{proof}
  The equation
  \[ (4-b \bar b) = -4au(1-d\bar d) - \bar b(b^2-4a\bar d) \]
  implies
  \begin{equation}
    v(4-b\bar b) \ge \min(\ell,\delta) \text{ with equality if } \ell \neq \delta.
    \label{eq:bb_odd}
  \end{equation}
  Hence, a priori \eqref{eq:bb_odd} suggests that we have a condition
  $n \le r + 2 \delta$ in addition to $n \le r + \ell$.
  However, by \Cref{prop:parameter_constraints}, we always have $\ell \le 2 \delta$,
  and consequently \eqref{eq:odd_ineq1} is redundant as well.
\end{proof}

Putting all of this together, we find that the valid pairs $(n,m)$ come in two cases.

\begin{description}
\item[Double sum for Case 1]
We sum over $(m,n)$ such that
\begin{equation}
  \begin{aligned}
    1 &\leq n \leq \ell + r, \\
    n-r &\leq m \leq \left\lceil \frac{n-r}{2} \right\rceil+\delta+r - 1
  \end{aligned}
  \label{eq:odd_range1}
\end{equation}
where each $(m,n)$ gives a volume contribution of
\[
  \begin{cases}
    q^{-n - \left\lceil \frac{n-r}{2} \right\rceil} \left( 1 - q\inv \right)
      & \text{if $n > r$} \\
    q^{-n} \left( 1 - q^{-2} \right)
      & \text{if $n \leq r$}.
  \end{cases}
\]

\item[Double sum for Case 2]
We sum over $(m,n)$ such that
\begin{equation}
  \begin{aligned}
    1 &\leq n \leq \ell + r, \\
    \max\left(n-r, \left\lceil \frac{n-r}{2} \right\rceil+\delta+r \right)
    &\leq m \leq \min(n,\lambda)+\delta+r.
  \end{aligned}
  \label{eq:odd_range2}
\end{equation}
where each $(m,n)$ gives a volume contribution of
\[
  \begin{cases}
    q^{-n - (m-\delta-r)} \left( 1 - q\inv \right)
      & \text{if $m > \delta + r$} \\
    q^{-n} \left( 1 - q^{-2} \right)
      & \text{if $m \le \delta + r$}.
  \end{cases}
\]
Notice that $m \leq \delta + r$ could only occur when $n \leq r$.
\end{description}

\subsubsection{Analysis of Case 3 and 4 assuming $n > 0$ and $v(b) = v(d) = 0$.}
Suppose $\ell \geq 0$ is even.
Using the identity
\[ (au-\bar d \bar u)^2 - \tau^2 = 4a\bar d(1- u\bar u) \]
we agree now to fix the choice of the square root of $\tau$ such that
\begin{equation}
  v(au-\bar d \bar u + \tau) = \lambda - \half \ell \quad\text{and}\quad
  v(au-\bar d \bar u - \tau) = \half \ell.
  \label{eq:tau_choice}
\end{equation}

As before we consider $n > 0$ and $n-r \le m \le n+\delta+r$ as fixed,
and seek to compute the volume of the set of $t$
for which $n = v(1-t\bar t)$ and $\mathbf{1}_{\le r}(\gamma,t,m) = 0$.

From $v(b) = v(d) = 0$ and \eqref{eq:dos}, we have
\[ \ell = 2v(\tau) = 2v(au - \bar d \bar u) < \lambda. \]
This lets us invoke \cite[Lemma 4.7]{ref:AFL} to evaluate $v(4-\Norm(b \pm \tau))$:
we have the calculation
\begin{align*}
  v\left( 4 - \Norm(b+\tau) \right) &= \lambda + \delta - \ell \\
  v\left( 4 - \Norm(b-\tau) \right) &= \delta.
\end{align*}

\begin{description}
\ii[Double sum for Case 3\ts+]
Suppose $n > \ell + r$,
$m < n - \frac{\ell}{2} + \delta$, and we choose $\frac{b+\tau}{2a}$.
Then \Cref{lem:quadruple_ineq} gives a nonzero contribution if and only if
\begin{align*}
  \lambda + \delta - \ell = v(4-\Norm(b+\tau)) &\geq n - \frac{\ell}{2} - r \\
  \lambda - \frac{\ell}{2} = v(au-\bar d \bar u + \tau) & \geq m - \delta - r.
\end{align*}
Compiling all seven constraints gives that the valid pairs $(m,n)$ are those for which
\begin{align*}
  \max(1, \ell+r+1) &\leq n \leq -\frac{\ell}{2} + \delta + \lambda + r, \\
  n-r &\leq m \leq \min\left( n+\delta+r, n - \frac{\ell}{2}+\delta - 1,
    \lambda-\frac{\ell}{2}+\delta+r \right)
\end{align*}
which from $\ell, \delta, r \ge 0$ can be simplified to just
\begin{equation}
  \begin{aligned}
    \ell+r+1 &\leq n \leq  -\frac{\ell}{2} + \delta + \lambda + r, \\
    n-r &\leq m \leq \min(n-1, \lambda-r) - \frac{\ell}{2} + \delta
  \end{aligned}
  \label{eq:even_case3_plus}
\end{equation}
Each $(m,n)$ gives a volume contribution of
\[ q^{-n - (n - \frac{\ell}{2} - r)} \left( 1 - q\inv \right). \]

\ii[Double sum for Case 3\ts-]
Suppose $n > \ell + r$,
$m < n - \frac{\ell}{2} + \delta$, and we choose $\frac{b-\tau}{2a}$.
Then \Cref{lem:quadruple_ineq} gives a nonzero contribution if and only if
\begin{align*}
  \delta = v(4-\Norm(b-\tau)) &\geq n - \frac{\ell}{2} - r \\
  \frac{\ell}{2} = v(au-\bar d \bar u - \tau) & \geq m - \delta - r.
\end{align*}
Compiling all seven constraints gives that the valid pairs $(m,n)$ are those for which
\begin{align*}
  \ell + r + 1 &\leq n \leq \frac{\ell}{2}+\delta+r, \\
  n-r &\leq m \leq \min\left( n - \frac{\ell}{2}+\delta - 1,
    \frac{\ell}{2} + \delta + r, n + \delta + r \right)
\end{align*}
This simplifies to
\begin{equation}
  \begin{aligned}
    \ell+r+1 &\leq n \leq \frac{\ell}{2}+\delta+r, \\
    n-r &\leq m \leq \frac{\ell}{2} + \delta + r.
  \end{aligned}
  \label{eq:even_case3_minus}
\end{equation}
As in the previous case, $(m,n)$ gives a volume contribution of
\[ q^{-n - (n - \frac{\ell}{2} - r)} \left( 1 - q\inv \right). \]

\ii[Double sum for Case 4\ts+]
Suppose $n > \ell + r$,
$m \ge n - \frac{\ell}{2} + \delta$
(which implies $m \ge n-r$ since $r \ge 0$ and $0 \le \ell \le 2 \delta$),
and we choose $\frac{b+\tau}{2a}$.
Then \Cref{lem:quadruple_ineq} gives a nonzero contribution if and only if
\begin{align*}
  \lambda &\geq m - \delta - r \\
  \lambda - \frac{\ell}{2} = v(au-\bar d \bar u + \tau) & \geq n - \frac{\ell}{2} - r.
\end{align*}
Rearranging gives that the valid pairs $(m,n)$ are those for which
\begin{equation}
  \begin{aligned}
    \ell + r + 1 &\leq n \leq \lambda + r \\
    n - \frac{\ell}{2} + \delta &\leq m \leq \min(n, \lambda) + \delta + r.
  \end{aligned}
  \label{eq:even_case4_plus}
\end{equation}
Here, each $(m,n)$ gives a volume contribution of
\[ q^{-n - (m - \delta - r)} \left( 1 - q\inv \right). \]

\ii[Double sum for Case 4\ts-]
Suppose $n > \ell + r$,
$m \ge n - \frac{\ell}{2} + \delta$, and we choose $\frac{b-\tau}{2a}$.
Then \Cref{lem:quadruple_ineq} gives a nonzero contribution if and only if
\begin{align*}
  \lambda &\geq m - \delta - r \\
  \frac{\ell}{2} = v(au-\bar d \bar u - \tau) & \geq n - \frac{\ell}{2} - r.
\end{align*}
The latter inequality contradicts the assumption that $n > \ell + r$,
so in fact this case can never occur.
\end{description}

\section{Evaluation of the orbital integral for $S_3(F)$}
\label{sec:orbital2}

We now put together the sums we found in the previous section
to come up with the expression for the orbital integral.

\subsection{Region where $n \leq 0$ for all values of $\ell$}
\begin{proposition}
  The contribution to the integral $\Orb(\gamma, \mathbf{1}_{K'_{S, \le r}}, s)$ over $n \leq 0$ is exactly
  \[ I_{n \le 0} \coloneqq q^{2(\delta+r)s} \sum_{j=0}^{\delta+2r} q^{-2js}
    = q^{-2rs} + \dots + q^{2(\delta+r)s}. \]
\end{proposition}
\begin{proof}
  For $n = 0$ we get a contribution of
  \begin{align*}
    &\phantom= \kappa \int_{t, t_1 \in E} \mathbf{1}[n=0] \mathbf{1}_{\le r}(\gamma,t,m)
      q^{2s \cdot m} q^{-2m} \odif t \odif t_1 \\
    &= \kappa \Vol(t: n =0) \sum_{m=-r}^{\delta+r} \Vol(t_1: -v(t_1)=m) q^{2m(s-1)} \\
    &= \kappa \left( 1 - \frac{q+1}{q^2} \right) \sum_{m=-r}^{\delta+r}
    \left( q^{2m} \left( 1-q^{-2} \right) \right) q^{2m(s-1)} \\
    &= \kappa \left( 1 - \frac{q+1}{q^2} \right) \left( 1-q^{-2} \right)
    \sum_{m=-r}^{\delta+r} q^{2ms}.
  \end{align*}
  For the region where $v(t) = -k < 0$, for each individual $k > 0$,
  \begin{align*}
    &\phantom= \kappa \int_{t, t_1 \in E} \mathbf{1}[v(t)=-k] \mathbf{1}_{\le r}(\gamma,t,m)
      q^{s(2m-n)} q^{2n-2m} \odif t \odif t_1 \\
    &= \kappa \Vol(t: v(t)=-k) \sum_{m=-r-k}^{\delta+r-k}
      \Vol(t_1: -v(t_1)=m) q^{s(2m+2k)-4k-2m} \\
    &= \kappa q^{2k} \left( 1 - q^{-2} \right) \sum_{m=-r-k}^{\delta+r-k}
      \left( q^{2m} \left( 1-q^{-2} \right) \right) q^{s(2m+2k)-4k-2m} \\
    &= \kappa q^{-2k} \left( 1 - q^{-2} \right)^2
      \sum_{m=-r-k}^{\delta+r-k} q^{2(m+k)s} \\
    &= \kappa q^{-2k} \left( 1 - q^{-2} \right)^2 \sum_{i=-r}^{\delta+r} q^{2is}.
  \end{align*}
  Since $\sum_{k > 0} q^{-2k} = \frac{q^{-2}}{1-q^{-2}}$,
  we find that the total contribution across both
  the $n=0$ case and the $k > 0$ case is
  \begin{align*}
    &\phantom= \left( \left( 1 - \frac{q+1}{q^2} \right) \left( 1-q^{-2} \right)
      + q^{-2}(1-q^{-2})  \right) \kappa \sum_{i=-r}^{\delta+r} q^{2is} \\
    &= \left( 1-q\inv \right) \left(1-q^{-2} \right)
      \kappa \sum_{i=-r}^{\delta+r} q^{2is} \\
    &= \sum_{i=-r}^{\delta+r} q^{2is}.
  \end{align*}
  This equals the claimed sum above.
  (We write it over $0 \le j \le \delta+2r$ for consistency with a later part.)
\end{proof}

\subsection{Contribution from Case 1 and Case 2 assuming $v(b)=v(d)=0$}
Again using $\Vol(t_1:-v(t_1)=m) = q^{2m} (1-q^{-2})$,
summing all the cases gives the following contribution within
$\kappa \int_{t, t_1 \in E} \mathbf{1}[n > 0] \mathbf{1}_{\le r}(\gamma,t,m)$:
\begin{align*}
  I_{n > 0}^{\text{1+2}}
  % -----------------------------------------------------------
  &\coloneqq \kappa \sum_{n=1}^{r}
    \sum_{m=n-r}^{\left\lceil \frac{n-r}{2} \right\rceil+\delta+r-1}
    q^{-n} \left( 1 - q^{-2} \right) \\
    &\qquad\qquad\cdot \Big( (-1)^n q^{s(2m-n)} q^{2n-2m} \Big) \Big( q^{2m}(1-q^{-2}) \Big) \\
  &\qquad+ \kappa \sum_{n=r+1}^{\ell+r}
    \sum_{m=n-r}^{\left\lceil \frac{n-r}{2} \right\rceil+\delta+r-1}
    q^{-n - \left\lceil \frac{n-r}{2} \right\rceil} \left( 1 - q\inv \right) \\
    &\qquad\qquad\cdot \Big( (-1)^n q^{s(2m-n)} q^{2n-2m} \Big) \Big( q^{2m}(1-q^{-2}) \Big) \\
  &\qquad+ \kappa \sum_{n=1}^{r}
    \sum_{m=\max\left(n-r, \left\lceil \frac{n-r}{2} \right\rceil+\delta+r\right)}^{\delta+r}
    q^{-n} \left( 1 - q^{-2} \right) \\
    &\qquad\qquad\cdot \Big( (-1)^n q^{s(2m-n)} q^{2n-2m} \Big) \Big( q^{2m}(1-q^{-2}) \Big) \\
    \\
  &\qquad+ \kappa \sum_{n=1}^{\ell+r}
    \sum_{m=\max\left(n-r, \left\lceil \frac{n-r}{2} \right\rceil+\delta+r, \delta+r+1 \right)}^{\min(n,\lambda)+\delta+r}
    q^{-n - (m-\delta-r)} \left( 1 - q\inv \right) \\
    &\qquad\qquad\cdot \Big( (-1)^n q^{s(2m-n)} q^{2n-2m} \Big) \Big( q^{2m}(1-q^{-2}) \Big)
    \\
  % -----------------------------------------------------------
  &= \sum_{n=1}^{r}
    \sum_{m=n-r}^{\left\lceil \frac{n-r}{2} \right\rceil+\delta+r-1}
    q^{n} \left( 1 + q\inv \right)
      \cdot (-1)^n q^{s(2m-n)} \\
  &\qquad+ \sum_{n=r+1}^{\ell+r}
    \sum_{m=n-r}^{\left\lceil \frac{n-r}{2} \right\rceil+\delta+r-1}
    q^{\left\lfloor \frac{n+r}{2} \right\rfloor}
      \cdot (-1)^n q^{s(2m-n)} \\
  &\qquad+ \sum_{n=1}^{r}
    \sum_{m=\max\left(n-r, \left\lceil \frac{n-r}{2} \right\rceil+\delta+r\right)}^{\delta+r}
    q^{n} \left( 1 + q\inv \right)
      \cdot (-1)^n q^{s(2m-n)} \\
  &\qquad+ \sum_{n=1}^{\ell+r}
    \sum_{m=\max\left(n-r, \left\lceil \frac{n-r}{2} \right\rceil+\delta+r, \delta+r+1 \right)}^{\min(n,\lambda)+\delta+r}
    q^{n - (m-\delta-r)}
      \cdot (-1)^n q^{s(2m-n)}.
\end{align*}
To simplify the expressions, we replace the summation variable $m$ with
\[ j \coloneqq (n + \delta + r) - m \geq 0. \]
In that case,
\[ 2m-n = 2(\delta+n+r-j)-n = n + 2\delta + 2r - 2j. \]
Then the expression rewrites as
\begin{align*}
  I_{n > 0}^{\text{1+2}}
  % -----------------------------------------------------------
  &= \sum_{n=1}^{r}
    \sum_{j = \left\lfloor \frac{n+r}{2} \right\rfloor+ 1}^{\delta + 2r}
    q^{n} \left( 1 + q\inv \right) \cdot (-1)^n q^{s(n+2\delta+2r-2j)} \\
  &\qquad+ \sum_{n=r+1}^{\ell+r}
    \sum_{j = \left\lfloor \frac{n+r}{2} \right\rfloor+ 1}^{\delta + 2r}
    q^{\left\lfloor \frac{n+r}{2} \right\rfloor} \cdot (-1)^n q^{s(n+2\delta+2r-2j)} \\
  &\qquad+ \sum_{n=1}^{r}
    \sum_{j=n}^{\min\left( \delta+2r, \left\lfloor \frac{n+r}{2} \right\rfloor \right)}
    q^{n} \left( 1 + q\inv \right) \cdot (-1)^n q^{s(n+2\delta+2r-2j)} \\
  &\qquad+ \sum_{n=1}^{\ell+r}
    \sum_{j=\max(0,n-\lambda)}^{\min(\delta+2r, \left\lfloor \frac{n+r}{2} \right\rfloor, n-1)}
    q^{j} \cdot (-1)^n q^{s(n+2\delta+2r-2j)}.
\end{align*}

We interchange the order of summation so that it is first over $j$ and then $n$.
There are four double sums to interchange.

\begin{itemize}
  \ii The first double sum runs from $j=\left\lfloor \frac{r+1}{2} \right\rfloor+1$ to $j=\delta+2r$.
  In addition to $1 \le n \le r$,
  we need $\left\lfloor \frac{n+r}{2} \right\rfloor + 1 \leq j$,
  which solves to $\frac{n+r}{2} \leq j-\half$ or $n \leq 2j-1-r$.
  Thus the condition on $n$ is
  \[ 1 \leq n \leq \min(2j-1-r, r). \]

  \ii The second double sum runs from $j=r+1$ to $\delta+2r$.
  We also need $r+1 \le n \le \ell+r$ and $n \le 2j-1-r$.
  Hence, the desired condition on $n$ is
  \[ r+1 \leq n \leq \min(2j-1-r, \ell+r). \]

  \ii The third double sum runs from $j=1$ to $j=r$.
  Meanwhile, the values of $n$ need to satisfy $1 \le n \le r$, $n \leq j$
  and $j \leq \left\lfloor \frac{n+r}{2} \right\rfloor \implies n \geq 2j-r$,
  consequently we just obtain
  \[ \max(1, 2j-r) \leq n \leq j. \]

  \ii The fourth double sum runs $j=0$ to
  \[ j=\min\left( \delta+2r, \left\lfloor \frac{\ell}{2} \right\rfloor + r, \ell+r-1 \right)
    = \left\lfloor \frac{\ell}{2} \right\rfloor + r - \mathbf{1}[\ell = 0] \]
  again because of $\ell < 2\delta$.
  Meanwhile, we require $1 \le n \le \ell+r$, $j \ge n-\lambda$, $j \le n-1$,
  as well as $j \le \left\lfloor \frac{n+r}{2} \right\rfloor
  \iff n \ge 2j-r$.
  Putting these four conditions together gives
  \[ \max(j+1, 2j-r) \le n \le \min(\lambda+j, \ell+r). \]
\end{itemize}
Hence we get
\begin{align*}
  I_{n > 0}^{\text{1+2}}
  &= \sum_{j = \left\lfloor \frac{r+1}{2} \right\rfloor+ 1}^{\delta+2r}
    \sum_{n=1}^{\min(2j-1-r, r)}
    q^{n} \left( 1 + q\inv \right) \cdot (-1)^n q^{s(n+2\delta+2r-2j)} \\
  &\qquad+ \sum_{j = r+1}^{\delta + 2r}
    \sum_{n=r+1}^{\min(2j-1-r, \ell+r)}
    q^{\left\lfloor \frac{n+r}{2} \right\rfloor} \cdot (-1)^n q^{s(n+2\delta+2r-2j)} \\
  &\qquad+ \sum_{j=1}^{r}
    \sum_{n=\max(1,2j-r)}^{j}
    q^{n} \left( 1 + q\inv \right) \cdot (-1)^n q^{s(n+2\delta+2r-2j)} \\
  &\qquad+ \sum_{j=0}^{\left\lfloor \frac{\ell}{2} \right\rfloor + r - \mathbf{1}[\ell = 0]}
    \sum_{n=\max(j+1, 2j-r)}^{\min(\lambda+j, \ell+r)}
    q^{j} \cdot (-1)^n q^{s(n+2\delta+2r-2j)}.
\end{align*}

At this point, we can unify the sum over $j$ by noting that for $j$ outside of the
summation range, the inner sum is empty anyway.
Specifically, note that:
\begin{itemize}
  \ii In the first and second double sum,
  the inner sum over $n$ is empty anyway when $j < r$.
  \ii In the third double sum, adding $j=0$ does not introduce new terms.
  Moreover, when $j > r$ the inner sum over $n$ is also empty anyway.
  \ii In the fourth double sum,
  \begin{itemize}
    \ii If $\ell = 0$ and $j \ge r$, then $j+1 \ge 0+r$; and
    \ii If $\ell > 0$ and $j > \frac{\ell}{2} + r$, then $2j-r \ge \ell+r$.
  \end{itemize}
  So no new terms are introduced in this case either.
\end{itemize}
So we can unify all four double sums to run over $0 \le j \le \delta + 2r$,
simplifying the expression to just

\begin{align*}
  I_{n > 0}^{\text{1+2}}
  = q^{2(\delta+r)s}
  \sum_{j = 0}^{\delta + 2r} \Bigg(
    & \sum_{n=1}^{\min(2j-1-r, r)}
      q^{n} \left( 1 + q\inv \right) \cdot (-1)^n q^{s(n-2j)} \\
    & + \sum_{n=r+1}^{\min(2j-1-r, \ell+r)}
      q^{\left\lfloor \frac{n+r}{2} \right\rfloor} \cdot (-1)^n q^{s(n-2j)} \\
    &+ \sum_{n=\max(1,2j-r)}^{j}
      q^{n} \left( 1 + q\inv \right) \cdot (-1)^n q^{s(n-2j)} \\
    &+ \sum_{n=\max(j+1, 2j-r)}^{\min(\lambda+j, \ell+r)} q^{j} \cdot (-1)^n q^{s(n-2j)} \Bigg).
\end{align*}

\subsection{Merging of $I_{n \le 0}$ with $I_{n > 0}^{\text{1+2}}$}
We continue assuming $v(b) = v(d) = 0$.
It turns out that $I_{n \le 0} + I_{n > 0}^{\text{1+2}}$ can be rewritten more compactly
(giving a simple answer especially when $\ell$ is odd).
Then
\[
  I_{n \le 0} + I_{n > 0}^{\text{1+2}}
\]
can be rewritten to the collation
\begin{align*}
  = q^{2(\delta+r)s}
  \sum_{j = 0}^{\delta + 2r} \Bigg(
    q^{-2js}
    &+ \sum_{n=1}^{\min(2j-1-r, r)}
      q^{n} \left( 1 + q\inv \right) \cdot (-1)^n q^{s(n-2j)} \\
    &+ \sum_{n=r+1}^{\min(2j-1-r, \ell+r)}
      q^{\left\lfloor \frac{n+r}{2} \right\rfloor} \cdot (-1)^n q^{s(n-2j)} \\
    &+ \sum_{n=\max(1,2j-r)}^{j}
      q^{n} \left( 1 + q\inv \right) \cdot (-1)^n q^{s(n-2j)} \\
    &+ \sum_{n=\max(j+1, 2j-r)}^{\min(\lambda+j, \ell+r)} q^{j} \cdot (-1)^n q^{s(n-2j)} \Bigg).
\end{align*}
Note that when $r=0$ and $\ell = \lambda \equiv 1 \pmod 2$
we recover \cite[equation (4.13)]{ref:AFL}.

The above expression can be considered as a Laurent polynomial in $-q^s$,
whose coefficients are nonnegative polynomials in $q$ (note that $(-1)^n = (-1)^{n-2j}$).
Now we are going to extract the coefficient of $(-q^s)^k$, for each integer $k$.
First, note that
\begin{itemize}
  \ii The initial term before the sums adds $1$
  if $k$ is even and $-2r \le k \le 2\delta + 2r$, and $0$ otherwise.
\end{itemize}
We move on to the inner sums and calculate their contributions.
For a fixed $k \in \ZZ$, we want to consider $(n,j)$ with $n-2j + 2(\delta+r) = k$,
that is, $2j = n + 2\delta + 2r - k$, or $n = 2j + k - 2\delta - 2r$.
The condition that $j \in \ZZ$ and $0 \le j \le \delta+2r$ is then equivalent to
\begin{equation}
  k-2\delta-2r \le n \le k+2r \qquad\text{and}\qquad n \equiv k \pmod 2.
  \label{eq:outer_assumption}
\end{equation}
We note also that
\begin{equation}
  n < 2j-r \iff n < n + 2 \delta + r - k \iff k < 2\delta + r
  \label{eq:outer_half_assumption_12}
\end{equation}
which needs to hold for the first two sums to contribute.
Conversely, in the latter two sums, we will assume that
\begin{equation}
  n \ge 2j-r \iff k \ge 2\delta + r.
  \label{eq:outer_half_assumption_34}
\end{equation}
Now we are ready for the main calculation.
In what follows $i \% 2 \in \{0,1\}$ means the remainder when $i$ is divided by $2$.
Moreover, any ellipses of the form $q^i + \dots + q^{i'}$
will be abbreviate for $q^i + q^{i-1} + \dots + q^{i'}$
(i.e.\ within any ellipses, the exponents are understood to decrease by $1$,
and the sums are always nonempty, meaning $i \ge i'$).

In the region where $k < 2\delta + r$, the first two sums contribute:
\begin{itemize}
  \ii The first sum contributes if and only if \eqref{eq:outer_assumption} holds,
  $1 \le n \le r$ and \eqref{eq:outer_half_assumption_12} is true.
  Hence, the contribution only occurs when $k < 2\delta + r$.
  In that case, all $1 \le n \le \min(r,k+2r)$ with $n \equiv k \pmod 2$ appear.
  Since the contribution of a given $n$ is $q^n + q^{n-1}$
  and the $n$ are incrementing by $2$, our final total is
  \[
    \begin{cases}
      q^{k+2r} + \dots + q^{(k-1)\%2} & \text{if } -2r < k \leq -r \\
      q^{r-(k-r)\%2} + \dots + q^{(k-1)\%2} & \text{if } -r \le k < 2\delta + r \\
      0 & \text{otherwise}.
    \end{cases}
  \]

  \ii The second sum contributes if and only if \eqref{eq:outer_assumption} holds,
  \eqref{eq:outer_half_assumption_12} holds and $r+1 \le n \le \ell + r$.
  The hypothesis $n > r$ means we need $k \geq -r$.
  Since $\ell < 2\delta$, the upper bound for $n$ is $n \le \min(\ell+r, k+2r)$
  which we split into two cases.

  In the case where $k+2r \le \ell+r$, then since $\ell < 2\delta$,
  the inequality $k < 2\delta + r$ holds automatically.
  We have the largest term $n=k+2r \equiv k \pmod 2$,
  so the largest exponent $q$ that appears is
  $\left\lfloor \frac{(k+2r)+r}{2} \right\rfloor$.

  In the other case $\ell+r \le k+2r$,
  we obtain largest exponents of
  \[ \left\lfloor \frac{(\ell+r-(\ell+r-k)\%2) + r}{2} \right\rfloor
    = r + \left\lfloor \frac{\ell-(\ell+r-k)\%2}{2} \right\rfloor.
  \]
  Thus, we obtain
  \[
    \begin{cases}
      q^{\left\lfloor \frac{k+3r}{2} \right\rfloor} + \dots + q^{r+(r+1-k)\%2}
        & \text{if } -r \le k \le \ell - r \\
      q^{r + \left\lfloor \frac{\ell-(\ell+r-k)\%2}{2} \right\rfloor}
        + \dots + q^{r + (r+1-k)\%2}
        & \text{if } \ell - r \leq k < 2\delta + r \\
      0 & \text{otherwise}.
    \end{cases}
  \]
\end{itemize}

In the region where $k \ge 2\delta + r$, the latter two sums are in play:
\begin{itemize}
  \ii In the third sum, we assume \eqref{eq:outer_half_assumption_34};
  then the other constraints on $n$ are
  \begin{align*}
    n &\ge 1 \\
    n \le j \iff 2n \le n+2\delta+2r-k \iff n &\le 2\delta + 2r - k
  \end{align*}
  which implies $k < 2 \delta + 2r$ for this range to be nonempty.
  In this case, \eqref{eq:outer_assumption} is actually redundant already.
  That means our contribution can be described as
  \[
    \begin{cases}
      q^{2\delta+2r-k} + \dots + q^{(k-1)\%2} & \text{if } 2\delta+r \le k < 2\delta+2r \\
      0 & \text{otherwise}.
    \end{cases}
  \]
  \ii Unlike the other sums, the $j$ is in the exponent in the fourth sum,
  so \eqref{eq:outer_assumption} will not be useful to us.
  Instead our goal is to detect the values of $j$ for which the corresponding value of
  \[ n = 2j-2\delta-2r+k \]
  lies in the desired interval.
  That is, we get a contribution of $q^j$ if and only if
  \eqref{eq:outer_half_assumption_34} holds and
  \begin{align*}
    &\phantom{\iff} 0 \le j \le \delta + 2r \\
    j < 2j-2\delta-2r+k &\implies 2\delta+2r-k < j \\
    2j-2\delta-2r+k \le \lambda+j &\iff j \le 2\delta+2r+\lambda-k \\
    2j-2\delta-2r+k \le \ell+r &\iff j \le \delta+\frac{3r+\ell-k}{2}
  \end{align*}
  The values of $k$ for which there is any valid index $j$ is given by
  \[ 2\delta + r \le k \le 2 \delta + \min \left( \lambda + 2r, \ell + 3r \right). \]
  The breakpoint for the two upper bounds on $j$ occurs when
  \[ 2\delta+2r-k+\lambda \le \delta+\frac{3r+\ell-k}{2}
    \iff k \ge 2\delta + 2\lambda - \ell + r. \]

  Comparing all the bounds, we find there are three possible scenarios.
  \begin{itemize}
    \ii If $\lambda \le \frac{\ell+r}{2}$, then we get
    \[
      \begin{cases}
        q^{\delta+\left\lfloor\frac{3r+\ell-k}{2}\right\rfloor} + \dots + q^{2\delta+2r-k+1}
          & \text{if } 2\delta+r \le k \le 2\delta+2r  \\
        q^{\delta+\left\lfloor\frac{3r+\ell-k}{2}\right\rfloor} + \dots + q^{0}
          & \text{if } 2\delta+2r < k \le 2\delta+2\lambda-\ell+r  \\
        q^{2\delta+2r+\lambda-k} + \dots + q^0
          & \text{if } 2\delta+2\lambda-\ell+r \le k \le 2\delta+\lambda+2r \\
        0 & \text{otherwise}.
      \end{cases}
    \]

    \ii If $\frac{\ell+r}{2} < \lambda \le \ell + r$, then we get
    \[
      \begin{cases}
        q^{\delta+\left\lfloor\frac{3r+\ell-k}{2}\right\rfloor} + \dots + q^{2\delta+2r-k+1}
          & \text{if } 2\delta+r \le k \le 2\delta+2\lambda-\ell+r  \\
        q^{\delta+\left\lfloor\frac{3r+\ell-k}{2}\right\rfloor} + \dots + q^{2\delta+2r-k+1}
          & \text{if } 2\delta+2\lambda-\ell+r \le k \le 2\delta+2r  \\
        q^{2\delta+2r+\lambda-k} + \dots + q^0
          & \text{if } 2\delta+2r < k \le 2\delta+\lambda+2r \\
        0 & \text{otherwise}.
      \end{cases}
    \]

    \ii If $\ell+r < \lambda$, then we get
    \[
      \begin{cases}
        q^{\delta+\left\lfloor\frac{3r+\ell-k}{2}\right\rfloor} + \dots + q^{2\delta+2r-k+1}
          & \text{if } 2\delta+r \le k \le 2\delta+2r \\
        q^{\delta+\left\lfloor\frac{3r+\ell-k}{2}\right\rfloor} + \dots + q^{0}
          & \text{if } 2\delta+2r < k \le 2\delta+\ell+3r \\
        0 & \text{otherwise}.
      \end{cases}
    \]
  \end{itemize}
\end{itemize}
This completes the analysis of the four sums above.
For later purposes, it will be more symmetric to rewrite the exponent as
\[ \delta + \left\lfloor \frac{3r+\ell-k}{2} \right\rfloor
  = r + \left\lfloor \frac{(2\delta+\ell+r)-k}{2} \right\rfloor. \]

Now we can piece together all the parts below.
It turns out that for every value of $k$,
the coefficient of $(-q^s)^k$ is an expression
of the form $1 + q + q^2 + \dots + q^{\nn_\gamma(k)}$ for some $k$.
Indeed,
\begin{itemize}
  \ii When $k = -2r$ the only term is $q^0$.
  \ii For $-2r < k < r$, only the first sum contributes
  $q^{k+2r} + \dots + q^{(k-1)\%2}$,
  which is then completed by the $q^0$ contribution from $I_{n \le 0}$ when $k$ is even
  with a possible $q^0$.
  \ii For $-r \le k < 2\delta+r$,
  the first and second sum actually fit together with a ``seam'' near $q^r$,
  which is for even $k$ then completed by
  the single $q^0$ contribution from $I_{n \le 0}$ (only when $k$ is even).
  \ii For $2\delta+r \le k \le 2\delta+2r$,
  the same holds with piecing the third and fourth sum together
  (where the seam is near $q^k$ this time).
  \ii Finally, only the fourth sum contributes for $k \ge 2\delta+2r$,
  and it is of the desired form.
\end{itemize}
This gives us a succinct description of the orbital integral.

If $\ell$ is even, we now have the following intermediate result.
\begin{proposition}
  Suppose $\ell$ is even.
  Then we have the intermediate result
  \[
    I_{n \le 0} + I_{n > 0}^{\text{1+2}}
    = \sum_{k = -2r}^{2\delta + \min(\lambda+2r, \ell+3r)}
    (-1)^k \left( 1 + q + q^2 + \dots + q^{\nn^{1+2}_{\gamma}(k)}  \right) (q^s)^k
  \]
  where the piecewise function $\nn_\gamma \colon \ZZ \to \ZZ_{\ge 0}$ is defined by
  \[
    \nn^{1+2}_\gamma(k) =
    \begin{cases}
      k + 2r & \text{if } {-2r} \le k \le -r \\
      \left\lfloor \frac{k+r}{2} \right\rfloor + r & \text{if }{-r} \le k \le \ell-r \\
      \frac{\ell}{2} + r - (k-r)\%2 & \text{if } \ell - r \le k \le 2\delta + r \\
      \left\lfloor \frac{(2\delta+\ell+r)-k}{2} \right\rfloor + r & \text{if } 2\delta + r \le k \le 2\delta + \ell + 3r
    \end{cases}
  \]
  in the case $\lambda \ge \ell+r$, and
  \[
    \nn^{1+2}_\gamma(k) =
    \begin{cases}
      k + 2r & \text{if } {-2r} \le k \le -r \\
      \left\lfloor \frac{k+r}{2} \right\rfloor + r & \text{if }{-r} \le k \le \ell-r \\
      \frac{\ell}{2} + r - (k-r) \% 2 & \text{if } \ell - r \le k \le 2\delta + r \\
      \left\lfloor \frac{(2\delta+\ell+r)-k}{2} \right\rfloor + r & \text{if } 2\delta + r \le k \le 2\delta + 2\lambda - \ell + r \\
      (2\delta + \lambda + 2r) - k & \text{if } 2 \delta + 2 \lambda - \ell + r \le k \le 2 \delta + \lambda + 2r.
    \end{cases}
  \]
  in the case $\lambda \le \ell+r$.
\end{proposition}

We describe the final result for odd $\ell$ in the next subsection.

\subsection{Final formula for $\ell$ odd}
If $\ell$ is odd, so $\lambda = \ell$, then
the orbital integral can be expressed succinctly in the following way.
\begin{theorem}
  [Orbital integral in the odd case for $\mathbf{1}_{K'_{S, \le r}}$]
  Let the representative
  \[ \gamma = \begin{bmatrix}
      a & 0 & 0 \\
      b & - \bar d & 1 \\
      -a \bar b + b d & 1 - d \bar d & d
    \end{bmatrix} \in S_3(F)\rs \]
  be paired to an element in $\U(\VV_n)$.
  Assume $v(b) = v(d) = 0$.
  Let $\delta = v(1 - d \bar d)$
  and $\ell = v(b^2 - 4 a \bar d)$ be the parameters defined earlier.
  Assume further that $\ell$ is odd, and define
  \[ \nn_\gamma \coloneqq \Arch_{[-2r, 2\delta + \ell + 2r]}(r, \ell). \]
  Then for any $r \ge 0$ we have the formula:
  \[
    \Orb(\gamma, \mathbf{1}_{K'_{S, \le r}}, s)
    = \sum_{k = -2r}^{2\delta + \ell + 2r}
    (-1)^k \left( 1 + q + q^2 + \dots + q^{\nn_{\gamma}(k)}  \right) (q^s)^k
  \]
\end{theorem}
\begin{remark}
  To make $\nn_\gamma$ fully explicit, one could also expand the arch shorthand to
  \[
    \nn_\gamma(k)
    \coloneqq \begin{cases}
      k + 2r & \text{if } {-2r} \le k \le -r \\
      \left\lfloor \frac{k+r}{2} \right\rfloor + r & \text{if }{-r} \le k \le \ell-r \\
      \frac{\ell-1}{2} + r & \text{if } \ell - r \le k \le 2\delta + r \\
      \left\lfloor \frac{(2\delta+\ell+r)-k}{2} \right\rfloor + r & \text{if } 2\delta + r \le k \le 2\delta + \ell + r\\
      (2\delta + \ell + 2r) - k & \text{if } 2\delta + \ell + r \le k \le 2\delta + \ell + 2r.
    \end{cases} \]
\end{remark}

From the identity
\[
  \Orb(\gamma, \mathbf{1}_{K'_{S, r}}, s)
  = \Orb(\gamma, \mathbf{1}_{K'_{S, \le r}}, s)
  - \Orb(\gamma, \mathbf{1}_{K'_{S, \le (r-1)}}, s)
\]
we can then write the following equivalent formulation.
\begin{theorem}
  [Orbital integral in the odd case for $\mathbf{1}_{K'_{S,r}}$]
  Retaining the setting of the previous theorem, we have for any $r \ge 1$ the formula
  \[
    \Orb(\gamma, \mathbf{1}_{K'_{S, r}}, s)
    = \sum_{k = -2r}^{2\delta + \ell + 2r}
    (-1)^k q^{\nn_{\gamma}(k)}
    (1+q^{-1})^{\mathbf{1}[k \in \mathcal I_{\gamma, r}]}
    (q^s)^k
  \]
  where $\mathcal I_{\gamma, r}$ is the set of indices defined by
  \begin{align*}
    \mathcal{I}_{\gamma, r}
    &\coloneqq \left\{ -(2r-1), -(2r-2), -(2r-3), \dots, -(r+1) \right\} \\
    &\sqcup \{-r, -r+2, -r+4 \dots, -r+\ell-1 \} \\
    &\sqcup \{ 2\delta+r+1, 2\delta+r+3, \dots, 2\delta+r+1, 2\delta+r+3, \dots, 2\delta+\ell+r \} \\
    &\sqcup \{ 2\delta+\ell+r+1, 2\delta+\ell+r+2, \dots, 2\delta+\ell+2r-1 \}.
  \end{align*}
\end{theorem}

\begin{example}
  If $r=3$, $\ell=5$, and $\delta=100$, the formulas above read
  \begin{align*}
  \Orb(\gamma, \mathbf{1}_{K'_{S, \le 3}}, s)
  &= q^{-6s} \\
  &- (q+1) \cdot q^{-5s} \\
  &+ (q^2+q+1) \cdot q^{-4s} \\
  &- (q^3+q^2+q+1) \cdot q^{-3s} \\
  &+ (q^3+q^2+q+1) \cdot q^{-2s} \\
  &- (q^4+q^3+q^2+q+1) \cdot q^{-s} \\
  &+ (q^4+q^3+q^2+q+1) \cdot q^{0} \\
  &- (q^5+q^4+q^3+q^2+q+1) \cdot q^{s} \\
  &+ (q^5+q^4+q^3+q^2+q+1) \cdot q^{2s} \\
  &\vdotswithin= \\
  &+ (q^5+q^4+q^3+q^2+q+1) \cdot q^{204s} \\
  &- (q^4+q^3+q^2+q+1) \cdot q^{205s} \\
  &+ (q^4+q^3+q^2+q+1) \cdot q^{206s} \\
  &- (q^3+q^2+q+1) \cdot q^{207s} \\
  &+ (q^3+q^2+q+1) \cdot q^{208s} \\
  &- (q^2+q+1) \cdot q^{209s} \\
  &+ (q+1) \cdot q^{210s} \\
  &- q^{211s}.
  \end{align*}
  In the ellipses, all the omitted terms
  have the same coefficient $q^5+q^4+q^3+q^2+q+1$
  and alternate sign.
\end{example}

\begin{example}
  Continuing the previous example with
  $r=3$, $\ell=5$, and $\delta=100$, we have
  \begin{align*}
  \Orb(\gamma, \mathbf{1}_{K'_{S, 3}}, s)
  &= q^{-6s} \\
  &- (q+1) \cdot q^{-5s} \\
  &+ (q^2+q) \cdot q^{-4s} \\
  &- (q^3+q^2) \cdot q^{-3s} \\
  &+ q^3 \cdot q^{-2s} \\
  &- (q^4+q^3) \cdot q^{-s} \\
  &+ q^4 \cdot q^{0} \\
  &- (q^5+q^4) \cdot q^{s} \\
  &+ q^5 \cdot q^{2s} \\
  &- q^5 \cdot q^{3s} \\
  &+ q^5 \cdot q^{4s} \\
  &\vdotswithin= \\
  &+ q^5 \cdot q^{202s} \\
  &- q^6 \cdot q^{203s} \\
  &+ (q^5+q^4) \cdot q^{204s} \\
  &- q^4 \cdot q^{205s} \\
  &+ (q^4+q^3) \cdot q^{206s} \\
  &- q^3 \cdot q^{207s} \\
  &+ (q^3+q^2) \cdot q^{208s} \\
  &- (q^2+q) \cdot q^{209s} \\
  &+ (q+1) \cdot q^{210s} \\
  &- q^{211s}.
  \end{align*}
\end{example}

\subsection{Contribution from Case 3\ts{+}, 3\ts{-}, 4\ts{+} assuming $v(b) = v(d) = 0$}
These cases only appear when $\ell$ is even and we assume this for this subsection.
We consider the contribution of these cases within
$\kappa \int_{t, t_1 \in E} \mathbf{1}[n > 0] \mathbf{1}_{\le r}(\gamma,t,m)$
(using \eqref{eq:even_case3_plus}, \eqref{eq:even_case3_minus}, \eqref{eq:even_case4_plus})
and put $\Vol(t_1:-v(t_1)=m) = q^{2m} (1-q^{-2})$ to get:
\begin{align*}
  I_{n > 0}^{\text{3+4}}
  % -----------------------------------------------------------
  &\coloneqq \kappa \sum_{n=\ell+r+1}^{-\frac{\ell}{2}+\delta+\lambda+r}
    \sum_{m=n-r}^{\min(n-1, \lambda+r)-\frac{\ell}{2}+\delta}
    q^{-n-(n-\frac{\ell}{2}-r)} \left( 1 - q^{-1} \right) \\
    &\qquad\qquad\cdot \Big( (-1)^n q^{s(2m-n)} q^{2n-2m} \Big) \Big( q^{2m}(1-q^{-2}) \Big) \\
  &\qquad+ \kappa \sum_{n=\ell+r+1}^{\frac{\ell}{2}+\delta+r}
    \sum_{m=n-r}^{\frac{\ell}{2}+\delta+r}
    q^{-n-(n-\frac{\ell}{2}-r)} \left( 1 - q^{-1} \right) \\
  &\qquad\qquad\cdot \Big( (-1)^n q^{s(2m-n)} q^{2n-2m} \Big) \Big( q^{2m}(1-q^{-2}) \Big) \\
  &\qquad+ \kappa \sum_{n=\ell+r+1}^{\lambda+r}
    \sum_{m=n-\frac{\ell}{2}+\delta}^{\min(n,\lambda)+\delta+r}
    q^{-n-(m-\delta-r)} \left( 1 - q^{-1} \right) \\
    &\qquad\qquad\cdot \Big( (-1)^n q^{s(2m-n)} q^{2n-2m} \Big) \Big( q^{2m}(1-q^{-2}) \Big) \\
  &= \sum_{n=\ell+r+1}^{-\frac{\ell}{2}+\delta+\lambda+r}
    \sum_{m=n-r}^{\min(n-1, \lambda+r)-\frac{\ell}{2}+\delta}
    q^{-n-(n-\frac{\ell}{2}-r)}
      \cdot \Big( (-1)^n q^{s(2m-n)} q^{2n} \Big) \\
  &\qquad+ \sum_{n=\ell+r+1}^{\frac{\ell}{2}+\delta+r}
    \sum_{m=n-r}^{\frac{\ell}{2}+\delta+r}
    q^{-n-(n-\frac{\ell}{2}-r)}
      \cdot \Big( (-1)^n q^{s(2m-n)} q^{2n} \Big) \\
  &\qquad+ \sum_{n=\ell+r+1}^{\lambda+r}
    \sum_{m=n-\frac{\ell}{2}+\delta}^{\min(n,\lambda)+\delta+r}
    q^{-n-(m-\delta-r)}
      \cdot \Big( (-1)^n q^{s(2m-n)} q^{2n} \Big) \\
  &= \sum_{n=\ell+r+1}^{-\frac{\ell}{2}+\delta+\lambda+r}
    \sum_{m=n-r}^{\min(n-1, \lambda+r)-\frac{\ell}{2}+\delta}
    q^{\frac{\ell}{2}+r} \cdot (-1)^n q^{s(2m-n)} \\
  &\qquad+ \sum_{n=\ell+r+1}^{\frac{\ell}{2}+\delta+r}
    \sum_{m=n-r}^{\frac{\ell}{2}+\delta+r}
    q^{\frac{\ell}{2}+r} \cdot (-1)^n q^{s(2m-n)} \\
  &\qquad+ \sum_{n=\ell+r+1}^{\lambda+r}
    \sum_{m=n-\frac{\ell}{2}+\delta}^{\min(n,\lambda)+\delta+r}
    q^{n-m+\delta+r} \cdot (-1)^n q^{s(2m-n)} \\
\end{align*}

\todo{ok this collation is going to be\dots fun\dots}

\subsection{Final formula for even $\ell$, when $v(b) = v(d) = 0$}
\begin{theorem}
  [Orbital integral in the even case for $\mathbf{1}_{K'_{S, \le r}}$]
  Let the representative
  \[ \gamma = \begin{bmatrix}
      a & 0 & 0 \\
      b & - \bar d & 1 \\
      -a \bar b + b d & 1 - d \bar d & d
    \end{bmatrix} \in S_3(F)\rs \]
  be paired to an element in $\U(\VV_n)$.
  Assume further $v(b) = v(d) = 0$.
  Let $\delta = v(1 - d \bar d)$, $\ell = v(b^2 - 4 a \bar d)$ and
  $\lambda = v((1 - d \bar d)^2 - c \bar c) - 2\delta$
  (where $c = -a \bar b + b d $) be the parameters defined earlier.
  (Necessarily $\theta \equiv 1 \pmod 2$.)
  Assume further $\ell$ is even, and define
  \begin{align*}
    \nn_\gamma &\coloneqq \Arch_{[-2r, 2\delta + \lambda + 2r]}(r, \ell) \\
    \cc_\gamma &\coloneqq \Arch_{[\ell-r, 2\delta + \lambda - \ell + r]}
    (\delta - \ell/2, \min(\lambda-\ell-1, 2r)).
  \end{align*}
  Then for any $r \ge 0$ we have the formula:
  \begin{align*}
    \Orb(\gamma, \mathbf{1}_{K'_{S, \le r}}, s)
    &= \sum_{k = \ell-r}^{2\delta + \lambda - \ell + r} \cc_\gamma(k) (-1)^k q^{\frac{\ell}{2}+r} (q^s)^k \\
    &+ \sum_{k = -2r}^{2\delta + \lambda + 2r}
    (-1)^k \left( 1 + q + q^2 + \dots + q^{\nn_{\gamma}(k)}  \right) (q^s)^k.
  \end{align*}
\end{theorem}
\begin{remark}
  One can expand the Arch notation for $\nn_\gamma$ to obtain
  \[
    \nn_\gamma(k) =
    \begin{cases}
      k + 2r & \text{if } {-2r} \le k \le -r \\
      \left\lfloor \frac{k+r}{2} \right\rfloor + r & \text{if }{-r} \le k \le \ell-r \\
      \frac{\ell}{2} + r & \text{if } \ell - r \le k \le 2\delta + \lambda - \ell + r \\
      \left\lfloor \frac{(2\delta+\lambda+r)-k}{2} \right\rfloor + r & \text{if } 2\delta + \lambda - \ell + r \le k \le 2\delta + \lambda + r \\
      (2\delta + \lambda + 2r) - k & \text{if } 2\delta + \lambda + r \le k \le 2\delta + \lambda + 2r.
    \end{cases}
  \]
  Then $\cc_\gamma$ can be similarly expanded, but the result is so notationally dense
  that it is hardly worth including.
  If one defines the shorthands
  \begin{align*}
    \mathsf{B}_\gamma &\coloneqq \delta + \frac{\ell}{2} - r \\
    \mathsf{T}_\gamma &\coloneqq (2\delta+\lambda) - \mathsf{B}_\gamma \\
    \mathsf{w}_\gamma &\coloneqq \min(\lambda-\ell-1, 2r)
  \end{align*}
  then it could be written out more fully as
  \[
    \cc_\gamma(k)
    =
    \begin{cases}
      k - (\ell - r)
        & \text{if } \ell-r \le k \le \mathsf{B}_\gamma \\
      \left\lfloor \frac{k - \mathsf{B}_\gamma}{2} \right\rfloor + \delta - \frac{\ell}{2}
        &\text{if } \mathsf{B}_\gamma \le k \le \mathsf{B}_\gamma + \mathsf{w}_\gamma \\
      \delta - \frac{\ell}{2} + \half \mathsf{w}_\gamma
        &\text{if } \mathsf{B}_\gamma + \mathsf{w}_\gamma \le k \le \mathsf{T}_\gamma - \mathsf{w}_\gamma \\
      \left\lfloor \frac{\mathsf{T}_\gamma - k}{2} \right\rfloor + \delta - \frac{\ell}{2}
        &\text{if } \mathsf{T}_\gamma - \mathsf{w}_\gamma \le k \le \mathsf{T}_\gamma \\
        (2\delta + \lambda - \ell + r) - k
        & \text{if } \mathsf{T}_\gamma \le k \le 2\delta + \lambda - \ell + r.
    \end{cases}
  \]
\end{remark}

\begin{example}
  We provide an example for $r=5$, $\ell=2$, $\delta=8$, $\lambda=9$ for concreteness:
  \begin{align*}
    \Orb(\gamma, \mathbf{1}_{K'_{S, \le r}}, s)
    &= q^{-10s} \\
    &- (q + 1) \cdot q^{-9s} \\
    &+ (q^2 + q + 1) \cdot q^{-8s} \\
    &- (q^3 + q^2 + q + 1) \cdot q^{-7s} \\
    &+ (q^4 + q^3 + q^2 + q + 1) \cdot q^{-6s} \\
    &- (q^5 + q^4 + q^3 + q^2 + q + 1) \cdot q^{-5s} \\
    &+ (q^5 + q^4 + q^3 + q^2 + q + 1) \cdot q^{-4s} \\
    &- (q^6 + q^5 + q^4 + q^3 + q^2 + q + 1) \cdot q^{-3s} \\
    &+ (2q^6 + q^5 + q^4 + q^3 + q^2 + q + 1) \cdot q^{-2s} \\
    &- (3q^6 + q^5 + q^4 + q^3 + q^2 + q + 1) \cdot q^{-s} \\
    &+ (4q^6 + q^5 + q^4 + q^3 + q^2 + q + 1) \cdot q^{0} \\
    &- (4q^6 + q^5 + q^4 + q^3 + q^2 + q + 1) \cdot q^{s} \\
    &+ (5q^6 + q^5 + q^4 + q^3 + q^2 + q + 1) \cdot q^{2s} \\
    &- (5q^6 + q^5 + q^4 + q^3 + q^2 + q + 1) \cdot q^{3s} \\
    &+ (6q^6 + q^5 + q^4 + q^3 + q^2 + q + 1) \cdot q^{4s} \\
    &- (6q^6 + q^5 + q^4 + q^3 + q^2 + q + 1) \cdot q^{5s} \\
    &+ (7q^6 + q^5 + q^4 + q^3 + q^2 + q + 1) \cdot q^{6s} \\
    &- (7q^6 + q^5 + q^4 + q^3 + q^2 + q + 1) \cdot q^{7s} \\
    &+ (7q^6 + q^5 + q^4 + q^3 + q^2 + q + 1) \cdot q^{8s} \\
    &- (7q^6 + q^5 + q^4 + q^3 + q^2 + q + 1) \cdot q^{9s} \\
    &+ (7q^6 + q^5 + q^4 + q^3 + q^2 + q + 1) \cdot q^{10s} \\
    &- (7q^6 + q^5 + q^4 + q^3 + q^2 + q + 1) \cdot q^{11s} \\
    &+ (6q^6 + q^5 + q^4 + q^3 + q^2 + q + 1) \cdot q^{12s} \\
    &- (6q^6 + q^5 + q^4 + q^3 + q^2 + q + 1) \cdot q^{13s} \\
    &+ (5q^6 + q^5 + q^4 + q^3 + q^2 + q + 1) \cdot q^{14s} \\
    &- (5q^6 + q^5 + q^4 + q^3 + q^2 + q + 1) \cdot q^{15s} \\
    &+ (4q^6 + q^5 + q^4 + q^3 + q^2 + q + 1) \cdot q^{16s} \\
    &- (4q^6 + q^5 + q^4 + q^3 + q^2 + q + 1) \cdot q^{17s} \\
    &+ (3q^6 + q^5 + q^4 + q^3 + q^2 + q + 1) \cdot q^{18s} \\
    &- (2q^6 + q^5 + q^4 + q^3 + q^2 + q + 1) \cdot q^{19s} \\
    &+ (q^6 + q^5 + q^4 + q^3 + q^2 + q + 1) \cdot q^{20s} \\
    &- (q^5 + q^4 + q^3 + q^2 + q + 1) \cdot q^{21s} \\
    &+ (q^5 + q^4 + q^3 + q^2 + q + 1) \cdot q^{22s} \\
    &- (q^4 + q^3 + q^2 + q + 1) \cdot q^{23s} \\
    &+ (q^3 + q^2 + q + 1) \cdot q^{24s} \\
    &- (q^2 + q + 1) \cdot q^{25s} \\
    &+ (q + 1) \cdot q^{26s} \\
    &- q^{27s}.
  \end{align*}
\end{example}

\section{Synopsis of the orbital integral
  $\Orb((\gamma, \uu, \vv^\top), \phi \otimes \mathbf{1}_{\OO_F^2 \times (\OO_F^2)^\vee}, s)$
  for $(\gamma, \uu, \vv^\top) \in S_2(F) \times V'_2(F)$ and $\phi \in \HH(S_2(F), K')$}

Throughout this section, $H = \GL_n(F)$ (rather than $H = \GL_{n-1}(F)$)
and $K' = \GL_n(\OO_F)$.
For the concrete calculation, we are mostly interested in the case $n = 2$.

\subsection{Definition}
We will not need to work in the generality of a function
on all of $S_n(F) \times V_n'(F)$, although it could be done.
Instead, it will be enough to define the orbital integral
in the case where our function is of the form
\[ \phi \otimes \mathbf{1}_{\OO_F^n \times (\OO_F^n)^\vee} \]
where $\phi \in \HH(S_n(F), K')$ is the left component, and
the right component is the indicator function defined in the obvious way:
\begin{align*}
  \mathbf{1}_{\OO_F^n \times (\OO_F^n)^\vee} \colon V'_n(F) &\to \{0,1\} \\
  (\uu, \vv^\top) &\mapsto
  \begin{cases}
    1 & \uu \text{ and } \vv^\top \text{ have } \OO_F \text{-entries} \\
    0 & \text{otherwise}.
  \end{cases}
\end{align*}

Then, unsurprisingly from the definition of our action as
\[ h \cdot (\gamma, \uu, \vv^\top) = (h\gamma h\inv, h\uu, \vv^\top h\inv) \]
we analogously define the orbital integral as follows.
\begin{definition}
  For brevity let $\eta(h) \coloneqq \eta(\det h)$ for $h \in H$.
  For $(\gamma, \uu, \vv^\top) \in S_n(F) \times V'_n(F)$,
  $\phi \in \HH(S_n(F), K')$, and $s \in \CC$,
  we define the orbital integral by
  \begin{align*}
    & \Orb((\gamma, \uu, \vv^\top), \phi \otimes \mathbf{1}_{\OO_F^n \times (\OO_F^n)^\vee}, s) \\
    &\coloneqq
    \int_{h \in H} \phi(h\inv \gamma h)
    \mathbf{1}_{\OO_F^n \times (\OO_F^n)^\vee}(h \uu, \vv^\top h^{-1})
    \eta(h) \left\lvert \det(h) \right\rvert_F^{-s} \odif h
  \end{align*}
\end{definition}

\subsection{Basis for the indicator functions in $\HH(S_2(F), K')$}
From now on assume $n = 2$.
This section is almost an exact analog of \Cref{sec:orbital0_hecke_basis},
so we will be slightly terser.
Again set
\[ S_2(F) \coloneqq \left\{ g \in \GL_2(E) \mid g \bar{g} = \id_2 \right\}. \]
We again have a Cartan decomposition indexed by a single integer $r \ge 0$:
\begin{lemma}
  [Cartan decomposition of $S_2(F)$]
  For each integer $r \ge 0$ let
  \[ K'_{S,r} \coloneqq \GL_3(\OO_E) \cdot
    \begin{bmatrix} 0 & \varpi^r \\ \varpi^{-r} & 0 \end{bmatrix} \]
  denote the orbit of
  $\begin{bmatrix} 0 & \varpi^r \\ \varpi^{-r} & 0 \end{bmatrix}$
  under the left action of $\GL_2(\OO_E)$.
  Then we have a decomposition
  \[ S_2(F) = \coprod_{r \geq 0} K'_{S,r}. \]
\end{lemma}
Like last time, $K'_{S,r}$ is the part of $S_2(F)$
for which the most negative valuation among the nine entries is $-r$.
And as before we abbreviate the $r = 0$ term specifically:
\begin{align*}
  K'_S
  &\coloneqq K'_{S,0} \\
  &= \GL_2(\OO_E) \cdot \begin{bmatrix} & 1 \\ 1 \end{bmatrix} \\
  &= \GL_2(\OO_E) \cdot \id_2 = S_2(F) \cap \GL_2(\OO_E).
\end{align*}

Repeating the definition
\[ K'_{S, \le r} \coloneqq S_2(F) \cap \varpi^{-r} \GL_2(\OO_E)
  = K'_{S,0} \sqcup K'_{S,1} \sqcup \dots \sqcup K'_{S,r} \]
we get a basis of indicator functions for the Hecke algebra $\HH(S_2(F), K')$:
\begin{proposition}
  For $r \ge 0$, the indicator functions $\mathbf{1}_{K'_{S, \le r}}$
  form a basis of $\HH(S_2(F), K')$.
\end{proposition}

\subsection{Parametrization of $\gamma$}
From now on assume $n = 2$,
and that $(\gamma, \uu, \vv^\top)$ is regular when viewed
as an element $\begin{bmatrix} \gamma & \uu \\ \vv^\top & 0 \end{bmatrix} \in \GL_3(F)$
(cf.\ \Cref{def:regular}).

\subsubsection{Identifying an orbit representative}
The orbital integral depends only on the $H$-orbit of $(\gamma, \uu, \vv^\top)$.
Consequently, we may assume without loss of generality
(via multiplication by a suitable change-of-basis $h \in H = \GL_2(F)$) that
\[ \uu = \begin{bmatrix} 0 \\ 1 \end{bmatrix}, \qquad
  \vv^\top = \begin{bmatrix} 0 & \theta \end{bmatrix} \qquad \theta \in F. \]
(We know $\uu$ is not the zero vector from the regular condition
applied on $(\gamma, \uu, \vv^\top)$.)

Meanwhile, we will let
$\gamma = \begin{bmatrix} a & b \\ c & d \end{bmatrix} \in \GL_2(F)$
for $a,b,c,d \in F$.
Then, viewed as an element of $\GL_3(F)$ via the embedding we described earlier, we have
\[
  (\gamma, \uu, \vv^\top)
  \mapsto \begin{bmatrix}
    a & b & 0 \\
    c & d & 1 \\
    0 & \theta & 0
  \end{bmatrix} \in \GL_3(F).
\]
Thus, our definition of regular requires that
$\begin{bmatrix} 0 \\ 1 \end{bmatrix}$
is linearly independent from $\begin{bmatrix} b \\ d \end{bmatrix}$
and
$\begin{bmatrix} 0 & \theta \end{bmatrix}$
is linearly independent from $\begin{bmatrix} c & d \end{bmatrix}$.
This is just saying that $b$, $c$, $\theta$ are all nonzero.
We also know that $\gamma \in S_2(F)$, which gives us relations on $a$, $b$, $c$, $d$,
(the same as \cite[equation (7.3.2)]{ref:AFLspherical}); we have
\[
  \begin{bmatrix} 1 & 0 \\ 0 & 1 \end{bmatrix}
  = \begin{bmatrix} a & b \\ c & d \end{bmatrix} \begin{bmatrix} \bar a & \bar b \\ \bar c & \bar d \end{bmatrix}
  \implies
  \begin{aligned}
    \bar b c = b \bar c &= 1 - a \bar a \\
    \text{and } d &= - \bar a c / \bar c = -\bar a b / \bar b.
  \end{aligned}
\]

\begin{remark}
  Note that for each integer $n$, the quantity $\vv^\top \gamma^n \uu$
  is invariant under the action of $h$.
\end{remark}

\subsubsection{Simplification due to the matching of non-quasi-split unitary group}
Like before, we focus on the case where regular $(\gamma, \uu, \vv^\top)$
matches an element in the non-quasi-split unitary group.
\todo{I need to ask Wei exactly what's up here}
This is controlled by the parity of $v(\Delta)$, where
\[ \Delta = \det \left[ \vv^\top \gamma^{i+j} \uu \right]_{0 \le i,j \le n-1}. \]
When $n=2$, for the representatives we described before,
we have
\[ \left[ \vv^\top \gamma^{i+j} \uu \right]_{0 \le i,j \le n-1}
  = \begin{bmatrix} e & de \\ de & bce + d^2e \end{bmatrix} \]
so
\[ \Delta = bce^2 = \frac{b}{\bar b}(1-a \bar a) e^2 . \]
Hence, $v(\Delta)$ is odd if and only if $v(1-a \bar a)$ is odd.
Thus, we restrict attention to the following situation:
\begin{assume}
  We will assume that
  \[ v(1-a \bar a) \equiv 1 \pmod 2. \]
  \label{assume:a_odd}
\end{assume}
In particular, $a$ must be a unit.
And since $d = -\bar a c / \bar c$, it follows $d$ is a unit.
In other words, \Cref{assume:a_odd} gives the direct corollary
\[ v(a) = v(d) = 0. \]

\subsection{Quantities to state the answer in terms of}

\section{Setup of the orbital integral for $S_2(F) \times V_2'(F)$}
\subsection{Iwasawa decomposition}
The overall method is to take the Iwasawa decomposition in $KAN$ form:
every element in $h \in \GL_2(F)$ may be parametrized as
\[ h = k \begin{bmatrix} x_1 & 0 \\ 0 & x_2 \end{bmatrix}
  \begin{bmatrix} 1 & y \\ 0 & 1 \end{bmatrix} \]
where $k \in K' = \GL_2(\OO_F)$, $x_1, x_2 \in \OO_F^\times$ and $y \in \OO_F$.
Because the orbits are invariant under conjugation by $K'$,
the parameter $k$ can be discarded.
The Haar measure integrated over is then given just by
\[ \odif[\times] x_1 \odif[\times] x_2 \odif y \]
i.e.\ we take multiplicative Haar measure on $F^\times$
(normalized so that $\OO_F^\times$ has volume $1$)
and additive Haar measure on $F$
(so $\OO_F$ has volume $1$).

\subsection{Action of upper triangular matrices on $(\gamma, \uu, \vv^\top)$.}
We now compute the action of an arbitrary
\[ h = \begin{bmatrix} x_1 & 0 \\ 0 & x_2 \end{bmatrix}
  \begin{bmatrix} 1 & y \\ 0 & 1 \end{bmatrix} \]
on $(\gamma, \uu, \vv^\top)$.
The main term is given by
\begin{align*}
  h \gamma h^{-1}
  &=
  \begin{bmatrix} x_1 & 0 \\ 0 & x_2 \end{bmatrix}
  \begin{bmatrix} 1 & y \\ 0 & 1 \end{bmatrix}
  \begin{bmatrix} a & b \\ c & d \end{bmatrix}
  \begin{bmatrix} 1 & -y \\ 0 & 1 \end{bmatrix}
  \begin{bmatrix} x_1^{-1} & 0 \\ 0 & x_2^{-1} \end{bmatrix} \\
  &=
  \begin{bmatrix} x_1 & 0 \\ 0 & x_2 \end{bmatrix}
  \begin{bmatrix} cy + a & -cy^2+(d-a)y+b \\ c & -cy+d \end{bmatrix}
  \begin{bmatrix} x_1^{-1} & 0 \\ 0 & x_2^{-1} \end{bmatrix} \\
  &=
  \begin{bmatrix} cy + a & \frac{x_1}{x_2} \cdot \left( -cy^2+(d-a)y+b \right) \\
    \frac{x_2}{x_1} \cdot c & -cy+d \end{bmatrix}
\end{align*}
Meanwhile, we have
\begin{align*}
  h \uu &=
    \begin{bmatrix} x_1 & 0 \\ 0 & x_2 \end{bmatrix}
    \begin{bmatrix} 1 & y \\ 0 & 1 \end{bmatrix}
    \begin{bmatrix} 0 \\ 1 \end{bmatrix}
    = \begin{bmatrix} x_1 y \\ x_2 \end{bmatrix} \\
  \vv^\top h^{-1} &=
    \begin{bmatrix} 0 & e \end{bmatrix}
    \begin{bmatrix} 1 & -y \\ 0 & 1 \end{bmatrix}
    \begin{bmatrix} x_1^{-1} & 0 \\ 0 & x_2^{-1} \end{bmatrix}
    = \begin{bmatrix} 0 & \frac{e}{x_2} \end{bmatrix}.
\end{align*}

\subsection{Description of support}
From now on we fix the notation
\begin{align*}
  n_1 &\coloneqq v(x_1) \\
  n_2 &\coloneqq v(x_2).
\end{align*}

For a given $r \ge 0$, we find that $h$ contributes to the integral exactly
if $h\uu$ and $\vv^\top h\inv$ have $\OO_F$-entries,
and all the entries of $h \gamma h\inv$ are in $\varpi^{-r}\OO_F$.
The former condition is just saying that
\begin{align*}
  v(y) &\ge -n_1 \\
  0 &\le n_2 \le v(e).
\end{align*}
Now we consider the entries of $h\gamma h\inv$.
First, because $a$ and $d$ are units by \Cref{assume:a_odd},
and $r \ge 0$, it follows that
\begin{align*}
  cy + a, -cy + d \in \varpi^{-r}\OO_F
  &\iff cy \in \varpi^{-r}\OO_F \\
  &\iff v(y) \ge -r - v(c).
\end{align*}
Moreover,
\[ \frac{x_2}{x_1} \cdot c \in \varpi^{-r} \OO_F
  \iff n_2 + v(c) \ge r + n_1. \]
As for the quadratic constraint, let us write
\[
  -cy^2 + (d-a)y + b
  = c\left( -\left( y - \frac{d-a}{2c} \right)^2 + \frac bc \right).
\]
Because $b \bar c = 1 - a \bar a$ has odd valuation,
it follows that $\frac b c = \frac{1-a \bar a}{c\bar c}$ has odd valuation to.
On the other hand, the valuation of
$\left( y - \frac{d-a}{2c} \right)^2$ is necessarily even.
So the terms $\left( y - \frac{d-a}{2c} \right)^2$ and $\frac bc$
never have the same valuation and hence
\[ v\left( -\left( y - \frac{d-a}{2c} \right)^2 + \frac bc  \right)
  = \min \left\{ 2v\left( y - \frac{d-a}{2c} \right), v\left( \frac bc \right) \right\}. \]
This implies
\begin{align*}
  &\phantom= v\left( \frac{x_1}{x_2} \cdot \left( -cy^2+(d-a)y+b \right) \right) \\
  &= n_1 - n_2 + \min \left\{ 2v\left( y - \frac{d-a}{2c} \right), v\left( \frac bc \right) \right\}.
\end{align*}

We now collate all the constraints we obtained together.
In summary, the triple $(x_1, x_2, y) \in \OO_F^\times \times \OO_F^\times \times \OO_F$
contributes to the orbital integral exactly if the following identities hold:
\begin{align*}
  0 &\le n_2 \le v(e) \\
  n_2  - v(b) - r &\le n_1 \le n_2 + v(c) + r \\
  v(y) &\ge \max\left(-n_1, -r-v(c)\right) \\
  v\left( y - \frac{d-a}{2c} \right) &\ge \frac{n_2 - n_1 - v(c) - r}{2}.
\end{align*}

\section{Evaluation of the orbital integral for $S_2(F) \times V'_2(F)$}
\label{sec:orbitalFJ2}

For odd $\theta$ I seem to be getting the formula
\[
  \Orb\left( (\gamma, \uu, \vv^\top), \mathbf{1}_{K'_S, \le r} \otimes \oneV \right)
\]
equal to
\[
  \sum_{n_2=0}^{v(e)}
  \sum_{m=0}^{\theta + 2r}
  q^{m - 2n_2 - \max \left\{ m - n_2, \left\lceil \frac m2 \right\rceil, 0 \right\} }
  (-1)^{v(c) + r - m}
  q^{s(2n_2 + v(c) + r - m)}
\]
which seems really weird. Really?
That is a bizarre answer.

Example where $r = 1$, $v(b) = 4$, $v(c) = 5$, $v(e) = 3$, $\theta = 9$:
\begin{align*}
  &- q^{-5s} \\
  &+ q^{-4s} \\
  &- \left( 1 + q^{-1} \right) q^{-3s} \\
  &+ \left( 1 + q^{-1} \right) q^{-2s} \\
  &- \left( 1 + q^{-1} + q^{-2} \right) q^{-s} \\
  &+ \left( 1 + q^{-1} + q^{-2} \right) \\
  &- \left( 1 + q^{-1} + q^{-2} + q^{-3} \right) q^{s} \\
  &+ \left( 1 + q^{-1} + q^{-2} + q^{-3} \right) q^{2s} \\
  &- \left( 1 + q^{-1} + q^{-2} + q^{-3} \right) q^{3s} \\
  &+ \left( 1 + q^{-1} + q^{-2} + q^{-3} \right) q^{4s} \\
  &- \left( 1 + q^{-1} + q^{-2} + q^{-3} \right) q^{5s} \\
  &+ \left( 1 + q^{-1} + q^{-2} + q^{-3} \right) q^{6s} \\
  &- \left( q^{-2} + q^{-3} + q^{-4} \right) q^{7s} \\
  &+ \left( q^{-2} + q^{-3} + q^{-4} \right) q^{8s} \\
  &- \left( q^{-4} + q^{-5} \right) q^{9s} \\
  &+ \left( q^{-4} + q^{-5} \right) q^{10s} \\
  &- \left( q^{-6} \right) q^{11s} \\
  &+ \left( q^{-6} \right) q^{12s} \\
\end{align*}

It doesn't have the same symmetry that I saw in $S_3(F)$ formula.
Also, it has all exponents negative rather than positive.

\chapter{Base change}
\label{ch:satake}

The goal of this section is to make completely explicit the base change
\[ \BC_{S_n}^{\eta^{n-1}} \colon \HH(S_n(F)) \to \HH(\U(\VV_n^-)) \]
in the special case $n = 3$ (in which case $\BC_{S_n}^{\eta^{n-1}} = \BC_{S_n}$ as $\eta^2 = 1$).
Note that the case $n = 2$ was already done in \cite[\S7.1]{ref:AFLspherical}.
When it is not more difficult, some of the results will be stated for all $n$,
rather than $n = 3$ specifically.

Throughout this section we have the following additional notation.
\begin{definition}
  Denote by $\rproj \colon \GL_n(E) \surjto S_n(F)$ the projection defined by
  \[ \rproj(g) \coloneqq g \bar{g}\inv. \]
\end{definition}
We also let $\Sym(n)$ denotes the symmetric group in $n$ variables with order $n!$
(since $S_n(F) \subseteq \GL_n(E)$ is already reserved for the symmetric space).

\section{Background on the Satake transformation in transformation}
We recall a general form of the Satake transformation, which will be used later.

For this subsection, $G$ will denote an arbitrary connected reductive group
over some non-Archimedean local field $F$.
We will not distinguish between $G$ and $G(F)$ when there is no confusion.

To simplify things, we will assume $G$ is unramified;
but we do \emph{not} assume $G$ is split.
Introduce the following notation:
\begin{itemize}
  \ii Let $K$ be a hyperspecial maximal compact subgroup of $G$
  (it exists because $G$ is unramified).
  \ii Let $A$ denote a maximal $F$-split torus in $G$.
  All the maximal $F$-split tori in $G$ are conjugate; let $A$ denote one of them.
  \ii Let $M$ be the centralizer of $A$; this is itself a maximal torus in $G$.
  \ii Let $\prescript{\circ}{} M \coloneqq M(F) \cap K$
  be the maximal compact subgroup of $M$.
  \ii Let $P$ denote a minimal $F$-parabolic containing $A$.
  \ii Let $\delta$ denotes the modulus character of $P$.
  It can be describes as follows.
  Let $\varpi$ denote a uniformizer for $F$ and $q$ the residue characteristic.
  Then if $\rho$ is the Weyl vector and $\mu$ is a positive cocharacter, then
  \[ \delta(\mu(\varpi)) = q^{- \left< \mu, \rho\right>}. \]
  \ii Let $N$ denote the unipotent radical of $P$.
  \ii Let $W$ be the relative Weyl group for the pair $(G,A)$,
  which acts on $\HH(M, \prescript{\circ}{} M)$.
\end{itemize}
We can now state the Satake isomorphism.
\begin{definition}
  The \emph{Satake transform} is a canonical isomorphism of Hecke algebras
  \[ \Sat \colon \HH(G, K) \to \HH(M, \prescript{\circ}{} M)^W \]
  which is given by defining
  \[ (\Sat(f))(t) \coloneqq \delta(t)^\half \int_N f(nt) \odif n  \]
  for each $t \in M$.
\end{definition}
We are going to apply this momentarily in two situations:
once when $G$ is the general linear group (which is split),
and once when $G$ is a unitary group.


\section{The Satake transformation for $\HH(\GL_n(E))$ and $\HH(\U(\VV_n^-))$}
To take the Satake transform of $\HH(\U(\VV_n^-))$, we define the following abbreviations.
\begin{itemize}
  \ii Let $T$ denote the split diagonal torus of $\GL_n$.
  \ii Let
  \[ N' \coloneqq \left\{ \begin{bmatrix}
      1 & \ast & \dots & \ast \\
        & 1 & \dots & \ast \\
        &   & \ddots & \vdots \\
        &   &   & 1 \end{bmatrix}\right\} \subseteq \GL_n(E) \]
  denote the unipotent upper-triangular matrices.
\end{itemize}
Similarly for $\HH(\U(\VV_n^-))$:
\begin{itemize}
  \ii Set $m \coloneqq \left\lfloor n/2 \right\rfloor$ for brevity.
  \ii Let
  \[ A \coloneqq \left\{
    \diag(x_1, \dots, x_m, 1_{n-2m}, x_m\inv, \dots, x_1\inv) \right\} \]
  so that $A(F)$ is a maximal $F$-split torus of $\U(\VV_n^-)$.
  \ii Let $N \coloneqq N' \cap G$ denote the unipotent upper triangular matrices
  which are also unitary.
  \ii For brevity, let $W_m \coloneqq (\ZZ/2\ZZ)^m \rtimes \Sym(m)$
  be the relative Weyl group of $(G,A)$.
\end{itemize}

We can now introduce the Satake transform for our two
\emph{bona fide} Hecke algebras, using the data in Table~\ref{tab:satakestuff}.

\begin{table}[ht]
  \centering
  \begin{tabular}{lll}
    \toprule
    Group & $G' = \GL_n(E)$ & $G = \U(\VV_n^-)$ \\ \midrule
    Local field & $E$ & $F$ \\\hline
    Hyperspecial compact & $K' = \GL_n(\OO_E)$ & $K = G \cap \GL_n(\OO_E)$ \\\hline
    Max'l split torus & $T(E)$ & $A(F)$ \\\hline
    Centralizer of split torus & $T(\OO_E)$ & $A(\OO_F)$ \\\hline
    Parabolic (Borel) & Upper tri in $G'$ & Upper tri in $G$ \\\hline
    Unipotent rad.\ of parabolic & $N'$ (unipot.\ upper tri) & $N$ (unipot.\ upper tri) \\\hline
    Relative Weyl group & $\Sym(n)$ & $W_m = (\ZZ/2\ZZ)^m \rtimes \Sym(m)$ \\
    %Cocharacter group & $\ZZ^{\oplus n}$ & $\ZZ^{\oplus m}$?? \\
    %Weyl vector & $\left< \frac{n-1}{2}, \frac{n-3}{2}, \dots, -\frac{n-1}{2} \right>$
    %            & $\left< \frac{m-1}{2}, \frac{m-3}{2}, \dots, -\frac{m-1}{2} \right>$??
    \bottomrule
  \end{tabular}
  \caption{Data needed to run the Satake transformation.}
  \label{tab:satakestuff}
\end{table}

Hence, the Satake transformations obtained can be viewed as
\begin{align*}
  \Sat &\colon \HH(\GL_n(E)) \xrightarrow{\sim} \QQ[T(E) / T(\OO_E)]^{\Sym(n)} \\
  \Sat &\colon \HH(\U(\VV_n^-))\xrightarrow{\sim} \QQ[A(F) / A(\OO_F)]^{W_m}
\end{align*}
(In both cases, the modular character $\delta^{1/2}$ gives rational values,
so it is okay to work over $\QQ$.)

To make this further concrete, we remark that the cocharacter groups
involved are free abelian groups with known bases.
This identification lets us rewrite the right-hand sides above as concrete polynomials.
Specifically, we identify
\[ \QQ[T(E) / T(\OO_E)]^{\Sym(n)}
  \xrightarrow{\sim} \QQ[X_1^\pm, \dots, X_n^\pm]^{\Sym(n)} \]
by identifying $X_i$ with the
cocharacter corresponding to injection into the $i$\ts{th} factor.
Similarly, we identify
\[ \QQ[A(F) / A(\OO_F)]^{W_m}
  \xrightarrow{\sim} \QQ[Y_1^{\pm}, \dots, Y_m^{\pm}]^{W_m} \]
by identifying $Y_i + Y_i^{-1}$
with the cocharacter corresponding to
\[ x \mapsto \diag(1, \dots, x, \dots, x\inv, \dots, 1) \]
where $x$ is in the $i$\ts{th} position and $x\inv$ is in the $(n-i)$\ts{th} position,
and all other positions are $1$.
Here $\QQ[Y_1^{\pm}, \dots, Y_m^{\pm}]^{W_m}$
denotes the ring of symmetric polynomials in $Y_i + Y_i^{-1}$.

So, henceforth, we will consider
\begin{align*}
  \Sat &\colon \HH(\GL_n(E)) \xrightarrow{\sim} \QQ[X_1^\pm, \dots, X_n^\pm]^{\Sym(n)} \\
  \Sat &\colon \HH(\U(\VV_n^-)) \xrightarrow{\sim} \QQ[Y_1^{\pm}, \dots, Y_m^{\pm}]^{W_m}.
\end{align*}

\section{Relation of Satake transformation to base change}
Let
\[ \BC \colon \HH(\GL_n(E)) \to \HH(\U(\VV_n^-)) \]
denote the stable base change morphism from $\GL_n(E)$ to the unitary group $\U$.
The relevance of the Satake transformation is that
(see e.g.\ \cite[Proposition 3.4]{ref:leslie})
it gives a way to make this $\BC$ completely explicit:
we have a commutative diagram
\begin{center}
\begin{tikzcd}
  \HH(\GL_n(E))  \ar[r, "\sim"', "\Sat"] \ar[d, "\BC"]
    & \QQ[X_1^\pm, \dots, X_n^\pm]^{\Sym(n)} \ar[d, "\BC"] \\
  \HH(\U(\VV_n^-)) \ar[r, "\sim"', "\Sat"]
    & \QQ[Y_1^\pm, \dots, Y_m^\pm]^{W_m}
\end{tikzcd}
\end{center}
Here the right arrow is also denoted $\BC$ following \cite{ref:AFLspherical}
(although it is denoted $\nu$ in \cite{ref:leslie}).
This gives a way in which we can concretely calculate the map $\BC$
in some situations.

\section{The map $\BC^{\eta^{n-1}}_{S_n}$}
Before we can define the map $\BC^{\eta^{n-1}}_{S_n}$
we need one more piece of notation.
Our prior map $\rproj \colon \GL_n(E) \surjto S_n(F)$ induces a map
\begin{align*}
  \rproj_\ast \colon \HH(\GL_n(E)) &\to \HH(S_n(F)) \\
  \rproj_\ast(f')\left( g\bar{g}\inv \right) &= \int_{\GL_n(F)} f'(gh) \odif h
\end{align*}
by integration on the fibers.
A similar twisted version by $\eta$
\begin{align*}
  \rproj_\ast^\eta \colon \HH(\GL_n(E)) &\to \HH(S_n(F)) \\
  \rproj_\ast^\eta(f')\left( g\bar{g}\inv \right) &= \int_{\GL_n(F)} f'(gh) \eta(gh) \odif h
\end{align*}
is defined analogously,
where as before $\eta(g) = (-1)^{v(\det g)}$ in a slight abuse of notation.

Then Leslie \cite{ref:leslie} shows the following result.
\begin{theorem}
  [{\cite[Theorem 3.2 and Proposition 3.4]{ref:leslie}}]
  Both maps $\rproj_\ast$ and $\rproj_\ast^\eta$ induce isomorphisms
  \begin{align*}
    \BC_{S_n} \colon \HH(S_n(F)) &\xrightarrow{\sim} \HH(\GL_n(E)) \\
    \BC^{\eta^{n-1}}_{S_n} \colon \HH(S_n(F)) &\xrightarrow{\sim} \HH(\GL_n(E))
  \end{align*}
  such that
  \begin{align*}
    \BC &= \BC_{S_n} \circ \rproj_\ast \\
    \BC &= \BC^{\eta^{n-1}}_{S_n} \circ \rproj_\ast^{\eta^{n-1}}.
  \end{align*}
\end{theorem}
We take these isomorphisms promised by this theorem
as the definition of $\BC_{S_n}$ and $\BC^{\eta^{n-1}}_{S_n}$ in our conjectures
(noting when $n$ is odd they coincide, as $\eta^{n-1} = 1$).

When combined with the Satake information we have, we get the following diagram.
\begin{center}
\begin{tikzcd}
  \HH(\GL_n(E)) \ar[dd, "\rproj_\ast^{\eta^{n-1}}"', bend right = 60] \ar[r, "\sim"', "\Sat"] \ar[d, "\BC"]
    & \QQ[X_1^\pm, \dots, X_n^\pm]^{\Sym(n)} \ar[d, "\BC"] \\
  \HH(\U(\VV_n^-)) \ar[r, "\sim"', "\Sat"]
    & \QQ[Y_1^\pm, \dots, Y_m^\pm]^{W_m} \\
    \HH(S_n(F)) \ar[u, "\sim", "\BC_{S_n}^{\eta^{n-1}}"']
\end{tikzcd}
\end{center}

\section{Calculation of $\BC_{S_3}$}
Specialize now to $n=3$ (thus $m = \left\lfloor n/2 \right\rfloor = 1$).
Our goal is to make completely explicit the arrow $\BC_{S_n}$
in the diagram above.
The completed result is \Cref{prop:BC_S3}.

\subsection{Overview}
Throughout this subsection, we use the shorthand
\[ \varpi^{(n_1, n_2, n_3)} \coloneqq \diag(\varpi^{n_1}, \varpi^{n_2}, \varpi^{n_3}). \]
As a $\QQ$-module, the spaces $\HH(\U(\VV_n^-))$ and $\HH(S_n(F))$
have a canonical basis of indicator functions indexed by $\ZZ$:
\begin{itemize}
  \ii $\HH(S_n(F))$ has $\QQ$-module basis $\mathbf{1}_{K'_{S,j}}$ for $j \ge 0$.
  \ii $\HH(\U(\VV_n^-))$ has a $\QQ$-module basis given by the indicator functions
  \[ \mathbf{1}_{\varpi^{-r} \Mat(\OO_E) \cap \U(\VV_n^-)} \]
  for $r \ge 0$.
\end{itemize}
On the other hand, the natural $\QQ$-module basis for $\HH(\GL_n(E))$, namely
\[ \mathbf{1}_{K'\varpi^{(n_1, n_2, n_3)}K'} \]
is given by triples of integers $n_1 \ge n_2 \ge n_3 \ge 0$, and is much larger.
So explicit calculations for the $\rproj_\ast$ or the Satake transforms viewed in
$\CC[X_1, X_2, X_3]^{\Sym(n)}$ is nontrivial if one works with the entire basis.

Hence the overall strategy, to reduce the amount of work we have to do,
is to focus on only the $\ZZ$-indexed elements
\[
  \mathbf{1}_{\Mat_3(\OO_E), v\circ\det=r}
  = \sum_{\substack{n_1 \ge n_2 \ge n_3 \\ n_1 + n_2 + n_3 = r}}
  \mathbf{1}_{K'\varpi^{(n_1, n_2, n_3)}K'} \in \HH(\GL_n(E)).
\]
This aggregated indicator function is easier to compute,
because given an explicit matrix it is somewhat easier
to evaluate \[ \mathbf{1}_{\Mat_3(\OO_E), v\circ\det=r} \]
at it (one only needs to check it has $\OO_E$ entries
and that the determinant has valuation $r$,
rather than determining the exact coset $K'\varpi^{(n_1, n_2, n_3)}K'$).

\subsection{Satake transform of the determinant characteristic function on the top arrow}
This is the easiest calculation, and we do it for all $n$ rather than just $n = 3$.
\begin{proposition}
  For every integer $r \ge 0$, we have
  \[ \Sat(\mathbf{1}_{\Mat_n(\OO_E), v\circ\det=r})
    = q^{(n-1)r} \sum_{e_1 \dots + e_n = r} X_1^{e_1} \dots X_n^{e_n}. \]
\end{proposition}
\begin{proof}
  We evaluate the coefficient $X_1^{e_1} \dots X_n^{e_n}$.
  Choose a cocharacter $\mu$,
  and suppose $\mu(\varpi) = \varpi^{(e_1, \dots, e_n)}$ with $n_1 \ge n_2 \ge n_3$.
  Let $q_E = q^2$ be the residue characteristic of $E$.
  Take the upper triangular matrices as our Borel subgroup as usual,
  so the unipotent radical of this Borel subgroup
  are the unipotent upper triangulars $N'$ which we describe as
  \[ N' \coloneqq \left\{
      \begin{bmatrix}
      1 & y_{12} & y_{13} & \dots & y_{1n} \\
        & 1 & y_{23} & \dots & y_{2n} \\
        &   & 1 & \dots & y_{3n} \\
        &   &   & \ddots & \vdots  \\
        &   &   &   & 1
      \end{bmatrix}
    \mid y_{12}, \dots, y_{(n-1)n}\in E \right\} \]
  and with additive Haar measure is $\odif{y_{12}, y_{23} \dotso, y_{(n-1)n}}$.
  Recall also the Weyl vector for $\GL_n(E)$ is just
  \[ \rho_{\GL_n(E)} = \left< \frac{n-1}{2}, \frac{n-3}{2}, \dots, -\frac{n-1}{2} \right>. \]
  Compute
  \begin{align*}
    &\Sat(\mathbf{1}_{\Mat_n(\OO_E), v\circ\det=r})(\mu(\varpi)) \\
    &= \delta(\mu(\varpi))^\half \int_{n' \in N'}
      \mathbf{1}_{\Mat_n(\OO_E, v\circ \det = r)} (\mu(\varpi) n') \odif{n'} \\
    &= q_E^{-\left< \mu, \rho\right>}
    \underbrace{\int_{y_{12} \in E} \int_{y_{13} \in E} \dotso \int_{y_{(n-1)n} \in E}}_{\binom n2 \text{ integrals}} \\
    &\qquad
      \mathbf{1}_{\Mat_3(\OO_E), v \circ \det = r}
      \left( \begin{bmatrix}
        \varpi^{e_1} & \varpi^{e_1} y_{12} & \varpi^{e_1} y_{13} & \dots & \varpi^{e_1} y_{1n} \\
        & \varpi^{e_2} & \varpi^{e_2} y_{23} & \dots & \varpi^{e_2} y_{2n} \\
        &   & \varpi^{e_3} & \dots & \varpi^{e_3} y_{3n} \\
        &   &   & \ddots & \vdots  \\
        &   &   &   & \varpi^{e_n}
        \end{bmatrix} \right) \\
    &\qquad \odif{y_{12}, y_{23} \dotso, y_{(n-1)n}} \\
    &= q_E^{-\left(\frac{n-1}{2}e_1 + \frac{n-3}{2}e_2 + \dots + -\frac{n-1}{2} e_n \right)}
    \mathbf{1}_{e_1 + \dots + e_n = r}
    \underbrace{\int_{y_{12} \in E} \int_{y_{13} \in E} \dotso \int_{y_{(n-1)n} \in E}}_{\binom n2 \text{ integrals}} \\
    &\qquad \prod_{1 \le i < j \le n} \mathbf{1}_{\OO_E}(\varpi^{e_i} y_{ij}) \odif{y_{ij}} \\
    &= q_E^{-\left(\frac{n-1}{2}e_1 + \frac{n-3}{2}e_2 + \dots + -\frac{n-1}{2} e_n \right)}
    \mathbf{1}_{e_1 + \dots + e_n = r} \prod_{1 \le i < j \le n} q_E^{e_i} \\
    &= q_E^{-\left(\frac{n-1}{2}e_1 + \frac{n-3}{2}e_2 + \dots + -\frac{n-1}{2} e_n \right)}
    \mathbf{1}_{e_1 + \dots + e_n = r} \prod_{1 \le i \le n} q_E^{(n-i)e_i} \\
    &= \mathbf{1}_{e_1 + \dots + e_n = r} \prod_{1 \le i \le n}^n q_E^{\frac{n-1}{2} e_i} \\
    &= q_E^{\frac{n-1}{2} r} \mathbf{1}_{e_1 + \dots + e_n = r} \\
    &= \begin{cases}
      q^{\frac{n-1}{2} r} & \text{if } e_1 + \dots + e_n = r \\
      0 & \text{otherwise}.
    \end{cases}
  \end{align*}
  This gives the sum claimed earlier.
\end{proof}


\subsection{Satake transform of the indicator on the bottom arrow}
\begin{proposition}
  For each $r \ge 0$ we have
  \[ \Sat\left(\mathbf{1}_{\varpi^{-r} \Mat_3(\OO_E) \cap \U(\VV_3^-)}\right)
    = \sum_{i=0}^r q^{2r - \mathbf{1}_{r \equiv i \bmod 2}} Y_1^{\pm i} \]
  where we adopt the shorthand
  \[
    Y_1^{\pm i} \coloneqq
    \begin{cases}
      Y_1^i + Y_1^{-i} & i > 0 \\
      1 & i = 0 .
    \end{cases}
  \]
\end{proposition}
\begin{proof}
  We first need to describe \[ N = N' \cap \U(\VV_3^-) \] a little more carefully.
  For $n \in N'$ we have
  \[
    n^\ast \beta n
    =
    \begin{bmatrix} 1 \\ \bar{y_1} & 1 \\ \bar{y_2} & \bar{y_3} & 1 \end{bmatrix}
    \beta
    \begin{bmatrix}
      1 & y_1 & y_2 \\
        & 1 & y_3 \\
        & & 1
    \end{bmatrix}
    = \begin{bmatrix}
      & & 1 \\
      & 1 & y_3 + \bar{y_1} \\
      1 & y_1 + \bar{y_3} & y_2 + \bar{y_2} + y_3 \bar{y_3}
    \end{bmatrix}.
  \]
  So $n \in N$ if and only if the above matrix equals $\beta$, which means
  \[ 0 = y_3 + \bar{y_1} = y_2 + \bar{y_2} + y_3 \bar{y_3}. \]
  Then we can re-parametrize by $z_1, z_2, z_3 \in F$ according to
  \begin{align*}
    y_3 &= z_1 + z_2 \sqrt\eps \\
    y_2 &= -\frac{z_1^2 + z_2^2 \eps}{2} + z_3\sqrt\eps \\
    y_1 &= -z_1 + z_2\sqrt\eps.
  \end{align*}
  Back to the original task.
  For each $i \ge 0$ we can evaluate the Satake transform at the element
  $\nu(\varpi) = \diag(\varpi^i, 1, \varpi^{-i})$, for the cocharacter $\nu$
  corresponding to $Y_1^i + Y_1^{-i}$:
  \begin{align*}
    &\Sat\left( \mathbf{1}_{\varpi^{-r} \Mat_3(\OO_E) \cap \U(\VV_3^-)}\right)
      \left( \nu(\varpi)  \right) \\
    &= \delta(\nu(\varpi))^\half \int_{n \in N}
      \mathbf{1}_{\varpi^{-r} \Mat_3(\OO_E) \cap \U(\VV_3^-)}
      \left( \nu(\varpi) n' \right) \odif n \\
    &= \delta(\nu(\varpi))^\half \int_{n \in N}
      \mathbf{1}_{{\varpi^{-r}} \Mat_3(\OO_E) \cap \U(\VV_3^-)}
      \left( \begin{bmatrix} \varpi^i & \varpi^i y_1 & \varpi^i y_2 \\
               & 1 & y_3 \\
               & & \varpi^{-i} \end{bmatrix} \right) \odif n
  \end{align*}
  The matrix itself is always in $\U(\VV_3^-)$, because it's the product of two unitary matrices.
  So the indicator needs to check whether all the entries have valuation at least $-r$.
  If we switch characterization to the coordinates $z_1$, $z_2$, $z_3$ we described earlier,
  we see that the conditions are
  \begin{align*}
    i &\le r, \\
    v(z_1) &\ge -r, \\
    v(z_2) &\ge -r,\\
    v(z_3) &\ge -(r+i),\\
    v(z_1^2 + z_2^2 \eps) &\ge -(r+i).
  \end{align*}
  Assume $i \le r$ henceforth.
  The condition for $z_1$ and $z_2$ then really says
  \[ \min(v(z_1), v(z_2)) \ge -\left\lfloor \frac{r+i}{2} \right\rfloor. \]
  So the integral factors as a triple integral
  \[
    \int_{z_1 \in F}
    \int_{z_2 \in F}
    \int_{z_3 \in F}
    \mathbf{1}_{\varpi^{-\left\lfloor \frac{r+i}{2} \right\rfloor} \OO_F}(z_1)
    \mathbf{1}_{\varpi^{-\left\lfloor \frac{r+i}{2} \right\rfloor} \OO_F}(z_2)
    \mathbf{1}_{\varpi^{-(r+i)} \OO_F}(z_3)
    \odif{z_1,z_2,z_3}
  \]
  which is equal to
  \[ q^{2\left\lfloor \frac{r+i}{2} \right\rfloor+r+i}. \]
  Meanwhile, $\delta(\nu(\varpi))^{\half} = q^{-2i}$.
  In summary,
  \[
    \Sat\left( \mathbf{1}_{\varpi^{-r} \Mat_3(\OO_E) \cap \U(\VV_3^-)}\right) \left( \nu(\varpi) \right)
    =
    \begin{cases}
      q^{2\left\lfloor \frac{r+i}{2} \right\rfloor - i + r} & i \le r \\
      0 & i > r
    \end{cases}
  \]
  Finally, since
  \[ 2\left\lfloor \frac{r+i}{2} \right\rfloor - i + r
    = \begin{cases}
      2r & r+i \text{ is even} \\
      2r-1 & r+i \text{ is odd}
    \end{cases}
  \]
  we get the formula claimed.
\end{proof}

\subsection{Integration over fiber}
\begin{proposition}
  For every integer $r \ge 0$, we have
  \begin{align*}
    &\rproj_\ast(\mathbf{1}_{\Mat_3(\OO_E), v\circ\det=r}) \\
    &= \sum_{j=0}^r \left(
      \sum_{i=0}^{2(r-j)} \min \left( 1 + \left\lfloor \frac i2 \right\rfloor,
        1 + \left\lfloor \frac{2(r-j)-i}{2} \right\rfloor \right) q^i \right)
        \mathbf{1}_{K'_{S,j}}.
  \end{align*}
\end{proposition}
\begin{proof}
  The coefficient of $\mathbf{1}_{K'_{S,j}}$ will be equal to
  the evaluation of the integral at any $g$ such that $g\bar{g} \in K'_{S,j}$.
  Fixing $j \ge 0$, we are going to take the choice
  \[
    g = \begin{bmatrix}
      1 &   & \varpi^{-j} \sqrt{\eps} \\
      & 1 \\
      &   & 1
    \end{bmatrix}.
  \]
  We need to check this choice of $g$ indeed satisfies $g\bar{g}\inv \in K'_{S,j}$.
  This follows as
  \[ \bar{g} = \begin{bmatrix} 1 &   & -\varpi^{-j} \sqrt{\eps} \\ & 1 \\ &   & 1 \end{bmatrix}
    \implies \bar{g}\inv = \begin{bmatrix} 1 &   & \varpi^{-j} \sqrt{\eps} \\ & 1 \\ &   & 1 \end{bmatrix}
  \]
  and therefore
  \[
    g\bar{g}\inv = \begin{bmatrix}
      1 &   & 2\varpi^{-j} \sqrt{\eps} \\
      & 1 \\
      &   & 1
    \end{bmatrix} \in K'_{S,j}
  \]
  as needed.

  Having chosen the representative $g$, we aim to calculate the right-hand side of
  \[
    \rproj_\ast(\mathbf{1}_{\Mat_3(\OO_E), v\circ\det=r})(g\bar{g})
    = \int_{h \in \GL_3(F)} \mathbf{1}_{\Mat_3(\OO_E), v\circ\det=r}(gh) \odif h.
  \]
  We take (non-Archimedean) Iwasawa decomposition of $h \in \GL_3(F)$ to rewrite it as
  \[
    h =
    \begin{bmatrix} x_1 \\ & x_2 \\ && x_3 \end{bmatrix}
    \begin{bmatrix} 1 & y_1 & y_2 \\ & 1 & y_3 \\ & & 1 \end{bmatrix}
    k
  \]
  for $k \in \GL_3(\OO_F) \subseteq K'$, which does not affect the indicator function.
  Here $x_1, x_2, x_3 \in F^\times$ and $y_1, y_2, y_3 \in F$.
  In that case, note that
  \begin{align*}
    gh
    &=
    \begin{bmatrix}
      1 &   & \varpi^{-j}\sqrt\eps \\
      & 1 \\
      &   & 1
    \end{bmatrix}
    \begin{bmatrix} x_1 \\ & x_2 \\ && x_3 \end{bmatrix}
    \begin{bmatrix} 1 & y_1 & y_2 \\ & 1 & y_3 \\ & & 1 \end{bmatrix} k \\
    &=
    \begin{bmatrix}
      1 &   & \varpi^{-j}\sqrt\eps \\
      & 1 \\
      &   & 1
    \end{bmatrix}
    \begin{bmatrix} x_1 & x_1 y_1 & x_1 y_2 \\ & x_2 & x_2 y_3 \\ & & x_3 \end{bmatrix} k \\
    &=
    \begin{bmatrix}
      x_1 & x_1 y_1 & x_1 y_2 + x_3 \varpi^{-j} \sqrt\eps \\
      & x_2 & x_2 y_3 \\
      & & x_3
    \end{bmatrix}
    k.
  \end{align*}
  Hence, we can rewrite the $\rproj_\ast(\mathbf{1}_{\Mat_3(\OO_E), v\circ\det=r})$
  as a six-fold integral
  \begin{align*}
    &\rproj_\ast(\mathbf{1}_{\Mat_3(\OO_E), v\circ\det=r}) \\
    &= \int_{x_1 \in F^\times} \int_{x_2 \in F^\times} \int_{x_3 \in F^\times}
    \int_{y_1 \in F} \int_{y_2 \in F} \int_{y_3 \in F} \\
    &\quad \mathbf{1}_{\Mat_3(\OO_E), v\circ\det=r} \left(
    \begin{bmatrix}
      x_1 & x_1 y_1 & x_1 y_2 + x_3 \varpi^{-j} \sqrt\eps \\
      & x_2 & x_2 y_3 \\
      & & x_3
    \end{bmatrix}
    \right) \\
    &\quad \odif[{\times,\times,\times}]{x_1,x_2,x_3,y_1,y_2,y_3}.
  \end{align*}
  Apparently the indicator function only depends on the valuations,
  so accordingly we rewrite the six-fold integral as a discrete sum over the valuations
  $\alpha_i \coloneqq v(x_i)$.
  Then the conditions are that
  \begin{align*}
    &\alpha_1 \ge 0, \quad \alpha_2 \ge 0, \quad \alpha_3 \ge j \\
    &v(y_1) \ge - \alpha_1, \quad v(y_2) \ge - \alpha_1, \quad v(y_3) \ge -\alpha_2.
  \end{align*}
  We have $\Vol(\varpi^{\alpha_i} \OO_F^\times) = 1$
  and $\Vol(\varpi^{-\alpha_i} \OO_F) = q^{\alpha_i}$.
  Hence the integral can be rewritten as the discrete sum
  \begin{align*}
    \sum_{\substack{\alpha_1 + \alpha_2 + \alpha_3 = r \\ \alpha_1 \ge 0 \\ \alpha_2 \ge 0 \\ \alpha_3 \ge j}}
    q^{\alpha_1} \cdot q^{\alpha_1} \cdot q^{\alpha_2}
    &= \sum_{\substack{\alpha_1 + \alpha_2 \le r-j \\ \alpha_1 \ge 0 \\ \alpha_2 \ge 0}}
    q^{2\alpha_1+\alpha_2} \\
    &= \sum_{i=0}^{2(r-j)}
    \min \left( 1 + \left\lfloor \frac i2 \right\rfloor,
      1 + \left\lfloor \frac{2(r-j)-i}{2} \right\rfloor
    \right) q^i
  \end{align*}
  as desired.
\end{proof}

\subsection{Base change from $\HH(\U(\VV_3^-))$ to $\HH(S_3(F))$}
We first need to determine an element of $\HH(\U(\VV_n^-))$
which is in the pre-image of
\[ \mathbf{1}_{\varpi^{-r} \Mat_3(\OO_E) \cap \U(\VV_3^-)} \]
under $\BC \colon \HH(\GL_3(E)) \to \HH(\U(\VV_3^-))$.

For convenience, we define the shorthand
\[
  \HH(\GL_3(E)) \ni
  f'_r \coloneqq \begin{cases}
    \mathbf{1}_{\Mat_3(\OO_E), v \circ \det = r} & r \ge 0 \\
    0 & r < 0
  \end{cases}
\]
for every integer $r$.
We start with the following intermediate calculation.
\begin{align*}
  &\BC\left( \Sat \left( f'_r - q^2 f'_{r-1} \right) \right) \\
  &= \BC \left(
    q^{2r} \sum_{n_1+n_2+n_3=r} X_1^{n_1} X_2^{n_2} X_3^{n_3}
    - q^2 \cdot q^{2(r-1)} \sum_{n_1+n_2+n_3=(r-1)} X_1^{n_1} X_2^{n_2} X_3^{n_3} \right) \\
  &= q^{2r} \left( \sum_{n_1+n_2+n_3=r} Y_1^{n_1-n_3} - \sum_{n_1+n_2+n_3=(r-1)} Y_1^{n_1-n_3} \right) \\
  &= q^{2r} \left( \sum_{n_1+n_3=r} Y_1^{n_1-n_3} \right) \\
  &= q^{2r} \left( Y_1^{r} + Y_1^{r-2} + \dots + Y_1^{-r} \right).
  \intertext{Replacing $r$ with $r-1$ gives}
  &\BC\left( \Sat \left(f'_{r-1} - q^2 f'_{r-2} \right) \right) \\
  &= q^{2r-2} \left( Y_1^{r-1} + Y_1^{r-3} + \dots + Y_1^{-(r-1)} \right).
  \intertext{Adding the former equation to $q$ times the latter gives}
  &\BC\left( \Sat\left( f'_r + (q-q^2) f'_{r-1} - q^3 f'_{r-2} \right) \right) \\
  &= q^{2r} \left( Y_1^r + Y_1^{r-2} + \dots + Y_1^{-r} \right)
  + q^{2r-1} \left( Y_1^{r-1} + Y_1^{r-3} + \dots + Y_1^{-(r-1)} \right) \\
  &= \Sat(\mathbf{1}_{\varpi^{-r} \Mat_3(\OO_E) \cap \U(\VV_3^-)}).
\end{align*}
This shows that
\[ \BC(f'_r + (q-q^2) f'_{r-1} - q^3 f'_{r-2}) =
  \mathbf{1}_{\varpi^{-r} \Mat_3(\OO_E) \cap \U(\VV_3^-)} \]
so indeed $f'_r + (q-q^2) f'_{r-1} - q^3 f'_{r-2}$
lies in the desired pre-image of the map $\BC \colon \HH(\GL_3(E)) \to \HH(\U(\VV_3^-))$.

On the other hand, it is easy to check that
\begin{align*}
  &\rproj_\ast(f'_r -q^2 f'_{r-1}) \\
  &= \sum_{j=0}^r \Bigg[
      \sum_{i=0}^{2(r-j)} \min \left( 1 + \left\lfloor \frac i2 \right\rfloor,
      1 + \left\lfloor \frac{2(r-j)-i}{2} \right\rfloor \right) q^i \\
  &\qquad - \sum_{i=0}^{2(r-1-j)} \min \left( 1 + \left\lfloor \frac i2 \right\rfloor,
    1 + \left\lfloor \frac{2((r-1)-j)-i}{2} \right\rfloor \right) q^{i+2}
  \Bigg] \mathbf{1}_{K'_{S,j}} \\
  &= \sum_{j=0}^r \left[ 1+q+q^2+\dots+q^{r-j} \right] \mathbf{1}_{K'_{S,j}} \\
  \intertext{so}
  &\rproj_\ast(f'_r -q^2 f'_{r-1} + q \left( f'_{r-1} - q^2 f'_{r-3} \right)) \\
  &= \sum_{j=0}^r \left[ (1+q+q^2+\dots+q^{r-j})+(q+q^2+\dots+q^{r-j}) \right] \mathbf{1}_{K'_{S,j}} \\
  &= \sum_{j=0}^r \left[ 1 + 2q + 2q^2 + \dots + 2q^{r-j} \right] \mathbf{1}_{K'_{S,j}}.
\end{align*}
To summarize, the completed commutative diagram can be written in full as
\begin{center}
\begin{tikzcd}
  \begin{tabular}{c} $f'_r + (q-q^2) f'_{r-1}$ \\ $- q^3 f'_{r-2} \in \HH(\GL_3(E))$ \end{tabular}
    \ar[dd, "\rproj_\ast"', mapsto, bend right = 50]
    \ar[r, "\Sat", mapsto] \ar[d, "\BC", mapsto]
    & \dots \in \QQ[X_1^\pm, X_2^\pm, X_3^\pm]^{\Sym(3)} \ar[d, "\BC", mapsto] \\
  \begin{tabular}{c} $\mathbf{1}_{\varpi^{-r} \Mat_3(\OO_E) \cap \U(\VV_3^-)}$ \\ $\in \HH(\U(\VV_3^-))$ \end{tabular}
    \ar[r, "\Sat", mapsto]
    & \begin{tabular}{l}
      $q^{2r} \left( Y_1^{\pm r} + \dotsb + Y_1^{\mp r} \right)$ \\
      $+ q^{2r-1} \left( Y_1^{\pm(r-1)} + \dotsb + Y_1^{\mp(r-1)} \right)$ \\
      $\in \QQ[Y_1^\pm]^{W_1}$
      \end{tabular} \\
  \begin{tabular}{c}
    $\sum_{j=0}^r \big[ 1 + 2q + 2q^2$ \\
    $+ \dots + 2q^{r-j} \big] \mathbf{1}_{K'_{S,j}}$ \\
    $\in \HH(S_3(F))$
  \end{tabular} \ar[u, "\sim", "\BC_{S_3}"', mapsto]
\end{tikzcd}
\end{center}

Thus, we arrive at the following:
\begin{proposition}
  \label{prop:BC_S3}
  For $n = 3$, we have
  \begin{align*}
    \BC_{S_3} \left( \sum_{j=0}^r \left[ 1 + 2q + 2q^2 + \dots + 2q^{r-j} \right]
    \mathbf{1}_{K'_{S,j}} \right)
    &= \mathbf{1}_{\varpi^{-r} \Mat_3(\OO_E) \cap \U(\VV_3^-)} \\
    \BC_{S_3} \left( \mathbf{1}_{K'_{S,r}}
    + \sum_{j=0}^{r-1} 2q^{r-j} \mathbf{1}_{K'_{S,j}} \right)
    &= \mathbf{1}_{K\varpi^{(r,0,-r)}K}
  \end{align*}
  for every integer $r \ge 0$.
\end{proposition}
\begin{proof}
  The first equation is the one we just proved.
  The second one follows by noting that
  \[
    \mathbf{1}_{K\varpi^{(r,0,-r)}K}
    = \mathbf{1}_{\varpi^{-r} \Mat_3(\OO_E) \cap \U(\VV_3^-)}
    - \mathbf{1}_{\varpi^{-(r-1)} \Mat_3(\OO_E) \cap \U(\VV_3^-)}
  \]
  so one merely subtracts the left-hand sides evaluated at $r$ and $r-1$ for $r \ge 1$
  to get
  \begin{align*}
    &\phantom= \sum_{j=0}^r \left[ 1 + 2q + 2q^2 + \dots + 2q^{r-j} \right] \mathbf{1}_{K'_{S,j}}
    - \sum_{j=0}^{r-1} \left[ 1 + 2q + 2q^2 + \dots + 2q^{(r-1)-j} \right] \mathbf{1}_{K'_{S,j}} \\
    &= \mathbf{1}_{K'_{S,r}} +
      \sum_{j=0}^{r-1} \left[ 1 + 2q + 2q^2 + \dots + 2q^{r-j} \right] \mathbf{1}_{K'_{S,j}}
    - \sum_{j=0}^{r-1} \left[ 1 + 2q + 2q^2 + \dots + 2q^{(r-1)-j} \right] \mathbf{1}_{K'_{S,j}} \\
    &= \mathbf{1}_{K'_{S,r}} + \sum_{j=0}^{r-1} \left[ 2q^{r-j}\mathbf{1}_{K'_{S,j}} \right].
  \end{align*}
  as claimed.
\end{proof}


\printbibliography[title=References]

\end{document}
