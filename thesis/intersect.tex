\chapter{Intersection numbers for $\Int((g,u), f)$ for $n = 2$}
This chapter is dedicated to computing intersection numbers
for the semi-Lie version of AFL in the special case $n = 2$.

\section{Background}
\subsection{Notation}
We continue using the notation $(\EE, \iota_\EE, \lambda_\EE)$ as the triple over $\FF$
whose Rosati involution has signature $(1,0)$.

Moreover we let $\DD$ be a quaternion division algebra over $F$,
with a fixed maximal order $\OO_\DD$ realized by
\[ \End(\EE) \simeq \OO_\DD \]
see \cite[Remark 2.5]{ref:KR}.
We also have an identification
\[ \VV_n^+ \simeq \DD. \]

In any case, we will take an orthogonal decomposition of our quaternion division algebra as
\[ \DD \simeq E \oplus E \Pi \]
where $\Pi$ is a uniformizer selected so that $\Pi^2 = \varpi$.
In general, for $u \in \DD$, we write $u = u_0 + u_-$ where $u_0 \in E$ and $u_- \in E \Pi$.
We also define
\[ \DT \coloneqq \left\{ u \in \DD \mid \Tr u = 0 \right\} \]
which has codimension $1$ inside $\DD$ (i.e.\ is three-dimensional).
Since $\Tr(u) = \Tr_{E/F}(u_0)$,
we could also write $\DT = \{ u \in \DD \mid \Tr_{E/F}(u_0) = 0 \}$.

\subsection{The Rapoport-Zink space $\RZ_2$}
For integers $m \ge 0$, if we consider the functions
\[ f_m \coloneqq \mathbf{1}_{\varpi^{-m} \Mat_2(\OO_E) \cap G} \in \HH(G, K) \]
then these functions produce a basis for the Hecke algebra $\HH(G, K)$.
Let us abbreviate
\[ \mathcal{T}_m \coloneqq \mathcal{T}_{\mathbf{1}_K \otimes f_m} \]
moving forward.
Then as described in the previous section we have a diagram
\begin{center}
\begin{tikzcd}
  & \ar[ld] \mathcal{T}_m \ar[rd] & \\
  \RZ_{2} && \RZ_{2}.
\end{tikzcd}
\end{center}

\subsection{The Lubin-Tate space $\MM_2$ and the divisor $\ZO4$ on $\MM_2 \times \MM_2$}
We introduce the notation $\MM_2$ for the \emph{Lubin-Tate space} for $n = 2$.
It's parametrized by tuples $(Y, \rho)$
where $Y$ is a one-dimensional strict formal $\OO_F$-module of height $2$,
and $\rho$ is a framing of a fixed Lubin-Tate module $\mathbb{Y}$.\todo{should this be $\EE$?}

\begin{proposition}
  [{\cite[Example 5.5.6]{ref:AFLspherical}}]
  The Serre tensor construction produces an identification
  \[ \Serre \colon \RZ_2 \xrightarrow{\sim} \MM_2. \]
  By abuse of notation we will also use the same symbol for the map
  \[ \Serre \colon \RZ_{2,2} \xrightarrow{\sim} \MM_2 \times \MM_2. \]
\end{proposition}

Now we define the special orthogonal divisor $\ZO4(u)$
on $\MM_2 \times \MM_2$ as follows.
\begin{definition}
  Let $u \in \OO_\DD$.
  Then we define the divisor $\ZO4(u)$ to be
  the pairs $(X, X') \in \MM_2 \times \MM_2$
  for which we have a diagram\todo{Is there a better term than ``diagram'' here?}
  \begin{center}
  \begin{tikzcd}
    X \ar[r, "\varphi"] \ar[d, dash] & X' \ar[d, dash] \\
    \EE \ar[r, "u"] & \EE.
  \end{tikzcd}
  \end{center}
\end{definition}
Then the Serre tensor construction gives an isomorphism
\[ \mathcal{T}_m \simeq \ZO4(\varpi^m) \]
such that we get an analogous diagram
\begin{center}
\begin{tikzcd}
  & \ar[ld] \ZO4(\varpi^m) \ar[rd] & \\
  \MM_{2} && \MM_{2}.
\end{tikzcd}
\end{center}

\subsection{The divisor $\ZO3$ on $\MM_2$}
Turning to $\MM_2 \times \MM_2$, we will henceforth always identify $\MM_2$
with its image under the diagonal map
\begin{center}
\begin{tikzcd}
  \ZO4(1) \ar[r, "\sim"] & \MM_2 \ar[r, hook, "\Delta_{\MM_2}"] & \MM_2 \times \MM_2.
\end{tikzcd}
\end{center}

\begin{definition}
  Suppose now $u \in \DT$.
  Then we define the divisor $\ZO3(u)$ to be those $X \in \MM_2$
  for which we have a diagram
  \begin{center}
  \begin{tikzcd}
    X \ar[r, "\varphi"] \ar[d, dash] & X \ar[d, dash] \\
    \EE \ar[r, "u"] & \EE.
  \end{tikzcd}
  \end{center}
  Note that basically by definition, for $u \in \OO_\DD$ and $\Tr u = 0$ we have
  \[ \ZO3(u) \simeq \ZO4(u) \cap \ZO4(1) \]
  when we identify $\MM_2$ with its image in $\MM_2 \times \MM_2$.
\end{definition}

\section{Comparison of the unitary and orthogonal special divisors}
We now relate $\ZD(u)$ to $\ZO3(u)$ through our
isomorphism $\RZ_{2,2} \xrightarrow{\sim} \MM_2 \times \MM_2$.
Recall that we have the notation
\[ \ZD(u)^\circ \coloneqq \ZD(u) - \ZD\left( \frac{u}{\varpi} \right). \]
Define $\ZO4(u)^\circ$ and $\ZO3(u)^\circ$ similarly.

\begin{lemma}
  Let $u \in \VV_n^+$, and consider it as an element $u \in \DD \simeq E \oplus E \Pi$.
  Accordingly write $u = u_0 + u_-$, where $u_0 \in E$ and $u_- \in E \Pi$.
  Then pullback along the Serre tensor construction identifies
  \[ \Serre^\ast \ZD(u)^\circ \simeq \ZO3(u_-)^\circ. \]
\end{lemma}
\begin{proof}
  \todo{find a reference}
\end{proof}


\section{The intersection number as a triple product}
We return to the intersection number
\[ \chi_{\RZ_{n,n}} \left(
      \Sheaf_{\TT_{\mathbf{1} \otimes f_m}(\Delta_{\ZD(u)})}
      \jiao_{\Sheaf_{\RZ_{n,n}}} \SO_{\Gamma_g} \right) \]
which we will rewrite more succinctly using angle brackets as
\[ \Big\langle (1,g) \cdot \mathcal{T}_m, \; \Delta(\ZD(u)) \Big\rangle_{\RZ_{2,2}} \]
in analogy to \cite[\S6.1]{ref:AFLspherical}.
(Note that $\Delta$ here is the diagonal map $\RZ_2 \to \RZ_{2,2}$.)
For this calculation, it would be sufficient to split
\[ \ZD(u) = \sum_{i \ge 0} \ZD(u/\varpi^i)^\circ. \]
So, we proceed to compute
\begin{align*}
  \Big\langle (1,g) \cdot \mathcal{T}_m, \; \Delta(\ZD(u)^\circ) \Big\rangle_{\RZ_{2,2}}
  &= \left< \ZO4(\varpi^m g), \; \Delta(\Serre^\ast(\ZD(u))^\circ) \right>_{\MM_2 \times \MM_2} \\
  &= \left< \ZO4(\varpi^m g), \; \ZO3(u_-)^\circ \right>_{\MM_2 \times \MM_2} \\
  &= \left< \ZO4(\varpi^m g), \; \ZO4(1)^\circ, \; \ZO4(u_-)^\circ \right>_{\MM_2 \times \MM_2}.
\end{align*}
