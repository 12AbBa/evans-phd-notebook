\chapter{Evaluation of the weighted orbital integral for $S_2(F) \times V'_2(F)$}
\label{ch:orbitalFJ2}

We now aggregate the supports we found in the previous section together with the
definition of the weighted orbital integral to extract the desired formulas.

Recall that the weighted orbital integral was defined as
\begin{align*}
  & \Orb(\guv, \phi \otimes \oneV, s) \\
  &\coloneqq
  \int_{h \in H} \phi(h\inv \gamma h)
  \oneV(h \uu, \vv^\top h^{-1})
  \eta(h) \left\lvert \det(h) \right\rvert_F^{-s} \odif h
\end{align*}
and that after taking Iwasawa decomposition as
\[ h = k \begin{bmatrix} x_1 & 0 \\ 0 & x_2 \end{bmatrix}
  \begin{bmatrix} 1 & y \\ 0 & 1 \end{bmatrix} \]
we broke the sum based on $n_1 = v(x_1)$ and $n_2 = v(x_2)$.
So the contribution to the weighted orbital integral looks like
For $h$ as above, we know that
\begin{align*}
  \eta(h) &= (-1)^{n_1 + n_2} \\
  \left\lvert \det(h) \right\rvert^{-s}_F &= (q^s)^{n_1 + n_2}.
\end{align*}
Applying \eqref{eq:n1_plus_n2_semi_lie} we find that
\[
  \eta(h)
  \left\lvert \det(h) \right\rvert^{-s}_F
  = (q^s)^{2n_ - m + v(c) + r}.
\]

\section{The contribution for Case 5}
We assume $\theta + 2r \ge 0$, because otherwise the entire sum is empty.
Hence, the total contribution for \textbf{Case 5} is
\begin{align*}
  I^{\text{5}}
  &\coloneqq \sum_{n_2 = 0}^{v(e)} \sum_{m = 0}^{\theta + 2r}
  q^{m - \max\left( m-n_2, \left\lceil m/2 \right\rceil \right)}
  (-q^s)^{2n_2 - m + v(c) + r} \\
  &\coloneqq \sum_{n_2 = 0}^{v(e)} \sum_{m = 0}^{\theta + 2r}
  q^{\min\left( n_2, \left\lfloor m/2 \right\rfloor \right)}
  (-q^s)^{2n_2 - m + v(c) + r}.
\end{align*}
We'll change the summation variable to
\[ k \coloneqq 2n_2 - m + v(c) + r
\iff m = 2n_2 - k + v(c) + r \]
Then
\begin{align*}
  I^{\text{5}}
  &\coloneqq \sum_{n_2 = 0}^{v(e)}
  \sum_{k = 2n_2 - \theta + v(c) - r}^{2n_2 + v(c) + r}
  q^{\min\left( n_2, n_2 + \left\lfloor \frac{v(c)+r-k}{2} \right\rfloor \right)} (-q^s)^{k} \\
  &= \sum_{n_2 = 0}^{v(e)}
  \sum_{k = 2n_2 - \theta + v(c) - r}^{2n_2 + v(c) + r}
  q^{n_2 - \max\left( 0, \left\lceil \frac{k-(v(c)+r)}{2} \right\rceil \right)} (-q^s)^{k}.
\end{align*}
We then interchange the order of summation so that $k$ is outside.
Then $k$ runs from the lowest value of $k = - \theta + v(c) - r$
to the largest value $k = 2v(e) + v(c) + r$ over all choices of $n_2$.
Since
\[ 2n_2 - \theta + v(c) - r \le k \le 2 n_2 + v(c) + r \]
then in addition to $0 \le n_2 \le v(e)$ we also need
\[ \frac{k-v(c)-r}{2} \le n_2 \le \frac{k + \theta - v(c) + r}{2}. \]
In other words, we obtain
\begin{align*}
  I^{\text{5}}
  &= \sum_{k = - \theta + v(c) - r}^{2v(e) + v(c) + r}
  (-1)^k (q^s)^k \sum_{n_2 = \max\left(0, \left\lceil \frac{k - v(c) - r}{2} \right\rceil \right)}
  ^{\min\left(v(e), \left\lfloor \frac{k + \theta - v(c) + r}{2} \right\rfloor\right)}
  q^{n_2 - \max\left( 0, \left\lceil \frac{k-(v(c)+r)}{2} \right\rceil \right)} \\
  &= \sum_{k = - \theta + v(c) - r}^{2v(e) + v(c) + r}
  (-1)^k (q^s)^k
  \left( q^{\min\left( v(e), \left\lfloor \frac{k+\theta-v(c)+r}{2} \right\rfloor \right) - \max\left( 0, \left\lceil \frac{k-v(c)-r}{2} \right\rceil \right)} + \dots + q^0 \right).
\end{align*}
Here, we retain the convention from \Cref{ch:orbital2} that ellipses of the form
\[ q^i + \dots + q^{i'} \]
will denote the expression $q^i + q^{i-1} + \dots + q^{i'}$
(i.e.\ within any ellipses, the exponents are understood to decrease by $1$,
and the sums are always nonempty, meaning $i \ge i'$).

To simplify the exponent, write
\begin{equation}
  \begin{aligned}
    &\min\left( v(e), \left\lfloor \tfrac{k+\theta-v(c)+r}{2} \right\rfloor \right)
    - \max\left( 0, \left\lceil \tfrac{k-v(c)-r}{2} \right\rceil \right) \\
    &= \min\left( v(e), \left\lfloor \tfrac{k+\theta-v(c)+r}{2} \right\rfloor \right)
    + \min\left( 0, \left\lfloor \tfrac{v(c)+r-k}{2} \right\rfloor \right) \\
    &= \min\left( \left\lfloor \tfrac{k+\theta-v(c)+r}{2} \right\rfloor,
      v(e) + \left\lfloor \tfrac{v(c)+r-k}{2} \right\rfloor,
      v(e),
      \left\lfloor \tfrac{k+\theta-v(c)+r}{2} \right\rfloor
      + \left\lfloor \tfrac{v(c)+r-k}{2} \right\rfloor \right).
  \end{aligned}
  \label{eq:case_5_exponent}
\end{equation}
This already completes \Cref{thm:semi_lie_formula}
in the situation when $\theta$ is odd since
\textbf{Case 6\ts+} and \textbf{Case 6\ts-} do not appear at all.
However, let's turn to the remaining cases first.

\section{The contribution for Case 6\ts+ and Case 6\ts-}
Herein we assume $\theta = 2v(d-a) > v(b) + v(c)$ is even,
and in particular $\theta \ge 0$.
We get a contribution of
\begin{align*}
  I^{\text{6+}}
  &\coloneqq \sum_{n_2 = 0}^{v(e)}
  \sum_{m = \theta + 2r + 1}
  ^{\max\left( r, n_2 - \frac{\theta}{2} \right) + v(b) + v(c) + r}
    q^{\min\left( n_2, \frac{\theta}{2} + r \right)}
    (-q^s)^{2n_2 - m + v(c) + r} \\
  I^{\text{6-}}
  &\coloneqq \sum_{n_2 = 0}^{v(e)}
  \sum_{m = \theta + 2r + 1}^{n_2 + \frac{\theta}{2} + r}
    q^{\min\left( n_2, \frac{\theta}{2} + r \right)}
    (-q^s)^{2n_2 - m + v(c) + r}.
\end{align*}
We split both contributions by cases on whether
$n_2 \le \frac{\theta}{2} + r$ and $n_2 > \frac{\theta}{2} + r$.

In $I^{\text{6-}}$ this is easier to do, because if $n_2 \le \frac{\theta}{2} + r$
then the inner sum of $I^{\text{6-}}$ has empty range anyway.
Consequently, we can simply write
\begin{align*}
  I^{\text{6-}}
  &= q^{\frac{\theta}{2} + r} \sum_{n_2 = \frac{\theta}{2} + r + 1}^{v(e)}
  \sum_{m = \theta + 2r + 1}^{n_2 + \frac{\theta}{2} + r} (-q^s)^{2n_2 - m + v(c) + r}
\end{align*}
which in particular is nonempty only when $v(e) > \frac{\theta}{2} + r$
(that is $v(e) > v(d-a) + r$).
In that case, simplifying the inner sum gives
\[
  I^{\text{6-}}
  = q^{\frac{\theta}{2} + r} \sum_{n_2 = \frac{\theta}{2} + r + 1}^{v(e)}
  \left(
    (-q^s)^{2n_2 - \theta + v(c) - r - 1}
    + \dots
    + (-q^s)^{n_2 - \frac{\theta}{2} + v(c)}
  \right).
\]
We collect the coefficient of $(-q^s)^k$ for each $k$.
The lowest value of $k$ which appears is $k = v(c) + r + 1$;
the highest one is $k = 2v(e) - \theta + v(c) - r - 1$.
For these $k$,
the coefficient is the number of integers $n_2$ such that
\[ \frac{\theta}{2} + r + 1 \le n_2 \le v(e) \]
and
\begin{align*}
  n_2 - \frac{\theta}{2} + v(c) &\le k \le 2n_2 - \theta - r - 1 + v(c) \\
  \iff \frac{k + \theta - v(c) + r + 1}{2} &\le n_2 \le k + \frac{\theta}{2} - v(c).
\end{align*}
Note we already have $\frac{k + \theta - v(c) + r + 1}{2} \ge \frac{\theta}{2}+r+1$
for $k$ in the desired range.
Hence we have
\begin{align*}
  I^{\text{6-}}
  &=
  q^{\frac{\theta}{2} + r}
  \sum_{k = v(c) + r + 1}^{2v(e) - \theta + v(c) - r - 1}
  \bigg( 1 + \min\left( v(e), k + \frac{\theta}{2} - v(c) \right) \\
    &\hspace{16ex} - \max\left( \frac{\theta}{2} + r + 1,
      \left\lceil \frac{k + \theta - v(c) + r + 1}{2} \right\rceil \right) \bigg) (-q^s)^k \\
  &=
  q^{\frac{\theta}{2} + r}
  \sum_{k = v(c) + r + 1}^{2v(e) - \theta + v(c) - r - 1}
  \left( 1 + \min\left( v(e), k + \frac{\theta}{2} - v(c) \right)
    - \left\lceil \frac{k + \theta - v(c) + r + 1}{2} \right\rceil
  \right) (-q^s)^k.
\end{align*}

We now turn to splitting $I^{\text{6+}}$ this time:
\begin{align*}
  I^{\text{6+}}
  &= \sum_{n_2 = 0}^{\frac{\theta}{2} + r}
  \sum_{m = \theta + 2r + 1}^{v(b) + v(c) + 2r}
    q^{n_2} (-q^s)^{2n_2 - m + v(c) + r} \\
  &+ q^{\frac{\theta}{2} + r} \sum_{n_2 = \frac{\theta}{2} + r + 1}^{v(e)}
  \sum_{m = \theta + 2r + 1}^{n_2 - \frac{\theta}{2} + v(b) + v(c) + r}
    (-q^s)^{2n_2 - m + v(c) + r}.
\end{align*}
The second double sum of $I^{\text{6+}}$ is again nonempty only when $v(e) > \theta + 2r$.
If that's the case, we compute it
in a similar way to $I^{\text{6-}}$ by putting
\begin{align*}
  &q^{\frac{\theta}{2} + r} \sum_{n_2 = \frac{\theta}{2} + r + 1}^{v(e)}
  \sum_{m = \theta + 2r + 1}^{n_2 - \frac{\theta}{2} + v(b) + v(c) + r}
    (-q^s)^{2n_2 - m + v(c) + r} \\
  &= q^{\frac{\theta}{2} + r} \sum_{n_2 = \frac{\theta}{2} + r + 1}^{v(e)}
  \left(
    (-q^s)^{2n_2 - \theta + v(c) - r - 1}
    + \dots
    + (-q^s)^{n_2 + \frac{\theta}{2} - v(b)}
  \right).
\end{align*}
Again we calculate the coefficient of $(-q^s)^k$.
The values of $k$ run from the lowest value $k = \theta - v(b) + r + 1$
and end at the highest value $k = 2v(e) - \theta + v(c) - r - 1$.
In this range we need $\frac{\theta}{2} + r + 1 \le n_2 \le v(e)$ and
\begin{align*}
  n_2 + \frac{\theta}{2} - v(b) &\le k \le 2n_2 - \theta + v(c) - r - 1 \\
  \iff \frac{k + \theta - v(c) + r + 1}{2} &\le n_2 \le k - \frac{\theta}{2} + v(b).
\end{align*}
The double sum therefore becomes
\begin{align*}
  &\phantom=
  q^{\frac{\theta}{2} + r}
  \sum_{k = \theta - v(b) + r + 1}^{2v(e) - \theta + v(c) - r - 1}
  \bigg( 1 + \min\left( v(e), k - \frac{\theta}{2} + v(b) \right) \\
    &\hspace{16ex} - \max\left( \frac{\theta}{2} + r + 1,
      \left\lceil \frac{k + \theta - v(c) + r + 1}{2} \right\rceil \right) \bigg) (-q^s)^k
\end{align*}
It is natural to split this sum into $k \le v(c) + r$ and $k > v(c) + r$.
In the former case, we have both $k - \frac{\theta}{2} + v(b) \le v(e)$
and $\frac{\theta}{2} + r + 1 \ge \left\lceil \frac{k + \theta - v(c) + r + 1}{2} \right\rceil$;
in the latter case we have just
$\frac{\theta}{2} + r + 1 \le \left\lceil \frac{k + \theta - v(c) + r + 1}{2} \right\rceil$
instead.
Hence, the double sum simplifies further to
\begin{align*}
  &= q^{\frac{\theta}{2} + r}
  \sum_{k = \theta - v(b) + r + 1}^{v(c) + r}
  \bigg( 1 + \left( k - \frac{\theta}{2} + v(b) \right)
    - \left( \frac{\theta}{2} + r + 1 \right) \bigg) (-q^s)^k \\
  &+ q^{\frac{\theta}{2} + r}
  \sum_{k = v(c) + r + 1}^{2v(e) - \theta + v(c) - r - 1}
  \bigg( 1 + \min\left( v(e), k - \frac{\theta}{2} + v(b) \right)
    - \left\lceil \frac{k + \theta - v(c) + r + 1}{2} \right\rceil \bigg) (-q^s)^k \\
  &= q^{\frac{\theta}{2} + r}
  \sum_{k = \theta - v(b) + r + 1}^{v(c) + r}
  \left( k - \theta + v(b) - r \right) (-q^s)^k \\
  &+ q^{\frac{\theta}{2} + r}
  \sum_{k = v(c) + r + 1}^{2v(e) - \theta + v(c) - r - 1}
  \bigg( 1 + \min\left( v(e), k - \frac{\theta}{2} + v(b) \right)
    - \left\lceil \frac{k + \theta - v(c) + r + 1}{2} \right\rceil \bigg) (-q^s)^k.
\end{align*}
Meanwhile, the first sum within $I^{\text{6+}}$ can be computed as
\begin{align*}
  \sum_{n_2 = 0}^{\frac{\theta}{2} + r}
  \sum_{m = \theta + 2r + 1}^{v(b) + v(c) + 2r}
    q^{n_2} (-q^s)^{2n_2 - m + v(c) + r}
  &= \sum_{n_2 = 0}^{\frac{\theta}{2} + r} q^{n_2}
    \sum_{m = \theta + 2r + 1}^{v(b) + v(c) + 2r}
      (-q^s)^{2n_2 - m + v(c) + r} \\
  &= \sum_{n_2 = 0}^{\frac{\theta}{2} + r} q^{n_2}
    \sum_{k = 2n_2 - v(b) - r}^{2n_2 - \theta + v(c) - r - 1} (-q^s)^k.
\end{align*}
We now interchange the summation so that $k$ is outside,
running from the lowest value $k = -v(b) - r$
to the highest value $k = v(c) + r - 1$.
From
\[ 2n_2 - v(b) - r \le k \le 2n_2 - \theta + v(c) - r - 1 \]
we require that $0 \le n_2 \le \frac{\theta}{2} + r$ and
\[ \frac{k + \theta - v(c) + r + 1}{2} \le n_2 \le \frac{k + r + v(b)}{2}. \]
In other words, we get
\[
  \sum_{k = - v(b) - r}^{v(c) + r - 1}
  (-q^s)^k
  \sum_{n_2 = \max\left(0, \left\lceil \frac{k + \theta - v(c) + r + 1}{2} \right\rceil \right)}
  ^{\min\left( \frac{\theta}{2} + r, \left\lfloor \frac{v(b) + r + k}{2} \right\rfloor \right) } q^{n_2}.
\]
Hence the total contribution from Case 6 can be written as
\begin{align*}
  I^{\text{6+}} + I^{\text{6-}}
  &=
  \sum_{k = - v(b) - r}^{v(c) + r - 1} (-q^s)^k \left(
    q^{\min\left( \frac{\theta}{2} + r, \left\lfloor \frac{v(b) + r + k}{2} \right\rfloor \right)}
    + \dots
    + q^{\max\left(0, \left\lceil \frac{k + \theta - v(c) + r + 1}{2} \right\rceil \right)}
    \right) \\
  &+ q^{\frac{\theta}{2} + r}
  \sum_{k = \theta - v(b) + r + 1}^{v(c) + r}
  \left( k - \theta + v(b) - r \right) (-q^s)^k \\
  &+ q^{\frac{\theta}{2} + r}
  \sum_{k = v(c) + r + 1}^{2v(e) - \theta + v(c) - r - 1}
  \bigg( 2 +
    \min\left( v(e), k - \frac{\theta}{2} + v(b) \right)
    + \min\left( v(e), k + \frac{\theta}{2} - v(c) \right) \\
    &\hspace{16ex} - 2\left\lceil \frac{k + \theta - v(c) + r + 1}{2} \right\rceil \bigg) (-q^s)^k
\end{align*}
where the latter two double sums contributing to $q^{\frac{\theta}{2}+r}$ above
are present only when $v(e) > \theta + 2r$.

In the situation where $v(e) > \theta + 2r$,
we'd like to simplify the coefficient of $q^{\frac{\theta}{2}+r}$ as follows.
First, we may as well write
\begin{align*}
  2 - 2\left\lceil \frac{k + \theta - v(c) + r + 1}{2} \right\rceil
  &= 2 - \left( (k + \theta - v(c) + r + 1)
  + \mathbf{1}_{k + \theta + v(c) + r \equiv 1 \pmod 2} \right) \\
  &= \mathbf{1}_{k + \theta + v(c) + r \equiv 0 \pmod 2}
  + v(c) - \theta - k - r.
\end{align*}
Set aside the indicator function
$\mathbf{1}_{k + \theta + v(c) + r \equiv 0 \pmod 2}$ momentarily;
we will merge it in a moment.
To consolidate the minimum's in the third double sum,
note that we have
\[ v(c) + r + 1 \le v(e) + \frac{\theta}{2} - v(b)
< v(e) + v(c) - \frac{\theta}{2} \le 2v(e) - \theta + v(c) - r - 1. \]
Hence, based on the value of $k$, we get the following coefficients:
\begin{itemize}
  \ii If $v(c) + r + 1 \le k \le v(e) + \frac{\theta}{2} - v(b)$, we get
  \begin{align*}
    \left( v(c) - \theta - k - r \right)
    &+ \left( k - \frac{\theta}{2} + v(b) \right)
    + \left( k + \frac{\theta}{2} - v(c) \right)  \\
    &= k - \theta + v(b) - r.
  \end{align*}

  \ii If $v(e) + \frac{\theta}{2} - v(b) \le k \le v(e) + v(c) - \frac{\theta}{2}$, we get
  \begin{align*}
    \left( v(c) - \theta - k - r \right)
    &+ v(e)
    + \left( k + \frac{\theta}{2} - v(c) \right)  \\
    &= v(e) - \frac{\theta}{2} - r.
  \end{align*}

  \ii If $v(e) + v(c) - \frac{\theta}{2} \le k \le 2v(e) - \theta + v(c) - r$, we get
  \begin{align*}
    \left( v(c) - \theta - k - r \right)
    &+ v(e) + v(e) \\
    &= 2v(e) + v(c) - \theta - r.
  \end{align*}
\end{itemize}
Noting the expression in the first bullet also matches the coefficient of $(-q^s)^k$
for $\theta - v(b) + r + 1 \le k \le v(c) + r$, we can now write
\begin{align*}
  I^{\text{6+}} + I^{\text{6-}}
  &=
  \sum_{k = - v(b) - r}^{v(c) + r - 1} (-q^s)^k \left(
    q^{\min\left( \frac{\theta}{2} + r, \left\lfloor \frac{v(b) + r + k}{2} \right\rfloor \right)}
    + \dots
    + q^{\max\left(0, \left\lceil \frac{k + \theta - v(c) + r + 1}{2} \right\rceil \right)}
    \right) \\
  &+ q^{\frac{\theta}{2} + r}
  \sum_{k = \theta - v(b) + r + 1}^{2v(e) - \theta + v(c) - r - 1}
  \min \left( k - \theta + v(b) - r, v(e) - \frac{\theta}{2} - r, 2v(e) + v(c) - \theta - r \right) (-q^s)^k \\
  &+ q^{\frac{\theta}{2} + r}
  \sum_{k = v(c) + r + 1}^{2v(e) - \theta + v(c) - r - 1}
  \mathbf{1}_{k + \theta + v(c) + r \equiv 0 \pmod 2} (-q^s)^k
\end{align*}
as the overall contribution from Case 6.

\section{Completed formula for the overall orbital integral}
\begin{theorem}
  \label{thm:semi_lie_formula}
  Let the representative
  \[
    \guv = \left( \begin{bmatrix} a & b \\ c & d \end{bmatrix},
      \begin{bmatrix} 0 \\ 1 \end{bmatrix},
      \begin{bmatrix} 0 & e \end{bmatrix} \right)
    \in (S_2(F) \times V_2'(F))\rs.
  \]
  be paired with an element of $(\U(\VV_2) \times \VV_2)\rs$.
  If $v(e) < 0$ or $v(b) + v(c) < -2r$, the entire orbital integral is $0$.
  Otherwise define
  \[ \nn_\guv(k) \coloneqq \min\left( \left\lfloor \tfrac{k + (v(b)+r)}{2} \right\rfloor,
      \left\lfloor \tfrac{(2v(e)+v(c)+r)-k}{2} \right\rfloor,
      v(e), \left\lfloor \tfrac{v(b)+v(c)}{2} \right\rfloor + r,
      v(d-a) + r \right). \]
  Also, if $v(d-a) < v(e) - r$ and $v(b) + v(c) > 2v(d-a)$, then additionally define
  \begin{align*}
    \cc_\guv(k) &= \min\big( k - (2v(d-a)-v(b)+r), \\
      &\qquad (2v(e)+v(c)-2v(d-a)-r)-k, v(e)-v(d-a)-r \big).
  \end{align*}
  Otherwise define $\cc_\guv(k) = 0$.
  Then we have
  \begin{align*}
    &\phantom= \Orb(\guv, \mathbf{1}_{K'_{S, \le r}} \otimes \oneV, s) \\
    &= \sum_{k = -(v(b)+r)}^{2v(e)+v(c)+r} (-1)^k
    \left( 1 + q + q^2 + \dots + q^{\nn_\guv(k)} \right) (q^s)^k \\
    &+ \sum_{k = 2v(d-a)-v(b)+r}^{2v(e)+v(c)-2v(d-a)-r} (-1)^k \cc_\guv(k) q^{v(d-a) + r} (q^s)^k.
  \end{align*}
\end{theorem}
\begin{proof}
  First, suppose $\theta = v(b) + v(c) < 2v(d-a)$ is odd.
  Then
  \[ \nn_\guv(k) \coloneqq \min\left( \left\lfloor \tfrac{k + (v(b)+r)}{2} \right\rfloor,
      \left\lfloor \tfrac{(2v(e)+v(c)+r)-k}{2} \right\rfloor,
      v(e), \left\lfloor \tfrac{v(b)+v(c)}{2} \right\rfloor + r \right) \]
  and $\cc_\guv$ terms do not appear.
  Now, in that case, the exponent \eqref{eq:case_5_exponent} can be simplified, because
  \begin{align*}
    \left\lfloor \frac{k+\theta-v(c)+r}{2} \right\rfloor
    &= \left\lfloor \frac{k+v(b)+r}{2} \right\rfloor \\
    v(e) + \left\lfloor \frac{v(c)+r-k}{2} \right\rfloor
    &= \left\lfloor \frac{(2v(e) + v(c) + r) - k}{2} \right\rfloor \\
    \left\lfloor \frac{k+\theta-v(c)+r}{2} \right\rfloor
    + \left\lfloor \frac{v(c)+r-k}{2} \right\rfloor
    &= \frac{k+\theta-v(c)+r}{2} + \frac{v(c)+r-k}{2} - \half \\
    &= \frac{\theta-1}{2} + r \\
    &= \left\lfloor \frac{v(b)+v(c)}{2} \right\rfloor + r < v(d-a) + r.
  \end{align*}
  Hence, $\nn_\guv$ coincides with the exponent in \eqref{eq:case_5_exponent}.
  So the result is true in this case.
\end{proof}

\begin{example}
  When $v(e) = 0$ the expression is particularly simple.
  The assumption $v(d-a) \ge v(e)-r$ is automatically true, and
  $\nn_{\guv}$ is identically zero, so
  \[ \Orb(\guv, \mathbf{1}_{K'_{S, \le r}} \otimes \oneV, s)
    = \sum_{k=-(v(b)+r)}^{v(c)+r} (-q^s)^k. \]
\end{example}

\begin{example}
  Suppose $r = 14$, $v(b) = -5$, $v(c) = 100$, $v(e) = 3$.
  We have $v(d-a) \ge 0 > -11 = v(e) - r$.
  Hence the above formula reads
  \begin{align*}
    \Orb(\guv, \mathbf{1}_{K'_{S, \le 14}} \otimes \oneV, s)
    &= -q^{-9s} \\
    &+ q^{-8s} \\
    &- (q+1) \cdot q^{-7s} \\
    &+ (q+1) \cdot q^{-6s} \\
    &- (q^2+q+1) \cdot q^{-5s} \\
    &+ (q^2+q+1) \cdot q^{-4s} \\
    &- (q^3+q^2+q+1) \cdot q^{-3s} \\
    &+ (q^3+q^2+q+1) \cdot q^{-2s} \\
    &- (q^3+q^2+q+1) \cdot q^{-s} \\
    &+ (q^3+q^2+q+1) \cdot q^{0} \\
    &- (q^3+q^2+q+1) \cdot q^{s} \\
    &+ (q^3+q^2+q+1) \cdot q^{2s} \\
    &\vdotswithin= \\
    &- (q^3+q^2+q+1) \cdot q^{113s} \\
    &+ (q^3+q^2+q+1) \cdot q^{114s} \\
    &- (q^2+q+1) \cdot q^{115s} \\
    &+ (q^2+q+1) \cdot q^{116s} \\
    &- (q+1) \cdot q^{117s} \\
    &+ (q+1) \cdot q^{118s} \\
    &- q^{119s} \\
    &+ q^{120s}.
  \end{align*}
\end{example}
\begin{example}
  Suppose $r = 2$, $v(b) = -5$, $v(c) = 100$, $v(e) = 20$, $v(d-a) = 1$.
  Then we have
  \begin{align*}
    \Orb(\guv, \mathbf{1}_{K'_{S, \le 2}} \otimes \oneV, s)
    &= -q^{3s} \\
    &+ q^{4s} \\
    &- (q+1) \cdot q^{5s} \\
    &+ (q+1) \cdot q^{6s} \\
    &- (q^2+q+1) \cdot q^{7s} \\
    &+ (q^2+q+1) \cdot q^{8s} \\
    &- (q^3+q^2+q+1) \cdot q^{9s} \\
    &+ (2q^3+q^2+q+1) \cdot q^{10s} \\
    &- (3q^3+q^2+q+1) \cdot q^{9s} \\
    &+ (4q^3+q^2+q+1) \cdot q^{10s} \\
    &- (5q^3+q^2+q+1) \cdot q^{9s} \\
    &+ (6q^3+q^2+q+1) \cdot q^{10s} \\
    &\vdotswithin= \\
    &- (17q^3+q^2+q+1) \cdot q^{25s} \\
    &+ (18q^3+q^2+q+1) \cdot q^{26s} \\
    &- (18q^3+q^2+q+1) \cdot q^{27s} \\
    &+ (18q^3+q^2+q+1) \cdot q^{28s} \\
    &\vdotswithin= \\
    &- (18q^3+q^2+q+1) \cdot q^{117s} \\
    &+ (18q^3+q^2+q+1) \cdot q^{118s} \\
    &- (18q^3+q^2+q+1) \cdot q^{119s} \\
    &+ (17q^3+q^2+q+1) \cdot q^{120s} \\
    &- (16q^3+q^2+q+1) \cdot q^{121s} \\
    &+ (15q^3+q^2+q+1) \cdot q^{122s} \\
    &\vdotswithin= \\
    &+ (3q^3+q^2+q+1) \cdot q^{134s} \\
    &- (2q^3+q^2+q+1) \cdot q^{135s} \\
    &+ (q^3+q^2+q+1) \cdot q^{136s} \\
    &- (q^2+q+1) \cdot q^{137s} \\
    &+ (q^2+q+1) \cdot q^{138s} \\
    &- (q+1) \cdot q^{139s} \\
    &+ (q+1) \cdot q^{140s} \\
    &- q^{141s} \\
    &+ q^{142s}.
  \end{align*}
\end{example}
