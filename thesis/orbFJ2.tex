\chapter{Evaluation of the weighted orbital integral for $S_2(F) \times V'_2(F)$}
\label{ch:orbitalFJ2}

We now aggregate the supports we found in the previous section together with the
definition of the weighted orbital integral to extract the desired formulas.

Recall that the weighted orbital integral was defined as
\begin{align*}
  & \Orb(\guv, \phi \otimes \oneV, s) \\
  &\coloneqq
  \int_{h \in H} \phi(h\inv \gamma h)
  \oneV(h \uu, \vv^\top h^{-1})
  \eta(h) \left\lvert \det(h) \right\rvert_F^{-s} \odif h
\end{align*}
and that after taking Iwasawa decomposition as
\[ h = k \begin{pmatrix} x_1 & 0 \\ 0 & x_2 \end{pmatrix}
  \begin{pmatrix} 1 & y \\ 0 & 1 \end{pmatrix} \]
we broke the sum based on $n_1 = v(x_1)$ and $n_2 = v(x_2)$.
So the contribution to the weighted orbital integral looks like
For $h$ as above, we know that
\begin{align*}
  \eta(h) &= (-1)^{n_1 + n_2} \\
  \left\lvert \det(h) \right\rvert^{-s}_F &= (q^s)^{n_1 + n_2}.
\end{align*}
Applying \eqref{eq:n1_plus_n2_semi_lie} we find that
\[
  \eta(h)
  \left\lvert \det(h) \right\rvert^{-s}_F
  = (q^s)^{2n_ - m + v(c) + r}.
\]

\section{The contribution for Case 5}
We assume $\theta + 2r \ge 0$, because otherwise the entire sum is empty.
Hence, the total contribution for \textbf{Case 5} is
\begin{align*}
  I^{\text{5}}
  &\coloneqq \sum_{n_2 = 0}^{v(e)} \sum_{m = 0}^{\theta + 2r}
  q^{m - \max\left( m-n_2, \left\lceil m/2 \right\rceil \right)}
  (-q^s)^{2n_2 - m + v(c) + r} \\
  &\coloneqq \sum_{n_2 = 0}^{v(e)} \sum_{m = 0}^{\theta + 2r}
  q^{\min\left( n_2, \left\lfloor m/2 \right\rfloor \right)}
  (-q^s)^{2n_2 - m + v(c) + r}.
\end{align*}
We'll change the summation variable to
\[ k \coloneqq 2n_2 - m + v(c) + r
\iff m = 2n_2 - k + v(c) + r \]
Then
\begin{align*}
  I^{\text{5}}
  &\coloneqq \sum_{n_2 = 0}^{v(e)}
  \sum_{k = 2n_2 - \theta + v(c) - r}^{2n_2 + v(c) + r}
  q^{\min\left( n_2, n_2 + \left\lfloor \frac{v(c)+r-k}{2} \right\rfloor \right)} (-q^s)^{k} \\
  &= \sum_{n_2 = 0}^{v(e)}
  \sum_{k = 2n_2 - \theta + v(c) - r}^{2n_2 + v(c) + r}
  q^{n_2 - \max\left( 0, \left\lceil \frac{k-(v(c)+r)}{2} \right\rceil \right)} (-q^s)^{k}.
\end{align*}
We then interchange the order of summation so that $k$ is outside.
Then $k$ runs from the lowest value of $k = - \theta + v(c) - r$
to the largest value $k = 2v(e) + v(c) + r$ over all choices of $n_2$.
Since
\[ 2n_2 - \theta + v(c) - r \le k \le 2 n_2 + v(c) + r \]
then in addition to $0 \le n_2 \le v(e)$ we also need
\[ \frac{k-v(c)-r}{2} \le n_2 \le \frac{k + \theta - v(c) + r}{2}. \]
In other words, we obtain
\begin{align*}
  I^{\text{5}}
  &= \sum_{k = - \theta + v(c) - r}^{2v(e) + v(c) + r}
  (-1)^k (q^s)^k \sum_{n_2 = \max\left(0, \left\lceil \frac{k - v(c) - r}{2} \right\rceil \right)}
  ^{\min\left(v(e), \left\lfloor \frac{k + \theta - v(c) + r}{2} \right\rfloor\right)}
  q^{n_2 - \max\left( 0, \left\lceil \frac{k-(v(c)+r)}{2} \right\rceil \right)} \\
  &= \sum_{k = - \theta + v(c) - r}^{2v(e) + v(c) + r}
  (-1)^k (q^s)^k
  \left( q^{\min\left( v(e), \left\lfloor \frac{k+\theta-v(c)+r}{2} \right\rfloor \right) - \max\left( 0, \left\lceil \frac{k-v(c)-r}{2} \right\rceil \right)} + \dots + q^0 \right).
\end{align*}
Here, we retain the convention from \Cref{ch:orbital2} that ellipses of the form
\[ q^i + \dots + q^{i'} \]
will denote the expression $q^i + q^{i-1} + \dots + q^{i'}$
(i.e.\ within any ellipses, the exponents are understood to decrease by $1$,
and the sums are always nonempty, meaning $i \ge i'$).

To simplify the exponent, write
\begin{equation}
  \begin{aligned}
    &\min\left( v(e), \left\lfloor \tfrac{k+\theta-v(c)+r}{2} \right\rfloor \right)
    - \max\left( 0, \left\lceil \tfrac{k-v(c)-r}{2} \right\rceil \right) \\
    &= \min\left( v(e), \left\lfloor \tfrac{k+\theta-v(c)+r}{2} \right\rfloor \right)
    + \min\left( 0, \left\lfloor \tfrac{v(c)+r-k}{2} \right\rfloor \right) \\
    &= \min\left( \left\lfloor \tfrac{k+\theta-v(c)+r}{2} \right\rfloor,
      v(e) + \left\lfloor \tfrac{v(c)+r-k}{2} \right\rfloor,
      v(e),
      \left\lfloor \tfrac{k+\theta-v(c)+r}{2} \right\rfloor
      + \left\lfloor \tfrac{v(c)+r-k}{2} \right\rfloor \right).
  \end{aligned}
  \label{eq:case_5_exponent}
\end{equation}
This already completes \Cref{thm:semi_lie_formula}
in the situation when $\theta$ is odd since
\textbf{Case 6\ts+} and \textbf{Case 6\ts-} do not appear at all.
However, let's turn to the remaining cases first.

\section{The contribution for Case 6\ts+ and Case 6\ts-}
Herein we assume $\theta = 2v(d-a) > v(b) + v(c)$ is even,
and in particular $\theta \ge 0$.
We get a contribution of
\begin{align*}
  I^{\text{6+}}
  &\coloneqq \sum_{n_2 = 0}^{v(e)}
  \sum_{m = \theta + 2r + 1}
  ^{\max\left( r, n_2 - \frac{\theta}{2} \right) + v(b) + v(c) + r}
    q^{\min\left( n_2, \frac{\theta}{2} + r \right)}
    (-q^s)^{2n_2 - m + v(c) + r} \\
  I^{\text{6-}}
  &\coloneqq \sum_{n_2 = 0}^{v(e)}
  \sum_{m = \theta + 2r + 1}^{n_2 + \frac{\theta}{2} + r}
    q^{\min\left( n_2, \frac{\theta}{2} + r \right)}
    (-q^s)^{2n_2 - m + v(c) + r}.
\end{align*}
We will split $I^{\text{6+}}$ into two parts:
\begin{align*}
  I^{\text{6+}}
  &= \sum_{n_2 = 0}^{\frac{\theta}{2} + r}
  \sum_{m = \theta + 2r + 1}^{v(b) + v(c) + 2r}
    q^{n_2} (-q^s)^{2n_2 - m + v(c) + r} \\
  &+ q^{\frac{\theta}{2} + r} \sum_{n_2 = \frac{\theta}{2} + r + 1}^{v(e)}
  \sum_{m = \theta + 2r + 1}^{n_2 - \frac{\theta}{2} + v(b) + v(c) + r}
    (-q^s)^{2n_2 - m + v(c) + r}.
\end{align*}
Note that the second sum is nonempty only when $v(e) > \frac{\theta}{2} + r$.
So we consider cases on this in what follows.

\subsection{Sub-case where $v(e) \le \frac{\theta}{2} + r$}
First, suppose $v(e) \le \frac{\theta}{2} + r$.
Then the contribution of \textbf{Case 6\ts{-}} is void,
since the inner sum of $I^{\text{6-}}$ contributes only when $n_2 > \frac{\theta}{2} + r$.
We only need to consider
\begin{align*}
  I^{\text{6+}}
  &= \sum_{n_2=0}^{v(e)} \sum_{m=\theta+2r+1}^{v(b)+v(c)+2r}
    q^{n_2} (-q^s)^{2n_2-m+v(c)+r} \\
  &= \sum_{n_2=0}^{v(e)} \sum_{k=2n_2-v(b)-r}^{2n_2-\theta+v(c)-r-1}
    q^{n_2} (-q^s)^k.
\end{align*}
Swapping the summation order so that $k$ is outside,
the sum runs from the lowest value $k = -v(b) - r$
up to the highest value $k = 2v(e) - \theta + v(c) - r - 1$,
subject to $0 \le n_2 \le v(e)$ and
\begin{align*}
  2n_2 - v(b) - r &\le k \le 2n_2 - \theta + v(c) - r - 1 \\
  \iff \left\lceil \frac{k + \theta - v(c) + r + 1}{2} \right\rceil
  &\le n_2 \le \left\lfloor \frac{k + v(b) + r}{2} \right\rfloor.
\end{align*}
Thus,
\[ I^{\text{6+}}
  = \sum_{k = -v(b) - r}^{2v(e) - \theta + v(c) - r -1}
    \sum_{n_2 = \max(0, \left\lceil \frac{k + \theta - v(c) + r + 1}{2} \right\rceil)}
    ^{\min(v(e), \left\lfloor \frac{k + v(b) + r}{2} \right\rfloor)}
    q^{n_2} (-q^s)^k.
\]

\subsection{Sub-case where $v(e) > \frac{\theta}{2} + r$}
We start on $I^{\text{6-}}$; note if $n_2 \le \frac{\theta}{2} + r$
then the inner sum of $I^{\text{6-}}$ has empty range anyway.
Consequently, we can simply write
\begin{align*}
  I^{\text{6-}}
  &= q^{\frac{\theta}{2} + r} \sum_{n_2 = \frac{\theta}{2} + r + 1}^{v(e)}
  \sum_{m = \theta + 2r + 1}^{n_2 + \frac{\theta}{2} + r} (-q^s)^{2n_2 - m + v(c) + r}
\end{align*}
which in particular is nonempty.
In that case, simplifying the inner sum gives
\[
  I^{\text{6-}}
  = q^{\frac{\theta}{2} + r} \sum_{n_2 = \frac{\theta}{2} + r + 1}^{v(e)}
  \left(
    (-q^s)^{2n_2 - \theta + v(c) - r - 1}
    + \dots
    + (-q^s)^{n_2 - \frac{\theta}{2} + v(c)}
  \right).
\]
We collect the coefficient of $(-q^s)^k$ for each $k$.
The lowest value of $k$ which appears is $k = v(c) + r + 1$;
the highest one is $k = 2v(e) - \theta + v(c) - r - 1$.
For these $k$,
the coefficient is the number of integers $n_2$ such that
\[ \frac{\theta}{2} + r + 1 \le n_2 \le v(e) \]
and
\begin{align*}
  n_2 - \frac{\theta}{2} + v(c) &\le k \le 2n_2 - \theta - r - 1 + v(c) \\
  \iff \frac{k + \theta - v(c) + r + 1}{2} &\le n_2 \le k + \frac{\theta}{2} - v(c).
\end{align*}
Note we already have $\frac{k + \theta - v(c) + r + 1}{2} \ge \frac{\theta}{2}+r+1$
for $k$ in the desired range.
Hence we have
\begin{align*}
  I^{\text{6-}}
  &=
  q^{\frac{\theta}{2} + r}
  \sum_{k = v(c) + r + 1}^{2v(e) - \theta + v(c) - r - 1}
  \bigg( 1 + \min\left( v(e), k + \frac{\theta}{2} - v(c) \right) \\
    &\hspace{16ex} - \max\left( \frac{\theta}{2} + r + 1,
      \left\lceil \frac{k + \theta - v(c) + r + 1}{2} \right\rceil \right) \bigg) (-q^s)^k \\
  &=
  q^{\frac{\theta}{2} + r}
  \sum_{k = v(c) + r + 1}^{2v(e) - \theta + v(c) - r - 1}
  \left( 1 + \min\left( v(e), k + \frac{\theta}{2} - v(c) \right)
    - \left\lceil \frac{k + \theta - v(c) + r + 1}{2} \right\rceil
  \right) (-q^s)^k.
\end{align*}

The second double sum of $I^{\text{6+}}$ is again nonempty
since $v(e) > \frac{\theta}{2} + r$.
So we compute it
in a similar way to $I^{\text{6-}}$ by putting
\begin{align*}
  &q^{\frac{\theta}{2} + r} \sum_{n_2 = \frac{\theta}{2} + r + 1}^{v(e)}
  \sum_{m = \theta + 2r + 1}^{n_2 - \frac{\theta}{2} + v(b) + v(c) + r}
    (-q^s)^{2n_2 - m + v(c) + r} \\
  &= q^{\frac{\theta}{2} + r} \sum_{n_2 = \frac{\theta}{2} + r + 1}^{v(e)}
  \left(
    (-q^s)^{2n_2 - \theta + v(c) - r - 1}
    + \dots
    + (-q^s)^{n_2 + \frac{\theta}{2} - v(b)}
  \right).
\end{align*}
Again we calculate the coefficient of $(-q^s)^k$.
The values of $k$ run from the lowest value $k = \theta - v(b) + r + 1$
and end at the highest value $k = 2v(e) - \theta + v(c) - r - 1$.
In this range we need $\frac{\theta}{2} + r + 1 \le n_2 \le v(e)$ and
\begin{align*}
  n_2 + \frac{\theta}{2} - v(b) &\le k \le 2n_2 - \theta + v(c) - r - 1 \\
  \iff \frac{k + \theta - v(c) + r + 1}{2} &\le n_2 \le k - \frac{\theta}{2} + v(b).
\end{align*}
The double sum therefore becomes
\begin{align*}
  &\phantom=
  q^{\frac{\theta}{2} + r}
  \sum_{k = \theta - v(b) + r + 1}^{2v(e) - \theta + v(c) - r - 1}
  \bigg( 1 + \min\left( v(e), k - \frac{\theta}{2} + v(b) \right) \\
    &\hspace{16ex} - \max\left( \frac{\theta}{2} + r + 1,
      \left\lceil \frac{k + \theta - v(c) + r + 1}{2} \right\rceil \right) \bigg) (-q^s)^k
\end{align*}
It is natural to split this sum into $k \le v(c) + r$ and $k > v(c) + r$.
In the former case, we have both $k - \frac{\theta}{2} + v(b) \le v(e)$
and $\frac{\theta}{2} + r + 1 \ge \left\lceil \frac{k + \theta - v(c) + r + 1}{2} \right\rceil$;
in the latter case we have just
$\frac{\theta}{2} + r + 1 \le \left\lceil \frac{k + \theta - v(c) + r + 1}{2} \right\rceil$
instead.
Hence, the double sum simplifies further to
\begin{align*}
  &\phantom= q^{\frac{\theta}{2} + r}
  \sum_{k = \theta - v(b) + r + 1}^{v(c) + r}
  \bigg( 1 + \left( k - \frac{\theta}{2} + v(b) \right)
    - \left( \frac{\theta}{2} + r + 1 \right) \bigg) (-q^s)^k \\
  &+ q^{\frac{\theta}{2} + r}
  \sum_{k = v(c) + r + 1}^{2v(e) - \theta + v(c) - r - 1}
  \bigg( 1 + \min\left( v(e), k - \frac{\theta}{2} + v(b) \right)
    - \left\lceil \frac{k + \theta - v(c) + r + 1}{2} \right\rceil \bigg) (-q^s)^k \\
  &= q^{\frac{\theta}{2} + r}
  \sum_{k = \theta - v(b) + r + 1}^{v(c) + r}
  \left( k - \theta + v(b) - r \right) (-q^s)^k \\
  &+ q^{\frac{\theta}{2} + r}
  \sum_{k = v(c) + r + 1}^{2v(e) - \theta + v(c) - r - 1}
  \bigg( 1 + \min\left( v(e), k - \frac{\theta}{2} + v(b) \right)
    - \left\lceil \frac{k + \theta - v(c) + r + 1}{2} \right\rceil \bigg) (-q^s)^k.
\end{align*}
Meanwhile, the first sum within $I^{\text{6+}}$ can be computed as
\begin{align*}
  \sum_{n_2 = 0}^{\frac{\theta}{2} + r}
  \sum_{m = \theta + 2r + 1}^{v(b) + v(c) + 2r}
    q^{n_2} (-q^s)^{2n_2 - m + v(c) + r}
  &= \sum_{n_2 = 0}^{\frac{\theta}{2} + r} q^{n_2}
    \sum_{m = \theta + 2r + 1}^{v(b) + v(c) + 2r}
      (-q^s)^{2n_2 - m + v(c) + r} \\
  &= \sum_{n_2 = 0}^{\frac{\theta}{2} + r} q^{n_2}
    \sum_{k = 2n_2 - v(b) - r}^{2n_2 - \theta + v(c) - r - 1} (-q^s)^k.
\end{align*}
We now interchange the summation so that $k$ is outside,
running from the lowest value $k = -v(b) - r$
to the highest value $k = v(c) + r - 1$.
From
\[ 2n_2 - v(b) - r \le k \le 2n_2 - \theta + v(c) - r - 1 \]
we require that $0 \le n_2 \le \frac{\theta}{2} + r$ and
\[ \frac{k + \theta - v(c) + r + 1}{2} \le n_2 \le \frac{k + r + v(b)}{2}. \]
In other words, we get
\[
  \sum_{k = - v(b) - r}^{v(c) + r - 1}
  (-q^s)^k
  \sum_{n_2 = \max\left(0, \left\lceil \frac{k + \theta - v(c) + r + 1}{2} \right\rceil \right)}
  ^{\min\left( \frac{\theta}{2} + r, \left\lfloor \frac{v(b) + r + k}{2} \right\rfloor \right) } q^{n_2}.
\]
Hence the total contribution from \textbf{Case 6} can be written as
\begin{align*}
  I^{\text{6+}} + I^{\text{6-}}
  &=
  \sum_{k = - v(b) - r}^{v(c) + r - 1} (-q^s)^k \left(
    q^{\min\left( \frac{\theta}{2} + r, \left\lfloor \frac{v(b) + r + k}{2} \right\rfloor \right)}
    + \dots
    + q^{\max\left(0, \left\lceil \frac{k + \theta - v(c) + r + 1}{2} \right\rceil \right)}
    \right) \\
  &+ q^{\frac{\theta}{2} + r}
  \sum_{k = \theta - v(b) + r + 1}^{v(c) + r}
  \left( k - \theta + v(b) - r \right) (-q^s)^k \\
  &+ q^{\frac{\theta}{2} + r}
  \sum_{k = v(c) + r + 1}^{2v(e) - \theta + v(c) - r - 1}
  \bigg( 2 +
    \min\left( v(e), k - \frac{\theta}{2} + v(b) \right)
    + \min\left( v(e), k + \frac{\theta}{2} - v(c) \right) \\
    &\hspace{16ex} - 2\left\lceil \frac{k + \theta - v(c) + r + 1}{2} \right\rceil \bigg) (-q^s)^k.
\end{align*}
We'd like to further simplify the coefficient of $q^{\frac{\theta}{2}+r}$ as follows.
First, we may as well write
\begin{align*}
  2 - 2\left\lceil \frac{k + \theta - v(c) + r + 1}{2} \right\rceil
  &= 2 - \left( (k + \theta - v(c) + r + 1)
  + \mathbf{1}_{k + \theta + v(c) + r \equiv 1 \bmod 2} \right) \\
  &= \mathbf{1}_{k + \theta + v(c) + r \equiv 0 \bmod 2}
  + v(c) - \theta - k - r.
\end{align*}
Set aside the indicator function
$\mathbf{1}_{k + \theta + v(c) + r \equiv 0 \bmod 2}$ momentarily;
we will merge it in a moment.
To consolidate the minimum's in the third double sum,
note that we have
\[ v(c) + r + 1 \le v(e) + \frac{\theta}{2} - v(b)
< v(e) + v(c) - \frac{\theta}{2} \le 2v(e) - \theta + v(c) - r - 1. \]
Hence, based on the value of $k$, we get the following coefficients:
\begin{itemize}
  \ii If $v(c) + r + 1 \le k \le v(e) + \frac{\theta}{2} - v(b)$, we get
  \begin{align*}
    \left( v(c) - \theta - k - r \right)
    &+ \left( k - \frac{\theta}{2} + v(b) \right)
    + \left( k + \frac{\theta}{2} - v(c) \right)  \\
    &= k - \theta + v(b) - r.
  \end{align*}

  \ii If $v(e) + \frac{\theta}{2} - v(b) \le k \le v(e) + v(c) - \frac{\theta}{2}$, we get
  \begin{align*}
    \left( v(c) - \theta - k - r \right)
    &+ v(e)
    + \left( k + \frac{\theta}{2} - v(c) \right)  \\
    &= v(e) - \frac{\theta}{2} - r.
  \end{align*}

  \ii If $v(e) + v(c) - \frac{\theta}{2} \le k \le 2v(e) - \theta + v(c) - r$, we get
  \begin{align*}
    \left( v(c) - \theta - k - r \right)
    &+ v(e) + v(e) \\
    &= 2v(e) + v(c) - \theta - r.
  \end{align*}
\end{itemize}
Noting the expression in the first bullet also matches the coefficient of $(-q^s)^k$
for $\theta - v(b) + r + 1 \le k \le v(c) + r$, we can now write
\begin{align*}
  I^{\text{6+}} + I^{\text{6-}}
  &=
  \sum_{k = - v(b) - r}^{v(c) + r - 1} (-q^s)^k \left(
    q^{\min\left( \frac{\theta}{2} + r, \left\lfloor \frac{v(b) + r + k}{2} \right\rfloor \right)}
    + \dots
    + q^{\max\left(0, \left\lceil \frac{k + \theta - v(c) + r + 1}{2} \right\rceil \right)}
    \right) \\
  &+ q^{\frac{\theta}{2} + r}
  \sum_{k = \theta - v(b) + r + 1}^{2v(e) - \theta + v(c) - r - 1} \cc_\guv(k)(-q^s)^k \\
  &+ q^{\frac{\theta}{2} + r}
  \sum_{k = v(c) + r + 1}^{2v(e) - \theta + v(c) - r - 1}
  \mathbf{1}_{k + \theta + v(c) + r \equiv 0 \bmod 2} (-q^s)^k
\end{align*}
as the overall contribution from \textbf{Case 6}, where
\[ \cc_\guv(k) \coloneqq \min \left( k - \theta + v(b) - r,
  v(e) - \frac{\theta}{2} - r,
  2v(e) + v(c) - \theta - r \right). \]

\section{Proof of \Cref{thm:semi_lie_formula}}
\label{sec:proof_semi_lie_formula}
We now prove \Cref{thm:semi_lie_formula}.
For reference, we provide \Cref{fig:semi_lie_sketch} sketching the shapes
of $\nn_\guv$ and $\cc_\guv$, which may be easier to think about.

\begin{figure}
  \centering
  \begin{asy}
    usepackage("amsmath");
    usepackage("amssymb");
    usepackage("mathtools");
    size(16cm);
    draw((0,0)--(6,3)--(19,3)--(25,0), lightred);
    draw((-1,0)--(26,0), grey);
    draw((-1.4,0)--(-1.8,0)--(-1.8,3)--(-1.4,3), brown);
    label("$N$", (-1.8,1.5), dir(180), brown);
    draw((26.4,0)--(26.8,0)--(26.8,3)--(26.4,3), brown);
    label("$N$", (26.8,1.5), dir(0), brown);

    label("$2N$", (3,3.8), dir(90), brown);
    label("$2N$", (22,3.8), dir(90), brown);
    draw((0,3.4)--(0,3.8)--(6,3.8)--(6,3.4), brown);
    draw((19,3.4)--(19,3.8)--(25,3.8)--(25,3.5), brown);

    draw((0,0)--(0,-1.5), dotted, Margins);
    draw((6,3)--(6,-5), dotted, Margins);
    draw((19,3)--(19,-5), dotted, Margins);
    draw((25,0)--(25,-1.5), dotted, Margins);

    for (int i=0; i<=25; ++i) {
      real y = min(floor(i/2), floor((25-i)/2), 3);
      dot((i, y), red);
    }

    label("$k=\boxed{-v(b)-r}$", (0,-1.5), dir(-90));
    label("$k=\boxed{-v(b)-r+2N}$", (6,-5), dir(-90));
    label("$k=\boxed{2v(e)+v(c)+r-2N}$", (19,-5), dir(-90));
    label("$k=\boxed{2v(e)+v(c)+r}$", (25,-1.5), dir(-90));
    label("$\mathbf{n}_{(\gamma, \mathbf{u}, \mathbf{v}^\top)}$", (12.5,3), dir(90), red);

    real h = 18; // offset downwards
    draw((6,-h)--(11,5-h)--(14,5-h)--(19,-h), lightblue);
    draw((5,-h)--(20,-h), grey);
    for (int i=6; i<=19; ++i) {
      real y = min(i-6, 19-i, 5);
      dot((i, y - h), blue);
    }
    label("$\mathbf{c}_{(\gamma, \mathbf{u}, \mathbf{v}^\top)}$", (12.5,5-h), 5*dir(90), blue);
    label("(only if $N = v(d-a) + r$)", (12.5,5-h), dir(90), blue);

    draw((6,-h)--(6,-6.8), dotted, Margins);
    draw((19,-h)--(19,-6.8), dotted, Margins);
    draw((11,5-h)--(11,-4.5-h), dotted, Margins);
    draw((14,5-h)--(14,-4.5-h), dotted, Margins);
    label("$k=\boxed{-v(b)+v(e)+v(d-a)}$", (11,-4.5-h), dir(230));
    label("$k=\boxed{v(c)+v(e)-v(d-a)}$", (14,-4.5-h), dir(310));

    draw((6,-h-0.4)--(6,-h-0.8)--(10.8,-h-0.8)--(10.8,-h-0.4), purple);
    draw((14.2,-h-0.4)--(14.2,-h-0.8)--(19,-h-0.8)--(19,-h-0.4), purple);
    label("$\varkappa \coloneqq v(e)-v(d-a)-r$", (7, -h-0.8), dir(-90), purple);
    label("$\varkappa \coloneqq v(e)-v(d-a)-r$", (18, -h-0.8), dir(-90), purple);
    draw((4.6,-h)--(4.2,-h)--(4.2,5-h)--(4.6,5-h), purple);
    draw((20.4,-h)--(20.8,-h)--(20.8,5-h)--(20.4,5-h), purple);
    label((4.2,2.5-h), "$\varkappa \coloneqq v(e)-v(d-a)-r$", dir(180), purple);
    label((20.8,2.5-h), "$\varkappa \coloneqq v(e)-v(d-a)-r$", dir(0), purple);
  \end{asy}
  \caption{Sketch of the functions in \Cref{thm:semi_lie_formula}.
    The boxed numbers indicate values of $k$.}
  \label{fig:semi_lie_sketch}
\end{figure}

\begin{proof}[Proof of \Cref{thm:semi_lie_formula}]
  First, suppose $\theta = v(b) + v(c) < 2v(d-a)$ is odd.
  Then
  \[ \nn_\guv(k) \coloneqq \min\left( \left\lfloor \tfrac{k + (v(b)+r)}{2} \right\rfloor,
      \left\lfloor \tfrac{(2v(e)+v(c)+r)-k}{2} \right\rfloor,
      v(e), \tfrac{v(b)+v(c)-1}{2} + r \right) \]
  and $\cc_\guv$ terms do not appear.
  Now, in that case, the exponent \eqref{eq:case_5_exponent} can be simplified, because
  \begin{align*}
    \left\lfloor \frac{k+\theta-v(c)+r}{2} \right\rfloor
    &= \left\lfloor \frac{k+v(b)+r}{2} \right\rfloor \\
    v(e) + \left\lfloor \frac{v(c)+r-k}{2} \right\rfloor
    &= \left\lfloor \frac{(2v(e) + v(c) + r) - k}{2} \right\rfloor \\
    \left\lfloor \frac{k+\theta-v(c)+r}{2} \right\rfloor
    + \left\lfloor \frac{v(c)+r-k}{2} \right\rfloor
    &= \frac{k+\theta-v(c)+r}{2} + \frac{v(c)+r-k}{2} - \half \\
    &= \frac{\theta-1}{2} + r \\
    &= \frac{v(b)+v(c)-1}{2} + r < v(d-a) + r.
  \end{align*}
  Hence, $\nn_\guv$ coincides with the exponent in \eqref{eq:case_5_exponent}.
  So the result is true in this case.

  Now assume instead $\theta = 2v(d-a) < v(b) + v(c)$ is even.
  Notice that
  \[ \left\lfloor \frac{k+\theta-v(c)+r}{2} \right\rfloor
    + \left\lfloor \frac{v(c)+r-k}{2} \right\rfloor
    = \frac{\theta}{2} + r - \mathbf{1}_{k + v(c) + r \equiv 0 \bmod 2}. \]

  First assume that $v(e) \le \frac{\theta}{2} + r$ and
  consider \eqref{eq:case_5_exponent}.
  For the range of values of $k$ in $I^{\text{6+}}$, that is
  \[ -v(b) - r \le k \le 2v(e)-\theta+v(c)-r-1 \]
  we have the first term of \eqref{eq:case_5_exponent} is smallest, as
  \begin{align*}
    \left\lfloor \frac{k + \theta - v(c) + r}{2} \right\rfloor &< v(e) \\
    &\le v(e) + \frac{v(b)+v(c)-1}{2}
      \le v(e) + \left\lfloor \frac{v(c)+r-k}{2} \right\rfloor \\
    \left\lfloor \frac{k + \theta - v(c) + r}{2} \right\rfloor &\le v(e)-1
      \le \frac{\theta}{2} + r -1.
  \end{align*}
  So the contributions from \textbf{Case 5} and \textbf{Case 6}
  fit together to give
  \begin{align*}
    \sum_{j=0}^{\left\lfloor \frac{k+\theta-v(c)+r}{2} \right\rfloor} q^j
    &+
    \left(
    q^{\min\left( v(e), \left\lfloor \frac{v(b) + r + k}{2} \right\rfloor \right)}
    + \dots
    + q^{\max\left(0, \left\lceil \frac{k + \theta - v(c) + r + 1}{2} \right\rceil \right)}
    \right) \\
    &=
    q^{\min(v(e), \left\lfloor \frac{v(b) + r + k}{2} \right\rfloor)} + \dots + q^0
  \end{align*}
  which thus matches the formula for $\nn_\guv(k)$.

  Now suppose instead $v(e) > \frac{\theta}{2} + r$.
  First, a similar analysis gives that the first part of $I^{\text{6+}}$ fits
  together with $I^{\text{5}}$ again.
  Indeed if
  \[ -v(b) - r \le k \le v(c) + r - 1 \]
  then in \eqref{eq:case_5_exponent} we get the first exponent again, and hence
  we again get the fit
  \begin{align*}
    \sum_{j=0}^{\left\lfloor \frac{k+\theta-v(c)+r}{2} \right\rfloor} q^j
    &+
    \left(
    q^{\min\left( \frac{\theta}{2} + r, \left\lfloor \frac{v(b) + r + k}{2} \right\rfloor \right)}
    + \dots
    + q^{\max\left(0, \left\lceil \frac{k + \theta - v(c) + r + 1}{2} \right\rceil \right)}
    \right) \\
    &=
    q^{\min(\frac{\theta}{2}+r, \left\lfloor \frac{v(b) + r + k}{2} \right\rfloor)} + \dots + q^0
  \end{align*}
  which matches the claimed formula for $\nn_\guv$ in this range.

  The remaining contribution from \textbf{Case 6\ts{+} and Case 6\ts{-}} is
  \begin{align*}
  &\phantom+ q^{\frac{\theta}{2} + r}
  \sum_{k = \theta - v(b) + r + 1}^{2v(e) - \theta + v(c) - r - 1} \cc_\guv(k)(-q^s)^k \\
  &+ q^{\frac{\theta}{2} + r}
  \sum_{k = v(c) + r + 1}^{2v(e) - \theta + v(c) - r - 1}
  \mathbf{1}_{k + \theta + v(c) + r \equiv 0 \bmod 2} (-q^s)^k.
  \end{align*}

  The first sum matches the claimed coefficient $\cc_\guv$
  (except the summation in the theorem statement includes
  endpoints at $k = \theta - v(b) + r$
  and $k = 2v(e) - \theta + v(c) - r$,
  but $\cc_\guv(k) = 0$ at these two endpoints,
  so there is no change).

  Meanwhile the second sum accounts for the discrepancy between
  the final term of \eqref{eq:case_5_exponent} and the formula for $\nn_\guv$.
  That is, the range of $k$ for which \eqref{eq:case_5_exponent}
  achieves the last minimum is exactly
  \[ v(c) + r + 1 \le k \le 2v(e) - \theta + v(c) - r - 1 \]
  and only in those cases does \eqref{eq:case_5_exponent}
  differs from $\nn_\guv$ by exactly
  $\mathbf{1}_{k + \theta + v(c) + r \equiv 0 \bmod 2}$.
  This final step shows the claimed formulas coincide.
\end{proof}

\begin{example}
  When $v(e) = 0$ the expression is particularly simple.
  The assumption $v(d-a) \ge v(e)-r$ is automatically true, and
  $\nn_{\guv}$ is identically zero, so
  \[ \Orb(\guv, \mathbf{1}_{K'_{S, \le r}} \otimes \oneV, s)
    = \sum_{k=-(v(b)+r)}^{v(c)+r} (-q^s)^k. \]
\end{example}

\begin{example}
  Suppose $r = 14$, $v(b) = -5$, $v(c) = 100$, $v(e) = 3$.
  We have $v(d-a) \ge 0 > -11 = v(e) - r$.
  Hence the above formula reads
  \begin{align*}
    \Orb(\guv, \mathbf{1}_{K'_{S, \le 14}} \otimes \oneV, s)
    &= -q^{-9s} \\
    &+ q^{-8s} \\
    &- (q+1) \cdot q^{-7s} \\
    &+ (q+1) \cdot q^{-6s} \\
    &- (q^2+q+1) \cdot q^{-5s} \\
    &+ (q^2+q+1) \cdot q^{-4s} \\
    &- (q^3+q^2+q+1) \cdot q^{-3s} \\
    &+ (q^3+q^2+q+1) \cdot q^{-2s} \\
    &- (q^3+q^2+q+1) \cdot q^{-s} \\
    &+ (q^3+q^2+q+1) \cdot q^{0} \\
    &- (q^3+q^2+q+1) \cdot q^{s} \\
    &+ (q^3+q^2+q+1) \cdot q^{2s} \\
    &\vdotswithin= \\
    &- (q^3+q^2+q+1) \cdot q^{111s} \\
    &+ (q^3+q^2+q+1) \cdot q^{112s} \\
    &- (q^3+q^2+q+1) \cdot q^{113s} \\
    &+ (q^3+q^2+q+1) \cdot q^{114s} \\
    &- (q^2+q+1) \cdot q^{115s} \\
    &+ (q^2+q+1) \cdot q^{116s} \\
    &- (q+1) \cdot q^{117s} \\
    &+ (q+1) \cdot q^{118s} \\
    &- q^{119s} \\
    &+ q^{120s}.
  \end{align*}
\end{example}
\begin{example}
  Suppose $r = 2$, $v(b) = -5$, $v(c) = 100$, $v(e) = 20$, $v(d-a) = 1$.
  Then we have
  \begin{align*}
    \Orb(\guv, \mathbf{1}_{K'_{S, \le 2}} \otimes \oneV, s)
    &= -q^{3s} \\
    &+ q^{4s} \\
    &- (q+1) \cdot q^{5s} \\
    &+ (q+1) \cdot q^{6s} \\
    &- (q^2+q+1) \cdot q^{7s} \\
    &+ (q^2+q+1) \cdot q^{8s} \\
    &- (q^3+q^2+q+1) \cdot q^{9s} \\
    &+ (2q^3+q^2+q+1) \cdot q^{10s} \\
    &- (3q^3+q^2+q+1) \cdot q^{9s} \\
    &+ (4q^3+q^2+q+1) \cdot q^{10s} \\
    &- (5q^3+q^2+q+1) \cdot q^{9s} \\
    &+ (6q^3+q^2+q+1) \cdot q^{10s} \\
    &\vdotswithin= \\
    &- (17q^3+q^2+q+1) \cdot q^{25s} \\
    &+ (18q^3+q^2+q+1) \cdot q^{26s} \\
    &- (18q^3+q^2+q+1) \cdot q^{27s} \\
    &+ (18q^3+q^2+q+1) \cdot q^{28s} \\
    &\vdotswithin= \\
    &- (18q^3+q^2+q+1) \cdot q^{117s} \\
    &+ (18q^3+q^2+q+1) \cdot q^{118s} \\
    &- (18q^3+q^2+q+1) \cdot q^{119s} \\
    &+ (17q^3+q^2+q+1) \cdot q^{120s} \\
    &- (16q^3+q^2+q+1) \cdot q^{121s} \\
    &+ (15q^3+q^2+q+1) \cdot q^{122s} \\
    &\vdotswithin= \\
    &+ (3q^3+q^2+q+1) \cdot q^{134s} \\
    &- (2q^3+q^2+q+1) \cdot q^{135s} \\
    &+ (q^3+q^2+q+1) \cdot q^{136s} \\
    &- (q^2+q+1) \cdot q^{137s} \\
    &+ (q^2+q+1) \cdot q^{138s} \\
    &- (q+1) \cdot q^{139s} \\
    &+ (q+1) \cdot q^{140s} \\
    &- q^{141s} \\
    &+ q^{142s}.
  \end{align*}
\end{example}

\section{Proof of \Cref{cor:semi_lie_derivative_single}}
With \Cref{thm:semi_lie_formula} established, we aim to calculate the derivative of
\[ \Orb(\guv, \mathbf{1}_{K', S, \le r} \otimes \oneV, s) \]
now at $s = 0$, thus proving \Cref{cor:semi_lie_derivative_single}.

For this calculation it will be more convenient to reformat \Cref{thm:semi_lie_formula}
as a sum over $q^j$ rather than $(-1)^k (q^s)^k$.
To that, continuing to write
\[ N \coloneqq \min \left(
    v(e), \frac{v(b)+v(c)-1}{2} + r,
    v(d-a) + r \right) \]
consider any index $0 \le j \le N$.
Then
\[ \nn_\guv(k) \ge j \iff 2j - v(b) - r \le k \le 2v(e)+v(c)+r-2j. \]
In other words, the first part of \Cref{cor:semi_lie_combo} can be rewritten as
\[ \sum_{k = -(v(b)+r)}^{2v(e)+v(c)+r} (-1)^k
  \left( 1 + \dots + q^{\nn_\guv(k)} \right) (q^s)^k
  = q^j \sum_{j=0}^N \left( \sum_{k=2j-v(b)-r}^{2v(e)+v(c)+r-2j} (-q^s)^k \right). \]
If we take the derivative at $s = 0$ with respect to $k$, we get
\[ \log q \sum_{j=0}^N \left( q^j \sum_{k=2j-v(b)-r}^{2v(e)+v(c)+r-2j} (-1)^k k \right). \]
The number of terms inside the summation is
$2v(e)+v(b)+v(c)+2r-4j+1$, an even number.
Each consecutive pair differs by $(-1)^{v(c)+r}$.
Hence we get
\[ (-1)^{v(c)+r} \log q \sum_{j=0}^N \left( q^j
  \cdot \left( \frac{2v(e)+v(b)+v(c)+1}{2} + r - 2j \right) \right). \]
Now in the case that $v(e) > v(d-a) + r$ and $2v(d-a) < v(b) + v(c)$,
we have to handle the additional contribution obtained when we differentiate
\begin{equation}
  q^{v(d-a)+r}
  \sum_{k = 2v(d-a)-v(b)+r}^{2v(e)+v(c)-2v(d-a)-r} (-1)^k \cc_\guv(k) (-q^s)^k
  \label{eq:top_derivative}
\end{equation}
at $s = 0$.
For brevity, we will define
\[ \varkappa \coloneqq v(e) - v(d-a) - r \ge 0. \]
(We allow the degenerate case $\kappa = 0$ for convenience,
in which case $\cc_\guv$ is still identically zero.)
We will use the following extremely easy lemma:
\begin{lemma}
  If $a_0 \le a_1$ are integers then
  \begin{align*}
    \sum_{k=a_0}^{a_1} (k-a_0) \cdot k \cdot (-1)^k
    &= (-1)^{a_1} \cdot \frac{a_1(a_1-a_0+1)}{2} - \frac{(-1)^{a_0} + (-1)^{a_1}}{4} \cdot a_0 \\
    \sum_{k=a_0}^{a_1} (a_1-k) \cdot k \cdot (-1)^k
    &= (-1)^{a_0} \cdot \frac{a_0(a_1-a_0+1)}{2} - \frac{(-1)^{a_0} + (-1)^{a_1}}{4} \cdot a_1.
  \end{align*}
\end{lemma}
\begin{proof}
  This follows trivially by induction on $a_1$.
  (Alternatively, use a symbolic engine like WolframAlpha;
  see
  \href{https://www.wolframalpha.com/input?i=sum+\%28k-a\%29*k*\%28-1\%29\%5Ek+from+k\%3Da+to+b}{here}
  for the first sum and
  \href{https://www.wolframalpha.com/input?i=sum+\%28b-k\%29*k*\%28-1\%29\%5Ek+from+k\%3Da+to+b}{here}
  for the second sum.)
\end{proof}
We return to differentiating \eqref{eq:top_derivative}.
For the range of $k$ given, we split it into three parts.
\begin{itemize}
  \ii For $-v(b) + r + 2v(d-a) \le k < -v(b) + v(e) + v(d-a)$,
  apply the first part of the lemma to conclude that the contribution to the derivative is
  \begin{align*}
    \log q \cdot \Big( & (-1)^{v(b)+v(e)+v(d-a)-1} \cdot
      \frac{(-v(b)+v(e)+v(d-a)-1) \cdot \varkappa}{2} \\
      &
      - \frac{(-1)^{v(b)+v(e)+v(d-a)+1} + (-1)^{v(b)+r}}{4} \cdot (-v(b) + r + 2v(d-a))
    \Big).
  \end{align*}

  \ii For $v(c)+v(e)-v(d-a) < k \le 2v(e)+v(c)-2v(d-a)-r$
  apply the second part of the lemma to conclude that the contribution to the derivative is
  \begin{align*}
    \log q \cdot \Big( & (-1)^{v(c)+v(e)+v(d-a)+1} \cdot
      \frac{(v(c)+v(e)-v(d-a)+1) \cdot \varkappa}{2} \\
      &
      - \frac{(-1)^{v(c)+v(e)+v(d-a)+1} + (-1)^{v(c)+r}}{4} \cdot (2v(e)+v(c)-r-2v(d-a))
    \Big).
  \end{align*}

  \ii For the region $-v(b)+v(e)+v(d-a) \le k \le v(c)+v(e)-v(d-a)$,
  we have $\cc_{\guv}(k) = \varkappa$
  and the values of $k$ form $\left\lceil \frac{v(c)+v(b)-2v(d-a)}{2} \right\rceil$ consecutive pairs.
  So the contribution to the derivative here is exactly
  \[ \log q \cdot (-1)^{v(c)+v(e)+v(d-a)} \varkappa \cdot \frac{v(b)+v(c)-2v(d-a)+1}{2}. \]
\end{itemize}
If we sum all three,
we get a contribution of $\varkappa \log q \cdot  (-1)^{v(c)+v(e)+v(d-a)}$ times
\begin{align*}
  \frac{v(b)+v(c)-2v(d-a)+1}{2} &- \frac{v(c)+v(e)-v(d-a)+1}{2} \\
  &+ \frac{-v(b)+v(e)+v(d-a)-1}{2} = -\half
\end{align*}
If $\varkappa$ is even, then that's all she wrote; we simply get
\[ (-1)^{v(c)+r+1} \cdot \frac{\varkappa}{2} \cdot q^{v(d-a)+r} \log q. \]
On the other hand, if $\varkappa$ is odd we get instead
\begin{align*}
  &\phantom+ (-1)^{v(c)+r} \frac{\varkappa}{2} q^{v(d-a)+r} \log q \\
  &- \half q^{v(d-a)+r} \log q \cdot (-1)^{v(c)+v(e)+v(d-a)} \cdot
    \Big( \left( 2v(e)+v(c)-r-2v(d-a)\right) \\
    &\hspace{20ex} - \left( -v(b)+r+2v(d-a) \right) \Big) \\
  &= (-1)^{v(c)+r} q^{v(d-a)+r} \log q \cdot \left[
    -\left( v(e)+\frac{v(b)+v(c)}{2}-2v(d-a)-r \right) + \frac{\varkappa}{2} \right].
\end{align*}
These formulas match \Cref{cor:semi_lie_derivative_single}, proving it.
(Note that we changed $\varkappa > 0$ to $\varkappa \ge 0$,
which makes no change since then the contribution is zero anyway.)

\section{Proof of \Cref{cor:semi_lie_combo}}
We can now prove \Cref{cor:semi_lie_combo}
as a corollary of \Cref{cor:semi_lie_derivative_single}.
Observe that when we add the right-hand side of \Cref{cor:semi_lie_derivative_single}
to the same right-hand side with $r$ replaced by $r-1$, almost all the terms cancel.
Indeed, the main sum for $0 \le j \le N-1$ line up:
\begin{align*}
  &(-1)^{r+v(c)} \sum_{j=0}^{N-1} q^j \cdot \left( \frac{2v(e)+v(b)+v(c)+1}{2} + r - 2j \right) \\
  &\phantom+
  (-1)^{(r-1)+v(c)} \sum_{j=0}^{N-1} q^j \cdot \left( \frac{2v(e)+v(b)+v(c)+1}{2} + (r-1) - 2j \right) \\
  &= q^{N-1} + \dots + q^0.
\end{align*}
So we consider three cases:
\begin{itemize}
\ii In the case where $v(d-a)+r > v(e)$
and $\frac{v(b)+v(c)-1}{2} + r > v(e)$
then the value of $N = v(e)$ does not change when we decrease $r$ by one.
Hence the term for $j = N$ cancels in both sums and we get exactly
we get \[ (-1)^{r+v(c)} \log q (1 + q + \dots + q^N). \]

\ii Next suppose $v(e) \ge \frac{v(b)+v(c)-1}{2} + r$
and also $2v(d-a) > v(b) + v(c)$ (so the extra terms involving $\varkappa$ are absent).
In that case $N = \frac{v(b) + v(c) - 1}{2} + r$.
Then we are left with $(-1)^{r+v(c)} \log q$ times
\begin{align*}
  &\phantom= \sum_{j=0}^{N-1} (q^j) + \left( \frac{2v(e)+v(b)+v(c)+1}{2} + r - (v(b)+v(c)-1+2r) \right) q^N \\
  &= \sum_{j=0}^{N-1} (q^j) + \left( \frac{2v(e)-v(b)-v(c)-1}{2} - r \right) q^N \\
  &= \sum_{j=0}^{N} (q^j) + \left( v(e) - r - \frac{v(b)+v(c)-1}{2} \right) q^N.
\end{align*}
This matches \Cref{thm:semi_lie_formula}.

\ii Finally, suppose $\varkappa \ge 0$ and $v(b) + v(c) > 2v(d-a)$
so that $N = v(d-a) + r$.
Then we are left with $(-1)^{r+v(c)} \log q$ times
\begin{align*}
  & \sum_{j=0}^{N-1} (q^j) + \left( \frac{2v(e)+v(b)+v(c)+1}{2} + r - 2(v(d-a)+r) \right) q^N \\
  & + q^{N} \cdot
  \begin{cases}
    -\frac{\varkappa}{2} & \text{if }\varkappa \equiv 0 \pmod 2 \\
    \frac{\varkappa}{2} - \left( v(e)+\frac{v(b)+v(c)}{2}-2v(d-a)-r \right)
    & \text{if }\varkappa \equiv 1 \pmod 2 \\
  \end{cases} \\
  & + q^{N-1} \cdot
  \begin{cases}
    -\frac{\varkappa+1}{2} & \text{if }\varkappa+1 \equiv 0 \pmod 2 \\
    \frac{\varkappa+1}{2} - \left( v(e)+\frac{v(b)+v(c)}{2}-2v(d-a)-(r-1) \right)
    & \text{if }\varkappa+1 \equiv 1 \pmod 2
  \end{cases} \\
  &= \sum_{j=0}^{N-1} (q^j)
   + q^{N} \cdot
  \begin{cases}
    \left( \frac{2v(e)+v(b)+v(c)+1}{2} + r - 2(v(d-a)+r) \right) -\frac{\varkappa}{2} & \text{if }\varkappa \equiv 0 \pmod 2 \\
    \half + \frac{\varkappa}{2} & \text{if }\varkappa \equiv 1 \pmod 2
  \end{cases} \\
  & - q^{N-1} \cdot
  \begin{cases}
    \frac{\varkappa-1}{2} - \left( v(e)+\frac{v(b)+v(c)}{2}-2v(d-a)-(r-1) \right)
    & \text{if }\varkappa \equiv 1 \pmod 2 \\
    -\frac{\varkappa+1}{2} & \text{if }\varkappa \equiv 1 \pmod 2
  \end{cases} \\
  &= \sum_{j=0}^{N} (q^j)
   + q^{N} \cdot
  \begin{cases}
    \left( \frac{2v(e)+v(b)+v(c)-1}{2} + r - 2(v(d-a)+r) \right) -\frac{\varkappa}{2} & \text{if }\varkappa \equiv 0 \pmod 2 \\
    \frac{\varkappa-1}{2} & \text{if }\varkappa \equiv 1 \pmod 2
  \end{cases} \\
  & - q^{N-1} \cdot
  \begin{cases}
    \frac{\varkappa-1}{2} - \left( v(e)+\frac{v(b)+v(c)}{2}-2v(d-a)-(r-1) \right)
    & \text{if }\varkappa \equiv 1 \pmod 2 \\
    -\frac{\varkappa+1}{2} & \text{if }\varkappa \equiv 1 \pmod 2
  \end{cases} \\
  &= \sum_{j=0}^{N} (q^j)
   + q^{N} \cdot
  \begin{cases}
    \left( \frac{v(e)+v(b)+v(c)-1}{2} - \frac32v(d-a) - \frac12r \right) & \text{if }\varkappa \equiv 0 \pmod 2 \\
    \frac{\varkappa+1}{2} & \text{if }\varkappa \equiv 1 \pmod 2
  \end{cases} \\
  & - q^{N-1} \cdot
  \begin{cases}
    -\frac{1}{2} - \left( \frac12v(e)+\frac{v(b)+v(c)}{2}-\frac32v(d-a)-\frac12r \right)
    & \text{if }\varkappa \equiv 1 \pmod 2 \\
    -\frac{\varkappa+1}{2} & \text{if }\varkappa \equiv 1 \pmod 2
  \end{cases} \\
  &= \sum_{j=0}^{N} (q^j)
   + q^{N} \cdot
  \begin{cases}
    \left( \frac{\varkappa+v(b)+v(c)-1-2v(d-a)}{2}  \right) & \text{if }\varkappa \equiv 0 \pmod 2 \\
    \frac{\varkappa-1}{2} & \text{if }\varkappa \equiv 1 \pmod 2
  \end{cases} \\
  & + q^{N-1} \cdot
  \begin{cases}
    \frac{\kappa+v(b)+v(c)+1-2v(d-a)}{2}
    & \text{if }\varkappa \equiv 1 \pmod 2 \\
    -\frac{\varkappa+1}{2} & \text{if }\varkappa \equiv 1 \pmod 2.
  \end{cases}
\end{align*}
This matches \Cref{thm:semi_lie_formula}, and the proof is complete.
\end{itemize}
