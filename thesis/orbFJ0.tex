\chapter{Synopsis of the weighted orbital integral
  $\Orb((\gamma, \uu, \vv^\top), \phi \otimes \oneV, s)$
  for $(\gamma, \uu, \vv^\top) \in (S_2(F) \times V'_2(F))\rs$
  and $\phi \in \HH(S_2(F))$}
\label{ch:orbitalFJ0}

Throughout this section, $H = \GL_n(F)$ (rather than $H = \GL_{n-1}(F)$)
and $K' = \GL_n(\OO_F)$.
For the concrete calculation, we are mostly interested in the case $n = 2$.

\section{Definition}
We will not work in the generality of a function on all of $S_n(F) \times V_n'(F)$.
Instead, for our conjecture, it will be enough to define the weighted orbital integral
in the case where our function is of the form
\[ \phi \otimes \oneV \]
where $\phi \in \HH(S_n(F))$ is the left component, and
the right component is the indicator function defined in the obvious way:
\begin{align*}
  \mathbf{1}_{\OO_F^n \times (\OO_F^n)^\vee} \colon V'_n(F) &\to \{0,1\} \\
  (\uu, \vv^\top) &\mapsto
  \begin{cases}
    1 & \uu \text{ and } \vv^\top \text{ have } \OO_F \text{-entries} \\
    0 & \text{otherwise}.
  \end{cases}
\end{align*}

Then, unsurprisingly from the definition of our action as
\[ h \cdot \guv = (h\gamma h\inv, h\uu, \vv^\top h\inv) \]
we analogously define the weighted orbital integral as follows.
\begin{definition}
  [{\cite[\S1.3]{ref:liuFJ}}]
  \label{def:orbitalFJ}
  For brevity let $\eta(h) \coloneqq \eta(\det h)$ for $h \in H$.
  For $\guv \in S_n(F) \times V'_n(F)$,
  $\phi \in \HH(S_n(F))$, and $s \in \CC$,
  we define the weighted orbital integral by
  \begin{align*}
    & \Orb((\gamma, \uu, \vv^\top), \phi \otimes \oneV, s) \\
    &\coloneqq \int_{h \in H} \phi(h\inv \gamma h) \oneV(h \uu, \vv^\top h^{-1})
    \eta(h) \left\lvert \det(h) \right\rvert_F^{-s} \odif h.
  \end{align*}
\end{definition}

As before it seems this weighted orbital integral should be related to an ordinary one.
To define it, fix a self-dual lattice $\Lambda_n$ in $\VV_n^+$ of full rank.
First, if $(g,u) \in \U(\VV_n^+) \times \VV_n^+$ and $f \in \HH(\U(\VV_n^+))$,
then we define an orbital integral for $\U(\VV_n^+) \times \VV_n^+$ by
\begin{equation}
  \Orb^{\U(\VV_n^+) \times \VV_n^+}\left( (g,u), f \otimes \mathbf{1}_{\Lambda_n} \right)
  \coloneqq \int_{\U(\VV_n^+)} f(x\inv g x) \mathbf{1}_{\Lambda_n}(x^{-1} u) \odif x.
  \label{eq:unweighted_orbital_semi_lie}
\end{equation}
Then in the spirit of \cite[Conjecture 1.9]{ref:liuFJ}
and \Cref{thm:rel_fundamental_lemma}, we propose the following.
\begin{conjecture}
  \label{conj:rel_fundamental_lemma_semilie}
  Let $\phi \in \HH(S_n(F))$ and $\guv \in (S_n(F) \times V_n')\rs$.
  If $\guv$ matches an element of $(\U(\VV_n^-) \times \VV_n^-)\rs$, then
  \[ \omega\guv \Orb(\phi \otimes \oneV, \guv, 0) = 0. \]
  If $\guv$ instead matches an element $g \in (\U(\VV_n^+) \times \VV_n^+)\rs$, then
  \[ \omega\guv \Orb(\phi \otimes \oneV, \guv, 0)
    = \Orb^{\U(\VV_n^+) \times \VV_n^+}((g,u), \BC^{\eta^{n-1}}_{S_n}(\phi) \otimes \mathbf{1}_{\Lambda_n}) \]
  where the transfer factor $\omega$ is defined in \Cref{ch:geo}.
\end{conjecture}
Wei Zhang suggests that this conjecture can be proven by similar means
to \Cref{thm:rel_fundamental_lemma},
but since it is not necessary for this paper we do not pursue this proof here.

\section{Basis for the indicator functions in $\HH(S_2(F))$}
From now on assume $n = 2$.
This section is almost an exact analog of \Cref{ch:orbital0_hecke_basis},
so we will be slightly terser.
Again set
\[ S_2(F) \coloneqq \left\{ g \in \GL_2(E) \mid g \bar{g} = \id_2 \right\}. \]
We again have a Cartan decomposition indexed by a single integer $r \ge 0$:
\begin{lemma}
  [Cartan decomposition of $S_2(F)$]
  For each integer $r \ge 0$ let
  \[ K'_{S,r} \coloneqq \GL_2(\OO_E) \cdot
    \begin{bmatrix} 0 & \varpi^r \\ \varpi^{-r} & 0 \end{bmatrix} \]
  denote the orbit of
  $\begin{bmatrix} 0 & \varpi^r \\ \varpi^{-r} & 0 \end{bmatrix}$
  under the left action of $\GL_2(\OO_E)$.
  Then we have a decomposition
  \[ S_2(F) = \coprod_{r \geq 0} K'_{S,r}. \]
\end{lemma}
\begin{proof}
  \todo{reference}
\end{proof}

Like last time, $K'_{S,r}$ is the part of $S_2(F)$
for which the most negative valuation among the nine entries is $-r$.
And as before we abbreviate the $r = 0$ term specifically:
\begin{align*}
  K'_S
  &\coloneqq K'_{S,0} \\
  &= \GL_2(\OO_E) \cdot \begin{bmatrix} & 1 \\ 1 \end{bmatrix} \\
  &= \GL_2(\OO_E) \cdot \id_2 = S_2(F) \cap \GL_2(\OO_E).
\end{align*}

Repeating the definition
\[ K'_{S, \le r} \coloneqq S_2(F) \cap \varpi^{-r} \GL_2(\OO_E)
  = K'_{S,0} \sqcup K'_{S,1} \sqcup \dots \sqcup K'_{S,r} \]
we get a basis of indicator functions for the Hecke algebra $\HH(S_2(F))$:
\begin{corollary}
  For $r \ge 0$, the indicator functions $\mathbf{1}_{K'_{S, \le r}}$
  form a basis of $\HH(S_2(F))$.
\end{corollary}

\section{Parametrization of $\gamma$}
From now on assume $n = 2$,
and that $(\gamma, \uu, \vv^\top) \in (S_2(F) \times V'_2)\rs$ is regular.

\subsection{Identifying an orbit representative}
The weighted orbital integral depends only on the $H$-orbit of $(\gamma, \uu, \vv^\top)$.
Consequently, we may assume without loss of generality
(via multiplication by a suitable change-of-basis $h \in H = \GL_2(F)$) that
\[ \uu = \begin{bmatrix} 0 \\ 1 \end{bmatrix}, \qquad
  \vv^\top = \begin{bmatrix} 0 & e \end{bmatrix} \qquad e \in F. \]
(We know $\uu$ is not the zero vector from the regular condition
applied on $(\gamma, \uu, \vv^\top)$.)

Meanwhile, we will let
$\gamma = \begin{bmatrix} a & b \\ c & d \end{bmatrix} \in \GL_2(F)$
for $a,b,c,d \in F$.
Then, viewed as an element of $\GL_3(F)$ via the embedding we described earlier, we have
\[
  (\gamma, \uu, \vv^\top)
  \mapsto \begin{bmatrix}
    a & b & 0 \\
    c & d & 1 \\
    0 & e & 0
  \end{bmatrix} \in \GL_3(F).
\]
Thus, our definition of regular requires that
$\begin{bmatrix} 0 \\ 1 \end{bmatrix}$
is linearly independent from $\begin{bmatrix} b \\ d \end{bmatrix}$
and
$\begin{bmatrix} 0 & e \end{bmatrix}$
is linearly independent from $\begin{bmatrix} c & d \end{bmatrix}$.
This is just saying that $b$, $c$, $e$ are all nonzero.
We also know that $\gamma \in S_2(F)$, which gives us relations on $a$, $b$, $c$, $d$
(the same as \cite[equation (7.3.2)]{ref:AFLspherical}); we have
\[
  \begin{bmatrix} 1 & 0 \\ 0 & 1 \end{bmatrix}
  = \begin{bmatrix} a & b \\ c & d \end{bmatrix} \begin{bmatrix} \bar a & \bar b \\ \bar c & \bar d \end{bmatrix}
  \implies
  \begin{aligned}
    \bar b c = b \bar c &= 1 - a \bar a = 1 - d \bar d \\
    \text{and } d &= - \bar a c / \bar c = -\bar a b / \bar b.
  \end{aligned}
\]

\subsection{Simplification due to the matching of non-split unitary group}
Like before, we focus on the case where regular $(\gamma, \uu, \vv^\top)$
matches an element in the non-split unitary group.
As we described in \Cref{prop:valuation_delta_matching_semilie},
this is controlled by the parity of $v(\Delta)$, where
\[ \Delta = \det \left[ \vv^\top \gamma^{i+j} \uu \right]_{0 \le i,j \le n-1}. \]
When $n=2$, for the representatives we described before,
we have
\[ \left[ \vv^\top \gamma^{i+j} \uu \right]_{0 \le i,j \le n-1}
  = \begin{bmatrix} e & de \\ de & bce + d^2e \end{bmatrix} \]
so
\[ \Delta = bce^2 = \frac{b}{\bar b}(1-a \bar a) e^2 . \]
Hence, $v(\Delta)$ is odd if and only if $v(1-a \bar a)$ is odd.
Thus, we restrict attention to the following situation:
\begin{assume}
  \label{assume:a_odd}
  We will assume that
  \[ v(1-a \bar a) \equiv 1 \pmod 2. \]
\end{assume}
In particular, $a$ must be a unit.
And since $d = -\bar a c / \bar c$, it follows $d$ is a unit.
In other words, \Cref{assume:a_odd} gives the direct corollary
\[ v(a) = v(d) = 0. \]

\section{Parameters used in the calculation of the weighted orbital integral}
The situation is simpler than \Cref{sec:param_orbital0}
and we will state our derivative in terms of the five integers
$r$, $v(b)$, $v(c)$, $v(e)$ and $v(d-a)$.
From \Cref{assume:a_odd}, we actually get that
\begin{assume}
  \label{assume:FJ}
  We have that
  \begin{itemize}
    \ii $v(b) + v(c)$ is an odd positive integer;
    \ii $v(d-a) \ge 0$.
  \end{itemize}
\end{assume}
These are the only constraints between these five numbers we will consider
(together with $r \ge 0$).
However, we mention that we will only be interested in the case when $v(e) \ge 0$
since in the case $v(e) < 0$ we will shortly see that
$\Orb(\guv, \phi \otimes \oneV, s) = 0$ identically in $s$ in that situation.

\section{Statement of the differentiated weighted orbital integral}
We can now state the following result:
\begin{theorem}
  \label{thm:semi_lie_derivative_single}
  Let the representative
  \[
    \guv = \left( \begin{bmatrix} a & b \\ c & d \end{bmatrix},
      \begin{bmatrix} 0 \\ 1 \end{bmatrix},
      \begin{bmatrix} 0 & e \end{bmatrix} \right)
    \in (S_2(F) \times V_2'(F))\rs.
  \]
  be paired with an element of $(\U(\VV_2) \times \VV_2)\rs$
  (in particular, \Cref{assume:FJ} holds).
  Let $r \ge 0$ and define
  \begin{align*}
    N &\coloneqq \min\left( v(e), \frac{v(b)+v(c)-1}{2} + r, v(d-a) + r \right) \\
    \varkappa &\coloneqq v(e) - (v(d-a)+r)
  \end{align*}
  Then
  \begin{align*}
    \frac{(-1)^{v(c)+r}}{\log q}
    &\left. \pdv{}{s} \right\rvert_{s=0}
    \Orb(\guv, (\mathbf{1}_{K'_{S, \le r}} + \mathbf{1}_{K'_{S, \le (r-1)}}) \otimes \oneV, s) \\
    &= \sum_{j=0}^N \left( \frac{2v(e)+v(b)+v(c)+1}{2} + r - 2j \right) \cdot q^j \\
    & + q^{v(d-a)+r} \cdot
    \begin{cases}
      -\frac{\varkappa}{2} & \text{if }\varkappa \equiv 0 \pmod 2 \\
      \frac{\varkappa}{2} - \left( v(e)+\frac{v(b)+v(c)}{2}-2v(d-a)-r \right)
      & \text{if }\varkappa \equiv 1 \pmod 2 \\
    \end{cases}
  \end{align*}
  where the second term is only present when $\varkappa \ge 0$ and $v(b)+v(c)>2v(d-a)$.
\end{theorem}

The formula looks somewhat simpler if one merges the contribution of
$\mathbf{1}_{K'_{S, \le r}}$ and $\mathbf{1}_{K'_{S, \le (r-1)}}$
(and in \Cref{ch:finale} we will see that
$\mathbf{1}_{K'_{S, \le r}} + \mathbf{1}_{K'_{S, \le (r-1)}}$
come up naturally again).
\begin{corollary}
  \label{cor:semi_lie_combo}
  Retain the notation of \Cref{thm:semi_lie_derivative_single} and assume $r > 0$.
  Define
  \begin{align*}
    C &\coloneqq
    \begin{cases}
      \frac{\varkappa-1}{2}
        & \text{if } \varkappa > 0 \text{ is odd}
          \text{ and } v(b) + v(c) > 2v(d-a)  \\
      \frac{\varkappa+v(b)+v(c)-2v(d-a)-1}{2}
        & \text{if } \varkappa \ge 0 \text{ is even}
          \text{ and } v(b) + v(c) > 2v(d-a)  \\
      v(e) - N
        & \text{if } v(e) \ge \frac{v(b)+v(c)-1}{2} + r
        \text{ and } 2v(d-a) > v(b) + v(c) \\
      0 & \text{otherwise}
    \end{cases} \\
    C' &\coloneqq
    \begin{cases}
      C + 1 & \text{if } \varkappa \ge 0 \text{ and } v(b)+v(c) > 2v(d-a) \\
      0 & \text{otherwise}.
    \end{cases}
  \end{align*}
  Then
  \begin{align*}
    \frac{(-1)^{v(c)+r}}{\log q} &\left. \pdv{}{s} \right\rvert_{s=0}
    \Orb(\guv, (\mathbf{1}_{K'_{S, \le r}} + \mathbf{1}_{K'_{S, \le (r-1)}}) \otimes \oneV, s) \\
    &= \left( (q^N + q^{N-1} + \dots + 1) + C q^N + C' q^{N-1} \right).
  \end{align*}
\end{corollary}

We dedicate \Cref{ch:orbitalFJ1,ch:orbitalFJ2} to the calculation
of the full orbital integral and the above formulas.
We have the following examples to showcase \Cref{cor:semi_lie_combo}.

\begin{example}
  When $v(e) < 0$, the entire expression is zero
  (indeed in that case the orbital integral is identically zero).
\end{example}

\begin{example}
  When $r=5$, $v(b) = -20$, $v(c) = 37$, $v(e) = 35$ and $v(d-a) > \frac{v(b)+v(c)}{2} = 8.5$
  the derivative in \Cref{cor:semi_lie_combo} equals
  \[ \log q \cdot (23q^{13} + q^{12} + q^{11} + q^{10} + q^9 + \dots + q + 1). \]
\end{example}

\begin{example}
  When $r = 6$, $v(b) = 10$, $v(c) = 5$, $v(e) = 7$, $v(d-a) > v(e)-r = 1$,
  the derivative in \Cref{cor:semi_lie_combo} equals
  \[ -\log q \cdot (q^7 + q^6 + q^5 + \dots + q + 1). \]
\end{example}

\begin{example}
  When $r = 8$, $v(b) = -101$, $v(c) = 1000$, $v(e) = 29$, $v(d-a) = 11$,
  the derivative in \Cref{cor:semi_lie_combo} equals
  \[ \log q \cdot (444 q^{19} + 445q^{18} + q^{17} + q^{16} + q^{15} + \dots + q + 1). \]
\end{example}
