\chapter{Setup for the orbital integral for $S_3(F)$}
\label{ch:orbital1}

In this section we set up the framework of the orbital integral
based on the definitions in the previous section.
This involves rewriting the integral as an infinite discrete double sum
over two parameters $(n,m)$ that we will introduce later,
and determining the volume of the supports of the indicator function.

\section{Reparametrization in terms of valuations}
\subsection{Computation of value in indicator function}
We are integrating over $t_1 \in E$ and $t_2 \in E$.
Regarding $h' \in H'$ as an element of $\GL_3(E)$ as described before, we have
\[ h' = \begin{bmatrix}
  t_1 & t_2 & 0  \\
  \bar t_2 & \bar t_1 & 0 \\
  0 & 0 & 1
  \end{bmatrix}. \]
We therefore have
\[ \bar{h'}\inv = \begin{bmatrix}
  \frac{t_1}{t_1 \bar t_1 - t_2 \bar t_2} & \frac{-\bar t_2}{t_1 \bar t_1 - t_2 \bar t_2} & 0 \\
  \frac{-t_2}{t_1 \bar t_1 - t_2 \bar t_2} & \frac{\bar t_1}{t_1 \bar t_1 - t_2 \bar t_2} & 0 \\
  0 & 0 & 1 \end{bmatrix}. \]
Hence
\begin{align*}
  \bar{h'} \inv \gamma h'
  &=
  \begin{bmatrix}
  \frac{t_1}{t_1 \bar t_1 - t_2 \bar t_2} & \frac{-\bar t_2}{t_1 \bar t_1 - t_2 \bar t_2} & 0 \\
  \frac{-t_2}{t_1 \bar t_1 - t_2 \bar t_2} & \frac{\bar t_1}{t_1 \bar t_1 - t_2 \bar t_2} & 0 \\
  0 & 0 & 1 \end{bmatrix}
  \begin{bmatrix}
    a & 0 & 0 \\
    b & - \bar d & 1 \\
    c & 1 - d \bar d & d
  \end{bmatrix}
  \begin{bmatrix}
  t_1 & t_2 & 0  \\
  \bar t_2 & \bar t_1 & 0 \\
  0 & 0 & 1
  \end{bmatrix} \\
  &=
  \begin{bmatrix}
  \frac{t_1}{t_1 \bar t_1 - t_2 \bar t_2} & \frac{-\bar t_2}{t_1 \bar t_1 - t_2 \bar t_2} & 0 \\
  \frac{-t_2}{t_1 \bar t_1 - t_2 \bar t_2} & \frac{\bar t_1}{t_1 \bar t_1 - t_2 \bar t_2} & 0 \\
  0 & 0 & 1 \end{bmatrix}
  \begin{bmatrix}
    at_1 & at_2 & 0 \\
    bt_1 - \bar d \bar t_2 & b t_2 - \bar d \bar t_1 & 1 \\
    ct_1 + (1-d\bar d)\bar t_2 & ct_2 + (1-d \bar d) \bar t_1 & d
  \end{bmatrix}
  \\
  &=
  \begin{bmatrix}
    \dfrac{at_1^2 - bt_1 \bar t_2 + d \bar t_2^2}{t_1 \bar t_1 - t_2 \bar t_2}
    & \dfrac{at_1t_2 - bt_2 \bar t_2 + \bar d \bar t_1 \bar t_2}{t_1 \bar t_1 - t_2 \bar t_2}
    & \dfrac{-\bar t_2}{t_1 \bar t_1 - t_2 \bar t_2} \\[2ex]
    \dfrac{-at_1t_2+bt_1\bar t_1-\bar d \bar t_1 \bar t_2}{t_1 \bar t_1 - t_2 \bar t_2}
    & \dfrac{-at_2^2+b\bar t_1 t_2-d\bar t_1^2}{t_1 \bar t_1 - t_2 \bar t_2}
    & \dfrac{\bar t_1}{t_1 \bar t_1 - t_2 \bar t_2} \\[2ex]
    ct_1 + (1-d\bar d)\bar t_2 & ct_2 + (1-d \bar d) \bar t_1 & d
  \end{bmatrix}
\end{align*}
At this point, note that the entry $d$ appears by itself at the bottom-right.
Consequently, the orbital integral is zero if $v(d) < -r$.
This justifies \Cref{assume:vd_ge_minus_r} from before.

Let us define
\[ t = t_2 \bar t_1 \inv \iff t_2 = t \bar t_1. \]
This lets us rewrite everything in terms of the ratio $t$ and $t_1 \in E$:
\[
  \bar{h'} \inv \gamma h'
  =
  \begin{bmatrix}
    \dfrac{t_1^2(a-b\bar t+\bar d \bar t^2)}{t_1 \bar t_1(1-t \bar t)}
    & \dfrac{t_1 \bar t_1(at-bt\bar t+\bar d \bar t)}{t_1 \bar t_1(1-t \bar t)}
    & \dfrac{t_1 \cdot (-\bar t)}{t_1 \bar t_1 (1-t \bar t)} \\[2ex]
    \dfrac{t_1\bar t_1(-at+b-\bar d \bar t)}{t_1 \bar t_1(1-t \bar t)}
    & \dfrac{\bar t_1^2(-at^2+bt-\bar d)}{t_1 \bar t_1(1-t \bar t)}
    & \dfrac{-\bar t_1}{t_1 \bar t_1(1-t \bar t)} \\[2ex]
    t_1(c + (1-d\bar d)\bar t) & \bar t_1(ct + (1-d \bar d)) & d
  \end{bmatrix}
\]
This new parametrization is better because $t_1$ only plays the role of
a scale factor on the outside, with ``interesting'' terms only involving $t$.
To make this further explicit, we write
\[ t_1 = \varpi^{-m} \epsilon \]
for $m \in \ZZ$ and $\epsilon \in \OO_E^\times$.
Then
\begin{align*}
  &\phantom=
  \begin{bmatrix} \bar\epsilon \\ & \epsilon \\ & & 1 \end{bmatrix}
  \bar{h'} \inv \gamma h'
  \begin{bmatrix} \epsilon\inv \\ & \bar\epsilon\inv \\ & & 1 \end{bmatrix} \\
  &=
  \begin{bmatrix}
  \dfrac{a-b\bar t+\bar d \bar t^2}{1-t \bar t}
  & \dfrac{at-bt\bar t+\bar d \bar t}{1-t \bar t}
  & \dfrac{-\varpi^m \bar t}{1-t \bar t} \\[2ex]
  \dfrac{-at+b-\bar d \bar t}{1-t \bar t}
  & \dfrac{-at^2+bt-\bar d}{1-t \bar t}
  & \dfrac{-\varpi^m}{1-t \bar t} \\[2ex]
  \dfrac{c + (1-d\bar d)\bar t}{\varpi^m} & \dfrac{ct + (1-d \bar d)}{\varpi^m} & d
  \end{bmatrix}.
\end{align*}
For brevity, we will let $\Gamma(\gamma, t, m)$ denote the right-hand matrix.
The conjugation by
$\left[ \begin{smallmatrix} \epsilon\inv \\ & \bar\epsilon\inv \\ & & 1 \end{smallmatrix} \right]$
has no effect on any of the $K'_{S, \le r}$, so that we can simply use
\[ \mathbf{1}_{K'_{S, \le r}}(\bar{h'}\inv \gamma h') = \mathbf{1}_{K'_{S, \le r}}(\Gamma(\gamma, t, m)) \]
in the work that follows.
For brevity, we abbreviate
\[ \mathbf{1}_{\le r}(\gamma, t, m) \coloneqq \mathbf{1}_{K'_{S, \le r}}(\Gamma(\gamma, t, m)). \]

\subsection{Reparametrizing the integral in terms of $t$ and $m$}
From now on, following \cite[\S4]{ref:AFL} we always fix the notation
\begin{align*}
  m &= m(t_1) \coloneqq -v(t_1) \\
  n &= n(t) \coloneqq v(1-t\bar t).
\end{align*}
(At the risk of stating the obvious, this $n$ is not the same $n$ in $\GL_n$, etc.)
We need to rewrite the integral, phrased originally via $\odif{h'}$,
in terms of the parameters $t$ (hence $n$), $m$, and $\gamma$.
We start by observing that
\[ \det h' = t_1 \bar t_1 - t_2 \bar t_2 = t_1 \bar t_1 (1 - t\bar t) \]
which means that
\[ v(\det h') = -2m + n \]
ergo
\begin{align*}
\left\lvert \det h' \right\rvert_F &= q^{-v(\det h')} = q^{2m-n} \\
\eta(h') &= (-1)^{v(\det h')} = (-1)^n.
\end{align*}
Meanwhile, from $t_2 = t \bar t_1$ we derive
\[ \odif t_2 = \left\lvert t_1 \right\rvert_E \odif t = q^{2m} \odif t. \]

Bringing this all into the orbital integral gives
\begin{align*}
  \Orb(\gamma, \mathbf{1}_{K'_{S, \le r}} s)
  &= \kappa \int_{t, t_1 \in E} \mathbf{1}_{\le r}(\gamma, t, m)
  (-1)^n \left( q^{2m-n} \right)^{s-2} \odif t_1 \cdot (q^{2m} \odif t) \\
  &= \kappa \int_{t, t_1 \in E} \mathbf{1}_{\le r}(\gamma, t, m)
  (-1)^n q^{s(2m-n)} \cdot q^{2n-2m} \odif t \odif t_1.
\end{align*}

\section{Description of the support of $\mathbf{1}_{\le r}$ when $n \le 0$}
\begin{proposition}
  Whenever $n = 0$ (this requires $v(t) \geq 0$),
  \[
    \mathbf{1}_{\le r}(\gamma, t, m) =
    \begin{cases}
      1 & \text{if } -r \le m \le \delta+r \\
      0 & \text{otherwise.}
    \end{cases}
  \]
\end{proposition}
\begin{proof}
  We have to consider the nine entries of $\Gamma(\gamma, t, m)$ in tandem.

  The upper $2 \times 2$ matrix is always in $\omega^{-r}\OO_E$,
  because $v(t) \geq 0$, $v(d) \geq -r$, $v(b) \geq -r$, and $v(a) = 0$ suffices.

  In the right column, since $v(t) \geq 0$ and $n = 0$, the condition is simply $m \ge -r$.

  In the bottom row, we need
  \begin{align*}
    v\left( c+(1-d\bar d) \bar t \right)-m &\geq -r, \\
    v\left( ct +(1-d\bar d) \right)-m &\geq -r.
  \end{align*}
  If $v(t) > 0$ this is equivalent to $m-r \leq \delta$.
  In the case where $v(t) = 0$ we instead use the observation that
  \begin{equation}
    \left[ c + (1-d \bar d) \bar t \right]
    - \bar t \left[ ct + (1-d \bar d) \right] = (1-t\bar t) c
    \label{eq:ctrick}
  \end{equation}
  which forces at least one of $ct + (1-d \bar d)$ and $c + (1-d \bar d) \bar t$ to
  have valuation $\delta$. So the claim follows now.
\end{proof}

\begin{proposition}
  Suppose $n = -2k < 0$, equivalently, $v(t) = -k < 0$, for some $k$.
  \[
    \mathbf{1}_{\le r}(\gamma, t, m) =
    \begin{cases}
      1 & \text{if } -r \le m+k \le \delta+r \\
      0 & \text{otherwise.}
    \end{cases}
  \]
\end{proposition}
\begin{proof}
  The proof is similar to the previous claim, but simpler.

  Since $k > 0$, the fraction $\frac{t^2}{1-t \bar t}$ has positive valuation,
  so the upper $2 \times 2$ submatrix of $\Gamma(\gamma, t, m)$ is always in $\varpi^{-r}\OO_E$.
  Turning to the right column, the condition reads exactly $m+k \geq -r$.
  Finally, in the bottom row, from $v(t) > 0$ and $v(c) = \delta$
  the condition is simply $-k+\delta-m \geq -r$.
\end{proof}

Note all the results in this section hold for any $b$ and $d$
with $v(b) \ge -r$ and $v(d) \ge -r$, i.e.~we do not yet need to consider cases
based on whether $b$ and $d$ are units or not.

\section{Description of the support of $\mathbf{1}_{\le r}$ when $n > 0$}
In this situation we evaluate over $n > 0$ only.
In this case $t$ is automatically a unit.

\subsection{Volume lemma}
The following two lemmas will be useful.

\begin{lemma}
  \label{lem:volume}
  Let $\xi \in \OO_E^\times$, $\rho \in \ZZ$, and $n \ge \max(\rho, 1)$ an integer.
  Then
  \begin{align*}
    &\Vol\left( \left\{ x \in E \mid v(1-x \bar x) = n,
      \; v(x-\xi) \ge \rho \right\} \right) \\
    &=
    \begin{cases}
      0 & \text{if } v(1-\xi\bar\xi) < \rho \\
      q^{-n}(1-q^{-2}) & \text{if } \rho \le 0 \\
      q^{-(n+\rho)}(1-q\inv) & \text{if } v(1-\xi\bar\xi) \ge \rho \ge 1.
    \end{cases}
  \end{align*}
\end{lemma}
\begin{proof}
  The case $\rho > 0$ is proved in \cite[Lemma 4.4]{ref:AFL}.

  When $\rho \le 0$, the condition $v(x - \xi) \ge \rho$ is vacuously true,
  so we just are computing
  $\Vol\left( \left\{ x \in E \mid v(1-x \bar x) = n \right\} \right)$.
  Follows from summing the previous lemma over $\rho = 1$
  and $\xi \in \OO_E / \varpi$ (there are $q_E-1 = q^2-1$ choices for $\xi$).
  \todo{This doesn't actually check out. Check margin of page 235 of notebook.}
\end{proof}

We also comment on the well-known fact that in an ultrametric space,
any two disks are either disjoint or one is contained in the other.
(In other words, the Mastercard logo cannot be drawn. See \Cref{fig:no_mastercard}.)
\begin{proposition}
  Choose $\xi_1, \xi_2 \in E$ and $\rho_1 \geq \rho_2$.
  Consider the two disks:
  \begin{align*}
    D_1 &= \left\{ x \in E \mid v(x-\xi_1) \ge \rho_1 \right\} \\
    D_2 &= \left\{ x \in E \mid v(x-\xi_2) \ge \rho_2 \right\}.
  \end{align*}
  Then, if $v(\xi_1-\xi_2) \geq \rho_2$, we have $D_1 \subseteq D_2$.
  If not, instead $D_1 \cap D_2 = \varnothing$.
  \label{prop:no_mastercard}
\end{proposition}
\begin{proof}
  Because $E$ is an ultrametric space and $\Vol(D_1) \leq \Vol(D_2)$,
  we either have $D_1 \subseteq D_2$ or $D_1 \cap D_2 = \varnothing$.
  The latter condition checks which case we are in by testing if $\xi_1 \in D_2$,
  since $\xi_1 \in D_1$.
\end{proof}
\todo{is it $q$ or $q_E$}

\begin{figure}
\centering
\begin{asy}
  defaultpen(fontsize(12pt));
  size(7cm);
  pair O1 = 0.57*dir(-30);
  pair O2 = (0,0);
  real r1 = 0.4;
  real r2 = 1;
  filldraw(circle(O2, r2), rgb(0.9, 0.9, 0.9));
  filldraw(circle(O1, r1), rgb(0.8, 0.8, 0.9));
  pair P1 = O1+r1*dir(230);
  pair P2 = O2+r2*dir( 70);
  draw(O1--P1);
  draw(O2--P2);
  dot("$\xi_1$", O1, dir(90));
  dot("$\xi_2$", O2, dir(180));
  label("$q^{-\rho_1}$", midpoint(O1--P1), dir(P1-O1)*dir(90));
  label("$q^{-\rho_2}$", midpoint(O2--P2), dir(P2-O2)*dir(90));
\end{asy}
\caption{Figure corresponding to \Cref{prop:no_mastercard}.}
\label{fig:no_mastercard}
\end{figure}

We package both of these results together
lemma that will be used repeatedly.
\begin{lemma}
  \label{lem:quadruple_ineq}
  Let $\xi_1, \xi_2 \in \OO_E^\times$ and let $\rho_1 \ge \rho_2$ be integers.
  Also let $n \ge \max(\rho_1, 1)$ be an integer.
  Then the set of points $x \in E$ satisfying all of the equations
  \begin{align*}
    v(x - \xi_1) &\ge \rho_1 \\
    v(x - \xi_2) &\ge \rho_2 \\
    v(1 - x \ol x) &= n
  \end{align*}
  has positive volume if and only if
  \[ v(1 - \xi_1 \bar{\xi_1}) \ge \rho_1, \qquad \rho_2 \le v(\xi_1 - \xi_2). \]
  In that case, the volume is equal to
  \[
    \begin{cases}
      q^{-(n+\rho_1)}(1-q\inv) & \text{if } \rho_1 \ge 1 \\
      q^{-n}(1-q^{-2}) & \text{if } \rho_1 \le 0.
    \end{cases}
  \]
\end{lemma}
In the situation where $\xi_i \notin \OO_E^\times$,
the quantity $v(x-\xi_i) = \min(0, v(\xi_i))$
becomes independent of the value of $x$,
and so \Cref{lem:quadruple_ineq} becomes unnecessary
(\Cref{lem:volume} will suffice).
We will deal with this situation when it arises;
it turns out this will only occur when $v(b) \neq 0$.

\subsection{Setup}
Consider the upper $2 \times 2$ submatrix of $\Gamma(\gamma, t, m)$.
Using the identities
\begin{align*}
  \dfrac{a-b\bar t+\bar d \bar t^2}{1-t \bar t}
    - \bar t \cdot \dfrac{at-bt\bar t+\bar d \bar t}{1-t \bar t}
    &= a-b\bar t \in \varpi^{-r} \OO_E \\[2ex]
  \dfrac{a-b\bar t+\bar d \bar t^2}{1-t \bar t}
    + \bar t \cdot \dfrac{-at+b-\bar d \bar t}{1-t \bar t}
    &= a \in \varpi^{-r} \OO_E \\[2ex]
  \dfrac{-at+b-\bar d \bar t}{1-t \bar t}
    - \bar t \cdot \dfrac{-at^2+bt-\bar d}{1-t \bar t}
    &= -a+b \in \varpi^{-r} \OO_E,
\end{align*}
it follows that as soon as one entry in the upper $2 \times 2$
submatrix is in $\varpi^{-r} \OO_E$, then all four are.
Focusing on the fraction with numerator $-at^2+bt-\bar d$
we use the identity
\[ (2at-b)^2 - (b^2-4a\bar d) = -4a(-at^2+bt-\bar d) \]
to rewrite $v(-at^2+bt-\bar d) \geq n-r$ as
\[ v\left( (2at-b)^2 - (b^2-4a\bar d) \right) \geq n-r. \]
This takes care of all four of the entries in the upper $2 \times 2$ submatrix.

Meanwhile, the requirements on the other entries amount to
\begin{align}
  m & \geq n - r \\
  v\left( c+(1-d \bar d) \bar t \right) &\geq m-r \label{eq:cddtop} \\
  v\left( ct+(1-d \bar d) \right) &\geq m-r \label{eq:cddbot}
\end{align}
and of course $v(d) \ge -r$ which we assume.
According to the earlier identity \eqref{eq:ctrick}, if \eqref{eq:cddtop} is assumed true,
then \eqref{eq:cddbot} is equivalent to
\[ \delta + v(1-t \bar t) \ge m-r. \]
Meanwhile, since $v(c+(1-d \bar d) \bar t) = v(\bar c + (1-d \bar d)t)$,
\eqref{eq:cddtop} is itself equivalent to
\[ v(t+u) + \delta \geq m-r \]
by reading the definition of \eqref{eq:u}.

In summary:
\begin{proposition}
  Assume $t$ is such that $n = v(1-t \bar t) > 0$.
  Then $\mathbf{1}_{\le r}(\gamma, t, m) = 1$ if and only if
  \[ n - r \leq m \leq n + \delta + r \]
  and $t$ lies in the set specified by
  \begin{equation}
    \begin{aligned}
      v\left( (2at-b)^2 - (b^2-4a\bar d) \right) &\geq n-r \\
      v(t+u) &\ge m-\delta-r.
    \end{aligned}
    \label{eq:two_disks_S3}
  \end{equation}
\end{proposition}

\section{Rewriting the quadratic constraint on the valuation of $t$ in the $n > 0$ situation}
We now analyze the inequality
\begin{equation}
  v\left( (2at-b)^2 - (b^2-4a\bar d) \right) \geq n-r
  \label{eq:2atb_dos}
\end{equation}
and divide it into several (disjoint) possibilities.
Recalling that $\ell = b^2 - 4 a \bar d$, there are several possibilities:
\begin{itemize}
  \ii If $\ell \ge n-r$, then \eqref{eq:2atb_dos} is equivalent to
  \begin{equation}
    2v(2at-b) \ge n-r
    \iff v\left( t - \frac{b}{2a} \right) \ge \left\lceil \frac{n-r}{2} \right\rceil.
    \label{eq:t_disk_case_1_2}
  \end{equation}
  We will further subdivide this into two cases.
  \begin{itemize}
    \ii \textbf{Case 1} is the situation where $\left\lceil \frac{n-r}{2} \right\rceil \ge m - \delta - r$.
    \ii \textbf{Case 2} is the situation where $\left\lceil \frac{n-r}{2} \right\rceil < m - \delta - r$.
  \end{itemize}

  \ii If $\ell < n-r$, then \eqref{eq:2atb_dos} could only hold if $v(2at-b) = \frac{\ell}{2}$.
  Note that in particular, this requires $\ell$ to be even.

  If this happens, then as we saw in \Cref{prop:parameter_constraints}
  the quantity $b^2 - 4 a \bar d$ must be a square and we denote it $\tau^2$.
  Thus, \eqref{eq:2atb_dos} then reads
  \[ v(2at-b+\tau) + v(2at-b-\tau) \ge n-r. \]
  Since we are assuming $n > \ell + r$,
  it must be the case that one of the two factors
  $v(2at-b\mp\tau)$ is equal to $v(\tau) = \ell / 2$ exactly.

  Suppose $v(2at-b+\tau) = \frac{\ell}{2}$, so we need
  \[ v\left( t - \frac{b+\tau}{2a} \right) = v(2at-b-\tau) \ge n - \frac{\ell}{2} - r. \]
  We further subdivide this into two cases:
  \begin{itemize}
    \ii \textbf{Case 3\ts+} is the situation where $n - \frac{\ell}{2} - r > m - \delta - r$.
    \ii \textbf{Case 4\ts+} is the situation where $n - \frac{\ell}{2} - r \le m - \delta - r$.
  \end{itemize}
  Replacing $\tau$ with $-\tau$ above gives us two additional cases
  which we denote \textbf{Case 3\ts-} and \textbf{Case 4\ts-}.
\end{itemize}

This gives us six cases, with each $t \in E$ satisfying at most one of them.
(If $\ell$ is odd, only \textbf{Case 1} and \textbf{Case 2} are used.)
In each case, for a given pair $(n,m)$ we are interested in the volume of $t$
such that two disk inequalities hold together with the assumption $n = v(1-t \bar t)$.

We rewrite these six cases in the format specified by \Cref{lem:quadruple_ineq},
noting that each possibility will actually split into two sub-cases
(although the lemma will only apply in the cases where the centers
$\xi_i$ are actually in $\OO_E^\times$;
this which will be true in the main case $v(b) = v(d) = 0$).
This gives \Cref{tab:orbital_cases}.

\begin{table}[h]
  \centering
  \begin{tabular}{ll cc}
    \toprule
    & Assume & $\xi_1$ and $\xi_2$ & $\rho_1$ and $\rho_2$ \\\midrule
    \textbf{1}
      & $n \le \ell+r$
      & $\begin{aligned}
        \xi_1 &= \tfrac{b}{2a} \\
        \xi_2 &= -u \\
        v(\xi_1 - \xi_2) &= v(au - \bar d \bar u) \\
        v(1-\xi_1 \bar{\xi_1}) &= v(4-b\bar{b})
        \end{aligned}$
      & $\begin{aligned}
        &\phantom{\ge} \left\lceil \tfrac{n-r}{2} \right\rceil \\
        &\ge m-\delta-r
        \end{aligned}$
        \\\hline
    \textbf{2}
      & $n \le \ell+r$
      & $\begin{aligned}
        \xi_1 &= -u \\
        \xi_2 &= \tfrac{b}{2a} \\
        v(\xi_1-\xi_2) &= v(au- \bar d \bar u) \\
        v(1-\xi_1 \bar{\xi_1}) &= \lambda
      \end{aligned}$
      & $\begin{aligned}
        &\phantom> m-\delta-r \\
        &> \left\lceil \tfrac{n-r}{2} \right\rceil
      \end{aligned}$ \\\hline
    \textbf{3\ts+}
      & $n > \ell + r$
      & $\begin{aligned}
        \xi_1 &= \tfrac{b+\tau}{2a} \\
        \xi_2 &= -u \\
        v(\xi_1-\xi_2) &= v(au - \bar d \bar u + \tau) \\
        v(1 - \xi_1 \bar{\xi_1}) &= v(1-\tfrac{\Norm(b+\tau)}{4})
      \end{aligned}$
      & $\begin{aligned}
        &\phantom> n-\frac{\ell}{2}-r \\
        &> m-\delta-r
      \end{aligned}$ \\\hline
    \textbf{3\ts-}
      & $n > \ell + r$
      & $\begin{aligned}
        \xi_1 &= \tfrac{b-\tau}{2a} \\
        \xi_2 &= -u \\
        v(\xi_1-\xi_2) &= v(au-\bar d \bar u-\tau) \\
        v(1 - \xi_1 \bar{\xi_1}) &= v(1-\tfrac{\Norm(b-\tau)}{4})
      \end{aligned}$
      & $\begin{aligned}
        &\phantom> n-\frac{\ell}{2}-r \\
        &> m-\delta-r
      \end{aligned}$ \\\hline
    \textbf{4\ts+}
      & $n > \ell + r$
      & $\begin{aligned}
        \xi_1 &= -u \\
        \xi_2 &= \tfrac{b+\tau}{2a} \\
        v(\xi_1-\xi_2) &= v(au-\bar d \bar u+\tau) \\
        v(1 - \xi_1 \bar{\xi_1}) &= \lambda
        \end{aligned}$
      & $\begin{aligned}
        &\phantom\ge m-\delta-r \\
        &\ge n-\frac{\ell}{2}-r
        \end{aligned}$ \\\hline
    \textbf{4\ts-}
      & $n > \ell + r$
      & $\begin{aligned}
        \xi_1 &= -u \\
        \xi_2 &= \tfrac{b-\tau}{2a} \\
        v(\xi_1-\xi_2) &= v(au-\bar d \bar u-\tau) \\
        v(1 - \xi_1 \bar{\xi_1}) &= \lambda
        \end{aligned}$
      & $\begin{aligned}
        &\phantom\ge m-\delta-r \\
        &\ge n-\frac{\ell}{2}-r
      \end{aligned}$ \\\bottomrule
  \end{tabular}

  \caption{The six cases for calculating the orbital integral for $S_3(F)$,
    in the inhomogeneous group version of the AFL.}
  \label{tab:orbital_cases}
\end{table}

Note that in generating \Cref{tab:orbital_cases}, we did the calculations
\begin{align*}
  v\left( u + \frac{b}{2a} \right)
  &= v\left( \frac{au-\bar d \bar u}{2a} \right)
  = v(au - \bar d \bar u) \\
  v\left( u + \frac{b \pm \tau}{2} \right)
  &= v(au - \bar d \bar u \pm \tau).
\end{align*}
to populate the entries for $v(\xi_1 - \xi_2)$,
as well as the identity
\[ 1 - \frac{b \pm \tau}{2a} \cdot \frac{\bar b \pm \bar \tau}{2\bar a}
= \frac{4 - \Norm(b \pm \tau)}{4} \]
to calculate $v(1-\xi_1\bar{\xi_1})$ entries in the latter four cases.

\section{Case analysis when $v(b) \ge 0$ and $v(d) \ge 0$}
Up until now the analysis has been valid for all five cases of
\Cref{prop:parameter_constraints}.
However, starting from now on we will have to be a little more careful
and think about which situation of \Cref{prop:parameter_constraints} we are in.

\subsection{Analysis of Case 1 and 2 assuming $n > 0$ and $v(b) = 0$}
We analyze Case 1 and 2 assuming $v(b) = 0$.

Considering $n > 0$ and $n-r \le m \le n+\delta+r$ as fixed,
we compute the volume of the set of $t$
for which $n = v(1-t\bar t)$ and $\mathbf{1}_{\le r}(\gamma,t,m) = 1$.

In addition to the constraint $n \le \ell + r$,
we see that the two cases have the following additional requirements:
\begin{description}
  \ii[Case 1] If $m < \left\lceil \frac{n-r}{2} \right\rceil + \delta + r$ then we need
  \begin{align}
    v(4-b\bar b) &\ge \left\lceil \frac{n-r}{2} \right\rceil \label{eq:odd_ineq1} \\
    v(au - \bar d \bar u) &\ge m - \delta - r \label{eq:odd_ineq2}.
  \end{align}

  \ii[Case 2] If $m \geq \left\lceil \frac{n-r}{2} \right\rceil + \delta + r$ then we need
  \begin{align}
    \lambda = v(1-u \bar u) &\ge m-\delta-r \label{eq:odd_ineq3} \\
    v(au - \bar d \bar u) &\ge \left\lceil \frac{n-r}{2} \right\rceil \label{eq:odd_ineq4}.
  \end{align}
\end{description}

We will now show that some of these inequalities are redundant and can be ignored.
\begin{proposition}
  If $v(b) = 0$, then \eqref{eq:odd_ineq2} and \eqref{eq:odd_ineq4} are redundant
  i.e.\ they are automatically true for $0 < n \le \ell + r$.
\end{proposition}
\begin{proof}
  First, assume that $\ell$ is odd.
  Then \eqref{eq:dos} together with $v(b) = v(d) = 0$ gives
  \begin{equation}
    \lambda = \ell = v(1-u \bar u) < 2v(au - \bar d \bar u).
    \label{eq:odd_center_distance}
  \end{equation}
  On the other hand, suppose $\ell$ is even.
  In the case where $v(d) = 0$, \eqref{eq:dos} implies that
  \begin{equation}
    \ell = 2v(au - \bar d \bar u) < v(1-u\bar u) = \lambda
    \label{eq:even_center_distance_case12}
  \end{equation}
  Meanwhile, if $v(d) > 0$, then $\ell = 0$ and $\lambda \ge 1$,
  so \eqref{eq:dos} implies $v(au - \bar d \bar u) = 0$.
  So in every situation we always have
  \[ v(au - \bar d \bar u) \ge \frac{\ell}{2} \ge \frac{n-r}{2}. \qedhere \]
\end{proof}

\begin{proposition}
  If $v(b) = 0$, then \eqref{eq:odd_ineq1} is redundant.
\end{proposition}
\begin{proof}
  Regardless of whether $v(d) = 0$ or $v(d) > 0$, the equation
  \[ (4-b \bar b) = -4au(1-d\bar d) - \bar b(b^2-4a\bar d) \]
  always implies
  \begin{equation}
    v(4-b\bar b) \ge \min(\ell,\delta) \text{ with equality if } \ell \neq \delta
    \label{eq:bb_odd}
  \end{equation}
  sinec $-4au$ and $\bar b$ are units.
  Hence, a priori \eqref{eq:bb_odd} suggests that we have a condition
  $n \le r + 2 \delta$ in addition to $n \le r + \ell$.
  However, by \Cref{prop:parameter_constraints}, we always have $\ell \le 2 \delta$,
  and consequently \eqref{eq:odd_ineq1} is redundant as well.
\end{proof}

Putting all of this together, we find that the valid pairs $(n,m)$ come in two cases.

\begin{description}
\item[Double sum for Case 1]
We sum over $(m,n)$ such that
\begin{equation}
  \begin{aligned}
    1 &\leq n \leq \ell + r, \\
    n-r &\leq m \leq \left\lceil \frac{n-r}{2} \right\rceil+\delta+r - 1
  \end{aligned}
  \label{eq:odd_range1}
\end{equation}
where each $(m,n)$ gives a volume contribution of
\begin{equation}
  \begin{cases}
    q^{-n - \left\lceil \frac{n-r}{2} \right\rceil} \left( 1 - q\inv \right)
      & \text{if $n > r$} \\
    q^{-n} \left( 1 - q^{-2} \right)
      & \text{if $n \leq r$}.
  \end{cases}
  \label{eq:case_1_vol_contrib}
\end{equation}

\item[Double sum for Case 2]
We sum over $(m,n)$ such that
\begin{equation}
  \begin{aligned}
    1 &\leq n \leq \ell + r, \\
    \max\left(n-r, \left\lceil \frac{n-r}{2} \right\rceil+\delta+r \right)
    &\leq m \leq \min(n,\lambda)+\delta+r
  \end{aligned}
  \label{eq:odd_range2}
\end{equation}
where each $(m,n)$ gives a volume contribution of
\begin{equation}
  \begin{cases}
    q^{-n - (m-\delta-r)} \left( 1 - q\inv \right)
      & \text{if $m > \delta + r$} \\
    q^{-n} \left( 1 - q^{-2} \right)
      & \text{if $m \le \delta + r$}.
  \end{cases}
  \label{eq:case_2_vol_contrib}
\end{equation}
Notice that $m \leq \delta + r$ could only occur when $n \leq r$.
\end{description}

\subsection{Analysis of Case 1 and 2 assuming $n > 0$ and $v(b) > 0$}
If $v(b) > 0$ instead, so that $\ell = 0$ but still $\delta \ge 0$.
Then \eqref{eq:t_disk_case_1_2} becomes independent of the unit $t$ because we must have
\[ v\left( t - \frac{b}{2a} \right) = 0. \]
Consequently, in this situation, it is not necessary to distinguish between Cases 1 and 2.
Instead, \eqref{eq:two_disks_S3} merely requires that
\begin{align*}
  0 &\ge n - r \\
  v(t+u) &\ge m - \delta - r
\end{align*}
In this situation, by \Cref{lem:volume},
we get a nonzero contribution in the merged case if and only if
\begin{equation}
  \begin{aligned}
    1 &\le n \le r \\
    n-r &\le m \le n + \delta + r \\
    \lambda &\ge m - \delta - r.
  \end{aligned}
  \label{eq:vb_pos_range}
\end{equation}
In that case the volume contribution is given by the same expression as \eqref{eq:case_2_vol_contrib}.

To avoid having to consider the situation $v(b) > 0$ separately,
we make the following observation:
\begin{proposition}
  Consider the two (disjoint) ranges \eqref{eq:odd_range1} and \eqref{eq:odd_range2}
  in the special case $\ell = 0$.
  These two ranges collectively cover exactly the same elements $(m,n)$
  as \eqref{eq:vb_pos_range}.
  Moreover, the volume contribution \eqref{eq:case_1_vol_contrib}
  equals that of \eqref{eq:case_2_vol_contrib} for pairs $(m,n)$ in \eqref{eq:odd_range1}.
\end{proposition}
\begin{proof}
  If we combine \eqref{eq:odd_range1} and \eqref{eq:odd_range2}, they say that
  \begin{align*}
    1 &\le n \le \ell + r = r \\
    n-r &\le m \le \min(n,\lambda) + \delta + r = \min(n, \lambda) + r
  \end{align*}
  which matches \eqref{eq:vb_pos_range} exactly.
  So it remains to verify that the volume contributions match.

  Now suppose that $m < \left\lceil \frac{n-r}{2} \right\rceil + \delta + r$.
  Since $n \le r$, it follows that $m \le \delta + r$.
  So on the one hand we have the $n \le r$ case in \eqref{eq:case_1_vol_contrib}
  and on the other hand we have the $m \le \delta + r$ case in \eqref{eq:case_2_vol_contrib},
  which both equal $q^{-n}(1-q^{-2})$.
  Hence the proof is complete.
\end{proof}
Because of this proposition then we can fold the $v(b) > 0$
result into the case $v(b) = 0$ too in future calculations.
In other words the double sums in Case 1 and 2 mentioned in the previous subsection
will work verbatim even in the $v(b) > 0$ situation, which is convenient.

\begin{remark}
  Note that in the original paper \cite{ref:AFL} only $r = 0$ is considered
  and in this case all three ranges \eqref{eq:odd_range1}, \eqref{eq:odd_range2}
  and \eqref{eq:vb_pos_range} are empty.
  Therefore the corresponding step in the calculation of \cite{ref:AFL}
  is much simpler, consisting only of the one-line observation that
  Case 1 and Case 2 cannot occur at all if $r = 0$.
  In contrast once $r > 0$ then the ranges are not necessarily empty
  and therefore one needs to ensure that the terms arising actually match.
\end{remark}

\subsection{Analysis of Case 3\ts+, 3\ts-, 4\ts+, 4\ts- assuming $n > 0$ and $v(b) \ge 0$ and $v(d) \ge 0$}
Suppose $\ell \geq 0$ is even.
As before we consider $n > 0$ and $n-r \le m \le n+\delta+r$ as fixed,
and seek to compute the volume of the set of $t$
for which $n = v(1-t\bar t)$ and $\mathbf{1}_{\le r}(\gamma,t,m) = 0$.

Moving forward, we need to fix the choice of the square root $\tau$.
\begin{definition}
  [Fixing the choice of $\tau$]
  Using \eqref{eq:dos} in the form
  \[ (au-\bar d \bar u)^2 - \tau^2 = 4a\bar d(1- u\bar u) \]
  we agree now to fix the choice of the square root of $\tau$ such that
  \begin{equation}
    \begin{aligned}
      v(au-\bar d \bar u + \tau) &= \lambda + v(d) - \half \ell > 0, \\
      v(au-\bar d \bar u - \tau) &= \half \ell.
    \end{aligned}
    \label{eq:tau_choice}
  \end{equation}
\end{definition}

\begin{proposition}
  \label{prop:four_norm_choice}
  With this choice of $\tau$, we have
  \begin{equation}
    \begin{aligned}
      v\left( 4 - \Norm(b+\tau) \right) &= \lambda + \delta - \ell \\
      v\left( 4 - \Norm(b-\tau) \right) &= \delta.
    \end{aligned}
    \label{eq:four_norm_choice}
  \end{equation}
\end{proposition}
\begin{proof}
  We consider which of $b$ and $d$ are units.
  \begin{itemize}
  \ii If $v(b) = v(d) = 0$ then from \eqref{eq:dos} we have
  \[ \ell = 2v(\tau) = 2v(au - \bar d \bar u) < \lambda \]
  and thus \cite[Lemma 4.7]{ref:AFL} applies to give \eqref{eq:four_norm_choice}, verabtim.

  \ii Now suppose $v(d) > 0$ but still $v(b) = v(\tau) = 0$.
  We begin with the observation that
  \[ 4 a \bar d = b^2 - \tau^2 = (b + \tau)(b - \tau) \]
  and so $\{v(b+\tau), v(b-\tau)\} = \{0, v(d)\}$.
  We need to determine which is which.
  However, note that we may write
  \[ au - \bar d \bar u - \tau = -(b + \tau) - 2 \bar d \bar u. \]
  Since $v(au - \bar d \bar u + \tau) = 0$ and $v(d) > 0$,
  it follows we must have $v(b+\tau) = 0$.
  And thus $v(b-\tau) = v(d)$.
  Hence $v(4 - \Norm(b-\tau)) = 0 = \delta$,
  and we have obtained the bottom equation of \eqref{eq:four_norm_choice}.

  It remains to show that $v(4-\Norm(b+\tau)) = \lambda$ to complete the proof.
  We quote \cite[Lemma 4.6]{ref:AFL} which states more generally that
  \begin{align*}
    2\delta + \lambda
    &= v(4-\Norm(b+\tau)) + v(4-\Norm(b-\tau)) \\
    &\qquad+ v(16 + 16 d \bar d - 8 b \bar b + 8 \tau \bar\tau).
  \end{align*}
  In our case $\delta = 0$, $v(4-\Norm(b-\tau)) = 0$, and
  since $v(\tau - b) > 0$ we get $v(\tau\bar\tau - b \bar b) > 0$\todo{I think},
  so that the final term on the right-hand side is $0$ too.

  \ii Finally assume $v(b) > 0$.
  As mentioned on \cite[p.\ 242]{ref:AFL}, the identity \eqref{eq:four_norm_choice}
  is still true in this situation too.
  This completes the final case.
  \qedhere
  \end{itemize}
\end{proof}

\begin{description}
\ii[Double sum for Case 3\ts+]
Suppose $n > \ell + r$,
$m < n - \frac{\ell}{2} + \delta$, and we choose $\frac{b+\tau}{2a}$.
Then \Cref{lem:quadruple_ineq} gives a nonzero contribution if and only if
\begin{align*}
  \lambda + \delta - \ell = v(4-\Norm(b+\tau)) &\geq n - \frac{\ell}{2} - r \\
  \lambda + v(d) - \frac{\ell}{2} = v(au-\bar d \bar u + \tau) & \geq m - \delta - r.
\end{align*}
Compiling all seven constraints gives that the valid pairs $(m,n)$ are those for which
\begin{align*}
  \max(1, \ell+r+1) &\leq n \leq -\frac{\ell}{2} + \delta + \lambda + r, \\
  n-r &\leq m \leq \min\left( n+\delta+r, n - \frac{\ell}{2}+\delta - 1,
    \lambda + v(d) -\frac{\ell}{2}+\delta+r \right)
\end{align*}
which from $\ell, \delta, r \ge 0$ can be simplified to just
\begin{align*}
  \ell+r+1 &\leq n \leq  -\frac{\ell}{2} + \delta + \lambda + r, \\
  n-r &\leq m \leq \min(n-1, \lambda + v(d) + r) - \frac{\ell}{2} + \delta
\end{align*}
Moreover, in any situation where $v(d) > 0$, we have $\ell = \delta = 0$,
and hence $n-1 < \lambda + r < \lambda + v(d) + r$ is obvious.
Thus we may drop the $v(d)$ term from the inequality and obtain
\begin{equation}
  \begin{aligned}
    \ell+r+1 &\leq n \leq  -\frac{\ell}{2} + \delta + \lambda + r, \\
    n-r &\leq m \leq \min(n-1, \lambda + r) - \frac{\ell}{2} + \delta
  \end{aligned}
  \label{eq:even_case3_plus}
\end{equation}
Each $(m,n)$ gives a volume contribution of
\[ q^{-n - (n - \frac{\ell}{2} - r)} \left( 1 - q\inv \right). \]

\ii[Double sum for Case 3\ts-]
Suppose $n > \ell + r$,
$m < n - \frac{\ell}{2} + \delta$, and we choose $\frac{b-\tau}{2a}$.
Then \Cref{lem:quadruple_ineq} gives a nonzero contribution if and only if
\begin{align*}
  \delta = v(4-\Norm(b-\tau)) &\geq n - \frac{\ell}{2} - r \\
  \frac{\ell}{2} = v(au-\bar d \bar u - \tau) & \geq m - \delta - r.
\end{align*}
Compiling all seven constraints gives that the valid pairs $(m,n)$ are those for which
\begin{align*}
  \ell + r + 1 &\leq n \leq \frac{\ell}{2}+\delta+r, \\
  n-r &\leq m \leq \min\left( n - \frac{\ell}{2}+\delta - 1,
    \frac{\ell}{2} + \delta + r, n + \delta + r \right)
\end{align*}
This simplifies to
\begin{equation}
  \begin{aligned}
    \ell+r+1 &\leq n \leq \frac{\ell}{2}+\delta+r, \\
    n-r &\leq m \leq \frac{\ell}{2} + \delta + r.
  \end{aligned}
  \label{eq:even_case3_minus}
\end{equation}
As in the previous case, $(m,n)$ gives a volume contribution of
\[ q^{-n - (n - \frac{\ell}{2} - r)} \left( 1 - q\inv \right). \]

\ii[Double sum for Case 4\ts+]
Suppose $n > \ell + r$,
$m \ge n - \frac{\ell}{2} + \delta$
(which implies $m \ge n-r$ since $r \ge 0$ and $0 \le \ell \le 2 \delta$),
and we choose $\frac{b+\tau}{2a}$.
Then \Cref{lem:quadruple_ineq} gives a nonzero contribution if and only if
\begin{align*}
  \lambda &\geq m - \delta - r \\
  \lambda + v(d) - \frac{\ell}{2} = v(au-\bar d \bar u + \tau) & \geq n - \frac{\ell}{2} - r.
\end{align*}
Rearranging gives that the valid pairs $(m,n)$ are those for which
\begin{align*}
  \ell + r + 1 &\leq n \leq \lambda + v(d) + r \\
  n - \frac{\ell}{2} + \delta &\leq m \leq \min(n, \lambda) + \delta + r.
\end{align*}
However, in any situation where $v(d) > 0$ and hence $\ell = \delta = 0$,
the inequality on $m$ already implies that $n \le \lambda + r$.
Hence we may drop the $v(d)$ term to get instead the inequalities
\begin{equation}
  \begin{aligned}
    \ell + r + 1 &\leq n \leq \lambda + v(d) + r \\
    n - \frac{\ell}{2} + \delta &\leq m \leq \min(n, \lambda) + \delta + r.
  \end{aligned}
  \label{eq:even_case4_plus}
\end{equation}
Here, each $(m,n)$ gives a volume contribution of
\[ q^{-n - (m - \delta - r)} \left( 1 - q\inv \right). \]

\ii[Double sum for Case 4\ts-]
Suppose $n > \ell + r$,
$m \ge n - \frac{\ell}{2} + \delta$, and we choose $\frac{b-\tau}{2a}$.
Then \Cref{lem:quadruple_ineq} gives a nonzero contribution if and only if
\begin{align*}
  \lambda &\geq m - \delta - r \\
  \frac{\ell}{2} = v(au-\bar d \bar u - \tau) & \geq n - \frac{\ell}{2} - r.
\end{align*}
The latter inequality contradicts the assumption that $n > \ell + r$,
so in fact this case can never occur.
\end{description}

\section{Case analysis when $v(b) = v(d) < 0$}
