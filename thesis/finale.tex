\chapter{Proof of \Cref{thm:semi_lie_n_equals_2}}
We now put together all the results from the previous chapters to
prove \Cref{thm:semi_lie_n_equals_2}.

\section{Matching $\guv$ and $(g,u)$, and the invariants for the matching}
\subsection{The invariants for the orbit of $\guv$}
From \Cref{ch:orbitalFJ0}, recall that we considered
\[
  (\gamma, \uu, \vv^\top)
  =
  \left( \begin{bmatrix} a & b \\ c & d \end{bmatrix}
    \begin{bmatrix} 0 \\ 1 \end{bmatrix}, \begin{bmatrix} 0 & e \end{bmatrix} \right)
  \in (S_2(F) \times V'_2(F))\rs
\]
subject to the conditions
\begin{align*}
  \bar b c = b \bar c &= 1 - a \bar a = 1 - d \bar d \\
  \text{and } d &= - \bar a c / \bar c = -\bar a b / \bar b.
\end{align*}
It will again be enough to consider the situation in which $\det \gamma = 1$.
The invariants in this case as described in \Cref{def:matching_semi_lie} are:
\begin{itemize}
  \ii $\Tr \gamma = a + d$
  \ii $\det \gamma = ad - bc$
  \ii $\vv^\top \uu = e$
  \ii $\vv^\top \gamma \uu = \begin{bmatrix} 0 & e \end{bmatrix}
  \begin{bmatrix} a & b \\ c & d \end{bmatrix} \begin{bmatrix} 0 \\ 1 \end{bmatrix} = de$.
\end{itemize}
Note that the parameters $b$ and $c$ are absent; but we have
\[ v(b) + v(c) = v(\det \gamma - a d). \]

\subsection{Matching}
We now take these results and line them up with \Cref{sec:g_u_invariants}.

\begin{lemma}
  \label{lem:finale_match}
  Let $g = \alpha + \beta \Pi$ and $u = s + t \Pi$ as above
  and suppose $(g,u)$ matches with
  \[ (\gamma, \uu, \vv^\top) = \left( \begin{bmatrix} a & b \\ c & d \end{bmatrix}
    \begin{bmatrix} 0 \\ 1 \end{bmatrix}, \begin{bmatrix} 0 & e \end{bmatrix} \right)
    \in (S_2(F) \times V'_2(F))\rs. \]
  Then we have
  \begin{align*}
    \alpha_1 &= \frac{d-a}{2} \\
    \Nm u &= e \\
    \det g - \alpha \bar \alpha &= -bc.
  \end{align*}
\end{lemma}
\begin{proof}
  Setting the invariants above equal we obtain
  \begin{align*}
    a + d &= \alpha + \bar\alpha \\
    \det \gamma = ad - bc &= \det g \\
    e &= \Nm u \\
    de &= \alpha \Nm u.
  \end{align*}
  Hence it follows $d = \alpha$ and $a = \bar\alpha$.
  In particular
  \[ \alpha_1 = \frac{\alpha - \bar\alpha}{2\sqrt\eps} = \frac{d-a}{2\sqrt\eps}. \]
  The lemma follows directly from this.
\end{proof}

\section{Translation of the Gross-Keating data to the orbital side}
We combine the results of \Cref{sec:GK} with \Cref{lem:finale_match}.
Retaining the notation
\begin{align*}
  x &\coloneqq u^\ast \sqrt{\eps} u = (s \bar s + t \bar t \varpi) \sqrt{\eps} \in \ODT \\
  y &\coloneqq \varpi^r g_- = (\alpha_1\sqrt\eps + \beta \Pi) \in \ODT
\end{align*}
from \Cref{sec:GK}, we obtain the following:
\begin{align*}
  v(\left\langle x,x \right\rangle _0)
    &= 2 v(\Nm u) \\
    &= 2v(e) \\
  v\left( \left\langle x,y \right\rangle _0 \right)
    &= r + v(\alpha_1) + v(\Nm u) \\
    &= r + v(d-a) + v(e) \\
  v(
    \left\langle x,x \right\rangle _0 \left\langle y,y \right\rangle _0
    - \left\langle x,y \right\rangle^2_0
  )
    &= 2r + 2 v(\Nm u) + v(\det g - \alpha \bar \alpha) \\
    &= 2r + 2v(e) + v(b) + v(c).
\end{align*}
Notice that the last valuation is odd.
Therefore we can always extract $v(\left\langle y,y \right\rangle _0)$ by writing
\begin{align*}
  v\left(\left\langle x,x \right\rangle _0\right) + v\left(\left\langle y,y \right\rangle _0\right)
  &= \min \left( v\left( \left\langle x,x \right\rangle _0 \left\langle y,y \right\rangle _0 - \left\langle x,y \right\rangle^2_0 \right),
    v(\left\langle x,y \right\rangle^2_0) \right) \\
  \implies  v\left(\left\langle y,y \right\rangle _0\right)
  &= \min(2r + 2v(e) + v(b) + v(c), 2r + 2v(d-a) + 2v(e)) - 2v(e) \\
  &= 2r + \min(v(b) + v(c), 2v(d-a)).
\end{align*}

Recall the right-hand side of
\[
  \begin{cases}
    \sum_{j=0}^{\frac{n_1-1}{2}} (n_1+n_2-4j) p^j & \text{if } n_1 \equiv 1 \pmod 2 \\
    \frac{n_2-n_1+1}{2} p^{n_1/2} + \sum_{j=0}^{n_1/2-1} (n_1+n_2-4j) p^j & \text{if } n_1 \equiv 0 \pmod 2.
  \end{cases}
\]
for $0 \le n_1 \le n_2$.
Then apply \Cref{prop:GK} to obtain that
\[
  \left< \ZO4(1), \; \ZO4(x)^\circ, \; \ZO4(y)^\circ \right>_{\MM_2 \times \MM_2}
\]
is equal to the above formula applied at
\begin{align*}
  n_1 &\coloneqq \min(2v(e), r+v(d-a)+v(e), 2r+\min(v(b)+v(c)), 2v(d-a)) \\
  n_2 &\coloneqq 2r + 2v(e) + v(b) + v(c) - n_1.
\end{align*}
For brevity, we henceforth denote the above sum by
\[ \GK(r, v(b), v(c), v(e), v(d-a)). \]

\section{Base change}
The base change for $n=2$ was already calculated in \cite{ref:AFLspherical}
and we simply recall the result here.

As in \Cref{ch:jiao}, we define
\[ f_r \coloneqq \mathbf{1}_{\varpi^{-r} \Mat_2(\OO_E) \cap \U(\VV_n^+)} \in \HH(\U(\VV_n^+)) \]
\begin{lemma}
  [{\cite[Lemma 7.1.1]{ref:AFLspherical}}]
  For $n = 2$ and $r \ge 1$ we have
  \[
    \BC_S^{\eta}(
      \mathbf{1}_{K'_{S, \le r}}
      + \mathbf{1}_{K'_{S, \le (r-1)}}
    ) = f_r.
  \]
\end{lemma}
\begin{proof}
  This follows directly from \cite[equation (7.1.9)]{ref:AFLspherical}.
\end{proof}

\section{Comparison to orbital formula}
In what follows, we denote
\[ \partial\Orb(r, v(b), v(c), v(e), v(d-a)) \]
as the value of the orbital integral described in \Cref{thm:semi_lie_combo}; that is,
it represents the value of
\[ \left. \pdv{}{s} \right\rvert_{s=0}
  \Orb(\guv, (\mathbf{1}_{K'_{S, \le r}} + \mathbf{1}_{K'_{S, \le (r-1)}}) \otimes \oneV, s) \]
as a function of the parameters in $\guv$.
We tacitly assume $r \ge 1$ in what follows.

The main claim is that the following formula holds:
\begin{proposition}
  \label{prop:finale_by_one}
  We have
  \begin{align*}
    &\frac{\pm1}{\log q}\Big(\partial\Orb(r, v(b), v(c), v(e), v(d-a))
      + \partial\Orb(r, v(b), v(c), v(e)-1, v(d-a))\Big) \\
    &= \GK(r, v(b), v(c), v(e), v(d-a)) - \GK(r-1, v(b), v(c), v(e), v(d-a)).
  \end{align*}
\end{proposition}
\begin{proof}
  \todo{write this up}
\end{proof}

This proposition immediately implies the following.
\begin{corollary}
  \label{cor:finale_by_two}
  We have
  \begin{align*}
    &\frac{\pm1}{\log q}\Big(\partial\Orb(r, v(b), v(c), v(e), v(d-a)) - \partial\Orb(r, v(b), v(c), v(e)-2, v(d-a))\Big) \\
    &= \GK(r, v(b), v(c), v(e), v(d-a)) - \GK(r-1, v(b), v(c), v(e), v(d-a)) \\
    &- \GK(r, v(b), v(c), v(e)-1, v(d-a)) + \GK(r-1, v(b), v(c), v(e)-1, v(d-a)).
  \end{align*}
\end{corollary}
\begin{proof}
  This is obtained by subtracting \Cref{prop:finale_by_one} from itself,
  with $v(e)$ decremented by one.
\end{proof}

\section{Finishing touches}
