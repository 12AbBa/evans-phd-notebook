\chapter{Proof of \Cref{thm:semi_lie_n_equals_2}}
\label{ch:finale}
We now put together all the results from the previous chapters to
prove \Cref{thm:semi_lie_n_equals_2}.

\section{Matching $\guv$ and $(g,u)$, and the invariants for the matching}
\subsection{The invariants for the orbit of $\guv$}
From \Cref{ch:orbitalFJ0}, recall that we considered
\[
  (\gamma, \uu, \vv^\top)
  =
  \left( \begin{bmatrix} a & b \\ c & d \end{bmatrix}
    \begin{bmatrix} 0 \\ 1 \end{bmatrix}, \begin{bmatrix} 0 & e \end{bmatrix} \right)
  \in (S_2(F) \times V'_2(F))\rs
\]
subject to the conditions
\begin{align*}
  \bar b c = b \bar c &= 1 - a \bar a = 1 - d \bar d \\
  \text{and } d &= - \bar a c / \bar c = -\bar a b / \bar b.
\end{align*}
It will again be enough to consider the situation in which $\det \gamma = 1$.
The invariants in this case as described in \Cref{def:matching_semi_lie} are:
\begin{itemize}
  \ii $\Tr \gamma = a + d$
  \ii $\det \gamma = ad - bc$
  \ii $\vv^\top \uu = e$
  \ii $\vv^\top \gamma \uu = \begin{bmatrix} 0 & e \end{bmatrix}
  \begin{bmatrix} a & b \\ c & d \end{bmatrix} \begin{bmatrix} 0 \\ 1 \end{bmatrix} = de$.
\end{itemize}
Note that the parameters $b$ and $c$ are absent; but we have
\[ v(b) + v(c) = v(\det \gamma - a d). \]

\subsection{Matching}
We now take these results and line them up with \Cref{sec:g_u_invariants}.

\begin{lemma}
  \label{lem:finale_match}
  Let $g = \alpha + \beta \Pi$ and $u = s + t \Pi$ as above
  and suppose $(g,u)$ matches with
  \[ (\gamma, \uu, \vv^\top) = \left( \begin{bmatrix} a & b \\ c & d \end{bmatrix}
    \begin{bmatrix} 0 \\ 1 \end{bmatrix}, \begin{bmatrix} 0 & e \end{bmatrix} \right)
    \in (S_2(F) \times V'_2(F))\rs. \]
  Then we have
  \begin{align*}
    \alpha_1 &= \frac{d-a}{2} \\
    \Nm u &= e \\
    \det g - \alpha \bar \alpha &= -bc.
  \end{align*}
\end{lemma}
\begin{proof}
  Setting the invariants above equal we obtain
  \begin{align*}
    a + d &= \alpha + \bar\alpha \\
    \det \gamma = ad - bc &= \det g \\
    e &= \Nm u \\
    de &= \alpha \Nm u.
  \end{align*}
  Hence it follows $d = \alpha$ and $a = \bar\alpha$.
  In particular
  \[ \alpha_1 = \frac{\alpha - \bar\alpha}{2\sqrt\eps} = \frac{d-a}{2\sqrt\eps}. \]
  The lemma follows directly from this.
\end{proof}

\section{Translation of the Gross-Keating data to the orbital side}
We combine the results of \Cref{sec:GK} with \Cref{lem:finale_match}.
Retaining the notation
\begin{align*}
  x &\coloneqq \bar u \sqrt{\eps} u = (s \bar s + t \bar t \varpi) \sqrt{\eps} \in \ODT \\
  y &\coloneqq \varpi^r g_- = (\alpha_1\sqrt\eps + \beta \Pi) \in \ODT
\end{align*}
from \Cref{sec:GK}, we obtain the following:
\begin{align*}
  v(\left\langle x,x \right\rangle _0)
    &= 2 v(\Nm u) \\
    &= 2v(e) \\
  v\left( \left\langle x,y \right\rangle _0 \right)
    &= r + v(\alpha_1) + v(\Nm u) \\
    &= r + v(d-a) + v(e) \\
  v(
    \left\langle x,x \right\rangle _0 \left\langle y,y \right\rangle _0
    - \left\langle x,y \right\rangle^2_0
  )
    &= 2r + 2 v(\Nm u) + v(\det g - \alpha \bar \alpha) \\
    &= 2r + 2v(e) + v(b) + v(c).
\end{align*}
Notice that the last valuation is odd.
Therefore we can always extract $v(\left\langle y,y \right\rangle _0)$ by writing
\begin{align*}
  v\left(\left\langle x,x \right\rangle _0\right) + v\left(\left\langle y,y \right\rangle _0\right)
  &= \min \left( v\left( \left\langle x,x \right\rangle _0 \left\langle y,y \right\rangle _0 - \left\langle x,y \right\rangle^2_0 \right),
    v(\left\langle x,y \right\rangle^2_0) \right) \\
  \implies  v\left(\left\langle y,y \right\rangle _0\right)
  &= \min(2r + 2v(e) + v(b) + v(c), 2r + 2v(d-a) + 2v(e)) - 2v(e) \\
  &= 2r + \min(v(b) + v(c), 2v(d-a)).
\end{align*}
Hence, we have
\begin{align*}
  &\phantom= \min\left(
    v(\left\langle x,x \right\rangle _0),
    v(\left\langle x,y \right\rangle _0),
    v(\left\langle y,y \right\rangle _0),
  \right) \\
  &= \min\left( 2v(e), r+v(d-a)+v(e), 2r+\min(v(b)+v(c), 2v(d-a)) \right) \\
  &= \min\left( 2v(e), r+v(d-a)+v(e), 2r+v(b)+v(c), 2r+2v(d-a) \right) \\
  &= \min\left( 2v(e), v(b)+v(c)+2r, 2v(d-a)+2r \right)
\end{align*}
where we can drop $r+v(d-a)+v(e)$ from the minimum because it equals
$\frac{2v(e) + (2v(d-a)+2r)}{2}$.

Now recall the right-hand side of \Cref{prop:GK}, that is
\[
  \begin{cases}
    \sum_{j=0}^{\frac{n_1-1}{2}} (n_1+n_2-4j) \cdot q^j & \text{if } n_1 \equiv 1 \pmod 2 \\
    \frac{n_2-n_1+1}{2} q^{n_1/2} + \sum_{j=0}^{n_1/2-1} (n_1+n_2-4j) \cdot q^j & \text{if } n_1 \equiv 0 \pmod 2.
  \end{cases}
\]
for $0 \le n_1 \le n_2$.
Then apply \Cref{prop:GK} to obtain that
\[
  \left< \ZO4(1), \;
    \ZO4(\bar u \sqrt{\eps} u), \;
    \ZO4(\varpi^r g_-) \right>_{\MM_2 \times \MM_2}
\]
is equal to the above formula applied at
\begin{align*}
  n_1 &\coloneqq \min(2v(e), v(b)+v(c)+2r, 2v(d-a)+2r) \\
  n_2 &\coloneqq 2v(e) + v(b) + v(c) + 2r - n_1.
\end{align*}
Note that
\[ n_1 + n_2 = 2v(e) + v(b) + v(c) + 2r. \]
For brevity, we henceforth introduce the symbol $\GK$ for the sum above;
in other words, \Cref{conj:serre_pullback_space} and \Cref{conj:serre_pullback_divisor}
together amount to asserting that
\[ \left< \ZO4(1), \; \ZO4(\bar u \sqrt{\eps} u), \; \ZO4(\varpi^r g_-) \right>_{\MM_2 \times \MM_2}
    = \GK(r, v(b), v(c), v(e), v(d-a)). \]
In that case we also have
\[
  \left< \ZO4(1), \;
    \ZO4\left(\frac{\bar u \sqrt{\eps} u}{\varpi}\right), \;
    \ZO4(\varpi^r g_-) \right>_{\MM_2 \times \MM_2}
    = \GK(r, v(b), v(c), v(e)-1, v(d-a)). \]
Subtracting the two gives
\begin{equation}
  \begin{aligned}
    \Int^\circ((g,u), \fr) &=
      \left< \ZO4(1), \;
        \ZO4(\bar u \sqrt{\eps} u)^\circ, \;
        \ZO4(\varpi^r g_-) \right>_{\MM_2 \times \MM_2} \\
    &= \GK(r, v(b), v(c), v(e), v(d-a)) \\
    &\qquad- \GK(r, v(b), v(c), v(e)-1, v(d-a)).
  \end{aligned}
  \label{eq:int_subtract}
\end{equation}

\section{Base change}
The base change for $n=2$ was already calculated in \cite{ref:AFLspherical}
and we simply recall the result here.

As in \Cref{ch:jiao}, we define
\[ \fr \coloneqq \mathbf{1}_{\varpi^{-r} \Mat_2(\OO_E) \cap \U(\VV_n^+)} \in \HH(\U(\VV_n^+)) \]
\begin{lemma}
  [{\cite[Lemma 7.1.1]{ref:AFLspherical}}]
  \label{lem:finale_base_change}
  For $n = 2$ and $r \ge 1$ we have
  \[
    \BC_S^{\eta}(
      \mathbf{1}_{K'_{S, \le r}}
      + \mathbf{1}_{K'_{S, \le (r-1)}}
      ) = (-1)^r(\mathbf{1}_{K, \le r} - \mathbf{1}_{K, \le (r-1)})
  \]
\end{lemma}
\begin{proof}
  This follows directly from \cite[equation (7.1.9)]{ref:AFLspherical}.
\end{proof}

\section{Transfer factor}
For $n = 2$, the transfer factor is
\begin{align*}
  \omega\guv
  &= \eta\left( \det
    \left[ \gamma^0 \uu, \gamma^1 \uu \right]
  \right) \\
  &= \eta\left( \det
    \begin{bmatrix} 0 & 1 \\ b & c \end{bmatrix}
  \right) \\
  &= (-1)^{v(b)} = - (-1)^{v(c)}.
\end{align*}

\section{Comparison to orbital formula}
In what follows, we introduce the shorthand
\[ \partial\Orb(r, v(b), v(c), v(e), v(d-a))
  \coloneqq
  \left. \pdv{}{s} \right\rvert_{s=0}
  \Orb(\guv, \mathbf{1}_{K'_{S, \le r}} \otimes \oneV, s). \]
The main claim is that the following formula holds:
\begin{theorem}
  \label{thm:miracle}
  We have
  \begin{align*}
    &\frac{(-1)^{r+v(c)}}{\log q}\Big(\partial\Orb(r, v(b), v(c), v(e), v(d-a))
      + \partial\Orb(r, v(b), v(c), v(e)-1, v(d-a))\Big) \\
    &= \GK(r, v(b), v(c), v(e), v(d-a)).
  \end{align*}
\end{theorem}

We continue to use the notation $N$ and $\varkappa$ from \Cref{ch:orbitalFJ2} defined by
\begin{align*}
  N &\coloneqq \min \left( v(e),
    \left\lfloor \tfrac{v(b)+v(c)}{2} \right\rfloor + r, v(d-a) + r \right) \\
  \varkappa &\coloneqq v(e) - v(d-a) - r \ge 0
\end{align*}
and prove \Cref{thm:miracle} by exhausting the
cases based on which value of $N$ is smallest.

\subsection{Proof of \Cref{thm:miracle}
  when $v(e) \le \left\lfloor \tfrac{v(b)+v(c)}{2} \right\rfloor + r$
  and $v(e) \le v(d-a) + r$}
In the Gross-Keating formula we have simply
\begin{align*}
  n_1 &= 2v(e) \\
  n_2 &= v(b) + v(c) + 2r.
\end{align*}
Hence, we have
\begin{align*}
  \GK(r, v(b), v(c), v(e), v(d-a))
  &= \frac{-2v(e)+v(b)+v(c)+2r+1}{2} q^{v(e)} \\
  &\qquad+ \sum_{j=0}^{v(e)-1} (2v(e)+v(b)+v(c)+2r-4j) \cdot q^j.
\end{align*}
On the orbital side, we refer to \Cref{thm:semi_lie_derivative_single}
and compare it to the single instance of Gross-Keating above.
The exponent of $j$ runs up to $v(e)$ in one case and $v(e)-1$ in the second;
that is we need
\begin{align*}
  &\phantom= \sum_{j=0}^{v(e)} \left( \frac{2v(e)+v(b)+v(c)+1}{2} + r - 2j \right) \cdot q^j \\
  &\qquad+ \sum_{j=0}^{v(e)-1} \left( \frac{2(v(e)-1)+v(b)+v(c)+1}{2} + r - 2j \right) \cdot q^j \\
  &= \frac{-2v(e)+v(b)+v(c)+2r+1}{2} q^{v(e)}
  + \sum_{j=0}^{v(e)-1} (2v(e)+v(b)+v(c)+2r-4j) \cdot q^j
\end{align*}
which is obvious.

\subsection{Proof of \Cref{thm:miracle}
  when $\left\lfloor \tfrac{v(b)+v(c)}{2} \right\rfloor + r < v(e)$
  and $v(b)+v(c) < 2v(d-a)$}
\subsubsection{Case where $r > 0$}
Set $N = \left\lfloor \frac{v(b)+v(c)}{2} \right\rfloor + r$.
In the Gross-Keating formula we have simply
\begin{align*}
  n_1 &= 2N+1 \\
  n_2 &= 2v(e).
\end{align*}
Hence
\[ \GK(r, v(b), v(c), v(e), v(d-a))
  = \sum_{j=0}^N (2v(e)+v(b)+v(c)+2r-4j) \cdot q^j. \]
We compare this to \Cref{thm:semi_lie_derivative_single}; we check
\begin{align*}
  &\phantom= \sum_{j=0}^{N} \left( \frac{2v(e)+v(b)+v(c)+1}{2} + r - 2j \right) \cdot q^j \\
  &\qquad+ \sum_{j=0}^{N} \left( \frac{2(v(e)-1)+v(b)+v(c)+1}{2} + r - 2j \right) \cdot q^j \\
  &= \sum_{j=0}^N (2v(e)+v(b)+v(c)+2r-4j) \cdot q^j
\end{align*}
which is clear.

\subsection{Proof of \Cref{thm:miracle}
  when $v(d-a)+r < v(e)$ and $2v(d-a) < v(b)+v(c)$}
Hence $N = v(d-a) + r$ and $\varkappa \coloneqq v(e) - (v(d-a)+r) > 0$.
In the Gross-Keating side formula, we now have
\begin{align*}
  n_1 &\coloneqq 2v(d-a) + 2r = 2N \\
  n_2 &= 2v(e) + v(b) + v(c) - 2v(d-a).
\end{align*}
Hence
\begin{align*}
  \GK(r, v(b), v(c), v(e), v(d-a))
  &= \frac{2v(e)+v(b)+v(c)-4v(d-a)-2r+1}{2} q^{N} \\
    &\qquad+ \sum_{j=0}^{N-1} (2v(e)+v(b)+v(c)+2r-4j) \cdot q^j.
\end{align*}
This time, the relevant combination of \Cref{thm:semi_lie_derivative_single} is
\begin{align*}
  &\phantom= \sum_{j=0}^N q^j
  \cdot \left( \frac{2v(e)+v(b)+v(c)+1}{2} + r - 2j \right) \\
  &\qquad + q^{N} \cdot
  \begin{cases}
    -\frac{\varkappa}{2} & \text{if }\varkappa \equiv 0 \pmod 2 \\
    \frac{\varkappa}{2} - \left( v(e)+\frac{v(b)+v(c)}{2}-2v(d-a)-r \right)
    & \text{if }\varkappa \equiv 1 \pmod 2
  \end{cases} \\
  &\qquad + \sum_{j=0}^N q^j
  \cdot \left( \frac{2(v(e)-1)+v(b)+v(c)+1}{2} + r - 2j \right) \\
  &\qquad + q^{N} \cdot
  \begin{cases}
    -\frac{\varkappa-1}{2} & \text{if }\varkappa-1 \equiv 0 \pmod 2 \\
    \frac{\varkappa-1}{2} - \left( (v(e)-1)+\frac{v(b)+v(c)}{2}-2v(d-a)-r \right)
    & \text{if }\varkappa-1 \equiv 1 \pmod 2 \\
  \end{cases} \\
  &= \sum_{j=0}^N q^j
  \cdot \left( 2v(e)+v(b)+v(c) + 2r - 4j \right) \cdot q^j \\
  &\qquad + q^{N} \cdot
  \begin{cases}
    -\frac{\varkappa}{2} + \frac{\varkappa-1}{2} - \left( (v(e)-1) +\frac{v(b)+v(c)}{2}-2v(d-a)-r \right)
    & \text{if }\varkappa \equiv 0 \pmod 2 \\
    -\frac{\varkappa-1}{2} + \frac{\varkappa}{2} - \left( v(e)+\frac{v(b)+v(c)}{2}-2v(d-a)-r \right)
    & \text{if }\varkappa \equiv 1 \pmod 2 \\
  \end{cases} \\
  &= \sum_{j=0}^N q^j
  \cdot \left( 2v(e)+v(b)+v(c) + 2r - 4j \right) \cdot q^j \\
  &\qquad + q^{N} \cdot
    \left( \half - \left( v(e)+\frac{v(b)+v(c)}{2}-2v(d-a)-r \right) \right) \\
  &= \sum_{j=0}^{N-1} q^j
  \cdot \left( 2v(e)+v(b)+v(c) + 2r - 4j \right) \cdot q^j \\
  &\qquad + q^{N} \cdot
    \left( \left( 2v(e)+v(b)+v(c) + 2r - 4N \right) +
    \half - \left( v(e)+\frac{v(b)+v(c)}{2}-2v(d-a)-r \right) \right).
\end{align*}
The coefficient of $q^N$ is given by
\begin{align*}
  &\phantom= \left( 2v(e)+v(b)+v(c) + 2r - 4(v(d-a)-r) \right) +
  \half - \left( v(e)+\frac{v(b)+v(c)}{2}-2v(d-a)-r \right) \\
  &= \frac{2v(e)+v(b)+v(c)-4v(d-a)-2r+1}{2}
\end{align*}
which matches the one from Gross-Keating.
Hence \Cref{thm:miracle} is completely proved.

\section{Conclusion (proof of \Cref{thm:semi_lie_n_equals_2})}
From \Cref{thm:miracle} we have
\begin{align*}
  \GK(r, v(b), v(c), v(e), v(d-a))
  &= \frac{(-1)^{r+v(c)}}{\log q} \Big(
      \partial \Orb(r, v(b), v(c), v(e), v(d-a)) \\
      &\qquad + \partial \Orb(r, v(b), v(c), v(e)-1, v(d-a))
    \Big) \\
  \GK(r, v(b), v(c), v(e)-1, v(d-a))
  &= \frac{(-1)^{r+v(c)}}{\log q} \Big(
      \partial \Orb(r, v(b), v(c), v(e)-1, v(d-a)) \\
      &\qquad + \partial \Orb(r, v(b), v(c), v(e)-2, v(d-a))
    \Big)
\end{align*}
so subtraction (and recalling \eqref{eq:int_subtract}) gives
\begin{equation}
  \begin{aligned}
    \Int^\circ((g,u), \fr)
    &= \GK(r, v(b), v(c), v(e), v(d-a)) \\
    &\qquad- \GK(r, v(b), v(c), v(e)-1, v(d-a)) \\
    &= \frac{(-1)^{r+v(c)}}{\log q} \Big(
        \partial \Orb(r, v(b), v(c), v(e), v(d-a)) \\
        &\qquad- \partial \Orb(r, v(b), v(c), v(e)-2, v(d-a))
      \Big).
  \end{aligned}
  \label{eq:descent_by_two}
\end{equation}
We now show that \eqref{eq:descent_by_two} implies \Cref{thm:semi_lie_n_equals_2}.
Because $r = 0$ is known already, it suffices to verify for $r > 0$.

Suppose we sum \eqref{eq:descent_by_two} with $u$ replaced by
$u/\varpi^i$ for $i = 0, 1, \dots$.
The left-hand side equals $\Int((g,u), \fr)$ by \eqref{eq:int_to_int_circ}.
On the right-hand side this has the effect of decreasing $v(e)$ by $2$ since $e = \Nm u$.
Hence the sum of the right-hand sides telescopes and gives us the identity
\begin{equation}
  \Int((g,u), \mathbf{1}_{K, \le r})
  = \frac{(-1)^{v(c)+r}}{\log q}\left. \pdv{}{s} \right\rvert_{s=0}
    \Orb(\guv, \mathbf{1}_{K'_{S, \le r}} \otimes \oneV, s).
  \label{eq:match_penultimate}
\end{equation}
Subtracting the same equation from itself with $r$ replaced by $r-1$ gives
\begin{align*}
  &\Int((g,u), \mathbf{1}_{K, \le r} - \mathbf{1}_{K, \le (r-1)}) \\
  &= \frac{(-1)^{v(c)+r}}{\log q}\left. \pdv{}{s} \right\rvert_{s=0}
  \Orb(\guv, (\mathbf{1}_{K'_{S, \le r}} + \mathbf{1}_{K'_{S, \le (r-1)}}) \otimes \oneV, s).
\end{align*}
which, since $(-1)^{v(c)} = -\omega\varpi$, becomes
\begin{equation}
  \begin{aligned}
    &\Int((g,u), (-1)^r(\mathbf{1}_{K, \le r} - \mathbf{1}_{K, \le (r-1)})) \\
    &= \frac{-\omega\guv}{\log q}\left. \pdv{}{s} \right\rvert_{s=0}
    \Orb(\guv, (\mathbf{1}_{K'_{S, \le r}} + \mathbf{1}_{K'_{S, \le (r-1)}}) \otimes \oneV, s).
  \end{aligned}
  \label{eq:match_final}
\end{equation}
But \Cref{lem:finale_base_change} says that
$(-1)^r(\mathbf{1}_{K, \le r} - \mathbf{1}_{K, \le (r-1)}) \in \HH(\U(\VV_n^+))$
matches $\mathbf{1}_{K'_{S, \le r}} + \mathbf{1}_{K'_{S, \le (r-1)}} \in \HH(S_2(F))$
for any $r \ge 0$.
And hence from the $r = 0$ case we can inductively conclude
\Cref{thm:semi_lie_n_equals_2} for $r > 0$, completing the proof.
