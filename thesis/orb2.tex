\chapter{Evaluation of the weighted orbital integral for $S_3(F)$}
\label{ch:orbital2}

We now put together the sums we found in the previous section
to come up with the expression for the weighted orbital integral.

\section{Region where $n \leq 0$ for all values of $\ell$}
\begin{proposition}
  The contribution to the integral $\Orb(\gamma, \mathbf{1}_{K'_{S, \le r}}, s)$ over $n \leq 0$ is exactly
  \[ I_{n \le 0} \coloneqq q^{2(\delta+r)s} \sum_{j=0}^{\delta+2r} q^{-2js}
    = q^{-2rs} + \dots + q^{2(\delta+r)s}. \]
\end{proposition}
\begin{proof}
  For $n = 0$ we get a contribution of
  \begin{align*}
    &\phantom= \kappa \int_{t, t_1 \in E} \mathbf{1}[n=0] \mathbf{1}_{\le r}(\gamma,t,m)
      q^{2s \cdot m} q^{-2m} \odif t \odif t_1 \\
    &= \kappa \Vol(t: n =0) \sum_{m=-r}^{\delta+r} \Vol(t_1: -v(t_1)=m) q^{2m(s-1)} \\
    &= \kappa \left( 1 - \frac{q+1}{q^2} \right) \sum_{m=-r}^{\delta+r}
    \left( q^{2m} \left( 1-q^{-2} \right) \right) q^{2m(s-1)} \\
    &= \kappa \left( 1 - \frac{q+1}{q^2} \right) \left( 1-q^{-2} \right)
    \sum_{m=-r}^{\delta+r} q^{2ms}.
  \end{align*}
  For the region where $v(t) = -k < 0$, for each individual $k > 0$,
  \begin{align*}
    &\phantom= \kappa \int_{t, t_1 \in E} \mathbf{1}[v(t)=-k] \mathbf{1}_{\le r}(\gamma,t,m)
      q^{s(2m-n)} q^{2n-2m} \odif t \odif t_1 \\
    &= \kappa \Vol(t: v(t)=-k) \sum_{m=-r-k}^{\delta+r-k}
      \Vol(t_1: -v(t_1)=m) q^{s(2m+2k)-4k-2m} \\
    &= \kappa q^{2k} \left( 1 - q^{-2} \right) \sum_{m=-r-k}^{\delta+r-k}
      \left( q^{2m} \left( 1-q^{-2} \right) \right) q^{s(2m+2k)-4k-2m} \\
    &= \kappa q^{-2k} \left( 1 - q^{-2} \right)^2
      \sum_{m=-r-k}^{\delta+r-k} q^{2(m+k)s} \\
    &= \kappa q^{-2k} \left( 1 - q^{-2} \right)^2 \sum_{i=-r}^{\delta+r} q^{2is}.
  \end{align*}
  Since $\sum_{k > 0} q^{-2k} = \frac{q^{-2}}{1-q^{-2}}$,
  we find that the total contribution across both
  the $n=0$ case and the $k > 0$ case is
  \begin{align*}
    &\phantom= \left( \left( 1 - \frac{q+1}{q^2} \right) \left( 1-q^{-2} \right)
      + q^{-2}(1-q^{-2})  \right) \kappa \sum_{i=-r}^{\delta+r} q^{2is} \\
    &= \left( 1-q\inv \right) \left(1-q^{-2} \right)
      \kappa \sum_{i=-r}^{\delta+r} q^{2is} \\
    &= \sum_{i=-r}^{\delta+r} q^{2is}.
  \end{align*}
  This equals the claimed sum above.
  (We write it over $0 \le j \le \delta+2r$ for consistency with a later part.)
\end{proof}

\section{Contribution from Case 1 and Case 2 assuming $v(b) \ge 0$ and $v(d) \ge 0$}
Again using $\Vol(t_1:-v(t_1)=m) = q^{2m} (1-q^{-2})$,
summing all the cases gives the following contribution within
$\kappa \int_{t, t_1 \in E} \mathbf{1}[n > 0] \mathbf{1}_{\le r}(\gamma,t,m)$:
\begin{align*}
  I_{n > 0}^{\text{1+2}}
  % -----------------------------------------------------------
  &\coloneqq \kappa \sum_{n=1}^{r}
    \sum_{m=n-r}^{\left\lceil \frac{n-r}{2} \right\rceil+\delta+r-1}
    q^{-n} \left( 1 - q^{-2} \right) \\
    &\qquad\qquad\cdot \Big( (-1)^n q^{s(2m-n)} q^{2n-2m} \Big) \Big( q^{2m}(1-q^{-2}) \Big) \\
  &\qquad+ \kappa \sum_{n=r+1}^{\ell+r}
    \sum_{m=n-r}^{\left\lceil \frac{n-r}{2} \right\rceil+\delta+r-1}
    q^{-n - \left\lceil \frac{n-r}{2} \right\rceil} \left( 1 - q\inv \right) \\
    &\qquad\qquad\cdot \Big( (-1)^n q^{s(2m-n)} q^{2n-2m} \Big) \Big( q^{2m}(1-q^{-2}) \Big) \\
  &\qquad+ \kappa \sum_{n=1}^{r}
    \sum_{m=\max\left(n-r, \left\lceil \frac{n-r}{2} \right\rceil+\delta+r\right)}^{\delta+r}
    q^{-n} \left( 1 - q^{-2} \right) \\
    &\qquad\qquad\cdot \Big( (-1)^n q^{s(2m-n)} q^{2n-2m} \Big) \Big( q^{2m}(1-q^{-2}) \Big) \\
    \\
  &\qquad+ \kappa \sum_{n=1}^{\ell+r}
    \sum_{m=\max\left(n-r, \left\lceil \frac{n-r}{2} \right\rceil+\delta+r, \delta+r+1 \right)}^{\min(n,\lambda)+\delta+r}
    q^{-n - (m-\delta-r)} \left( 1 - q\inv \right) \\
    &\qquad\qquad\cdot \Big( (-1)^n q^{s(2m-n)} q^{2n-2m} \Big) \Big( q^{2m}(1-q^{-2}) \Big)
    \\
  % -----------------------------------------------------------
  &= \sum_{n=1}^{r}
    \sum_{m=n-r}^{\left\lceil \frac{n-r}{2} \right\rceil+\delta+r-1}
    q^{n} \left( 1 + q\inv \right)
      \cdot (-1)^n q^{s(2m-n)} \\
  &\qquad+ \sum_{n=r+1}^{\ell+r}
    \sum_{m=n-r}^{\left\lceil \frac{n-r}{2} \right\rceil+\delta+r-1}
    q^{\left\lfloor \frac{n+r}{2} \right\rfloor}
      \cdot (-1)^n q^{s(2m-n)} \\
  &\qquad+ \sum_{n=1}^{r}
    \sum_{m=\max\left(n-r, \left\lceil \frac{n-r}{2} \right\rceil+\delta+r\right)}^{\delta+r}
    q^{n} \left( 1 + q\inv \right)
      \cdot (-1)^n q^{s(2m-n)} \\
  &\qquad+ \sum_{n=1}^{\ell+r}
    \sum_{m=\max\left(n-r, \left\lceil \frac{n-r}{2} \right\rceil+\delta+r, \delta+r+1 \right)}^{\min(n,\lambda)+\delta+r}
    q^{n - (m-\delta-r)}
      \cdot (-1)^n q^{s(2m-n)}.
\end{align*}
To simplify the expressions, we replace the summation variable $m$ with
\[ j \coloneqq (n + \delta + r) - m \geq 0. \]
In that case,
\[ 2m-n = 2(\delta+n+r-j)-n = n + 2\delta + 2r - 2j. \]
Then the expression rewrites as
\begin{align*}
  I_{n > 0}^{\text{1+2}}
  % -----------------------------------------------------------
  &= \sum_{n=1}^{r}
    \sum_{j = \left\lfloor \frac{n+r}{2} \right\rfloor+ 1}^{\delta + 2r}
    q^{n} \left( 1 + q\inv \right) \cdot (-1)^n q^{s(n+2\delta+2r-2j)} \\
  &\qquad+ \sum_{n=r+1}^{\ell+r}
    \sum_{j = \left\lfloor \frac{n+r}{2} \right\rfloor+ 1}^{\delta + 2r}
    q^{\left\lfloor \frac{n+r}{2} \right\rfloor} \cdot (-1)^n q^{s(n+2\delta+2r-2j)} \\
  &\qquad+ \sum_{n=1}^{r}
    \sum_{j=n}^{\min\left( \delta+2r, \left\lfloor \frac{n+r}{2} \right\rfloor \right)}
    q^{n} \left( 1 + q\inv \right) \cdot (-1)^n q^{s(n+2\delta+2r-2j)} \\
  &\qquad+ \sum_{n=1}^{\ell+r}
    \sum_{j=\max(0,n-\lambda)}^{\min(\delta+2r, \left\lfloor \frac{n+r}{2} \right\rfloor, n-1)}
    q^{j} \cdot (-1)^n q^{s(n+2\delta+2r-2j)}.
\end{align*}

We interchange the order of summation so that it is first over $j$ and then $n$.
There are four double sums to interchange.

\begin{itemize}
  \ii The first double sum runs from $j=\left\lfloor \frac{r+1}{2} \right\rfloor+1$ to $j=\delta+2r$.
  In addition to $1 \le n \le r$,
  we need $\left\lfloor \frac{n+r}{2} \right\rfloor + 1 \leq j$,
  which solves to $\frac{n+r}{2} \leq j-\half$ or $n \leq 2j-1-r$.
  Thus the condition on $n$ is
  \[ 1 \leq n \leq \min(2j-1-r, r). \]

  \ii The second double sum runs from $j=r+1$ to $\delta+2r$.
  We also need $r+1 \le n \le \ell+r$ and $n \le 2j-1-r$.
  Hence, the desired condition on $n$ is
  \[ r+1 \leq n \leq \min(2j-1-r, \ell+r). \]

  \ii The third double sum runs from $j=1$ to $j=r$.
  Meanwhile, the values of $n$ need to satisfy $1 \le n \le r$, $n \leq j$
  and $j \leq \left\lfloor \frac{n+r}{2} \right\rfloor \implies n \geq 2j-r$,
  consequently we just obtain
  \[ \max(1, 2j-r) \leq n \leq j. \]

  \ii The fourth double sum runs $j=0$ to
  \[ j=\min\left( \delta+2r, \left\lfloor \frac{\ell}{2} \right\rfloor + r, \ell+r-1 \right)
    = \left\lfloor \frac{\ell}{2} \right\rfloor + r - \mathbf{1}[\ell = 0] \]
  again because of $\ell < 2\delta$.
  Meanwhile, we require $1 \le n \le \ell+r$, $j \ge n-\lambda$, $j \le n-1$,
  as well as $j \le \left\lfloor \frac{n+r}{2} \right\rfloor
  \iff n \ge 2j-r$.
  Putting these four conditions together gives
  \[ \max(j+1, 2j-r) \le n \le \min(\lambda+j, \ell+r). \]
\end{itemize}
Hence we get
\begin{align*}
  I_{n > 0}^{\text{1+2}}
  &= \sum_{j = \left\lfloor \frac{r+1}{2} \right\rfloor+ 1}^{\delta+2r}
    \sum_{n=1}^{\min(2j-1-r, r)}
    q^{n} \left( 1 + q\inv \right) \cdot (-1)^n q^{s(n+2\delta+2r-2j)} \\
  &\qquad+ \sum_{j = r+1}^{\delta + 2r}
    \sum_{n=r+1}^{\min(2j-1-r, \ell+r)}
    q^{\left\lfloor \frac{n+r}{2} \right\rfloor} \cdot (-1)^n q^{s(n+2\delta+2r-2j)} \\
  &\qquad+ \sum_{j=1}^{r}
    \sum_{n=\max(1,2j-r)}^{j}
    q^{n} \left( 1 + q\inv \right) \cdot (-1)^n q^{s(n+2\delta+2r-2j)} \\
  &\qquad+ \sum_{j=0}^{\left\lfloor \frac{\ell}{2} \right\rfloor + r - \mathbf{1}[\ell = 0]}
    \sum_{n=\max(j+1, 2j-r)}^{\min(\lambda+j, \ell+r)}
    q^{j} \cdot (-1)^n q^{s(n+2\delta+2r-2j)}.
\end{align*}

At this point, we can unify the sum over $j$ by noting that for $j$ outside of the
summation range, the inner sum is empty anyway.
Specifically, note that:
\begin{itemize}
  \ii In the first and second double sum,
  the inner sum over $n$ is empty anyway when $j < r$.
  \ii In the third double sum, adding $j=0$ does not introduce new terms.
  Moreover, when $j > r$ the inner sum over $n$ is also empty anyway.
  \ii In the fourth double sum,
  \begin{itemize}
    \ii If $\ell = 0$ and $j \ge r$, then $j+1 \ge 0+r$; and
    \ii If $\ell > 0$ and $j > \frac{\ell}{2} + r$, then $2j-r \ge \ell+r$.
  \end{itemize}
  So no new terms are introduced in this case either.
\end{itemize}
So we can unify all four double sums to run over $0 \le j \le \delta + 2r$,
simplifying the expression to just

\begin{align*}
  I_{n > 0}^{\text{1+2}}
  = q^{2(\delta+r)s}
  \sum_{j = 0}^{\delta + 2r} \Bigg(
    & \sum_{n=1}^{\min(2j-1-r, r)}
      q^{n} \left( 1 + q\inv \right) \cdot (-1)^n q^{s(n-2j)} \\
    & + \sum_{n=r+1}^{\min(2j-1-r, \ell+r)}
      q^{\left\lfloor \frac{n+r}{2} \right\rfloor} \cdot (-1)^n q^{s(n-2j)} \\
    &+ \sum_{n=\max(1,2j-r)}^{j}
      q^{n} \left( 1 + q\inv \right) \cdot (-1)^n q^{s(n-2j)} \\
    &+ \sum_{n=\max(j+1, 2j-r)}^{\min(\lambda+j, \ell+r)} q^{j} \cdot (-1)^n q^{s(n-2j)} \Bigg).
\end{align*}

\section{Merging of $I_{n \le 0}$ with $I_{n > 0}^{\text{1+2}}$}
We continue assuming $v(b) = v(d) = 0$.
It turns out that $I_{n \le 0} + I_{n > 0}^{\text{1+2}}$ can be rewritten more compactly
(giving a simple answer especially when $\ell$ is odd).
Then
\[
  I_{n \le 0} + I_{n > 0}^{\text{1+2}}
\]
can be rewritten to the collation
\begin{align*}
  = q^{2(\delta+r)s}
  \sum_{j = 0}^{\delta + 2r} \Bigg(
    q^{-2js}
    &+ \sum_{n=1}^{\min(2j-1-r, r)}
      q^{n} \left( 1 + q\inv \right) \cdot (-1)^n q^{s(n-2j)} \\
    &+ \sum_{n=r+1}^{\min(2j-1-r, \ell+r)}
      q^{\left\lfloor \frac{n+r}{2} \right\rfloor} \cdot (-1)^n q^{s(n-2j)} \\
    &+ \sum_{n=\max(1,2j-r)}^{j}
      q^{n} \left( 1 + q\inv \right) \cdot (-1)^n q^{s(n-2j)} \\
    &+ \sum_{n=\max(j+1, 2j-r)}^{\min(\lambda+j, \ell+r)} q^{j} \cdot (-1)^n q^{s(n-2j)} \Bigg).
\end{align*}
Note that when $r=0$ and $\ell = \lambda \equiv 1 \pmod 2$
we recover \cite[equation (4.13)]{ref:AFL}.

The above expression can be considered as a Laurent polynomial in $-q^s$,
whose coefficients are nonnegative polynomials in $q$ (note that $(-1)^n = (-1)^{n-2j}$).
Now we are going to extract the coefficient of $(-q^s)^k$, for each integer $k$.
First, note that
\begin{itemize}
  \ii The initial term before the sums adds $1$
  if $k$ is even and $-2r \le k \le 2\delta + 2r$, and $0$ otherwise.
\end{itemize}
We move on to the inner sums and calculate their contributions.
For a fixed $k \in \ZZ$, we want to consider $(n,j)$ with $n-2j + 2(\delta+r) = k$,
that is, $2j = n + 2\delta + 2r - k$, or $n = 2j + k - 2\delta - 2r$.
The condition that $j \in \ZZ$ and $0 \le j \le \delta+2r$ is then equivalent to
\begin{equation}
  k-2\delta-2r \le n \le k+2r \qquad\text{and}\qquad n \equiv k \pmod 2.
  \label{eq:outer_assumption}
\end{equation}
We note also that
\begin{equation}
  n < 2j-r \iff n < n + 2 \delta + r - k \iff k < 2\delta + r
  \label{eq:outer_half_assumption_12}
\end{equation}
which needs to hold for the first two sums to contribute.
Conversely, in the latter two sums, we will assume that
\begin{equation}
  n \ge 2j-r \iff k \ge 2\delta + r.
  \label{eq:outer_half_assumption_34}
\end{equation}
Now we are ready for the main calculation.
In what follows $i \% 2 \in \{0,1\}$ means the remainder when $i$ is divided by $2$.
Moreover, any ellipses of the form $q^i + \dots + q^{i'}$
will be abbreviate for $q^i + q^{i-1} + \dots + q^{i'}$
(i.e.\ within any ellipses, the exponents are understood to decrease by $1$,
and the sums are always nonempty, meaning $i \ge i'$).

In the region where $k < 2\delta + r$, the first two sums contribute:
\begin{itemize}
  \ii The first sum contributes if and only if \eqref{eq:outer_assumption} holds,
  $1 \le n \le r$ and \eqref{eq:outer_half_assumption_12} is true.
  Hence, the contribution only occurs when $k < 2\delta + r$.
  In that case, all $1 \le n \le \min(r,k+2r)$ with $n \equiv k \pmod 2$ appear.
  Since the contribution of a given $n$ is $q^n + q^{n-1}$
  and the $n$ are incrementing by $2$, our final total is
  \[
    \begin{cases}
      q^{k+2r} + \dots + q^{(k-1)\%2} & \text{if } -2r < k \leq -r \\
      q^{r-(k-r)\%2} + \dots + q^{(k-1)\%2} & \text{if } -r \le k < 2\delta + r \\
      0 & \text{otherwise}.
    \end{cases}
  \]

  \ii The second sum contributes if and only if \eqref{eq:outer_assumption} holds,
  \eqref{eq:outer_half_assumption_12} holds and $r+1 \le n \le \ell + r$.
  The hypothesis $n > r$ means we need $k \geq -r$.
  Since $\ell < 2\delta$, the upper bound for $n$ is $n \le \min(\ell+r, k+2r)$
  which we split into two cases.

  In the case where $k+2r \le \ell+r$, then since $\ell < 2\delta$,
  the inequality $k < 2\delta + r$ holds automatically.
  We have the largest term $n=k+2r \equiv k \pmod 2$,
  so the largest exponent $q$ that appears is
  $\left\lfloor \frac{(k+2r)+r}{2} \right\rfloor$.

  In the other case $\ell+r \le k+2r$,
  we obtain largest exponents of
  \[ \left\lfloor \frac{(\ell+r-(\ell+r-k)\%2) + r}{2} \right\rfloor
    = r + \left\lfloor \frac{\ell-(\ell+r-k)\%2}{2} \right\rfloor.
  \]
  Thus, we obtain
  \[
    \begin{cases}
      q^{\left\lfloor \frac{k+3r}{2} \right\rfloor} + \dots + q^{r+(r+1-k)\%2}
        & \text{if } -r \le k \le \ell - r \\
      q^{r + \left\lfloor \frac{\ell-(\ell+r-k)\%2}{2} \right\rfloor}
        + \dots + q^{r + (r+1-k)\%2}
        & \text{if } \ell - r \leq k < 2\delta + r \\
      0 & \text{otherwise}.
    \end{cases}
  \]
\end{itemize}

In the region where $k \ge 2\delta + r$, the latter two sums are in play:
\begin{itemize}
  \ii In the third sum, we assume \eqref{eq:outer_half_assumption_34};
  then the other constraints on $n$ are
  \begin{align*}
    n &\ge 1 \\
    n \le j \iff 2n \le n+2\delta+2r-k \iff n &\le 2\delta + 2r - k
  \end{align*}
  which implies $k < 2 \delta + 2r$ for this range to be nonempty.
  In this case, \eqref{eq:outer_assumption} is actually redundant already.
  That means our contribution can be described as
  \[
    \begin{cases}
      q^{2\delta+2r-k} + \dots + q^{(k-1)\%2} & \text{if } 2\delta+r \le k < 2\delta+2r \\
      0 & \text{otherwise}.
    \end{cases}
  \]
  \ii Unlike the other sums, the $j$ is in the exponent in the fourth sum,
  so \eqref{eq:outer_assumption} will not be useful to us.
  Instead our goal is to detect the values of $j$ for which the corresponding value of
  \[ n = 2j-2\delta-2r+k \]
  lies in the desired interval.
  That is, we get a contribution of $q^j$ if and only if
  \eqref{eq:outer_half_assumption_34} holds and
  \begin{align*}
    &\phantom{\iff} 0 \le j \le \delta + 2r \\
    j < 2j-2\delta-2r+k &\implies 2\delta+2r-k < j \\
    2j-2\delta-2r+k \le \lambda+j &\iff j \le 2\delta+2r+\lambda-k \\
    2j-2\delta-2r+k \le \ell+r &\iff j \le \delta+\frac{3r+\ell-k}{2}
  \end{align*}
  The values of $k$ for which there is any valid index $j$ is given by
  \[ 2\delta + r \le k \le 2 \delta + \min \left( \lambda + 2r, \ell + 3r \right). \]
  The breakpoint for the two upper bounds on $j$ occurs when
  \[ 2\delta+2r-k+\lambda \le \delta+\frac{3r+\ell-k}{2}
    \iff k \ge 2\delta + 2\lambda - \ell + r. \]

  Comparing all the bounds, we find there are three possible scenarios.
  \begin{itemize}
    \ii If $\lambda \le \frac{\ell+r}{2}$, then we get
    \[
      \begin{cases}
        q^{\delta+\left\lfloor\frac{3r+\ell-k}{2}\right\rfloor} + \dots + q^{2\delta+2r-k+1}
          & \text{if } 2\delta+r \le k \le 2\delta+2r  \\
        q^{\delta+\left\lfloor\frac{3r+\ell-k}{2}\right\rfloor} + \dots + q^{0}
          & \text{if } 2\delta+2r < k \le 2\delta+2\lambda-\ell+r  \\
        q^{2\delta+2r+\lambda-k} + \dots + q^0
          & \text{if } 2\delta+2\lambda-\ell+r \le k \le 2\delta+\lambda+2r \\
        0 & \text{otherwise}.
      \end{cases}
    \]

    \ii If $\frac{\ell+r}{2} < \lambda \le \ell + r$, then we get
    \[
      \begin{cases}
        q^{\delta+\left\lfloor\frac{3r+\ell-k}{2}\right\rfloor} + \dots + q^{2\delta+2r-k+1}
          & \text{if } 2\delta+r \le k \le 2\delta+2\lambda-\ell+r  \\
        q^{\delta+\left\lfloor\frac{3r+\ell-k}{2}\right\rfloor} + \dots + q^{2\delta+2r-k+1}
          & \text{if } 2\delta+2\lambda-\ell+r \le k \le 2\delta+2r  \\
        q^{2\delta+2r+\lambda-k} + \dots + q^0
          & \text{if } 2\delta+2r < k \le 2\delta+\lambda+2r \\
        0 & \text{otherwise}.
      \end{cases}
    \]

    \ii If $\ell+r < \lambda$, then we get
    \[
      \begin{cases}
        q^{\delta+\left\lfloor\frac{3r+\ell-k}{2}\right\rfloor} + \dots + q^{2\delta+2r-k+1}
          & \text{if } 2\delta+r \le k \le 2\delta+2r \\
        q^{\delta+\left\lfloor\frac{3r+\ell-k}{2}\right\rfloor} + \dots + q^{0}
          & \text{if } 2\delta+2r < k \le 2\delta+\ell+3r \\
        0 & \text{otherwise}.
      \end{cases}
    \]
  \end{itemize}
\end{itemize}
This completes the analysis of the four sums above.
For later purposes, it will be more symmetric to rewrite the exponent as
\[ \delta + \left\lfloor \frac{3r+\ell-k}{2} \right\rfloor
  = r + \left\lfloor \frac{(2\delta+\ell+r)-k}{2} \right\rfloor. \]

Now we can piece together all the parts below.
It turns out that for every value of $k$,
the coefficient of $(-q^s)^k$ is an expression
of the form $1 + q + q^2 + \dots + q^{\nn_\gamma(k)}$ for some $k$.
Indeed,
\begin{itemize}
  \ii When $k = -2r$ the only term is $q^0$.
  \ii For $-2r < k < r$, only the first sum contributes
  $q^{k+2r} + \dots + q^{(k-1)\%2}$,
  which is then completed by the $q^0$ contribution from $I_{n \le 0}$ when $k$ is even
  with a possible $q^0$.
  \ii For $-r \le k < 2\delta+r$,
  the first and second sum actually fit together with a ``seam'' near $q^r$,
  which is for even $k$ then completed by
  the single $q^0$ contribution from $I_{n \le 0}$ (only when $k$ is even).
  \ii For $2\delta+r \le k \le 2\delta+2r$,
  the same holds with piecing the third and fourth sum together
  (where the seam is near $q^k$ this time).
  \ii Finally, only the fourth sum contributes for $k \ge 2\delta+2r$,
  and it is of the desired form.
\end{itemize}
This gives us a succinct description of the weighted orbital integral.

If $\ell$ is even, we now have the following intermediate result.
\begin{proposition}
  Suppose $\ell$ is even.
  Then we have the intermediate result
  \[
    I_{n \le 0} + I_{n > 0}^{\text{1+2}}
    = \sum_{k = -2r}^{2\delta + \min(\lambda+2r, \ell+3r)}
    (-1)^k \left( 1 + q + q^2 + \dots + q^{\nn^{1+2}_{\gamma}(k)}  \right) (q^s)^k
  \]
  where the piecewise function $\nn_\gamma \colon \ZZ \to \ZZ_{\ge 0}$ is defined by
  \[
    \nn^{1+2}_\gamma(k) =
    \begin{cases}
      k + 2r & \text{if } {-2r} \le k \le -r \\
      \left\lfloor \frac{k+r}{2} \right\rfloor + r & \text{if }{-r} \le k \le \ell-r \\
      \frac{\ell}{2} + r - (k-r)\%2 & \text{if } \ell - r \le k \le 2\delta + r \\
      \left\lfloor \frac{(2\delta+\ell+r)-k}{2} \right\rfloor + r & \text{if } 2\delta + r \le k \le 2\delta + \ell + 3r
    \end{cases}
  \]
  in the case $\lambda \ge \ell+r$, and
  \[
    \nn^{1+2}_\gamma(k) =
    \begin{cases}
      k + 2r & \text{if } {-2r} \le k \le -r \\
      \left\lfloor \frac{k+r}{2} \right\rfloor + r & \text{if }{-r} \le k \le \ell-r \\
      \frac{\ell}{2} + r - (k-r) \% 2 & \text{if } \ell - r \le k \le 2\delta + r \\
      \left\lfloor \frac{(2\delta+\ell+r)-k}{2} \right\rfloor + r & \text{if } 2\delta + r \le k \le 2\delta + 2\lambda - \ell + r \\
      (2\delta + \lambda + 2r) - k & \text{if } 2 \delta + 2 \lambda - \ell + r \le k \le 2 \delta + \lambda + 2r.
    \end{cases}
  \]
  in the case $\lambda \le \ell+r$.
\end{proposition}

We describe the final result for odd $\ell$ in the next subsection.

\section{Arches}
We introduce one more piece of notation to compress
the particular shape our formulas are about to take.

\begin{definition}
  Suppose $\{a_0, a_0 + 1, \dots, a_1\}$ is an interval of integers for some $a_0 \le a_1$,
  and consider two more integers $w_1$ and $w_2$ such that $w_1 + w_2 \le \frac{a_1-a_0}{2}$.
  Then we can define a piecewise linear function
  \[ \Arch_{[a_0, a_1]}(w_1, w_2) \colon \{a_0, a_0+1, \dots, a_1\} \to \ZZ_{\ge 0} \]
  according to the following definition:
  \[
    k \mapsto
    \begin{cases}
      k - a_0 & \text{if }a_0 \le k \le a_0 + w_1 \\
      w_1 + \left\lfloor \frac{k-(a_0+w_1)}{2} \right\rfloor & \text{if } a_0 + w_1 \le k \le a_0 + w_1 + w_2 \\
      w_1 + \left\lfloor \frac{w_2}{2} \right\rfloor & \text{if } a_0 + w_1 + w_2 \le k \le a_1 - (w_1 + w_2)\\
      w_1 + \left\lfloor \frac{(a_1-w_1) - k}{2} \right\rfloor & \text{if } a_1 - (w_1 + w_2) \le k \le a_1 - w_1 \\
      a_1 - k & \text{if }a_1 - w_1 \le k \le a_1.
    \end{cases}
  \]
\end{definition}
The nomenclature is meant to be indicative of the shape of the graph,
which looks a little bit like an arch.
It is a function symmetric around $\frac{a_0+a_1}{2}$ defined piecewise.
The function grows linearly with slope $1$ at the far left for $w_1$ steps,
then changes to slope $1/2$ for $w_2$ steps (rounding down),
before stabilizing, then doing the symmetric descent on the right half.
\begin{figure}[ht]
  \begin{center}
  \begin{asy}
    size(12cm);
    draw((-1,0)--(20,0));
    draw((0,0)--(3,3)--(7,5)--(12,5)--(16,3)--(19,0), lightred);
    draw((3,3)--(3,0), grey);
    draw((7,5)--(7,0), grey);
    draw((12,5)--(12,0), grey);
    draw((16,3)--(16,0), grey);
    real eps = 0.3;
    void brack(string s, real x0, real x1) {
      draw((x0+0.1,-eps)--(x0+0.1,-2*eps)--(x1-0.1,-2*eps)--(x1-0.1,-eps), blue);
      label(s, ((x0+x1)/2, -2*eps), dir(-90), blue);
    }
    brack("$w_1 = 3$", 0, 3);
    brack("$w_2 = 4$", 3, 7);
    brack("$w_2 = 4$", 12, 16);
    brack("$w_1 = 3$", 16, 19);

    dotfactor *= 1.5;
    dot("$(0,0)$", (0,0), dir(225));
    dot((1,1), red);
    dot((2,2), red);
    dot("$(3,3)$", (3,3), dir(135));
    dot((4,3), red);
    dot((5,4), red);
    dot((6,4), red);
    dot("$(7,5)$", (7,5), dir(90));
    dot((8,5), red);
    dot((9,5), red);
    dot((10,5), red);
    dot((11,5), red);
    dot("$(12,5)$", (12,5), dir(90));
    dot((13,4), red);
    dot((14,4), red);
    dot((15,3), red);
    dot("$(16,3)$", (16,3), dir(45));
    dot((17,2), red);
    dot((18,1), red);
    dot("$(19,0)$", (19,0), dir(315));

    label(rotate(45)*"Slope $+1$", (1.5,1.5), dir(135), lightred);
    label(rotate(26.57)*"Slope $+\frac12$", (5,4), dir(125), lightred);
    label(rotate(-26.57)*"Slope $-\frac12$", (14,4), dir(55), lightred);
    label(rotate(-45)*"Slope $-1$", (17.5,1.5), dir(45), lightred);
    label("Slope $0$", (9.5,5), dir(90), lightred);
  \end{asy}
  \end{center}
  \caption{A plot of $\Arch_{[0,19]}(3,4)$.}
  \label{fig:arch}
\end{figure}

\section{Full explicit weighted orbital integral for the case where $\ell$ odd}
If $\ell$ is odd, so $\lambda = \ell$, then
the weighted orbital integral can be expressed succinctly in the following way.
\begin{theorem}
  [Weighted orbital integral in the odd case for $\mathbf{1}_{K'_{S, \le r}}$]
  \label{thm:full_orbital_ell_odd}
  Let the representative
  \[ \gamma = \begin{bmatrix}
      a & 0 & 0 \\
      b & - \bar d & 1 \\
      -a \bar b + b d & 1 - d \bar d & d
    \end{bmatrix} \in S_3(F)\rs \]
  be paired to an element in $\U(\VV_n^-)$.
  Assume $v(b) = v(d) = 0$.
  Let $\delta = v(1 - d \bar d)$
  and $\ell = v(b^2 - 4 a \bar d)$ be the parameters defined earlier.
  Assume further that $\ell$ is odd, and define
  \[ \nn_\gamma \coloneqq \Arch_{[-2r, 2\delta + \ell + 2r]}(r, \ell). \]
  Then for any $r \ge 0$ we have the formula:
  \[
    \Orb(\gamma, \mathbf{1}_{K'_{S, \le r}}, s)
    = \sum_{k = -2r}^{2\delta + \ell + 2r}
    (-1)^k \left( 1 + q + q^2 + \dots + q^{\nn_{\gamma}(k)}  \right) (q^s)^k
  \]
\end{theorem}
\begin{remark}
  To make $\nn_\gamma$ fully explicit, one could also expand the arch shorthand to
  \[
    \nn_\gamma(k)
    \coloneqq \begin{cases}
      k + 2r & \text{if } {-2r} \le k \le -r \\
      \left\lfloor \frac{k+r}{2} \right\rfloor + r & \text{if }{-r} \le k \le \ell-r \\
      \frac{\ell-1}{2} + r & \text{if } \ell - r \le k \le 2\delta + r \\
      \left\lfloor \frac{(2\delta+\ell+r)-k}{2} \right\rfloor + r & \text{if } 2\delta + r \le k \le 2\delta + \ell + r\\
      (2\delta + \ell + 2r) - k & \text{if } 2\delta + \ell + r \le k \le 2\delta + \ell + 2r.
    \end{cases} \]
\end{remark}

From the identity
\[
  \Orb(\gamma, \mathbf{1}_{K'_{S, r}}, s)
  = \Orb(\gamma, \mathbf{1}_{K'_{S, \le r}}, s)
  - \Orb(\gamma, \mathbf{1}_{K'_{S, \le (r-1)}}, s)
\]
we can then write the following equivalent formulation.
\begin{corollary}
  [Weighted orbital integral in the odd case for $\mathbf{1}_{K'_{S,r}}$]
  Retaining the setting of the previous theorem, we have for any $r \ge 1$ the formula
  \[
    \Orb(\gamma, \mathbf{1}_{K'_{S, r}}, s)
    = \sum_{k = -2r}^{2\delta + \ell + 2r}
    (-1)^k q^{\nn_{\gamma}(k)}
    (1+q^{-1})^{\mathbf{1}[k \in \mathcal I_{\gamma, r}]}
    (q^s)^k
  \]
  where $\mathcal I_{\gamma, r}$ is the set of indices defined by
  \begin{align*}
    \mathcal{I}_{\gamma, r}
    &\coloneqq \left\{ -(2r-1), -(2r-2), -(2r-3), \dots, -(r+1) \right\} \\
    &\sqcup \{-r, -r+2, -r+4 \dots, -r+\ell-1 \} \\
    &\sqcup \{ 2\delta+r+1, 2\delta+r+3, \dots, 2\delta+r+1, 2\delta+r+3, \dots, 2\delta+\ell+r \} \\
    &\sqcup \{ 2\delta+\ell+r+1, 2\delta+\ell+r+2, \dots, 2\delta+\ell+2r-1 \}.
  \end{align*}
\end{corollary}

\begin{example}
  If $r=3$, $\ell=5$, and $\delta=100$, the formulas above read
  \begin{align*}
  \Orb(\gamma, \mathbf{1}_{K'_{S, \le 3}}, s)
  &= q^{-6s} \\
  &- (q+1) \cdot q^{-5s} \\
  &+ (q^2+q+1) \cdot q^{-4s} \\
  &- (q^3+q^2+q+1) \cdot q^{-3s} \\
  &+ (q^3+q^2+q+1) \cdot q^{-2s} \\
  &- (q^4+q^3+q^2+q+1) \cdot q^{-s} \\
  &+ (q^4+q^3+q^2+q+1) \cdot q^{0} \\
  &- (q^5+q^4+q^3+q^2+q+1) \cdot q^{s} \\
  &+ (q^5+q^4+q^3+q^2+q+1) \cdot q^{2s} \\
  &\vdotswithin= \\
  &+ (q^5+q^4+q^3+q^2+q+1) \cdot q^{204s} \\
  &- (q^4+q^3+q^2+q+1) \cdot q^{205s} \\
  &+ (q^4+q^3+q^2+q+1) \cdot q^{206s} \\
  &- (q^3+q^2+q+1) \cdot q^{207s} \\
  &+ (q^3+q^2+q+1) \cdot q^{208s} \\
  &- (q^2+q+1) \cdot q^{209s} \\
  &+ (q+1) \cdot q^{210s} \\
  &- q^{211s}.
  \end{align*}
  In the ellipses, all the omitted terms
  have the same coefficient $q^5+q^4+q^3+q^2+q+1$
  and alternate sign.
\end{example}

\begin{example}
  Continuing the previous example with
  $r=3$, $\ell=5$, and $\delta=100$, we have
  \begin{align*}
  \Orb(\gamma, \mathbf{1}_{K'_{S, 3}}, s)
  &= q^{-6s} \\
  &- (q+1) \cdot q^{-5s} \\
  &+ (q^2+q) \cdot q^{-4s} \\
  &- (q^3+q^2) \cdot q^{-3s} \\
  &+ q^3 \cdot q^{-2s} \\
  &- (q^4+q^3) \cdot q^{-s} \\
  &+ q^4 \cdot q^{0} \\
  &- (q^5+q^4) \cdot q^{s} \\
  &+ q^5 \cdot q^{2s} \\
  &- q^5 \cdot q^{3s} \\
  &+ q^5 \cdot q^{4s} \\
  &\vdotswithin= \\
  &+ q^5 \cdot q^{202s} \\
  &- q^6 \cdot q^{203s} \\
  &+ (q^5+q^4) \cdot q^{204s} \\
  &- q^4 \cdot q^{205s} \\
  &+ (q^4+q^3) \cdot q^{206s} \\
  &- q^3 \cdot q^{207s} \\
  &+ (q^3+q^2) \cdot q^{208s} \\
  &- (q^2+q) \cdot q^{209s} \\
  &+ (q+1) \cdot q^{210s} \\
  &- q^{211s}.
  \end{align*}
\end{example}

\section{Contribution from Case 3\ts{+}, 3\ts{-}, 4\ts{+} assuming $v(b) \ge 0$ and $v(d) \ge 0$}
These cases only appear when $\ell$ is even and we assume this for this subsection.
We consider the contribution of these cases within
$\kappa \int_{t, t_1 \in E} \mathbf{1}[n > 0] \mathbf{1}_{\le r}(\gamma,t,m)$
(using \eqref{eq:even_case3_plus}, \eqref{eq:even_case3_minus}, \eqref{eq:even_case4_plus})
and put $\Vol(t_1:-v(t_1)=m) = q^{2m} (1-q^{-2})$ to get:
\begin{align*}
  I_{n > 0}^{\text{3+4}}
  % -----------------------------------------------------------
  &\coloneqq \kappa \sum_{n=\ell+r+1}^{-\frac{\ell}{2}+\delta+\lambda+r}
    \sum_{m=n-r}^{\min(n-1, \lambda+r)-\frac{\ell}{2}+\delta}
    q^{-n-(n-\frac{\ell}{2}-r)} \left( 1 - q^{-1} \right) \\
    &\qquad\qquad\cdot \Big( (-1)^n q^{s(2m-n)} q^{2n-2m} \Big) \Big( q^{2m}(1-q^{-2}) \Big) \\
  &\qquad+ \kappa \sum_{n=\ell+r+1}^{\frac{\ell}{2}+\delta+r}
    \sum_{m=n-r}^{\frac{\ell}{2}+\delta+r}
    q^{-n-(n-\frac{\ell}{2}-r)} \left( 1 - q^{-1} \right) \\
  &\qquad\qquad\cdot \Big( (-1)^n q^{s(2m-n)} q^{2n-2m} \Big) \Big( q^{2m}(1-q^{-2}) \Big) \\
  &\qquad+ \kappa \sum_{n=\ell+r+1}^{\lambda+r}
    \sum_{m=n-\frac{\ell}{2}+\delta}^{\min(n,\lambda)+\delta+r}
    q^{-n-(m-\delta-r)} \left( 1 - q^{-1} \right) \\
    &\qquad\qquad\cdot \Big( (-1)^n q^{s(2m-n)} q^{2n-2m} \Big) \Big( q^{2m}(1-q^{-2}) \Big) \\
  &= \sum_{n=\ell+r+1}^{-\frac{\ell}{2}+\delta+\lambda+r}
    \sum_{m=n-r}^{\min(n-1, \lambda+r)-\frac{\ell}{2}+\delta}
    q^{-n-(n-\frac{\ell}{2}-r)}
      \cdot \Big( (-1)^n q^{s(2m-n)} q^{2n} \Big) \\
  &\qquad+ \sum_{n=\ell+r+1}^{\frac{\ell}{2}+\delta+r}
    \sum_{m=n-r}^{\frac{\ell}{2}+\delta+r}
    q^{-n-(n-\frac{\ell}{2}-r)}
      \cdot \Big( (-1)^n q^{s(2m-n)} q^{2n} \Big) \\
  &\qquad+ \sum_{n=\ell+r+1}^{\lambda+r}
    \sum_{m=n-\frac{\ell}{2}+\delta}^{\min(n,\lambda)+\delta+r}
    q^{-n-(m-\delta-r)}
      \cdot \Big( (-1)^n q^{s(2m-n)} q^{2n} \Big) \\
  &= \sum_{n=\ell+r+1}^{-\frac{\ell}{2}+\delta+\lambda+r}
    \sum_{m=n-r}^{\min(n-1, \lambda+r)-\frac{\ell}{2}+\delta}
    q^{\frac{\ell}{2}+r} \cdot (-1)^n q^{s(2m-n)} \\
  &\qquad+ \sum_{n=\ell+r+1}^{\frac{\ell}{2}+\delta+r}
    \sum_{m=n-r}^{\frac{\ell}{2}+\delta+r}
    q^{\frac{\ell}{2}+r} \cdot (-1)^n q^{s(2m-n)} \\
  &\qquad+ \sum_{n=\ell+r+1}^{\lambda+r}
    \sum_{m=n-\frac{\ell}{2}+\delta}^{\min(n,\lambda)+\delta+r}
    q^{n-m+\delta+r} \cdot (-1)^n q^{s(2m-n)} \\
\end{align*}

\todo{ok this collation is going to be\dots fun\dots}

\section{Full explicit weighted orbital integral for the case where $\ell$ is even and $v(b) \ge 0$, $v(d) \ge 0$}
\begin{theorem}
  [Weighted orbital integral in the even case for $\mathbf{1}_{K'_{S, \le r}}$]
  \label{thm:full_orbital_ell_even}
  Let the representative
  \[ \gamma = \begin{bmatrix}
      a & 0 & 0 \\
      b & - \bar d & 1 \\
      -a \bar b + b d & 1 - d \bar d & d
    \end{bmatrix} \in S_3(F)\rs \]
  be paired to an element in $\U(\VV_n^-)$.
  Assume further $v(b) \ge 0$ and $v(d) \ge 0$.
  Let $\delta = v(1 - d \bar d)$, $\ell = v(b^2 - 4 a \bar d)$ and
  $\lambda = v((1 - d \bar d)^2 - c \bar c) - 2\delta$
  (where $c = -a \bar b + b d $) be the parameters defined earlier.
  (Necessarily $\lambda \equiv 1 \pmod 2$.)
  Assume further $\ell$ is even, and define
  \begin{align*}
    \nn_\gamma &\coloneqq \Arch_{[-2r, 2\delta + \lambda + 2r]}(r, \ell) \\
    \cc_\gamma &\coloneqq \Arch_{[\ell-r, 2\delta + \lambda - \ell + r]}
    (\delta - \ell/2, \min(\lambda-\ell-1, 2r)).
  \end{align*}
  Then for any $r \ge 0$ we have the formula:
  \begin{align*}
    \Orb(\gamma, \mathbf{1}_{K'_{S, \le r}}, s)
    &= \sum_{k = -2r}^{2\delta + \lambda + 2r}
    (-1)^k \left( 1 + q + q^2 + \dots + q^{\nn_{\gamma}(k)}  \right) (q^s)^k \\
    &+ \sum_{k = \ell-r}^{2\delta + \lambda - \ell + r} \cc_\gamma(k) (-1)^k q^{\frac{\ell}{2}+r} (q^s)^k.
  \end{align*}
\end{theorem}
\begin{remark}
  One can expand the Arch notation for $\nn_\gamma$ to obtain
  \[
    \nn_\gamma(k) =
    \begin{cases}
      k + 2r & \text{if } {-2r} \le k \le -r \\
      \left\lfloor \frac{k+r}{2} \right\rfloor + r & \text{if }{-r} \le k \le \ell-r \\
      \frac{\ell}{2} + r & \text{if } \ell - r \le k \le 2\delta + \lambda - \ell + r \\
      \left\lfloor \frac{(2\delta+\lambda+r)-k}{2} \right\rfloor + r & \text{if } 2\delta + \lambda - \ell + r \le k \le 2\delta + \lambda + r \\
      (2\delta + \lambda + 2r) - k & \text{if } 2\delta + \lambda + r \le k \le 2\delta + \lambda + 2r.
    \end{cases}
  \]
  Then $\cc_\gamma$ can be similarly expanded, but the result is so notationally dense
  that it is hardly worth including.
  If one defines the shorthands
  \begin{align*}
    \mathsf{B}_\gamma &\coloneqq \delta + \frac{\ell}{2} - r \\
    \mathsf{T}_\gamma &\coloneqq (2\delta+\lambda) - \mathsf{B}_\gamma \\
    \mathsf{w}_\gamma &\coloneqq \min(\lambda-\ell-1, 2r)
  \end{align*}
  then it could be written out more fully as
  \[
    \cc_\gamma(k)
    =
    \begin{cases}
      k - (\ell - r)
        & \text{if } \ell-r \le k \le \mathsf{B}_\gamma \\
      \left\lfloor \frac{k - \mathsf{B}_\gamma}{2} \right\rfloor + \delta - \frac{\ell}{2}
        &\text{if } \mathsf{B}_\gamma \le k \le \mathsf{B}_\gamma + \mathsf{w}_\gamma \\
      \delta - \frac{\ell}{2} + \half \mathsf{w}_\gamma
        &\text{if } \mathsf{B}_\gamma + \mathsf{w}_\gamma \le k \le \mathsf{T}_\gamma - \mathsf{w}_\gamma \\
      \left\lfloor \frac{\mathsf{T}_\gamma - k}{2} \right\rfloor + \delta - \frac{\ell}{2}
        &\text{if } \mathsf{T}_\gamma - \mathsf{w}_\gamma \le k \le \mathsf{T}_\gamma \\
        (2\delta + \lambda - \ell + r) - k
        & \text{if } \mathsf{T}_\gamma \le k \le 2\delta + \lambda - \ell + r.
    \end{cases}
  \]
\end{remark}

\begin{example}
  We provide an example for $r=5$, $\ell=2$, $\delta=4$, $\lambda=9$ for concreteness:
  \begin{align*}
    \Orb(\gamma, \mathbf{1}_{K'_{S, \le r}}, s)
    &= q^{-10s} \\
    &- (q + 1) \cdot q^{-9s} \\
    &+ (q^2 + q + 1) \cdot q^{-8s} \\
    &- (q^3 + q^2 + q + 1) \cdot q^{-7s} \\
    &+ (q^4 + q^3 + q^2 + q + 1) \cdot q^{-6s} \\
    &- (q^5 + q^4 + q^3 + q^2 + q + 1) \cdot q^{-5s} \\
    &+ (q^5 + q^4 + q^3 + q^2 + q + 1) \cdot q^{-4s} \\
    &- (q^6 + q^5 + q^4 + q^3 + q^2 + q + 1) \cdot q^{-3s} \\
    &+ (2q^6 + q^5 + q^4 + q^3 + q^2 + q + 1) \cdot q^{-2s} \\
    &- (3q^6 + q^5 + q^4 + q^3 + q^2 + q + 1) \cdot q^{-s} \\
    &+ (4q^6 + q^5 + q^4 + q^3 + q^2 + q + 1) \cdot q^{0} \\
    &- (4q^6 + q^5 + q^4 + q^3 + q^2 + q + 1) \cdot q^{s} \\
    &+ (5q^6 + q^5 + q^4 + q^3 + q^2 + q + 1) \cdot q^{2s} \\
    &- (5q^6 + q^5 + q^4 + q^3 + q^2 + q + 1) \cdot q^{3s} \\
    &+ (6q^6 + q^5 + q^4 + q^3 + q^2 + q + 1) \cdot q^{4s} \\
    &- (6q^6 + q^5 + q^4 + q^3 + q^2 + q + 1) \cdot q^{5s} \\
    &+ (7q^6 + q^5 + q^4 + q^3 + q^2 + q + 1) \cdot q^{6s} \\
    &- (7q^6 + q^5 + q^4 + q^3 + q^2 + q + 1) \cdot q^{7s} \\
    &+ (7q^6 + q^5 + q^4 + q^3 + q^2 + q + 1) \cdot q^{8s} \\
    &- (7q^6 + q^5 + q^4 + q^3 + q^2 + q + 1) \cdot q^{9s} \\
    &+ (7q^6 + q^5 + q^4 + q^3 + q^2 + q + 1) \cdot q^{10s} \\
    &- (7q^6 + q^5 + q^4 + q^3 + q^2 + q + 1) \cdot q^{11s} \\
    &+ (6q^6 + q^5 + q^4 + q^3 + q^2 + q + 1) \cdot q^{12s} \\
    &- (6q^6 + q^5 + q^4 + q^3 + q^2 + q + 1) \cdot q^{13s} \\
    &+ (5q^6 + q^5 + q^4 + q^3 + q^2 + q + 1) \cdot q^{14s} \\
    &- (5q^6 + q^5 + q^4 + q^3 + q^2 + q + 1) \cdot q^{15s} \\
    &+ (4q^6 + q^5 + q^4 + q^3 + q^2 + q + 1) \cdot q^{16s} \\
    &- (4q^6 + q^5 + q^4 + q^3 + q^2 + q + 1) \cdot q^{17s} \\
    &+ (3q^6 + q^5 + q^4 + q^3 + q^2 + q + 1) \cdot q^{18s} \\
    &- (2q^6 + q^5 + q^4 + q^3 + q^2 + q + 1) \cdot q^{19s} \\
    &+ (q^6 + q^5 + q^4 + q^3 + q^2 + q + 1) \cdot q^{20s} \\
    &- (q^5 + q^4 + q^3 + q^2 + q + 1) \cdot q^{21s} \\
    &+ (q^5 + q^4 + q^3 + q^2 + q + 1) \cdot q^{22s} \\
    &- (q^4 + q^3 + q^2 + q + 1) \cdot q^{23s} \\
    &+ (q^3 + q^2 + q + 1) \cdot q^{24s} \\
    &- (q^2 + q + 1) \cdot q^{25s} \\
    &+ (q + 1) \cdot q^{26s} \\
    &- q^{27s}.
  \end{align*}
\end{example}

\section{Full explicit weighted orbital integral for the case where $\ell = \delta = 2v(b) = 2v(d) < 0$}
\begin{theorem}
  \label{thm:full_orbital_ell_neg}
  Let the representative
  \[ \gamma = \begin{bmatrix}
      a & 0 & 0 \\
      b & - \bar d & 1 \\
      -a \bar b + b d & 1 - d \bar d & d
    \end{bmatrix} \in S_3(F)\rs \]
  be paired to an element in $\U(\VV_n^-)$.
  Assume that $v(b) = v(d) < 0$ and let
  $\lambda = v((1 - d \bar d)^2 - c \bar c) + 2|v(d)|$
  which is necessarily odd.

  If $|v(d)| > r$, the entire orbital integral is zero.

  Otherwise define
  \begin{align*}
    \nn_\gamma &\coloneqq \Arch_{[-2r, \lambda + 2(r-|v(d)|)]}(r-|v(d)|, 0) \\
    \cc_\gamma &\coloneqq \Arch_{[-r-|v(d)|, \lambda + r-|v(d)|]} (0, 2(r-|v(d)|)).
  \end{align*}
  Then for any $r \ge 0$ we have the formula:
  \begin{align*}
    \Orb(\gamma, \mathbf{1}_{K'_{S, \le r}}, s)
    &= \sum_{k = -2r}^{\lambda + 2(r-|v(d)|)}
    (-1)^k \left( 1 + q + q^2 + \dots + q^{\nn_{\gamma}(k)} \right) (q^s)^k \\
    &= \sum_{k = -r-|v(d)|}^{\lambda + r - |v(d)|} \cc_\gamma(k) (-1)^k q^{r-|v(d)|} (q^s)^k \\
  \end{align*}
\end{theorem}
