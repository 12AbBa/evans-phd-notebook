\chapter{Regular semi-simplicity and matching}
\label{ch:rs_matching}

\section{Regular semi-simple elements}
We first recall the notion of regularity
that first appeared in \cite[\S6]{ref:multoneconj}.

\begin{definition}
  \label{def:rs}
  Consider a $n \times n$ matrix
  \[ \begin{bmatrix} A & \uu \\ \vv^\top & d \end{bmatrix} \in \GL_n(E) \]
  in $\GL_n(E)$, where $A$ is an $(n-1) \times (n-1)$ matrix.
  Then we say this matrix is \emph{regular semi-simple} if
  \[ \left< \uu, A\uu, \dots, A^{n-2}\uu \right> \]
  and \[ \left< \vv^\top, \vv^\top A, \dots, \vv^\top A^{n-2} \right> \]
  are each a basis of $E^{n-1}$.
  Equivalently, the matrix
  \[ \left[ \vv^\top A^{i+j-2} \uu \right]_{i,j=1}^{n-1} \]
  should be nonsingular.
\end{definition}

\begin{remark}
  In \cite[Theorem 6.1]{ref:multoneconj}, this definition is shown to be equivalent to
  requiring that, under the action of conjugation by $\GL_{n-1}(E)$:
  \begin{itemize}
  \ii the matrix has trivial stabilizer; and
  \ii the $\GL_{n-1}(\ol E)$-orbit is a Zariski-closed subset of $\GL_n(\ol E)$.
  \end{itemize}
  Here $\ol E$ is as usual an algebraic closure of $E$.
\end{remark}

\begin{remark}
  [{\cite[Proposition 6.2]{ref:multoneconj}}]
  It turns out we can detect whether two regular semisimple elements
  \[
    \begin{bmatrix} A_1 & \uu_1 \\ \vv_1^\top & d_1 \end{bmatrix},
    \begin{bmatrix} A_2 & \uu_2 \\ \vv_2^\top & d_2 \end{bmatrix}
    \in \GL_{n}(E)
  \]
  are conjugate by an element of $\GL_{n-1}(E)$.
  This happens if and only if the following conditions all hold:
  \begin{itemize}
    \ii The matrices $A_1$ and $A_2$ have the same characteristic polynomial;
    \ii We have $\vv_1^\top A_1^i \uu_1 = \vv_2^\top A_2^i \uu_2$
    for every $i = 0, 1, \dots, n-2$; and
    \ii We have $d_1 = d_2$.
  \end{itemize}
  Thus, this gives a set of invariants that completely classify the orbits
  under the action of $\GL_{n-1}(E)$.

  Put another way, the invariants of
  \[ \begin{bmatrix} A & \uu \\ \vv^\top & d \end{bmatrix} \in \GL_n(E) \]
  are the (monic) characteristic polynomial of $A$
  (which has $n-1$ coefficients besides the leading coefficient),
  the values of $\vv^\top A^i \uu$ for $i = 0, \dots, n-2$
  and the number $d$ --- for a total of $2n-1$ numbers.
  \label{rem:invariants}
\end{remark}

We can now speak of regular-simplicity in each of the four
particular cases relevant to this paper.
\begin{definition}
  In the group version of the AFL:
  \begin{itemize}
    \ii We say $\gamma \in S_n(F)$ is regular semisimple
    if its image under the inclusion $S_n(F) \subseteq \GL_n(E)$ is regular semisimple.
    We write $\gamma \in S_n(F)\rs$.

    \ii For either $g \in G \coloneqq \U(V_0)$ (resp.\ $g \in \U(\VV_n)$),
    we say $g$ is regular semisimple
    if its image under the inclusion $\U(\VV_n) \subseteq \GL_n(E)$ is regular semisimple.
    We write $g \in \U(V_0)\rs$ (resp.\ $g \in \U(\VV_n)\rs$).
  \end{itemize}
  In the semi-Lie version of the AFL:
  \begin{itemize}
    \ii We say $(\gamma, \uu, \vv^\top) \in S_n(F) \times V_n'$
    is regular semisimple if its image under the embedding
    \begin{equation}
      \begin{aligned}
        S_n(F) \times V_n' &\hookrightarrow \GL_{n+1}(E) \\
        (\gamma, \uu, \vv^\top) &\mapsto \begin{bmatrix} \gamma & \uu \\ \vv^\top & 0 \end{bmatrix}
      \end{aligned}
      \label{eq:embed_FJ_analytic}
    \end{equation}
    is regular semisimple.
    In other words, we require that
    both of the sets
    $\left( \uu, \gamma \uu \dots, \gamma^{n-1}\uu \right)$
    and
    $\left( \vv^\top, \vv^\top \gamma, \dots, \vv^\top \gamma^{n-1} \right)$
    are bases of $E^n$.
    In this case we write $(\gamma, \uu, \vv^\top) \in (S_n(F) \times V_n')\rs$.

    \ii For either $(g,u) \in G \times V_0 \coloneqq \U(V_0) \times V_0$
    (resp.\ $(g, u) \in \U(\VV_n) \times \VV_n$),
    we say $(g, u)$
    is regular semisimple if its image under the embedding
    \begin{equation}
      \begin{aligned}
        \U(\VV_n) \times \VV_n &\hookrightarrow \GL_{n+1}(E) \\
        (\gamma, u, u^\ast) &\mapsto \begin{bmatrix} \gamma & u \\ u^\ast & 0 \end{bmatrix}
      \end{aligned}
      \label{eq:embed_FJ_geometric}
    \end{equation}
    is regular semisimple; here $u^\ast$ is the conjugate transpose.
    \todo{check with Wei}
    This is equivalent to the set $\left(  u, gu, \dots, g^{n-1}u \right)$
    being linearly independent (i.e.\ form a basis of $\VV_n$);
    in this case the independence of $\left( u^\ast, u^\ast g, \dots, u^\ast g^{n-1} \right)$
    is redundant, so it's enough to verify one condition.
    We write $(g,u) \in (\U(V_0) \times V_0)$ (resp.\ $(g,u) \in (\U(\VV_n) \times \VV_n)\rs$).
  \end{itemize}
  \label{def:regular}
\end{definition}

\section{Matching in the group version of the inhomogeneous AFL}
We now describe the matching condition used in the group version of AFL.
\begin{definition}
  We say $\gamma \in S_n(F)\rs$ matches the element $g \in \U(\VV_n)\rs$
  if $g$ is conjugate to $\gamma$ by an element of $\GL_{n-1}(E)$.
  By \Cref{rem:invariants}, this is an assertion that
  the invariants for $\gamma$ and $g$ coincide.
  The same definition applies verbatim for $V_0$ in place of $\VV_n$.
  \label{def:matching_inhomog}
\end{definition}
In that case, we have the following result.
\begin{proposition}
  [{\cite[Lemma 2.3]{ref:AFL}}; see also {\cite[(3.3.2)]{ref:AFLspherical}}]
  \label{prop:valuation_delta_matching_group}
  This definition of matching gives
  a bijection of regular semisimple orbits
  \[ [S_n(F)]\rs \xrightarrow{\simeq} [\U(V_0)]\rs \amalg [\U(\VV_n)]\rs. \]
  Moreover, we can detect whether $\gamma \in S_n(F)\rs$ matches an orbit of
  $\U(V_0)$ or $\U(\VV_n)$ as follows.
  Suppose we write $\gamma$ in the format of \Cref{def:rs} and consider
  \[ \Delta \coloneqq \left[ \vv^\top A^{i+j} \uu \right]_{i,j=1}^{n-1} \neq 0. \]
  Then
  \begin{itemize}
    \ii $\gamma$ matches an orbit in $\U(V_0)\rs$ if $v(\Delta)$ is even
    \ii $\gamma$ matches an orbit in $\U(\VV_n)\rs$ if $v(\Delta)$ is odd.
  \end{itemize}
\end{proposition}
In this paper, \Cref{conj:inhomog}
specifies that $\gamma$ should match an element of $\U(\VV_n)$
and consequently we will usually only be interested in the latter case.

\section{Matching in the semi-Lie version of the AFL}
For the semi-Lie version matching is defined analogously:
\begin{definition}
  [{\cite[\S1.3]{ref:liuFJ}}]
  We say $(\gamma, \uu, \vv^\top) \in (S_n(F) \times V_n')\rs$
  matches the element $(g, u) \in (\U(\VV_n) \times \VV_n)\rs$ if
  their images under the embeddings \eqref{eq:embed_FJ_analytic}
  and \eqref{eq:embed_FJ_geometric} are conjugate by an element of $\GL_n(E)$.
  Unwrapping this with \Cref{rem:invariants},
  an equivalent definition is to require both of the following conditions:
  \begin{itemize}
    \ii As elements of $\GL_n(E)$,
    both $g$ and $\gamma$ have the same characteristic polynomial.
    \ii We have $\vv^\top \gamma^i \uu = \left< g^i u, u \right>$ for all $0 \le i \le n-1$,
    where $\left< -,- \right>$ is the Hermitian form on $\VV_n$.
  \end{itemize}
  The same definition applies verbatim for $V_0$ in place of $\VV_n$.
  \label{def:matching_semi_lie}
\end{definition}
We have the following analogous criteria for matching.
\begin{proposition}
  [\cite{ref:liuFJ}]
  \label{prop:valuation_delta_matching_semilie}
  This definition of matching gives
  a bijection of regular semisimple orbits
  \[ [S_n(F) \times V_n']\rs \xrightarrow{\simeq} [\U(V_0) \times V_0]\rs \amalg [\U(\VV_n) \times \VV_n]\rs. \]
  Moreover, we can detect whether $\guv \in S_n(F)\rs$ matches an orbit of
  $\U(V_0)$ or $\U(\VV_n)$ as follows: consider the determinant
  \[ \Delta \coloneqq \det \left[ \vv^\top \gamma^{i+j-2} \right]_{i,j=1}^n \neq 0. \]
  Then
  \begin{itemize}
    \ii $\gamma$ matches an orbit in $(\U(V_0) \times V_0)\rs$ if $v(\Delta)$ is even
    \ii $\gamma$ matches an orbit in $(\U(\VV_n) \times \VV_n)\rs$ if $v(\Delta)$ is odd.
  \end{itemize}
\end{proposition}
In this paper, \Cref{conj:semi_lie_spherical}
specifies that $\guv$ should match an element of $\U(\VV_n) \times \VV_n$
and consequently we will usually only be interested in the latter case.

\section{The split and non-split cases}
\todo{Ask Wei what's up here}
