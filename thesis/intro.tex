\section{Introduction}
Throughout this whole paper, $p > 2$ is a prime,
$F$ is a finite extension of $\QQ_p$,
and $E/F$ is an unramified quadratic field extension.

\subsection{Brief history and motivation for the arithmetic fundamental lemma}
The primary motivation for this paper arises from
the study of conjectured variants of the arithmetic fundamental lemma
for spherical Hecke algebras proposed in \cite{ref:AFLspherical}.
This section briefly provides an overview of the historical context
that led to the formulation of these conjectures.
This history is also summarized in \Cref{fig:history}.

Because this subsection is meant for motivation only, in this survey we do not give
complete definitions or statements, being content to outline a brief gist.
A more detailed account can be found in \cite{ref:survey}.

\begin{figure}[ht]
  \centering
  \begin{tikzcd}
      \text{\cite{ref:waldspurger}} \ar[d] \\
      \text{\cite{ref:GP1,ref:GP2}} \ar[d] \\
    \text{GGP \cite{ref:GGP}}
      \ar[d, dotted, leftrightarrow, "\text{analog}"]
      & \ar[l, Rightarrow, "\text{used to prove}"] \text{FL \cite{ref:JR}}
        \ar[d, dotted, leftrightarrow, "\text{analog}"] \ar[r]
      & \text{\cite{ref:leslie}}
        \ar[d, dotted, leftrightarrow, "\text{analog}"] \\
    \text{Arith.\ GGP \cite{ref:GGP}}
      & \ar[l, Rightarrow, "\text{used to prove}"] \text{AFL \cite{ref:AFL}} \ar[r]
      & \text{\cite{ref:AFLspherical}} \\
    \text{\cite{ref:GZshimura}} \ar[u] \\
    \text{\cite{ref:gross_zagier}} \ar[u]
  \end{tikzcd}
  \caption{The history behind the fundamental lemma and its arithmetic counterpart.
    Unlabeled arrows denote generalizations.}
  \label{fig:history}
\end{figure}

\subsubsection{The GGP conjectures, and the fundamental lemma of Jacquet-Rallis}
In modern arithmetic geometry, a common theme is that there are deep connections
between geometric data with the values of related $L$-functions.

This story begins with a result of
Waldspurger \cite{ref:waldspurger} which showed a formula
relating the nonvanishing of an automorphic period integral
to the central value of the same $L$-functions.
Later, a conjecture that generalizes Waldspurger's formula
was proposed by Gross-Prasad in \cite{ref:GP1,ref:GP2}.
This was further generalized to a series of conjectures
now known as the Gan-Gross-Prasad (GGP) conjectures,
which were proposed in 2012 in \cite{ref:GGP};
they generalize the Gross-Prasad conjecture to different classical groups.
Specifically, the GGP conjecture predict the nonvanishing of a period integral
based on the values of the $L$-function of a certain cuspidal automorphic representation.

In 2011, Jacquet-Rallis \cite{ref:JR} proposed an approach to the Gross-Prasad conjectures
for unitary groups via a relative trace formula (RTF).
The idea is to compare a RTF for the general linear group to one for a unitary group.
This approach relies on a so-called \emph{fundamental lemma},
which links values of certain orbital integrals
over two reductive groups over a non-Archimedean local field.

Let's be a bit more precise about what this fundamental lemma says.
Let $V_0$ denote the split $E/F$-Hermitian space of dimension $n$ (unique up to isomorphism),
fix a unit vector $w_0$ in it,
and let $V_0^\flat$ be the orthogonal complement of the span of $w_0$.
Let $G'^\flat \coloneqq \GL_{n-1}(E)$, $G' \coloneqq \GL_n(E)$,
$G^\flat \coloneqq \U(V_0^\flat)(F)$ and $G \coloneqq \U(V_0)(F)$.
For certain
\[ \gamma \in G'^\flat \times G', \qquad g \in G^\flat \times G \]
the Jacquet-Rallis fundamental lemma proposes a relation between two orbital integrals.
Specifically, it supplies a relation between
\begin{itemize}
\item the orbital integral of $\gamma$ with respect to
  the indicator function $\mathbf{1}_{K'^\flat \times K'}$
  of the natural hyperspecial compact subgroup
  \[ K'^\flat \times K' \subset G'^\flat \times G' = \GL_{n-1}(E) \times \GL_n(E); \]
  and
\item the orbital integral of $g$ with respect to
  the indicator function $\mathbf{1}_{K^\flat \times K}$
  of the natural hyperspecial compact subgroup
  \[ K^\flat \times K \subset G^\flat \times G = \U(V_0^\flat)(F) \times \U(V_0)(F). \]
\end{itemize}
In other words, it states that
\begin{equation}
  \Orb(\gamma, \mathbf{1}_{K'^\flat \times K'}) = \omega(\gamma) \Orb(g, \mathbf{1}_{K^\flat \times K})
  \label{eq:old_FL}
\end{equation}
where $\omega(\gamma)$ is a suitable \emph{transfer factor}.
The fundamental lemma has since been proved completely;
a local proof was given by Beuzart-Plessis \cite{ref:BeuzartPlessis}
while a global proof was given for large characteristic by W.\ Zhang \cite{ref:Wei2021}.

\subsubsection{The arithmetic GGP conjectures, and the arithmetic fundamental lemma.}
At around the same time Waldspurger's formula was published,
Gross-Zagier \cite{ref:gross_zagier} proved a formula
relating the height of Heegner points
on certain modular curves to the derivative at $s=1$ of certain $L$-functions.
The Gross-Zagier formula was then generalized over several decades,
culminating in \cite{ref:GZshimura} where the formula is established
for Shimura curves over arbitrary totally real fields.

An arithmetic analogue of the original Gan-Gross-Prasad conjectures,
which we henceforth refer to as \emph{arithmetic GGP} \cite{ref:GGP},
can then be formulated, further generalizing Gross-Zagier's formula.
Here the modular curves in Gross-Zagier
are replaced with higher dimensional Shimura varieties.
Rather than the period integrals considered previously,
one instead takes intersection numbers of cycles on some Shimura varieties.
Specifically, if one considers the Shimura variety associated to a classical group,
the arithmetic GGP conjecture predicts a relation between intersection numbers
on this Simura variety with the central derivative of automorphic $L$-functions.

By analogy to the work Jacquet-Rallis \cite{ref:JR},
the arithmetic GGP conjectures should have a corresponding
\emph{arithmetic fundamental lemma} (henceforth AFL),
which was proposed by W.\ Zhang \cite[Conjecture 2.9]{ref:AFL}.
The arithmetic fundamental lemma then relates the derivative
of the orbital integral with respect to the indicator function
$\mathbf{1}_{K'^\flat \times K'} \in \HH(G'^\flat \times G, K'^\flat \times K')$, that is
\[ \left. \pdv{}{s} \right\rvert_{s=0} \Orb(\gamma, \mathbf{1}_{K'^\flat \times K'}, s) \]
for $\gamma \in G'^\flat \times G'$,
to arithmetic intersection numbers on a certain Rapoport-Zink formal moduli space.
The AFL in \cite{ref:AFL} has since been proven over $p$-adic fields for any prime $p$ in
Mihatsch-Zhang \cite{ref:MZ2021}, W.\ Zhang \cite{ref:Wei2021}, Z.\ Zhang \cite{ref:Zhiyu}.

\subsubsection{The semi-Lie version of the AFL proposed by Liu}
There is another different version of the AFL, proposed by Yifeng Liu in
\cite[Conjecture 1.12]{ref:liuFJ},
which is often referred to as the \emph{semi-Lie version} of the AFL.
Its statement has been shown to be equivalent to AFL
\cite[Remark 1.13]{ref:liuFJ}.
In contrast, the original AFL proposed by Zhang in \cite[Conjecture 2.9]{ref:AFL}
is sometimes referred to as the \emph{group version}.

A more detailed account of this equivalence is described in \cite[\S1.4]{ref:liuFJ}.

\subsubsection{Generalizations of FL and AFL to the full spherical Hecke algebra}
Recently it was shown by Leslie \cite{ref:leslie} that in fact
\eqref{eq:old_FL} holds in greater generality where the indicator function
$\mathbf{1}_{K^\flat \times K}$ can be replaced by any element in the spherical
Hecke algebra $\varphi \in \HH(G'^\flat \times G', K'^\flat \times K')$.
In that case, $\mathbf{1}_{K'^\flat \times K}$ needs to be replaced
by the corresponding element $\varphi'$ under a certain base change homomorphism
\begin{align*}
  \HH(G'^\flat \times G', K'^\flat \times K') &\to \HH(G^\flat \times G, K^\flat \times K) \\
  \varphi &\mapsto \varphi'
\end{align*}

In that case, the identity \eqref{eq:old_FL} still hold as
\begin{equation}
  \Orb(g, \varphi) = \omega(\gamma) \Orb(\gamma, \varphi').
  \label{eq:eq:leslie_FL}
\end{equation}
To complete the analogy illustrated in \Cref{fig:history},
there should thus be a generalization of the AFL in which
$\mathbf{1}_{K'^\flat \times K}$ is replaced by any element of the Hecke algebra
$\HH(G'^\flat \times G, K'^\flat \times K)$.
This formula is proposed by \cite{ref:AFLspherical},
and is the primary focus of this paper; we discuss it in the next section.

\subsection{The inhomogeneous version of the arithmetic fundamental lemma for spherical Hecke algebras}
In contrast to the vague motivational cheerleading in the previous section,
starting in this section we will give more precise statements,
even though we will necessarily need to reference definitions appearing in later sections.

Retain the notation $G' \coloneqq \GL_n(E)$, and $G \coloneqq \U(V_0)(F)$,
with $K' \subset G'$ and $K \subset G$ the natural hyperspecial compact subgroups.
Also, let $q$ denote the residue characteristic of $F$.
Moreover, define the symmetric space
\[ S_n(F) \coloneqq \left\{ g \in \GL_n(E) \mid g \bar{g} = \id_n \right\}. \]
Finally, let $\VV_n$ be the non-split Hermitian space of dimension $n$ (unique up to isomorphism),
and let $\U(\VV_n)$ denote its unitary group.

For concreteness, we focus on the inhomogeneous version
of the arithmetic fundamental lemma, which is \cite[Conjecture 6.2.1]{ref:AFLspherical}.
This allows us to deal with just $G'$ instead of $G'^\flat \times G'$, etc.,
so that the Hecke algebra $\HH(G'^\flat \times G', K'^\flat \times K')$
can be replaced by the simpler one $\HH(G', K')$.

The AFL conjecture provides a bridge between a geometric left-hand side
(given by an intersection number)
and an analytic right-hand side (given by an orbital integral).
Stating it requires several pieces of data.
We only mention these pieces by name here, with definitions given later:
\begin{itemize}
  \ii On the geometric side, we have an intersection number.
  It uses the following ingredients.
  \begin{itemize}
    \ii We choose a regular semisimple element $g \in \U(\VV_n)\rs$.
    (The notation $\U(\VV_n)\rs$ denotes the regular semisimple elements of $\U(\VV_n)$, etc.
    The notion of regular semisimple is defined in \Cref{sec:matching}.)
    \ii We choose a function $f \in \HH(G, K)$ from the spherical Hecke algebra,
    defined in \Cref{sec:background}.
    \ii \todo{intersection number}
  \end{itemize}

  \ii On the analytic side, we have an orbital integral.
  It uses the following ingredients.
  \begin{itemize}
    \ii We choose a regular semisimple element $\gamma \in S_n(F)\rs$.
    \ii We choose a test function $f'$ which comes
    from a certain $\HH(G',K')$-module that we will denote $\HH(S_n(F), K')$.
    This module $\HH(S_n(F), K')$ is defined in \Cref{sec:background}.
    \ii The orbital integral $\Orb(\gamma, f', s)$
    is itself defined in \Cref{sec:orbital0}.
  \end{itemize}

  \ii We need a way to connect the inputs between the two parts of our conjecture.
  Specifically, $f$ and $f'$ need to be linked, and $g$ and $\gamma$ need to be linked.
  This is done as follows.
  \begin{itemize}
    \ii Once the regular semisimple element $g \in \U(\VV_n)\rs$ is chosen,
    we require $\gamma \in S_n(F)\rs$ to be a \emph{matching} element.
    This matching is defined in \Cref{sec:matching}.

    \ii Once $f \in \HH(G,K)$ is chosen, we select
    \[ f' = \BC_{S_n}^{\eta^{n-1}}(f) \]
    to be the image of $f$ under a \emph{base change}.
    This base change is defined and then calculated explicitly in \Cref{sec:satake}.
  \end{itemize}
\end{itemize}
With all our protagonists now having names and references,
we can now state the conjecture proposed in \cite{ref:AFLspherical}.
\begin{conjecture}
  [Inhomogeneous version of the AFL for the full spherical Hecke algebra,
  {\cite[Conjecture 6.2.1]{ref:AFLspherical}}]
  \label{conj:inhomog}
  Let $f \in \HH(G, K)$ be any element of the Hecke algebra,
  and let $f' = \BC_{S_n}^{\eta^{n-1}}(f) \in \HH(S_n(F), K')$ be its image
  under base change as defined in \Cref{sec:satake}.
  Then for matching (as defined in \Cref{sec:matching}) regular semisimple elements
  \[ g \in \U(\VV_n)\rs \longleftrightarrow \gamma \in S_n(F)\rs \]
  we have
  \begin{equation}
    \Int\left( (1,g), \mathbf{1}_{K'^\flat} \otimes f \right) \log q
    = -\omega(\gamma) \left. \pdv{}{s} \right\rvert_{s=0} \Orb(\gamma, f', s)
    \label{eq:inhomog}
  \end{equation}
  where the intersection number $\Int(\dots)$ and
  the transfer factor $\omega$ are defined in \Cref{sec:geo}
  while the orbital integral $\Orb(\dots)$ is defined in \Cref{sec:orbital0}.
\end{conjecture}

At present, the (inhomogeneous) AFL is the case where $f = \mathbf{1}_K$,
and is thus proven.

The generalized conjecture is also proved in full for
$n = 2$ in \cite[Theorem 1.0.1]{ref:AFLspherical}
(in that reference, our $n$ denotes the $n+1$ in \emph{loc.\ cit.}).
The part of the calculation involving orbital integral has two parts:
\begin{itemize}
  \ii The calculation makes $\BC_{S_{n}}^{\eta^{n-1}}$
  completely explicit in a natural basis for $n = 2$.
  The result is \cite[Lemma 7.1.1]{ref:AFLspherical}.

  \ii The calculation makes explicit the value of the orbital integral
  \[ \Orb(\gamma, f') \]
  for any $\gamma \in S_n(F)\rs$ and $f' \in \HH(S_n(F), K')$,
  in terms of invariants of $\gamma$ and a decomposition of $f'$ in a natural basis.
  The result is \cite[Proposition 7.3.2]{ref:AFLspherical}.
\end{itemize}
Combining these two (hence obtaining the right-hand side of \eqref{eq:inhomog})
with a calculation of intersection numbers in \cite[Corollary 7.4.3]{ref:AFLspherical}
(which is the left-hand side of \eqref{eq:inhomog})
shows that \Cref{conj:inhomog} holds for $n = 2$,
cf.\ \cite[Theorem 7.5.1]{ref:AFLspherical}.

\subsection{The arithmetic fundamental lemma for spherical Hecke algebras
  in the Fourier-Jacobi case}

This paper also states an analogous conjecture for the semi-Lie version.
It serves to complete the analogy given in \Cref{tab:semi_lie_analogy}

\begin{table}[ht]
  \centering
  \begin{tabular}{lll}
    \toprule
    Version & AFL for $\mathbf{1}$ (now proven) & Full spherical Hecke \\
    \midrule
    Group & \cite[Conjecture 2.9]{ref:AFL} & \cite[Conjecture 6.2.1]{ref:AFLspherical} \\
    Semi-Lie & \cite[Conjecture 1.12]{ref:liuFJ} & \Cref{conj:semi_lie_spherical} \\
    \bottomrule
  \end{tabular}
  \caption{Table showing}
  \label{tab:semi_lie_analogy}
\end{table}

In this variation, as in \cite{ref:liuFJ},
rather than matching $g \in \U(\VV_n)\rs$ to $\gamma \in S_n(F)\rs$,
we consider an augmented space larger than $\U(\VV_n)$ and $S_n(F)$.
Specifically, one considers a matching between tuples of regular semisimple elements
\[ (g, u) \in (\U(\VV_n) \times \VV_n)\rs
  \longleftrightarrow (\gamma, \uu, \vv^\top) \in (S_n(F) \times V_n'(F))\rs \]
where $V_n'(F)$ (defined in \Cref{sec:background})
consists of pairs of column vectors and row vectors of length $n$,
and the space $\VV_n$ is defined in \Cref{sec:matching}.
The notion of \emph{matching} is defined in \Cref{sec:matching} as well.
Meanwhile, we still use the same test functions $f$ and $f'$,
as we did for \cite[Conjecture 6.2.1]{ref:AFLspherical}.

\begin{conjecture}
  [Semi-Lie version of the AFL for the full spherical Hecke algebra]
  Let $f \in \HH(G, K)$ be any element of the Hecke algebra,
  and let $f' = \BC_{S_n}^{\eta^{n-1}}(f) \in \HH(S_n(F), K')$ be its image
  under base change defined in \Cref{sec:satake}.
  Then for matching (as defined in \Cref{sec:matching}) regular semisimple elements
  \[ (g, u) \in (\U(\VV_n) \times \VV_n)\rs \longleftrightarrow
    (\gamma, \uu, \vv^\top) \in (S_n(F) \times V_n')\rs \]
  we have
  \begin{equation}
    \begin{aligned}
      &\phantom= \Int\left( (g,u), f \right) \log q \\
      &= -\omega(\gamma, \uu, \vv^\top) \left. \pdv{}{s} \right\rvert_{s=0}
      \Orb( (\gamma, \uu, \vv^\top), f' \otimes \oneV, s)
    \end{aligned}
  \end{equation}
  where the intersection number $\Int(\dots)$ and
  the transfer factor $\omega$ are defined in \Cref{sec:geo}
  while the orbital integral $\Orb(\dots)$ is defined in \Cref{sec:orbital0}.
  \label{conj:semi_lie_spherical}
\end{conjecture}

\subsection{Results}
The goal of this paper is to perform the analogous calculations
that correspond to \cite[Lemma 7.1.1 and Proposition 7.3.2]{ref:AFLspherical} in the situations:
\begin{itemize}
  \ii The case $n = 3$ for \Cref{conj:inhomog};
  \ii The case $n = 2$ of \Cref{conj:semi_lie_spherical}.
\end{itemize}
The methods, which are local in nature,
are rather similar to those employed in \cite{ref:AFL},
which shows how to compute the orbital integral for $n = 3$ when $f = \mathbf{1}_{K'}$.
We state these results as follows.

\begin{theorem}
\end{theorem}
\todo{write summary theorem}

Besides providing evidence for \Cref{conj:inhomog} and \Cref{conj:semi_lie_spherical},
these orbital integral formulas also establish some other results.
For example, the formula above
proves the $n = 3$ case of \cite[Conjecture 8.2.1]{ref:AFLspherical}
and an analogous version; we state this below.

\begin{theorem}
  \todo{kernel thingy}
\end{theorem}
\todo{need to figure out why there are all these semisimple restrictions}

\subsection{Roadmap}
The rest of the paper is organized as follows.
\begin{itemize}
  \ii In \Cref{sec:background} we provide some preliminary background
  on the spaces appearing in the overall paper and the Hecke modules that will be used.
  Further background is stated in \Cref{sec:matching},
  where we describe the matching of regular semisimple elements

  \ii In \Cref{sec:satake} we provide reminders on the Satake transform
  and use it to derive the formulas for base change when $n = 3$.

  \ii In \Cref{sec:orbital0} we introduce the orbital integral
  for the group version of the AFL for full spherical Hecke algebra,
  and dedicate \Cref{sec:orbital1,sec:orbital2} to their proof.

  \ii In \Cref{sec:orbitalFJ0} we introduce the orbital integral
  for the semi-Lie version of the AFL for full spherical Hecke algebra,
  and produce analogous proofs to \Cref{sec:orbitalFJ1,sec:orbitalFJ2}.

  \ii Finally in \Cref{sec:geo}, we describe the Rapoport-Zink spaces
  that the geometric side is based on.
\end{itemize}


\subsection{Acknowledgments}
\todo{write this}
