\chapter{Large and small kernels}
\label{ch:ker}

In this chapter
\begin{itemize}
  \ii TODO: GROUP AFL VERSION
  \ii We use \Cref{cor:semi_lie_combo} to prove
  both \Cref{thm:semi_lie_ker_trivial} and  \Cref{thm:semi_lie_ker_huge}.
\end{itemize}

\section{TODO}
\todo{GROUP AFL}

\section{In the semi-Lie case, the kernel is trivial if we allow $v(e)$ to vary, even if we fix $\gamma$}
We prove \Cref{thm:semi_lie_ker_trivial} in this section.

\begin{lemma}
  Fix any $\gamma$ such that $v(b) = 1$, $v(c) = 0$, $v(d-a) = 0$,
  and let $N \ge 0$ be a nonnegative integer
  We define an $(N+2) \times (N+1)$ matrix $M$ as follows:
  for $0 \le i \le N+1$ and $0 \le r \le N$,
  the $i$\ts{th} row and $r$\ts{j} column takes the value
  \[
    M_{i,r} \coloneqq
    \frac{(-1)^r}{\log q}
    \left. \pdv{}{s} \right\rvert_{s=0}
      \Orb\left( \left(\gamma, \begin{pmatrix} 0 \\ 1 \end{pmatrix},
      \begin{pmatrix} 0 & \varpi^i \end{pmatrix} \right),
      \varphi \otimes \oneV, s\right).
  \]
  Then $M$ has full rank.
\end{lemma}
\begin{proof}
  We perform row operations as follows:
  \begin{itemize}
    \ii For each $i=N,N-1,\dots,0$, subtract the $i$\ts{th} row of $M$
    from the $(i+1)$\ts{th} row of $M$.
    Denote the new matrix as $M'$.

    \ii For each $i=N-1,N-2,\dots,0$, subtract the $i$\ts{th} row of $M'$
    from the $(i+2)$\ts{nd} row of $M'$.
    Denote the new matrix as $M''$.
  \end{itemize}
  For example, for $N = 5$, we have
  \[
    M =
    \begin{pmatrix}
    1 & 2 & 3 & 4 & 5 \\
    1 & q + 3 & 2 \, q + 4 & 3 \, q + 5 & 4 \, q + 6 \\
    2 & q + 4 & q^{2} + 3 \, q + 5 & 2 \, q^{2} + 4 \, q + 6 & 3 \, q^{2} + 5 \, q + 7 \\
    2 & 2 \, q + 5 & q^{2} + 4 \, q + 6 & q^{3} + 3 \, q^{2} + 5 \, q + 7 & 2 \, q^{3} + 4 \, q^{2} + 6 \, q + 8 \\
    3 & 2 \, q + 6 & 2 \, q^{2} + 5 \, q + 7 & q^{3} + 4 \, q^{2} + 6 \, q + 8 & q^{4} + 3 \, q^{3} + 5 \, q^{2} + 7 \, q + 9 \\
    3 & 3 \, q + 7 & 2 \, q^{2} + 6 \, q + 8 & 2 \, q^{3} + 5 \, q^{2} + 7 \, q + 9 & q^{4} + 4 \, q^{3} + 6 \, q^{2} + 8 \, q + 10
    \end{pmatrix}
  \]
  hence
  \[ M'=
    \begin{pmatrix}
    1 & 2 & 3 & 4 & 5 \\
    0 & q + 1 & 2 \, q + 1 & 3 \, q + 1 & 4 \, q + 1 \\
    1 & 1 & q^{2} + q + 1 & 2 \, q^{2} + q + 1 & 3 \, q^{2} + q + 1 \\
    0 & q + 1 & q + 1 & q^{3} + q^{2} + q + 1 & 2 \, q^{3} + q^{2} + q + 1 \\
    1 & 1 & q^{2} + q + 1 & q^{2} + q + 1 & q^{4} + q^{3} + q^{2} + q + 1 \\
    0 & q + 1 & q + 1 & q^{3} + q^{2} + q + 1 & q^{3} + q^{2} + q + 1
    \end{pmatrix}
  \]
  and finally
  \[ M'' =
    \begin{pmatrix}
    1 & 2 & 3 & 4 & 5 \\
    0 & q + 1 & 2 \, q + 1 & 3 \, q + 1 & 4 \, q + 1 \\
    0 & -1 & q^{2} + q - 2 & 2 \, q^{2} + q - 3 & 3 \, q^{2} + q - 4 \\
    0 & 0 & -q & q^{3} + q^{2} - 2 \, q & 2 \, q^{3} + q^{2} - 3 \, q \\
    0 & 0 & 0 & -q^{2} & q^{4} + q^{3} - 2 \, q^{2} \\
    0 & 0 & 0 & 0 & -q^{3}
    \end{pmatrix}.
  \]
  In order to prove $M$ has full rank, it suffices to prove $M''$ has full rank.
  However, as the examples suggest, $M''$ is almost upper triangular.

  We now show this.
  In this region in \Cref{thm:semi_lie_derivative_single}, if $i > r$
  the ``extra'' coefficient on $q^r$ is given by
  \begin{itemize}
    \ii When $i \equiv r \pmod 2$ we get $-\frac{\varkappa}{2} = - \frac{i-r}{2}$;
    \ii When $i \equiv r+1 \pmod 2$ we get $\frac{\varkappa}{2} - (i + \half - r) = -\frac{i-r+1}{2}$
  \end{itemize}
  and so we can write
  \[
    M_{i,r}
    = - \max\left(0, \left\lceil \frac{i-r}{2} \right\rceil \right) q^{r}
    + \sum_{j=0}^{\min(i,r)} \left( i + 1 + r - 2j \right) q^j.
  \]
  Hence
  \begin{align*}
    M'_{i,r}
    &= M_{i,r} - M_{i-1,r} \\
    &= - \mathbf{1}_{i \ge r, i \not\equiv r \mod 2} q^r
    + \mathbf{1}_{i \le r} (i+1-r) q^i
    + \sum_{j=0}^{\min(i-1,r)} q^j
  \end{align*}
  Then for any $i > r > 0$ it follows that
  \begin{align*}
    M''_{i,r}
    &= M'_{i,r} - M'_{i-2,r} \\
    &= M'_{i,r} - M'_{i-2,r} \\
    \begin{cases}
      -q^{r-1} & \text{if } i \ge r+1, r \ge 1 \\
      0 & \text{if } i \ge r+2.
    \end{cases}
  \end{align*}
  Moreover, since $M'_{0,r} = \mathbf{1}_{r \equiv 0 \pmod 2}$
  we deduce
  $M''_{0,r} = M'_{0,r} - M'_{0,r-2} = 0$
  for any $r \ge 2$, and $M''_{0,0} = 1$.
  Hence the pattern suggested by $M''$ holds in general.

  Finally, delete the 2nd row of $M''$ to get a square matrix
  with $N+1$ rows and $N+1$ columns which is upper triangular,
  and whose determinant is $(-1)^N q^{1+2+\dots+(N-1)} \neq 0$.
  Hence $M''$ has full rank.
\end{proof}

Suppose we are given some function
\[ \phi = \sum_{r=0}^N (-1)^r c_r \mathbf{1}_{K', \le r} \in \HH(S_2(F)). \]
Letting $M$ be the matrix given in the above proposition, we are supposed to have
\[
  M \begin{pmatrix} c_0 \\ c_1 \\ \vdots \\ c_N \end{pmatrix}
  = \mathbf{0} \in \CC^{N+2}.
\]
Since $M$ has full rank, it follows that $c_0 = \dots = c_N = 0$.

\section{In the semi-Lie case, the kernel has finite codimension if $v(e)=0$}
We prove \Cref{thm:semi_lie_ker_huge} in this section,
where we fix $e = 1$ (or any $e$ with $v(e) = 0$.
For this section,
we define the following sequence of indicator functions for $r \ge 5$:
\begin{align*}
  \phi^{(1)}_r &\coloneqq
  \left( \mathbf{1}_{K' \le r} + \mathbf{1}_{K' \le (r-1)}
     - q^2 (\mathbf{1}_{K' \le (r-2)} + \mathbf{1}_{K' \le (r-3)}) \right) \\
  \phi^{(2)}_r &\coloneqq \phi^{(1)}_r + \phi^{(1)}_{r-1} \\
  \phi^{(3)}_r &\coloneqq \phi^{(2)}_r + q \phi^{(2)}_{r-1}.
\end{align*}

We prove the following statement,
which implies \Cref{thm:semi_lie_ker_huge} as a direct corollary.
\begin{theorem}
  Suppose $\guv \in (S_2(F) \times V_2'(F))\rs$
  pairs with an element of $(\U(\VV_2^-) \times \VV_2^-)\rs$.
  Then
  \[
    \left. \pdv{}{s} \right\rvert_{s=0}
    \Orb \left( \guv, \phi^{(3)}_r \otimes \oneV, s \right) = 0
  \]
  holds for all $r \ge 5$ with at most three exceptions,
  namely $v(e) - v(d-a) + 2 \le r \le v(e) - v(d-a) + 4$.
\end{theorem}
