\chapter{Synopsis of the weighted orbital integral $\Orb(\gamma, \phi, s)$ for $\gamma \in S_3(F)\rs$ and $\phi \in \HH(S_3(F))$}
\label{ch:orbital0}

This section defines the weighted orbital integral
and describes the parameters which we will use to express our answer.

\section{Initial definition of the weighted orbital integral for general $S_n(F)$}
Let $H = \GL_{n-1}(F)$.
Then $H$ has a natural embedding into $\GL_n(E)$ by
\[ h \mapsto \begin{pmatrix} h & 0 \\ 0 & 1 \end{pmatrix} \]
which endows it with an action $S_n(F)$.
Then our weighted orbital integral is defined as follows.
\begin{definition}
  [{\cite[Equation (3.2.3)]{ref:AFLspherical}}]
  \label{def:orbital0}
  For brevity let $\eta(h) \coloneqq \eta(\det h)$ for $h \in H$.
  For $\gamma \in S_n(F)$, $\phi \in \HH(S_n(F))$, and $s \in \CC$,
  we define the \emph{weighted orbital integral} by
  \[ \Orb(\gamma, \phi, s) \coloneqq
    \int_{h \in H} \phi(h\inv \gamma h) \eta(h)
    \left\lvert \det(h) \right\rvert_F^{-s} \odif h. \]
\end{definition}
\begin{definition}
  [The abbreviation $\partial \Orb(\gamma, \phi)$]
  From now on we will abbreviate
  \[ \partial \Orb(\gamma, \phi)
    \coloneqq \left. \pdv{}{s} \right\rvert_{s=0} \Orb(\gamma, \phi, s). \]
\end{definition}

We remark that this weighted orbital integral is related to
an (unweighted) orbital integral on the unitary side
by the so-called relative fundamental lemma.
Specifically, for $g \in \U(\VV_n^+)$ and $f \in \HH(\U(\VV_n^+))$,
we define the (unweighted) orbital integral by
\[ \Orb^{\U(\VV_n^+)}(g, f) \coloneqq \int_{\U(\VV_n^+)} f(x^{-1}gx) \odif x. \]
Then the following result is true.
\begin{theorem}
  [Relative fundamental lemma; {\cite[Theorem 1.1]{ref:leslie}}]
  \label{thm:rel_fundamental_lemma}
  Let $\phi \in \HH(S_n(F))$ and $\gamma \in S_n(F)\rs$.
  \[ \omega(\gamma) \Orb(\phi, \gamma, 0)
    = \begin{cases}
      0 & \text{if }\gamma \in S_n(F)\rs^- \\
      \Orb^{\U(\VV_n^+)}(g, \BC^{\eta^{n-1}}_{S_n}(\phi)) & \text{if } \gamma \in S_n(F)\rs^+
    \end{cases}
  \]
  where the transfer factor $\omega$ is defined in \Cref{ch:geo}.
\end{theorem}

\section{Basis for the indicator functions in $\HH(S_3(F))$}
\label{ch:orbital0_hecke_basis}
From now on assume $n = 3$.
We have the symmetric space
\[ S_3(F) \coloneqq \left\{ g \in \GL_3(E) \mid g \bar{g} = \id_3 \right\}. \]
which has a left action under $\GL_3(E)$ by $g \cdot s \mapsto gs\bar{g}\inv$.

Then $S_3(F)$ admits the following decomposition, which we will use:
\begin{lemma}
  [Cartan decomposition of $S_3(F)$]
  For each integer $r \ge 0$ let
  \[ K'_{S,r} \coloneqq \GL_3(\OO_E) \cdot \begin{pmatrix} 0 & 0 & \varpi^r \\ 0 & 1 & 0 \\ \varpi^{-r} & 0 & 0 \end{pmatrix} \]
  denote the orbit of
  $\begin{pmatrix} 0 & 0 & \varpi^r \\ 0 & 1 & 0 \\ \varpi^{-r} & 0 & 0 \end{pmatrix}$
  under the left action of $\GL_3(\OO_E)$.
  Then we have a decomposition
  \[ S_3(F) = \coprod_{r \geq 0} K'_{S,r}. \]
\end{lemma}
\begin{proof}
  See \cite[\S3]{ref:pacific_offen_sphere}.
\end{proof}

The $r=0$ case will be given a special shorthand,
and can be expressed in a few equivalent ways:
\begin{align*}
  K'_S
  &\coloneqq K'_{S,0} \\
  &= \GL_3(\OO_E) \cdot \begin{pmatrix} & & 1 \\ & 1 \\ 1 \end{pmatrix} \\
  &= \GL_3(\OO_E) \cdot \id_3 = S_3(F) \cap \GL_3(\OO_E).
\end{align*}
One can equivalently define $K'_{S,r}$ to be the part of $S_3(F)$
for which the most negative valuation among the nine entries is $-r$.

For $r \geq 0$, define
\[ K'_{S, \le r} \coloneqq S_3(F) \cap \varpi^{-r} \GL_3(\OO_E). \]
We can re-parametrize the problem according to the following.
\begin{corollary}
  We have a decomposition
  \[ K'_{S, \le r} = K'_{S,0} \sqcup K'_{S,1} \sqcup \dots \sqcup K'_{S,r}. \]
\end{corollary}
Then an integral over each $K'_{S, \le r}$ lets us extract the integrals over $K'_{S,r}$.
\begin{corollary}
  [Basis for $\HH(S_3(F))$]
  For $r \ge 0$, the indicator functions $\mathbf{1}_{K'_{S, \le r}}$
  form a basis of $\HH(S_3(F))$.
\end{corollary}

Then, our goal is to compute for
$\partial \Orb(\gamma, \mathbf{1}_{K'_{S, \le r}})$
at $s=0$ for any $r > 0$ as well;
note that the $r = 0$ case is already done in \cite{ref:AFL}.

\section{Parametrization of $\gamma$}
Again, assume $n = 3$.
Further assume $\gamma \in S_3(F)\rs$ is regular semisimple.
We identify some parameters for the orbit of $\gamma$
that we can use for our explicit calculations.

\subsection{Rewriting the weighted orbital integral as a double integral over $E$
  via the group $H' \cong \GL_2(F)$}
Our weighted orbital integral is at present a quadruple integral over $F$,
owing to $H = \GL_{2}(F)$ being a four-dimensional $F$-vector space.

It will be more economical to work with the weighted orbital integral
as a double integral with two coefficients in $E$, in the following sense.
As in \cite[\S4.1]{ref:AFL} define
\[ H' \coloneqq
  \left\{ \begin{pmatrix} t_1 & t_2 \\ \bar t_2 & \bar t_1 \end{pmatrix}
    \mid t_1, t_2 \in E \right\}
\]
which is indeed a four-dimensional $F$-algebra.
As before $H' \hookrightarrow \U(\VV_n^+)$ according to the same embedding
$\GL_2(E) \hookrightarrow \GL_3(E)$
and so $H'$ also acts on $S_n(E)$ by conjugation.

As an $F$-algebra, we have an isomorphism (see \cite[\S4.1]{ref:AFL})
\begin{align*}
  \iota_2 \colon H = \GL_2(F)
  &\xrightarrow{\cong} H' \\
  \begin{pmatrix} a_{11} & a_{12} \\ a_{21} & a_{22} \end{pmatrix}
  &\mapsto \begin{pmatrix} t_1 & t_2 \\ \bar t_2 & \bar t_1 \end{pmatrix} \\
  t_1 &= \half\left( a_{11} + a_{22} + \frac{a_{12}}{\sqrt{\eps}} + a_{21} \sqrt{\eps} \right) \\
  t_2 &= \half\left( a_{11} - a_{22} + \frac{a_{12}}{\sqrt{\eps}} - a_{21} \sqrt{\eps} \right).
\end{align*}
Under this isomorphism, we have
\[ h \gamma h^{-1} = \iota_2(h) \gamma \overline{\iota_2(h)^{-1}}. \]

This allows us to rewrite the weighted orbital integral over $H'$ instead.
If we write $h' = \overline{\iota_2(h)^{-1}}$,
then the following integral formula is obtained.
\begin{proposition}
  [{\cite[\S4.2]{ref:AFL}}]
  \label{prop:orbital_over_H_prime}
  For brevity let $\eta(h') \coloneqq \eta(\det h')$ for $h' \in H'$.
  For $\gamma \in S_3(F)$, $\phi \in \HH(S_3(F))$, and $s \in \CC$,
  the weighted orbital integral can instead be written as
  \[ \Orb(\gamma, \phi, s) =
    \int_{h' \in H'} \phi(\bar{h'}\inv \gamma h') \eta(h')
    \left\lvert \det(h') \right\rvert_F^{s} \odif{h'} \]
  where
  \[ \odif{h'} = \kappa \cdot \frac{\odif t_1 \odif t_2}
    {\left\lvert t_1 \bar t_1 - t_2 \bar t_2 \right\rvert_F^2} \]
  for the constant
  \[ \kappa \coloneqq \frac{1}{(1-q\inv)(1-q^{-2})}. \]
\end{proposition}

\subsection{Identifying a representative in the $H'$-orbit}
Write $\gamma \overset{H'}{\sim} \gamma'$ to mean that $\gamma$ and $\gamma'$
are in the same $H'$-orbit.
Evidently the weighted orbital integral $\Orb(\gamma, \phi, s)$ in \Cref{prop:orbital_over_H_prime}
only depends on such an orbit.
So it makes sense to pick a canonical representative for the $H'$-orbit to compute
the weighted orbital integral in terms of.

Since we assumed $\gamma \in S_3(F)\rs$ is regular semisimple,
we can invoke \cite[Proposition 4.1]{ref:AFL}
to assume that $\gamma$ that under the orbit of $H'$ we have
\[ \gamma \overset{H'}{\sim}
  \begin{pmatrix}
    a & 0 & 0 \\
    b & - \bar d & 1 \\
    c & 1 - d \bar d & d
  \end{pmatrix}
  \in S_3(F)\rs; \quad \text{where $c = -a \bar b + b d$} \]
over all $a \in E^1$, $b \in E$, $d \in E$ for which $(1-d\bar d)^2 - c \bar c \neq 0$.
In other words, the representatives described here cover all the regular $H'$-orbits in $S_3(F)\rs$.

\subsection{Simplification due to the matching of non-quasi-split unitary group}
In this calculation, we restrict attention to the case where our regular $\gamma$
matches an element in the non-quasi-split unitary group $\U(\VV_n^-)\rs$
(rather than $\U(\VV_n^+)\rs$).
As we described in \Cref{prop:valuation_delta_matching_group},
this is controlled by the parity of the invariant
\[ v\left( (1-d\bar d)^2 - c \bar c\right) \]
being odd.
Hence, we only have to consider this case:
\begin{assume}
  \label{assume:u_odd}
  We will assume that
  \[ v\left( (1-d\bar d)^2 - c \bar c\right) \equiv 1 \pmod 2. \]
\end{assume}
This is the same assumption made in \cite[Equation (4.3)]{ref:AFL}.

To summarize everything we've said.
\begin{lemma}
  [Parametrization for $S_3(F)\rs^-$ via $H'$-orbit]
  \label{lem:S3_abcd}
  Let $\gamma \in S_3(F)\rs^-$.
  Then there exists $a \in E^1$, $b \in E$, $d \in E$ such that
  \[ \gamma \overset{H'}{\sim}
    \begin{pmatrix}
      a & 0 & 0 \\
      b & - \bar d & 1 \\
      c & 1 - d \bar d & d
    \end{pmatrix}
    \in S_3(F)\rs; \quad \text{where $c = -a \bar b + b d$}. \]
  Moreover, $v\left( (1-d \bar d)^2 - c \bar c \right)$ is odd.
\end{lemma}

We will mostly be interested in the case where $v(b) = v(d) = 0$.
In fact, few other cases even occur at all given \Cref{assume:u_odd};
we will see momentarily that either $v(b) = v(d) < 0$,
or one of $\{v(b), v(d)\}$ is zero and the other is nonnegative.

\section{Parameters used in the calculation of the weighted orbital integral}
\label{sec:param_orbital0}

We now evaluate the orbital integral in terms of
\[ a \in E^1, \qquad b, d \in E, \qquad r \ge 0. \]
To simplify the notation in what follows,
it will be convenient to define several quantities that reappear frequently.
From \Cref{assume:u_odd}, we may define
\begin{equation}
  \delta \coloneqq v(1-d \bar d) = v(c) \neq -\infty.
  \label{eq:delta}
\end{equation}
Following \cite[Equation (4.3)]{ref:AFL} we will also define
\begin{equation}
  u \coloneqq \frac{\bar c}{1-d \bar d} \in \OO_E^\times
  \label{eq:u}
\end{equation}
so that $\nu(1-u \bar u) \equiv 1 \pmod 2$ and
\begin{equation}
  b = -au - \bar{d} \bar{u}.
  \label{eq:b}
\end{equation}
Note that this gives us the following repeatedly used identity
\begin{equation}
  b^2-4a\bar d = (au-\bar d \bar u)^2 - 4a\bar d(1-u\bar u).
  \label{eq:dos}
\end{equation}
Finally, define
\begin{equation}
  \ell \coloneqq v(b^2 - 4 a \ol d).
  \label{eq:ell}
\end{equation}
We will also define one additional parameter useful when $\ell$ is even
(but as we will see, redundant for odd $\ell$):
\begin{equation}
  \lambda \coloneqq v(1-u \bar u) \equiv 1 \pmod 2.
  \label{eq:lambda}
\end{equation}

Just as many pairs $(v(b), v(d))$ do not occur (given \Cref{assume:u_odd})
and $v(b) = v(d) = 0$ is the main case of interest,
the parameters $(\delta, \ell, \lambda)$ satisfy some additional relations.
We will now describe them.

\begin{lemma}
  [Constraints between $\ell$, $\delta$, $\lambda$]
  \label{lem:parameter_constraints}
  Let $a$, $b$, $c$, $d$ be as in \Cref{lem:S3_abcd},
  and let $\delta$, $\ell$, and $\lambda$ be as defined in
  \eqref{eq:delta}, \eqref{eq:ell}, \eqref{eq:lambda}.
  Then exactly one of the following situations is true:
  \begin{itemize}
    \ii $v(b) = v(d) = 0$, $\ell \ge 1$ is odd, $\ell < 2 \delta$, and $\lambda = \ell$.
    \ii $v(b) = v(d) = 0$, $\ell \ge 0$ is even, $\ell \le 2 \delta$, and $\lambda > \ell$ is odd.
    \ii $v(b) = 0$, $v(d) > 0$, $\ell = \delta = 0$, and $\lambda > 0$ is odd.
    \ii $v(b) > 0$, $v(d) = 0$, $\ell = 0$, $\delta \ge 0$, and $\lambda > 0$ is odd.
    \ii $v(b) = v(d) < 0$, $\ell = \delta = 2v(d) < 0$, and $\lambda > 0$ is odd.
  \end{itemize}
  See \Cref{tab:parameter_constraints}.
  Moreover, whenever $\ell$ is even,
  the quantity $b^2 - 4 a \bar d$ is a square of some element in $E$.
\end{lemma}

\begin{table}[ht]
  \centering
  \begin{tabular}{cccc}
    \toprule
    & $v(b) = 0$ & $v(b) > 0$ & $v(b) < 0$ \\
    \midrule
    $v(d) = 0$ & $0 \le \ell \le 2 \delta$ & $\ell = 0$, $\delta \ge 0$ & never \\
    $v(d) > 0$ & $\ell = \delta = 0$ & never & never \\
    $v(d) < 0$ & never & never & $v(b) = v(d) = \frac{\ell}{2} = \frac{\delta}{2} < 0$ \\
    \bottomrule
  \end{tabular}
  \caption{A table showing the five cases in \Cref{lem:parameter_constraints}.}
  \label{tab:parameter_constraints}
\end{table}

Before proving the lemma in full we first prove the following lemmas.
\begin{lemma}
  [$v(au-\bar d \bar u)$]
  \label{lem:au_minus_du}
  Assume $v(d) \ge 0$.
  For odd $\ell$, we have
  \[ 2 v(au - \bar d \bar u) > \ell = \lambda \]
  while for even $\ell$ we instead have
  \[ 2 v(au - \bar d \bar u) = \ell < \lambda. \]
\end{lemma}
\begin{proof}
  If $v(d) = 0$, this follows from \eqref{eq:dos} directly,
  since $v\left( (au - \bar d \bar u)^2 \right)$ is even, and hence
  can never equal $v(4a \bar d (1- u \bar u)) = \lambda \equiv 1 \pmod 2$.

  Meanwhile, if $v(d) > 0$, then from \eqref{eq:b}
  it follows $v(b) = 0$, and hence $\ell = 0$.
  And $v(au - \bar d \bar u) = 0$ in this case as well.
  Since $\lambda$ is a positive odd integer, the lemma is proved.
\end{proof}

\begin{lemma}
  [$\ell \le 2 \delta$]
  If $v(b) = v(d) = 0$ and $\ell \ge 0$, then $\ell \le 2 \delta$.
  \label{lem:ell_le_2_delta}
\end{lemma}
\begin{proof}
  If $\ell = 0$ there is nothing to prove so assume $\ell > 0$.
  Let us write
  \begin{equation}
    \OO_E^\times \ni \frac{a d u}{\bar u} = x + y \sqrt{\eps} \qquad x,y \in F
    \label{eq:ell_delta_adu_xy}
  \end{equation}
  which has norm
  \begin{equation}
    \OO_F^\times \ni x^2+y^2\eps
    = \frac{a d u}{\bar u} \cdot \frac{\bar a \bar d \bar u}{u} = d \bar d.
    \label{eq:ell_delta_adu_norm}
  \end{equation}

  Now, according to \Cref{lem:au_minus_du} we have that
  \[ 0 < \ell \le 2v(au - \bar d \bar u)
    = 2v\left( \frac{a d u}{\bar u} - d \bar d \right)
    = 2v\left(  (x - d \bar d) + y \sqrt{\eps}. \right) \]
  and since $d \bar d \in F$, it follows that
  \begin{align}
    v\left( x - d \bar d \right) &> \frac{\ell}{2} \label{eq:ell_delta_x_minus_d_bar_d} \\
    v(y) &> \frac{\ell}{2}. \label{eq:ell_delta_y}
  \end{align}
  In particular, \eqref{eq:ell_delta_y} implies $v(y) > 0$ which has two consequences:
  \begin{itemize}
    \ii From \eqref{eq:ell_delta_adu_xy} we get $v(x) = 0$.
    \ii From $v(y^2) > 0$ and \eqref{eq:ell_delta_adu_norm}
    we conclude \[ v(x^2 - d \bar d) = v(y^2) > \ell. \]
  \end{itemize}
  Putting \eqref{eq:ell_delta_x_minus_d_bar_d} together with the previous two bullets,
  \[ \frac{\ell}{2} \le v\left( x^2 - d \bar d - x(x - d \bar d) \right)
    = v(x) + v(1-d \bar d) = 0 + \delta \]
  and this proves $\ell \le 2\delta$.
\end{proof}

Now we can prove \Cref{lem:parameter_constraints}.
\begin{proof}[Proof of \Cref{lem:parameter_constraints}]
  It's clear the five bullets above are disjoint.
  \begin{itemize}
  \ii First assume $\ell$ is odd.
  We assert in this case we have $v(b) = v(d) = 0$.
  Indeed if $v(d) \neq 0$, then $b = -au-\bar d\bar u$ is a unit,
  and hence so is $b^2 - 4 a \bar d$, causing $\ell = 0$, contradiction.
  And if $d$ is a unit, $\ell \neq 0$ means $v(b) = 0$ too.
  In particular, $\ell > 0$.
  The rest of the claims follow by \Cref{lem:au_minus_du} and \Cref{lem:ell_le_2_delta}.
  \end{itemize}
  For the rest of the proof we only consider even $\ell$.
  Because $b = -au - \bar d \bar u$, it cannot be the case that $v(b) > 0$ and $v(d) > 0$;
  moreover if either $v(b) < 0$ or $v(d) < 0$,
  then in fact $v(b) = v(d)$.
  We consider each of the four possibilities.
  \begin{itemize}
  \ii Suppose and $v(b) = v(d) = 0$.
  Then \Cref{lem:au_minus_du} and \Cref{lem:ell_le_2_delta} imply the results.

  \ii If $v(b) = 0$ and $v(d) > 0$, then $b^2 - 4a \bar d$ is a unit
  and $1 - d \bar d$ are both units, ergo $\ell = \delta = 0$.

  \ii If $v(b) > 0$ and $v(d) = 0$ then $b^2 - 4 a \bar d$ is a unit,
  and there is nothing left to prove.

  \ii Finally suppose $v(b) = v(d) < 0$.
  Then $v(b^2) < v(4 a \bar d) < 0$, so indeed $\ell = 2 v(d) < 0$.
  And $v(1 - d \bar d) = 2v(d) < 0$ as well.
  \end{itemize}
  We now verify the last assertion that $b^2 - 4 a \bar d$
  is a square whenever $\ell$ is even.
  The proof in all cases uses \eqref{eq:dos} to show $b^2 - 4 a \bar d$
  is equal to $\varpi^\ell$ times a quadratic residue in $\OO_E^\times$.
  Indeed we need only verify that $v(4a \bar d(1 - u \bar u)) = v(d) + \lambda$
  has larger valuation than $v\left( (a u - \bar d \bar u)^2 \right) = \ell$.
  In the case $v(b) = 0$ this follows from $\lambda > \ell$.
  Whereas if $v(d) > 0$ we have $\ell = 0$
  and if $v(b) = v(d) < 0$ then $\ell = -2v(d)$;
  so in all these cases the claim is obvious too. \qedhere
\end{proof}

In the case where $\ell$ is odd (and hence $\ell \ge 1$ and $v(b) = v(d) = 0$),
we get \eqref{eq:dos} implying $\lambda = \ell$
and thus $\lambda$ will never be used --- the weighted orbital will be computed
as a function of $\ell$ and $\delta$ (and $r$).

However for even $\ell$ these numbers are never equal and our weighted orbital
integral will be stated in terms of $\ell$, $\delta$, and $\lambda$ (and $r$).
We just saw that in these situations $b^2 - 4 a \bar d$ is a square;
moving forward, we need to fix the choice of the square root $\tau$.
We do so as follows.
\begin{definition}
  [Fixing the choice of $\tau$]
  Assuming $\ell$ is even, using \eqref{eq:dos} in the form
  \[ (au-\bar d \bar u)^2 - \tau^2 = 4a\bar d(1- u\bar u) \]
  we agree now to fix the choice of the square root of $\tau$ such that
  \begin{equation}
    \begin{aligned}
      v(au-\bar d \bar u + \tau) &= \lambda + v(d) - \half \ell > 0, \\
      v(au-\bar d \bar u - \tau) &= \half \ell.
    \end{aligned}
    \label{eq:tau_choice}
  \end{equation}
\end{definition}
Here $\lambda + v(d) - \half \ell > 0$ is obvious when $v(d) \ge 0$
(since $\lambda = \ell > 0$ for odd $\ell$ and otherwise $\lambda > \ell$),
and for $v(d) < 0$ we have $v(d) = \half \ell$ anyway.

\begin{lemma}
  [$v(4 - \Norm(b\pm\tau)$]
  With this choice of $\tau$, we have
  \begin{equation}
    \begin{aligned}
      v\left( 4 - \Norm(b+\tau) \right) &= \lambda + \delta - \ell \\
      v\left( 4 - \Norm(b-\tau) \right) &= \delta.
    \end{aligned}
    \label{eq:four_norm_choice}
  \end{equation}
\end{lemma}
\begin{proof}
  We consider several cases.
  \begin{itemize}
  \ii If $v(b) = v(d) = 0$ then from \eqref{eq:dos} we have
  \[ \ell = 2v(\tau) = 2v(au - \bar d \bar u) < \lambda \]
  and thus \cite[Lemma 4.7]{ref:AFL} applies to give \eqref{eq:four_norm_choice}, verabtim.

  \ii Now suppose $v(d) > 0$ but still $v(b) = v(\tau) = 0$.
  We begin with the observation that
  \begin{equation}
    4 a \bar d = b^2 - \tau^2 = (b + \tau)(b - \tau)
    \label{eq:tau_which_which}
  \end{equation}
  and so $\{v(b+\tau), v(b-\tau)\} = \{0, v(d)\}$.
  We need to determine which is which.
  However, note that we may write
  \[ au - \bar d \bar u - \tau = -(b + \tau) - 2 \bar d \bar u. \]
  Since $v(au - \bar d \bar u - \tau) = 0$ and $v(d) > 0$,
  it follows we must have $v(b+\tau) = 0$.
  And thus $v(b-\tau) = v(d)$.
  Hence $v(4 - \Norm(b-\tau)) = 0 = \delta$,
  and we have obtained the bottom equation of \eqref{eq:four_norm_choice}.

  It remains to show that $v(4-\Norm(b+\tau)) = \lambda$ to complete the proof.
  We quote \cite[Lemma 4.6]{ref:AFL} which states more generally that
  \begin{align*}
    2\delta + \lambda
    &= v(4-\Norm(b+\tau)) + v(4-\Norm(b-\tau)) \\
    &\qquad+ v(16 + 16 d \bar d - 8 b \bar b + 8 \tau \bar\tau).
  \end{align*}
  In our case $\delta = 0$, $v(4-\Norm(b-\tau)) = 0$.
  Moreover since $v(\tau - b) > 0$ we get $v(\tau\bar\tau - b \bar b) > 0$.
  (Indeed, if $\tau = x_\tau + \eps y_\tau$ and $b = x_b + \eps y_b$,
  then $\tau\bar\tau - b \bar b = (x_\tau^2 - x_b^2 + \eps(y_\tau^2 - y_b^2))
  + \eps(x_\tau y_\tau - y_\tau y_b)$.
  Since $x_\tau \equiv x_b \pmod{\varpi}$ and $y_\tau \equiv y_b \pmod{\varpi}$,
  the conclusion is immediate.)
  Hence the final term on the right-hand side is $0$ too.

  \ii Consider $v(b) > 0$.
  As mentioned on \cite[p.\ 242]{ref:AFL}, the identity \eqref{eq:four_norm_choice}
  is still true in this situation too.

  \ii Finally assume $v(\tau) = v(b) = v(d) < 0$.
  Again from \eqref{eq:tau_which_which}
  we know $\{v(b+\tau), v(b-\tau)\} = \{0, v(d)\}$ and need to determine which is which.
  This time we write
  \[ au - \bar d \bar u + \tau = -(b - \tau) - 2 \bar d \bar u. \]
  Since $v(au - \bar d \bar u + \tau) = \lambda > 0$,
  but $v(2 \bar d \bar u) = v(d) < 0$,
  it follows we must have $v(b-\tau) < 0$,
  so in fact $v(b-\tau) = v(d)$ and $v(b+\tau) = 0$.
  So, in this case, we get $v(4 - \Norm(b-\tau)) = 2v(b) = \delta$,
  which is the bottom equation of \eqref{eq:four_norm_choice}.

  Then it remains to show that $v(4-\Norm(b+\tau)) = \lambda$,
  which is done in the same way as $v(d) > 0$ earlier.
  \qedhere
  \end{itemize}
\end{proof}
\section{Statement of the full orbital integral}
We now show the full orbital integrals that were previously only summarized as \Cref{thm:summary}.
The proof of these formulas is carried out in \Cref{ch:orbital1,ch:orbital2}.
\subsection{Arches}
We introduce one piece of notation to compress the particular shape our formulas are about to take.

\begin{definition}
  \label{def:arch}
  Suppose $\{a_0, a_0 + 1, \dots, a_1\}$ is an interval of integers for some $a_0 \le a_1$,
  and consider two more integers $w_1$ and $w_2$ such that $w_1 + w_2 \le \frac{a_1-a_0}{2}$.
  Then we can define a piecewise linear function
  \[ \Arch_{[a_0, a_1]}(w_1, w_2) \colon \{a_0, a_0+1, \dots, a_1\} \to \ZZ_{\ge 0} \]
  according to the following definition:
  \[
    k \mapsto
    \begin{cases}
      k - a_0 & \text{if }a_0 \le k \le a_0 + w_1 \\
      w_1 + \left\lfloor \frac{k-(a_0+w_1)}{2} \right\rfloor & \text{if } a_0 + w_1 \le k \le a_0 + w_1 + w_2 \\
      w_1 + \left\lfloor \frac{w_2}{2} \right\rfloor & \text{if } a_0 + w_1 + w_2 \le k \le a_1 - (w_1 + w_2)\\
      w_1 + \left\lfloor \frac{(a_1-w_1) - k}{2} \right\rfloor & \text{if } a_1 - (w_1 + w_2) \le k \le a_1 - w_1 \\
      a_1 - k & \text{if }a_1 - w_1 \le k \le a_1.
    \end{cases}
  \]
\end{definition}
The nomenclature is meant to be indicative of the shape of the graph,
which looks a little bit like an arch.
It is a function symmetric around $\frac{a_0+a_1}{2}$ defined piecewise.
The function grows linearly with slope $1$ at the far left for $w_1$ steps,
then changes to slope $1/2$ for $w_2$ steps (rounding down),
before stabilizing, then doing the symmetric descent on the right half.
\begin{figure}[ht]
  \begin{center}
  \begin{asy}
    size(12cm);
    draw((-1,0)--(20,0));
    draw((0,0)--(3,3)--(7,5)--(12,5)--(16,3)--(19,0), lightred);
    draw((3,3)--(3,0), grey);
    draw((7,5)--(7,0), grey);
    draw((12,5)--(12,0), grey);
    draw((16,3)--(16,0), grey);
    real eps = 0.3;
    void brack(string s, real x0, real x1) {
      draw((x0+0.1,-eps)--(x0+0.1,-2*eps)--(x1-0.1,-2*eps)--(x1-0.1,-eps), blue);
      label(s, ((x0+x1)/2, -2*eps), dir(-90), blue);
    }
    brack("$w_1 = 3$", 0, 3);
    brack("$w_2 = 4$", 3, 7);
    brack("$w_2 = 4$", 12, 16);
    brack("$w_1 = 3$", 16, 19);

    dotfactor *= 1.5;
    dot("$(0,0)$", (0,0), dir(225));
    dot((1,1), red);
    dot((2,2), red);
    dot("$(3,3)$", (3,3), dir(135));
    dot((4,3), red);
    dot((5,4), red);
    dot((6,4), red);
    dot("$(7,5)$", (7,5), dir(90));
    dot((8,5), red);
    dot((9,5), red);
    dot((10,5), red);
    dot((11,5), red);
    dot("$(12,5)$", (12,5), dir(90));
    dot((13,4), red);
    dot((14,4), red);
    dot((15,3), red);
    dot("$(16,3)$", (16,3), dir(45));
    dot((17,2), red);
    dot((18,1), red);
    dot("$(19,0)$", (19,0), dir(315));

    label(rotate(45)*"Slope $+1$", (1.5,1.5), dir(135), lightred);
    label(rotate(26.57)*"Slope $+\frac12$", (5,4), dir(125), lightred);
    label(rotate(-26.57)*"Slope $-\frac12$", (14,4), dir(55), lightred);
    label(rotate(-45)*"Slope $-1$", (17.5,1.5), dir(45), lightred);
    label("Slope $0$", (9.5,5), dir(90), lightred);
  \end{asy}
  \end{center}
  \caption{A plot of $\Arch_{[0,19]}(3,4)$.}
  \label{fig:arch}
\end{figure}



\subsection{Full explicit weighted orbital integral for the case where $\ell$ odd}
If $\ell$ is odd, so $\lambda = \ell$, then
the weighted orbital integral can be expressed succinctly in the following way.
\begin{theorem}
  [Weighted orbital integral for odd $\ell$]
  \label{thm:full_orbital_ell_odd}
  Let $r \ge 0$.
  Let $\gamma \in S_3(F)\rs$ match an element of $\U(\VV_3^-)$,
  and let $b$, $d$, $\delta$, $\ell$, be as in
  \Cref{lem:S3_abcd} and \Cref{lem:parameter_constraints}.
  If $\ell$ is odd, define
  \[ \nn_\gamma \coloneqq \Arch_{[-2r, \ell + 2\delta + 2r]}(r, \ell). \]
  Then for any $r \ge 0$ we have the formula:
  \[
    \Orb(\gamma, \mathbf{1}_{K'_{S, \le r}}, s)
    = \sum_{k = -2r}^{\ell + 2\delta + 2r}
    (-1)^k \left( 1 + q + q^2 + \dots + q^{\nn_{\gamma}(k)}  \right) (q^s)^k
  \]
\end{theorem}
\begin{remark}
  To make $\nn_\gamma$ fully explicit, one could also expand the arch shorthand to
  \[
    \nn_\gamma(k)
    \coloneqq \begin{cases}
      k + 2r & \text{if } {-2r} \le k \le -r \\
      \left\lfloor \frac{k+r}{2} \right\rfloor + r & \text{if }{-r} \le k \le \ell-r \\
      \frac{\ell-1}{2} + r & \text{if } \ell - r \le k \le 2\delta + r \\
      \left\lfloor \frac{(\ell+2\delta+r)-k}{2} \right\rfloor + r & \text{if } 2\delta + r \le k \le \ell + 2\delta + r\\
      (\ell + 2\delta + 2r) - k & \text{if } \ell + 2\delta + r \le k \le \ell + 2\delta + 2r.
    \end{cases} \]
\end{remark}

From the identity
\[
  \Orb(\gamma, \mathbf{1}_{K'_{S, r}}, s)
  = \Orb(\gamma, \mathbf{1}_{K'_{S, \le r}}, s)
  - \Orb(\gamma, \mathbf{1}_{K'_{S, \le (r-1)}}, s)
\]
we can then write the following equivalent formulation.
\begin{corollary}
  Retaining the setting of the previous theorem, we have for any $r \ge 1$ the formula
  \[
    \Orb(\gamma, \mathbf{1}_{K'_{S, r}}, s)
    = \sum_{k = -2r}^{\ell + 2\delta + 2r}
    (-1)^k q^{\nn_{\gamma}(k)}
    (1+q^{-1})^{\mathbf{1}[k \in \mathcal I_{\gamma, r}]}
    (q^s)^k
  \]
  where $\mathcal I_{\gamma, r}$ is the set of indices defined by
  \begin{align*}
    \mathcal{I}_{\gamma, r}
    &\coloneqq \left\{ -(2r-1), -(2r-2), -(2r-3), \dots, -(r+1) \right\} \\
    &\sqcup \{-r, -r+2, -r+4 \dots, -r+\ell-1 \} \\
    &\sqcup \{ 2\delta+r+1, 2\delta+r+3, \dots, 2\delta+r+1, 2\delta+r+3, \dots, 2\delta+\ell+r \} \\
    &\sqcup \{ \ell+2\delta+r+1, \ell+2\delta+r+2, \dots, \ell+2\delta+2r-1 \}.
  \end{align*}
\end{corollary}

\begin{example}
  If $r=3$, $\ell=5$, and $\delta=100$, the formulas above read
  \begin{align*}
  \Orb(\gamma, \mathbf{1}_{K'_{S, \le 3}}, s)
  &= q^{-6s} \\
  &- (q+1) \cdot q^{-5s} \\
  &+ (q^2+q+1) \cdot q^{-4s} \\
  &- (q^3+q^2+q+1) \cdot q^{-3s} \\
  &+ (q^3+q^2+q+1) \cdot q^{-2s} \\
  &- (q^4+q^3+q^2+q+1) \cdot q^{-s} \\
  &+ (q^4+q^3+q^2+q+1) \cdot q^{0} \\
  &- (q^5+q^4+q^3+q^2+q+1) \cdot q^{s} \\
  &+ (q^5+q^4+q^3+q^2+q+1) \cdot q^{2s} \\
  &\vdotswithin= \\
  &+ (q^5+q^4+q^3+q^2+q+1) \cdot q^{204s} \\
  &- (q^4+q^3+q^2+q+1) \cdot q^{205s} \\
  &+ (q^4+q^3+q^2+q+1) \cdot q^{206s} \\
  &- (q^3+q^2+q+1) \cdot q^{207s} \\
  &+ (q^3+q^2+q+1) \cdot q^{208s} \\
  &- (q^2+q+1) \cdot q^{209s} \\
  &+ (q+1) \cdot q^{210s} \\
  &- q^{211s}.
  \end{align*}
  In the ellipses, all the omitted terms
  have the same coefficient $q^5+q^4+q^3+q^2+q+1$
  and alternate sign.
\end{example}

\begin{example}
  Continuing the previous example with
  $r=3$, $\ell=5$, and $\delta=100$, we have
  \begin{align*}
  \Orb(\gamma, \mathbf{1}_{K'_{S, 3}}, s)
  &= q^{-6s} \\
  &- (q+1) \cdot q^{-5s} \\
  &+ (q^2+q) \cdot q^{-4s} \\
  &- (q^3+q^2) \cdot q^{-3s} \\
  &+ q^3 \cdot q^{-2s} \\
  &- (q^4+q^3) \cdot q^{-s} \\
  &+ q^4 \cdot q^{0} \\
  &- (q^5+q^4) \cdot q^{s} \\
  &+ q^5 \cdot q^{2s} \\
  &- q^5 \cdot q^{3s} \\
  &+ q^5 \cdot q^{4s} \\
  &\vdotswithin= \\
  &+ q^5 \cdot q^{202s} \\
  &- q^6 \cdot q^{203s} \\
  &+ (q^5+q^4) \cdot q^{204s} \\
  &- q^4 \cdot q^{205s} \\
  &+ (q^4+q^3) \cdot q^{206s} \\
  &- q^3 \cdot q^{207s} \\
  &+ (q^3+q^2) \cdot q^{208s} \\
  &- (q^2+q) \cdot q^{209s} \\
  &+ (q+1) \cdot q^{210s} \\
  &- q^{211s}.
  \end{align*}
\end{example}

\subsection{Full explicit weighted orbital integral for the case where $\ell \ge 0$ is even}
When $\ell$ is even the formula has a change to the leading coefficient as well.
\begin{theorem}
  [Weighted orbital integral for even $\ell \ge 0$]
  \label{thm:full_orbital_ell_even}
  Let $r \ge 0$.
  Let $\gamma \in S_3(F)\rs$ match an element of $\U(\VV_3^-)$,
  and let $b$, $d$, $\delta$, $\ell$, $\lambda$ be as in
  \Cref{lem:S3_abcd} and \Cref{lem:parameter_constraints}.
  Suppose also $\ell \ge 0$ is even.
  Define
  \begin{align*}
    \nn_\gamma &\coloneqq \Arch_{[-2r, \lambda + 2\delta + 2r]}(r, \ell) \\
    \cc_\gamma &\coloneqq \Arch_{[\ell-r, \lambda - \ell + 2\delta + r]}
    (\delta - \ell/2, \min(2r, \lambda-\ell)).
  \end{align*}
  Then for any $r \ge 0$ we have:
  \begin{align*}
    \Orb(\gamma, \mathbf{1}_{K'_{S, \le r}}, s)
    &= \sum_{k = -2r}^{\lambda + 2\delta +  2r}
    (-1)^k \left( 1 + q + q^2 + \dots + q^{\nn_{\gamma}(k)}  \right) (q^s)^k \\
    &+ \sum_{k = \ell-r}^{2\delta + \lambda - \ell + r} \cc_\gamma(k) (-1)^k q^{\frac{\ell}{2}+r} (q^s)^k.
  \end{align*}
\end{theorem}

\begin{example}
  We provide an example for $r=5$, $\ell=2$, $\delta=4$, $\lambda=9$ for concreteness:
  \begin{align*}
    \Orb(\gamma, \mathbf{1}_{K'_{S, \le r}}, s)
    &= q^{-10s} \\
    &- (q + 1) \cdot q^{-9s} \\
    &+ (q^2 + q + 1) \cdot q^{-8s} \\
    &- (q^3 + q^2 + q + 1) \cdot q^{-7s} \\
    &+ (q^4 + q^3 + q^2 + q + 1) \cdot q^{-6s} \\
    &- (q^5 + q^4 + q^3 + q^2 + q + 1) \cdot q^{-5s} \\
    &+ (q^5 + q^4 + q^3 + q^2 + q + 1) \cdot q^{-4s} \\
    &- (q^6 + q^5 + q^4 + q^3 + q^2 + q + 1) \cdot q^{-3s} \\
    &+ (2q^6 + q^5 + q^4 + q^3 + q^2 + q + 1) \cdot q^{-2s} \\
    &- (3q^6 + q^5 + q^4 + q^3 + q^2 + q + 1) \cdot q^{-s} \\
    &+ (4q^6 + q^5 + q^4 + q^3 + q^2 + q + 1) \cdot q^{0} \\
    &- (4q^6 + q^5 + q^4 + q^3 + q^2 + q + 1) \cdot q^{s} \\
    &+ (5q^6 + q^5 + q^4 + q^3 + q^2 + q + 1) \cdot q^{2s} \\
    &- (5q^6 + q^5 + q^4 + q^3 + q^2 + q + 1) \cdot q^{3s} \\
    &+ (6q^6 + q^5 + q^4 + q^3 + q^2 + q + 1) \cdot q^{4s} \\
    &- (6q^6 + q^5 + q^4 + q^3 + q^2 + q + 1) \cdot q^{5s} \\
    &+ (7q^6 + q^5 + q^4 + q^3 + q^2 + q + 1) \cdot q^{6s} \\
    &- (7q^6 + q^5 + q^4 + q^3 + q^2 + q + 1) \cdot q^{7s} \\
    &+ (7q^6 + q^5 + q^4 + q^3 + q^2 + q + 1) \cdot q^{8s} \\
    &- (7q^6 + q^5 + q^4 + q^3 + q^2 + q + 1) \cdot q^{9s} \\
    &+ (7q^6 + q^5 + q^4 + q^3 + q^2 + q + 1) \cdot q^{10s} \\
    &- (7q^6 + q^5 + q^4 + q^3 + q^2 + q + 1) \cdot q^{11s} \\
    &+ (6q^6 + q^5 + q^4 + q^3 + q^2 + q + 1) \cdot q^{12s} \\
    &- (6q^6 + q^5 + q^4 + q^3 + q^2 + q + 1) \cdot q^{13s} \\
    &+ (5q^6 + q^5 + q^4 + q^3 + q^2 + q + 1) \cdot q^{14s} \\
    &- (5q^6 + q^5 + q^4 + q^3 + q^2 + q + 1) \cdot q^{15s} \\
    &+ (4q^6 + q^5 + q^4 + q^3 + q^2 + q + 1) \cdot q^{16s} \\
    &- (4q^6 + q^5 + q^4 + q^3 + q^2 + q + 1) \cdot q^{17s} \\
    &+ (3q^6 + q^5 + q^4 + q^3 + q^2 + q + 1) \cdot q^{18s} \\
    &- (2q^6 + q^5 + q^4 + q^3 + q^2 + q + 1) \cdot q^{19s} \\
    &+ (q^6 + q^5 + q^4 + q^3 + q^2 + q + 1) \cdot q^{20s} \\
    &- (q^5 + q^4 + q^3 + q^2 + q + 1) \cdot q^{21s} \\
    &+ (q^5 + q^4 + q^3 + q^2 + q + 1) \cdot q^{22s} \\
    &- (q^4 + q^3 + q^2 + q + 1) \cdot q^{23s} \\
    &+ (q^3 + q^2 + q + 1) \cdot q^{24s} \\
    &- (q^2 + q + 1) \cdot q^{25s} \\
    &+ (q + 1) \cdot q^{26s} \\
    &- q^{27s}.
  \end{align*}
\end{example}


\subsection{Full explicit weighted orbital integral for the case $\ell < 0$}
In this case, $\ell = \delta = 2v(b) = 2v(d) < 0$.
We will just state the relevant theorem in terms of $v(b)$ and $v(d)$, omitting $\ell$ and $\delta$.
\begin{theorem}
  [Weighted orbital integral when $v(b)=v(d)<0$]
  \label{thm:full_orbital_ell_neg}
  Let $r \ge 0$.
  Let $\gamma \in S_3(F)\rs$ match an element of $\U(\VV_3^-)$,
  and let $b$, $d$, $\lambda$ be as in
  \Cref{lem:S3_abcd} and \Cref{lem:parameter_constraints}.
  Suppose also $v(b) = v(d) < 0$.

  Then if $|v(d)| > r$, the entire orbital integral is zero.
  Otherwise define
  \begin{align*}
    \nn_\gamma &\coloneqq \Arch_{[-2r, \lambda+2r-4|v(d)|]}(r-|v(d)|, 0) \\
    \cc_\gamma &\coloneqq \Arch_{[-r-|v(d)|, \lambda+r-3|v(d)|]} (0, \min(2r-2|v(d)|, \lambda)).
  \end{align*}
  Then for any $r \ge 0$ we have the formula:
  \begin{align*}
    \Orb(\gamma, \mathbf{1}_{K'_{S, \le r}}, s)
    &= \sum_{k = -2r}^{\lambda + 2r-4|v(d)|}
    (-1)^k \left( 1 + q + q^2 + \dots + q^{\nn_{\gamma}(k)} \right) (q^s)^k \\
    &+ \sum_{k = -r-|v(d)|}^{\lambda+r-3|v(d)|} \cc_\gamma(k) (-1)^k q^{r-|v(d)|} (q^s)^k \\
  \end{align*}
\end{theorem}

\begin{example}
  \label{ex:ell_neg_top_case}
  In the case where $r = |v(d)|$, the orbital simplifies to just
  \[ \Orb(\gamma, \mathbf{1}_{K'_{S, \le r}}, s)
    = \sum_{k=-2r}^{\lambda-2r} (-1)^k (q^s)^k. \]
\end{example}

\begin{example}
  If $\lambda = 7$, $v(d) = -5$ and $r = 12$ we have
  \begin{align*}
    \Orb(\gamma, \mathbf{1}_{K'_{S, \le r}}, s)
    &= q^{-24s} \\
    &- (q+1) \cdot q^{-23s} \\
    &+ (q^2+q+1) \cdot q^{-22s} \\
    &- (q^3+q^2+q+1) \cdot q^{-21s} \\
    &+ (q^4+q^3+q^2+q+1) \cdot q^{-20s} \\
    &- (q^5+q^4+q^3+q^2+q+1) \cdot q^{-19s} \\
    &+ (q^6+q^5+q^4+q^3+q^2+q+1) \cdot q^{-18s} \\
    &- (q^7+q^6+q^5+q^4+q^3+q^2+q+1) \cdot q^{-17s} \\
    &+ (2q^7+q^6+q^5+q^4+q^3+q^2+q+1) \cdot q^{-16s} \\
    &- (2q^7+q^6+q^5+q^4+q^3+q^2+q+1) \cdot q^{-15s} \\
    &+ (2q^7+q^6+q^5+q^4+q^3+q^2+q+1) \cdot q^{-14s} \\
    &- (2q^7+q^6+q^5+q^4+q^3+q^2+q+1) \cdot q^{-13s} \\
    &+ (3q^7+q^6+q^5+q^4+q^3+q^2+q+1) \cdot q^{-12s} \\
    &- (3q^7+q^6+q^5+q^4+q^3+q^2+q+1) \cdot q^{-11s} \\
    &+ (4q^7+q^6+q^5+q^4+q^3+q^2+q+1) \cdot q^{-10s} \\
    &- (4q^7+q^6+q^5+q^4+q^3+q^2+q+1) \cdot q^{-9s} \\
    &+ (4q^7+q^6+q^5+q^4+q^3+q^2+q+1) \cdot q^{-8s} \\
    &- (4q^7+q^6+q^5+q^4+q^3+q^2+q+1) \cdot q^{-7s} \\
    &+ (4q^7+q^6+q^5+q^4+q^3+q^2+q+1) \cdot q^{-6s} \\
    &- (4q^7+q^6+q^5+q^4+q^3+q^2+q+1) \cdot q^{-5s} \\
    &+ (4q^7+q^6+q^5+q^4+q^3+q^2+q+1) \cdot q^{-4s} \\
    &- (4q^7+q^6+q^5+q^4+q^3+q^2+q+1) \cdot q^{-3s} \\
    &+ (4q^7+q^6+q^5+q^4+q^3+q^2+q+1) \cdot q^{-2s} \\
    &- (3q^7+q^6+q^5+q^4+q^3+q^2+q+1) \cdot q^{-s} \\
    &+ (3q^7+q^6+q^5+q^4+q^3+q^2+q+1) \cdot q^{0} \\
    &- (2q^7+q^6+q^5+q^4+q^3+q^2+q+1) \cdot q^{s} \\
    &+ (2q^7+q^6+q^5+q^4+q^3+q^2+q+1) \cdot q^{2s} \\
    &- (q^7+q^6+q^5+q^4+q^3+q^2+q+1) \cdot q^{3s} \\
    &+ (q^7+q^6+q^5+q^4+q^3+q^2+q+1) \cdot q^{4s} \\
    &- (q^6+q^5+q^4+q^3+q^2+q+1) \cdot q^{5s} \\
    &+ (q^5+q^4+q^3+q^2+q+1) \cdot q^{6s} \\
    &- (q^4+q^3+q^2+q+1) \cdot q^{7s} \\
    &+ (q^3+q^2+q+1) \cdot q^{8s} \\
    &- (q^2+q+1) \cdot q^{9s} \\
    &+ (q+1) \cdot q^{10s} \\
    &- q^{11s}.
  \end{align*}
\end{example}
\begin{example}
  If $\lambda = 2025$, $v(d) = -5$ and $r = 12$ we have
  \begin{align*}
    \Orb(\gamma, \mathbf{1}_{K'_{S, \le r}}, s)
    &= q^{-24s} \\
    &- (q+1) \cdot q^{-23s} \\
    &+ (q^2+q+1) \cdot q^{-22s} \\
    &- (q^3+q^2+q+1) \cdot q^{-21s} \\
    &+ (q^4+q^3+q^2+q+1) \cdot q^{-20s} \\
    &- (q^5+q^4+q^3+q^2+q+1) \cdot q^{-19s} \\
    &+ (q^6+q^5+q^4+q^3+q^2+q+1) \cdot q^{-18s} \\
    &- (q^7+q^6+q^5+q^4+q^3+q^2+q+1) \cdot q^{-17s} \\
    &+ (2q^7+q^6+q^5+q^4+q^3+q^2+q+1) \cdot q^{-16s} \\
    &- (2q^7+q^6+q^5+q^4+q^3+q^2+q+1) \cdot q^{-15s} \\
    &+ (2q^7+q^6+q^5+q^4+q^3+q^2+q+1) \cdot q^{-14s} \\
    &- (2q^7+q^6+q^5+q^4+q^3+q^2+q+1) \cdot q^{-13s} \\
    &+ (3q^7+q^6+q^5+q^4+q^3+q^2+q+1) \cdot q^{-12s} \\
    &- (3q^7+q^6+q^5+q^4+q^3+q^2+q+1) \cdot q^{-11s} \\
    &+ (4q^7+q^6+q^5+q^4+q^3+q^2+q+1) \cdot q^{-10s} \\
    &- (4q^7+q^6+q^5+q^4+q^3+q^2+q+1) \cdot q^{-9s} \\
    &+ (5q^7+q^6+q^5+q^4+q^3+q^2+q+1) \cdot q^{-8s} \\
    &- (5q^7+q^6+q^5+q^4+q^3+q^2+q+1) \cdot q^{-7s} \\
    &+ (6q^7+q^6+q^5+q^4+q^3+q^2+q+1) \cdot q^{-6s} \\
    &- (6q^7+q^6+q^5+q^4+q^3+q^2+q+1) \cdot q^{-5s} \\
    &+ (7q^7+q^6+q^5+q^4+q^3+q^2+q+1) \cdot q^{-4s} \\
    &- (7q^7+q^6+q^5+q^4+q^3+q^2+q+1) \cdot q^{-3s} \\
    &+ (8q^7+q^6+q^5+q^4+q^3+q^2+q+1) \cdot q^{-2s} \\
    &- (8q^7+q^6+q^5+q^4+q^3+q^2+q+1) \cdot q^{-s} \\
    &+ (8q^7+q^6+q^5+q^4+q^3+q^2+q+1) \cdot q^{0} \\
    &- (8q^7+q^6+q^5+q^4+q^3+q^2+q+1) \cdot q^{s} \\
    &+ (8q^7+q^6+q^5+q^4+q^3+q^2+q+1) \cdot q^{2s} \\
    &- (8q^7+q^6+q^5+q^4+q^3+q^2+q+1) \cdot q^{3s} \\
    &+ (8q^7+q^6+q^5+q^4+q^3+q^2+q+1) \cdot q^{4s} \\
    &\vdotswithin= \\
    &- (8q^7+q^6+q^5+q^4+q^3+q^2+q+1) \cdot q^{2007s} \\
    &+ (8q^7+q^6+q^5+q^4+q^3+q^2+q+1) \cdot q^{2008s} \\
    &- (7q^7+q^6+q^5+q^4+q^3+q^2+q+1) \cdot q^{2009s} \\
    &+ (7q^7+q^6+q^5+q^4+q^3+q^2+q+1) \cdot q^{2010s} \\
    &- (6q^7+q^6+q^5+q^4+q^3+q^2+q+1) \cdot q^{2011s} \\
    &+ (6q^7+q^6+q^5+q^4+q^3+q^2+q+1) \cdot q^{2012s} \\
    &- (5q^7+q^6+q^5+q^4+q^3+q^2+q+1) \cdot q^{2013s} \\
    &+ (5q^7+q^6+q^5+q^4+q^3+q^2+q+1) \cdot q^{2014s} \\
    &- (4q^7+q^6+q^5+q^4+q^3+q^2+q+1) \cdot q^{2015s} \\
    &+ (4q^7+q^6+q^5+q^4+q^3+q^2+q+1) \cdot q^{2016s} \\
    &- (3q^7+q^6+q^5+q^4+q^3+q^2+q+1) \cdot q^{2017s} \\
    &+ (3q^7+q^6+q^5+q^4+q^3+q^2+q+1) \cdot q^{2018s} \\
    &- (2q^7+q^6+q^5+q^4+q^3+q^2+q+1) \cdot q^{2019s} \\
    &+ (2q^7+q^6+q^5+q^4+q^3+q^2+q+1) \cdot q^{2020s} \\
    &- (q^7+q^6+q^5+q^4+q^3+q^2+q+1) \cdot q^{2021s} \\
    &+ (q^7+q^6+q^5+q^4+q^3+q^2+q+1) \cdot q^{2022s} \\
    &- (q^6+q^5+q^4+q^3+q^2+q+1) \cdot q^{2023s} \\
    &+ (q^5+q^4+q^3+q^2+q+1) \cdot q^{2024s} \\
    &- (q^4+q^3+q^2+q+1) \cdot q^{2025s} \\
    &+ (q^3+q^2+q+1) \cdot q^{2026s} \\
    &- (q^2+q+1) \cdot q^{2027s} \\
    &+ (q+1) \cdot q^{2028s} \\
    &- q^{2029s}.
  \end{align*}
\end{example}
\subsection{Derivatives of the orbital integrals}
We do not explicitly write the derivatives for the above three theorems in this paper,
since they are not used later on.
However, we mention that these formulas could be easily extracted by using
\Cref{lem:derivative_nn} and \Cref{lem:derivative_cc} later on;
indeed, we use certain combinations of the derivative later on in
\Cref{thm:group_kernel_full}.
