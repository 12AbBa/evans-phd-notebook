\section{Setup of the orbital integral for $S_2(F) \times V_2'(F)$}
\label{sec:orbitalFJ1}

\subsection{Iwasawa decomposition}
The overall method is to take the Iwasawa decomposition in $KAN$ form:
every element in $h \in \GL_2(F)$ may be parametrized as
\[ h = k \begin{bmatrix} x_1 & 0 \\ 0 & x_2 \end{bmatrix}
  \begin{bmatrix} 1 & y \\ 0 & 1 \end{bmatrix} \]
where $k \in K' = \GL_2(\OO_F)$, $x_1, x_2 \in \OO_F^\times$ and $y \in \OO_F$.
Because the orbits are invariant under conjugation by $K'$,
the parameter $k$ can be discarded.
The Haar measure integrated over is then given just by
\[ \odif[\times] x_1 \odif[\times] x_2 \odif y \]
i.e.\ we take multiplicative Haar measure on $F^\times$
(normalized so that $\OO_F^\times$ has volume $1$)
and additive Haar measure on $F$
(so $\OO_F$ has volume $1$).

\subsection{Action of upper triangular matrices on $(\gamma, \uu, \vv^\top)$.}
We now compute the action of an arbitrary
\[ h = \begin{bmatrix} x_1 & 0 \\ 0 & x_2 \end{bmatrix}
  \begin{bmatrix} 1 & y \\ 0 & 1 \end{bmatrix} \]
on $(\gamma, \uu, \vv^\top)$.
The main term is given by
\begin{align*}
  h \gamma h^{-1}
  &=
  \begin{bmatrix} x_1 & 0 \\ 0 & x_2 \end{bmatrix}
  \begin{bmatrix} 1 & y \\ 0 & 1 \end{bmatrix}
  \begin{bmatrix} a & b \\ c & d \end{bmatrix}
  \begin{bmatrix} 1 & -y \\ 0 & 1 \end{bmatrix}
  \begin{bmatrix} x_1^{-1} & 0 \\ 0 & x_2^{-1} \end{bmatrix} \\
  &=
  \begin{bmatrix} x_1 & 0 \\ 0 & x_2 \end{bmatrix}
  \begin{bmatrix} cy + a & -cy^2+(d-a)y+b \\ c & -cy+d \end{bmatrix}
  \begin{bmatrix} x_1^{-1} & 0 \\ 0 & x_2^{-1} \end{bmatrix} \\
  &=
  \begin{bmatrix} cy + a & \frac{x_1}{x_2} \cdot \left( -cy^2+(d-a)y+b \right) \\
    \frac{x_2}{x_1} \cdot c & -cy+d \end{bmatrix}
\end{align*}
Meanwhile, we have
\begin{align*}
  h \uu &=
    \begin{bmatrix} x_1 & 0 \\ 0 & x_2 \end{bmatrix}
    \begin{bmatrix} 1 & y \\ 0 & 1 \end{bmatrix}
    \begin{bmatrix} 0 \\ 1 \end{bmatrix}
    = \begin{bmatrix} x_1 y \\ x_2 \end{bmatrix} \\
  \vv^\top h^{-1} &=
    \begin{bmatrix} 0 & e \end{bmatrix}
    \begin{bmatrix} 1 & -y \\ 0 & 1 \end{bmatrix}
    \begin{bmatrix} x_1^{-1} & 0 \\ 0 & x_2^{-1} \end{bmatrix}
    = \begin{bmatrix} 0 & \frac{e}{x_2} \end{bmatrix}.
\end{align*}

\subsection{Description of support}
From now on we fix the notation
\begin{align*}
  n_1 &\coloneqq v(x_1) \\
  n_2 &\coloneqq v(x_2).
\end{align*}

For a given $r \ge 0$, we find that $h$ contributes to the integral exactly
if $h\uu$ and $\vv^\top h\inv$ have $\OO_F$-entries,
and all the entries of $h \gamma h\inv$ are in $\varpi^{-r}\OO_F$.
The former condition is just saying that
\begin{align*}
  v(y) &\ge -n_1 \\
  0 &\le n_2 \le v(e).
\end{align*}
Now we consider the entries of $h\gamma h\inv$.
First, because $a$ and $d$ are units by \Cref{assume:a_odd},
and $r \ge 0$, it follows that
\begin{align*}
  cy + a, -cy + d \in \varpi^{-r}\OO_F
  &\iff cy \in \varpi^{-r}\OO_F \\
  &\iff v(y) \ge -r - v(c).
\end{align*}
Moreover,
\[ \frac{x_2}{x_1} \cdot c \in \varpi^{-r} \OO_F
  \iff n_2 + v(c) \ge r + n_1. \]
As for the quadratic constraint, let us write
\[
  -cy^2 + (d-a)y + b
  = c\left( -\left( y - \frac{d-a}{2c} \right)^2 + \frac bc \right).
\]
Because $b \bar c = 1 - a \bar a$ has odd valuation,
it follows that $\frac b c = \frac{1-a \bar a}{c\bar c}$ has odd valuation to.
On the other hand, the valuation of
$\left( y - \frac{d-a}{2c} \right)^2$ is necessarily even.
So the terms $\left( y - \frac{d-a}{2c} \right)^2$ and $\frac bc$
never have the same valuation and hence
\[ v\left( -\left( y - \frac{d-a}{2c} \right)^2 + \frac bc  \right)
  = \min \left\{ 2v\left( y - \frac{d-a}{2c} \right), v\left( \frac bc \right) \right\}. \]
This implies
\begin{align*}
  &\phantom= v\left( \frac{x_1}{x_2} \cdot \left( -cy^2+(d-a)y+b \right) \right) \\
  &= n_1 - n_2 + \min \left\{ 2v\left( y - \frac{d-a}{2c} \right), v\left( \frac bc \right) \right\}.
\end{align*}

We now collate all the constraints we obtained together.
In summary, the triple $(x_1, x_2, y) \in \OO_F^\times \times \OO_F^\times \times \OO_F$
contributes to the orbital integral exactly if the following identities hold:
\begin{align*}
  0 &\le n_2 \le v(e) \\
  n_2  - v(b) - r &\le n_1 \le n_2 + v(c) + r \\
  v(y) &\ge \max\left(-n_1, -r-v(c)\right) \\
  v\left( y - \frac{d-a}{2c} \right) &\ge \frac{n_2 - n_1 - v(c) - r}{2}.
\end{align*}
