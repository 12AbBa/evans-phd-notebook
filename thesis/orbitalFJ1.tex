\section{Setup of the orbital integral for $S_2(F) \times V_2'(F)$}
\label{sec:orbitalFJ1}

\subsection{Iwasawa decomposition}
The overall method is to take the Iwasawa decomposition in $KAN$ form:
every element in $h \in \GL_2(F)$ may be parametrized as
\[ h = k \begin{bmatrix} x_1 & 0 \\ 0 & x_2 \end{bmatrix}
  \begin{bmatrix} 1 & y \\ 0 & 1 \end{bmatrix} \]
where $k \in K' = \GL_2(\OO_F)$, $x_1, x_2 \in \OO_F^\times$ and $y \in \OO_F$.
Because the orbits are invariant under conjugation by $K'$,
the parameter $k$ can be discarded.
The Haar measure in these coordinates
\[ \left\lvert \frac{x_1}{x_2} \right\rvert \odif[\times] x_1 \odif[\times] x_2 \odif y \]
where we take multiplicative Haar measure on $F^\times$
(normalized so that $\OO_F^\times$ has volume $1$)
and additive Haar measure on $F$ (so $\OO_F$ has volume $1$).

\subsection{Action of upper triangular matrices on $(\gamma, \uu, \vv^\top)$.}
We now compute the action of an arbitrary
\[ h = \begin{bmatrix} x_1 & 0 \\ 0 & x_2 \end{bmatrix}
  \begin{bmatrix} 1 & y \\ 0 & 1 \end{bmatrix} \]
on $(\gamma, \uu, \vv^\top)$.
The main term is given by
\begin{align*}
  h \gamma h^{-1}
  &=
  \begin{bmatrix} x_1 & 0 \\ 0 & x_2 \end{bmatrix}
  \begin{bmatrix} 1 & y \\ 0 & 1 \end{bmatrix}
  \begin{bmatrix} a & b \\ c & d \end{bmatrix}
  \begin{bmatrix} 1 & -y \\ 0 & 1 \end{bmatrix}
  \begin{bmatrix} x_1^{-1} & 0 \\ 0 & x_2^{-1} \end{bmatrix} \\
  &=
  \begin{bmatrix} x_1 & 0 \\ 0 & x_2 \end{bmatrix}
  \begin{bmatrix} cy + a & -cy^2+(d-a)y+b \\ c & -cy+d \end{bmatrix}
  \begin{bmatrix} x_1^{-1} & 0 \\ 0 & x_2^{-1} \end{bmatrix} \\
  &=
  \begin{bmatrix} cy + a & \frac{x_1}{x_2} \cdot \left( -cy^2+(d-a)y+b \right) \\
    \frac{x_2}{x_1} \cdot c & -cy+d \end{bmatrix}
\end{align*}
Meanwhile, we have
\begin{align*}
  h \uu &=
    \begin{bmatrix} x_1 & 0 \\ 0 & x_2 \end{bmatrix}
    \begin{bmatrix} 1 & y \\ 0 & 1 \end{bmatrix}
    \begin{bmatrix} 0 \\ 1 \end{bmatrix}
    = \begin{bmatrix} x_1 y \\ x_2 \end{bmatrix} \\
  \vv^\top h^{-1} &=
    \begin{bmatrix} 0 & e \end{bmatrix}
    \begin{bmatrix} 1 & -y \\ 0 & 1 \end{bmatrix}
    \begin{bmatrix} x_1^{-1} & 0 \\ 0 & x_2^{-1} \end{bmatrix}
    = \begin{bmatrix} 0 & \frac{e}{x_2} \end{bmatrix}.
\end{align*}

\subsection{Description of support}
From now on we fix the notation
\begin{align*}
  n_1 &\coloneqq v(x_1) \\
  n_2 &\coloneqq v(x_2).
\end{align*}
Note that although $n_2 \ge 0$, the value of $n_1$ will often be non-positive.
In fact $n_1$ is not particularly simple to work with and we will
prefer to introduce the notation
\[ m \coloneqq n_2 + v(c) + r - n_1 \]
instead to use as a summation variable.
(This is chosen so that $\frac{x_2}{x_1} \cdot c \in \varpi^{-r} \OO_F \iff m \ge 0$.)

\subsubsection{Collating the linear constraints}
For a given $r \ge 0$, we find that $h$ contributes to the integral exactly
if $h\uu$ and $\vv^\top h\inv$ have $\OO_F$-entries,
and all the entries of $h \gamma h\inv$ are in $\varpi^{-r}\OO_F$.
The former condition is just saying that
\begin{align*}
  v(y) &\ge -n_1, \\
  0 &\le n_2 \le v(e).
\end{align*}
Now we consider the entries of $h\gamma h\inv$.
First, because $a$ and $d$ are units by \Cref{assume:a_odd},
and $r \ge 0$, it follows that
\begin{align*}
  cy + a, -cy + d \in \varpi^{-r}\OO_F
  &\iff cy \in \varpi^{-r}\OO_F \\
  &\iff v(y) \ge -v(c) - r.
\end{align*}
Moreover,
\[ \frac{x_2}{x_1} \cdot c \in \varpi^{-r} \OO_F
  \iff n_2 + v(c) -n_1 \ge - r \iff m \ge 0. \]
In summary, up until now we have the following requirements imposed:
\begin{equation}
  \begin{aligned}
  0 &\le n_2 \le v(e) \\
  0 &\le m \\
  v(y) &\ge \max(-n_1, -v(c) - r) \\
  % &= \max(m - n_2 - v(c) - r, - v(c) - r) \\
  &= \max(m-n_2, 0) - v(c) - r.
  \end{aligned}
  \label{eq:linear_constraints}
\end{equation}

As for the quadratic constraint, we seek $y$ such that
\begin{align*}
  \phantom\iff \frac{x_1}{x_2} \cdot (-cy^2+(d-a)y+b) &\in \varpi^{-r} \OO_F \\
  \iff v\left(-y^2+ \frac{d-a}{c} y + \frac bc \right) &\ge n_2 - n_1 - v(c) - r \\
  &= m - 2v(c) - 2r.
\end{align*}

\subsection{Analyzing the quadratic constraint}
As before, we complete the square:
\[
  -y^2+ \frac{d-a}{c} y + \frac bc
  = -\left( y - \frac{d-a}{2c} \right)^2 + \frac bc + \frac{(d-a)^2}{4c^2}
\]
Because $b \bar c = 1 - a \bar a$ has odd valuation,
it follows that $\frac b c = \frac{1-a \bar a}{c\bar c}$ has odd valuation to.
On the other hand, $\frac{(d-a)^2}{4c^2}$ has even valuation.

This motivates us to introduce the parameter
\[ \theta \coloneqq \min \left( v(b)+v(c), 2v(d-a) \right) \ge 0. \]
Note that $v(b) + v(c)$ is odd,
so $\theta$ takes the odd value if $v(b)+v(c) < 2v(d-a)$ and the even value otherwise.
This definition ensures that
\[ v \left( \frac bc + \frac{(d-a)^2}{4c^2} \right) = \theta - 2v(c). \]
Then we can consider two cases based on $\theta$.
(We start numbering these Case 5 and Case 6 to prevent confusion
with the cases introduced in \Cref{sec:orbital1}.)
\begin{description}
  \item[Case 5]
  Let's assume first that
  \[ \theta - 2v(c) \ge m - 2v(c) - 2r \iff m \le \theta + 2r. \]
  Then the only additional condition on $y$ is that
  \[
    v\left( y - \frac{d-a}{2c} \right)
    \ge \left\lceil \frac{m}{2} \right\rceil - v(c) - r.
  \]
  We refer to this as \textbf{Case 5}.

  \item[Case 6\ts+ / Case 6\ts-]
  Otherwise assume that
  \[ \theta - 2v(c) < m - 2v(c) - 2r \iff m > \theta + 2r. \]
  Then in order for $y$ to satisfy the constraint,
  we would need to be in a situation where $2v(y - \frac{d-a}{2c}) = \theta - 2v(c)$.
  So this case could only arise at all when $\theta$ is even, that is
  \[ 0 \le 2v(d-a) = \theta < v(b) + v(c) \]
  (note that $v(d-a) \ge 0$ because $a$ and $d$ are units).
  As the quantity $\frac bc + \frac{(d-a)^2}{4c^2}$ must be a perfect square,
  we denote it by $\tau^2$, with
  \[ v(\tau) = \frac{\theta}{2} - v(c). \]
  This gives us the factorization
  \[ \frac bc = \tau^2 - \frac{(d-a)^2}{4c^2}
    = \left( \tau - \frac{d-a}{2c} \right) \left( \tau + \frac{d-a}{2c} \right). \]
  The left-hand side has odd valuation $v(b) - v(c)$,
  so the two factors on the right have unequal valuations
  and hence exactly one of them has valuation the same as $v(\frac{d-a}{2c}) = v(\tau)$.
  Hence, we agree to fix the choice of the square root $\tau$ so that
  \begin{align*}
    v\left( \tau + \frac{d-a}{2c} \right) &= v(b) - v(c) - v(\tau) = v(b) - \frac{\theta}{2} \\
    v\left( \tau - \frac{d-a}{2c} \right) &= v(\tau) = \frac{\theta}{2} - v(c)
  \end{align*}
  and in particular
  $v\left( \tau + \frac{d-a}{2c} \right) > v\left( \tau - \frac{d-a}{2c} \right)$.

  In any case, the constraint on $y$ is that
  \begin{align*}
    v\left( y - \left( \frac{d-a}{2c} \pm \tau \right) \right)
      &\ge \left( m - 2v(c) - 2r \right) - v(\tau) \\
      &= \left( m - 2v(c) - 2r \right) - \left( \frac{\theta}{2} - v(c) \right) \\
      &= m - \frac{\theta}{2} - v(c) - 2r \\
    v\left( y - \left( \frac{d-a}{2c} \mp \tau \right) \right) &= v(\tau)
      = \frac{\theta}{2} - v(c).
  \end{align*}
  By assumption, the second equation is true
  whenever the first inequality is and we may disregard it.
  \textbf{Case 6\ts+} refers to the situation where the $\pm$ sign is $+$
  and \textbf{Case 6\ts-} refers to the situation where the $\mp$ sign is $-$.
  And these cases must be disjoint because the right-hand sides above are unequal.
\end{description}

\subsubsection{Analysis of Case 5}
The triple $(x_1, x_2, y) \in \OO_F^\times \times \OO_F^\times \times \OO_F$
contributes to the orbital integral in Case 5 exactly if the following identities hold:
\begin{align*}
  0 &\le n_2 \le v(e) \\
  0 &\le m \le \theta + 2r \\
  v(y) &\ge \max(m-n_2,0) - v(c) - r \\
  v\left( y - \frac{d-a}{2c} \right) &\ge \left\lceil \frac{m}{2} \right\rceil - v(c) - r.
\end{align*}
However, from the definitions we already know that
\[ v\left( \frac{d-a}{2c} - 0 \right)
  \ge \frac{\theta - 2v(c)}{2} \ge \frac{m}{2} - v(c) - r \]
so the disks in the last two conditions have nonempty intersection.
Hence the earlier \Cref{prop:no_mastercard} applies to tell us that
the locus of valid $y$ is a single disk whose volume in $\OO_F$ is given by
\[ q^{-\max\left( m-n_2, \left\lceil m/2 \right\rceil, 0 \right) + v(c) + r}. \]
The volume contribution for and $x_1 \in \OO_F^\times$ and $x_2 \in \OO_F^\times$
is also $1$, because $v(x_1)$ and $v(x_2)$ are fixed.
Hence the volume element contributed by this portion is exactly given by
\begin{align*}
  \left\lvert \frac{x_1}{x_2} \right\rvert q^{n_2 - n_1} \Vol(\{ y \mid \dots \})
  &= q^{n_2 - n_1 -\max\left( m-n_2, \left\lceil m/2 \right\rceil, 0 \right) + v(c) + r} \\
  &= q^{m - \max\left( m-n_2, \left\lceil m/2 \right\rceil, 0 \right)}.
\end{align*}
And again, this case is summed over
\[ 0 \le n_2 \le v(e), \qquad 0 \le m \le \theta + 2r. \]

\subsubsection{Analysis of Case 6\ts+ and Case 6\ts-}
Again, this case could only occur if $\theta$ is even.
The triple $(x_1, x_2, y) \in \OO_F^\times \times \OO_F^\times \times \OO_F$
contributes to the orbital integral in Case 6\ts+ and Case 6\ts-
exactly if the following identities hold:
\begin{align*}
  0 &\le n_2 \le v(e) \\
  \theta + 2r &< m \\
  v(y) &\ge \max(m-n_2,0) - v(c) - r \\
  v\left( y - \left( \frac{d-a}{2c} \pm \tau \right) \right) &\ge m - \frac{\theta}{2} - v(c) - 2r.
\end{align*}
The last two inequalities specify disks.
So in each case, via \Cref{prop:no_mastercard}
we get a nonzero contribution if and only if the distance between the centers
$0$ and $\frac{d-a}{2c} \pm \tau$ has valuation at least
that of the smaller of the two right-hand sides, that is
\begin{align*}
  v\left( \frac{d-a}{2c} \pm \tau \right)
  &\ge \min\left( \max(m-n_2,0) - v(c) - r, m - \frac{\theta}{2} - v(c) - 2r \right) \\
  &= \min\left( \max(m-n_2,0), m - \frac{\theta}{2} - r \right) - v(c) - r.
\end{align*}
Hence the upper bound on $m$ is given by two different requirements,
depending on which of the two values of
$v\left( \frac{d-a}{2c} \pm \tau \right) + v(c) + r$ is given by the case:
\begin{itemize}
  \ii In Case 6\ts+, we need at least one of the inequalities
  \[
    \begin{cases}
    \max(m-n_2, 0) \le v(b) - \frac{\theta}{2} + v(c) + r, \\
    m \le v(b) + v(c) + 2r
    \end{cases}
  \]
  to hold.
  Now the inequality $0 \le v(b) - \frac{\theta}{2} + v(c) + r$ is always true,
  as $\theta < v(b) + v(c)$, so we can disregard it.
  Therefore this can be rewritten as just
  \[ m \le \max\left( r, n_2 - \frac{\theta}{2} \right) + v(b) + v(c) + r. \]

  \ii In Case 6\ts-, we need at least one of the inequalities
  \[
    \begin{cases}
      \max(m-n_2, 0) \le \frac{\theta}{2} + r \\
      m \le \theta + 2r
    \end{cases}
  \]
  to hold.
  But $m \le \theta+2r$ is always false and $0 \le \frac{\theta}{2} + r$ is always true,
  so this simplifies to
  \[ m \le n_2 + \frac{\theta}{2} + r. \]
\end{itemize}
Assuming $m$ lies in the valid range so that the locus of valid $y$ is nonempty,
it follows that the volume is given exactly by
\[ q^{-\max(m-n_2, m - \frac{\theta}{2} - r, 0) - v(c) - r}. \]
Hence the volume contributed by this portion is exactly given by
\begin{align*}
  \left\lvert \frac{x_1}{x_2} \right\rvert q^{n_2 - n_1} \Vol(\{ y \mid \dots \})
  &= q^{n_2 - n_1 - \max\left( m-n_2, m/2 - \frac{\theta}{2} - r, 0 \right) + v(c) + r} \\
  &= q^{m - \max\left( m-n_2, m/2 - \frac{\theta}{2} - r, 0 \right)}.
\end{align*}
And this sum is over two ranges of $m$
(although the ranges obviously overlap, they sum different values of $y$):
\begin{align*}
  \theta + 2r & < m \le \max\left( r, n_2 - \frac{\theta}{2} \right) + v(b) + v(c) + r \\
  \theta + 2r & < m \le n_2 + \frac{\theta}{2} + r.
\end{align*}
