\section{Background}
\label{sec:background}

\subsection{Notation}
We provide a glossary of notation that will be used in this paper.
As mentioned in the introduction, $p > 2$ is a prime,
$F$ is a finite extension of $\QQ_p$,
and $E/F$ is an unramified quadratic field extension.

\begin{itemize}
  \ii Denote by $\varpi$ a uniformizer of $\OO_F$, such that $\bar \varpi = \varpi$.
  \ii Let $q \coloneqq |F/\varpi|$ be the residue characteristic.
  \ii Let $v$ be the associated valuation for $\varpi$.
  \ii Let $\eta$ be the quadratic character attached to $E/F$ by class field theory,
  so that $\eta(x) = -1^{v(x)}$.
  \ii Set $G' \coloneqq \GL_n(E)$.
  \ii Set $K' \coloneqq \GL_n(\OO_E) \subseteq G'$ as the hyperspecial maximal compact subgroup of $G$.
  \ii $V_0$ denotes a split $E/F$-Hermitian space of dimension $n$.
  \ii Let $\beta$ denote the $n \times n$ antidiagonal matrix
  \[ \beta \coloneqq \begin{bmatrix} && 1 \\ & \iddots \\ 1 \end{bmatrix} \]
  and pick the basis of $V_0$ such that the Hermitian form on $V_0$ is given by
  \[ V_0 \times V_0 \to E \qquad (x,y) \mapsto x^\ast \beta y. \]
  \ii Set
  \[ G \coloneqq \U(V_0) = \{ g \in \GL_n(\OO_E) \mid g^\ast \beta g = \beta\} \]
  the unitary group over $V_0$.
  Note that $\beta$ is \emph{antidiagonal}, in contrast to the convention $\beta = \id_n$
  that is often used for unitary matrices with entries in $\CC$.
  \ii Set
  \[ K \coloneqq G \cap \GL_n(\OO_E) \]
  as the natural hyperspecial maximal compact subgroup.

  \ii Define the symmetric space
  \[ S_n(F) \coloneqq \left\{ g \in \GL_n(E) \mid g \bar g = \id_n \right\}. \]
  It has a natural left action of $\GL_n(E)$ by
  \[ \GL_n(E) \times S_n(F) \to S_n(F) \text { by } g \cdot s \mapsto g s \bar g^{-1} \]

  \ii Following the notation in \cite{ref:highdim2024}, define
  \[ V'_n \coloneqq F^n \times (F^n)^\vee \]
  where $-^\vee$ denotes the $F$-dual space, i.e., $(F^n)^\vee = \Hom_F(F^n, F)$.
  Then we may also consider the augmented space
  \[ S_n(F) \times V'_n \]
  If we identify $F^n$ with column vectors of length $n$ and $(F^n)^\vee$
  with row vectors of length $n$ then we have a left action of $\GL_n(F)$ by
  \[ \GL_n(F) \times S_n(F) \times V_n' \to S_n(F) \times V_n'
    \text { by } h \cdot (s, u_1, u_2) \mapsto (hsh^{-1}, h^{-1}u_1, u_2h). \]

  \ii Let \[ K'_S \coloneqq S_n(F) \cap \GL_n(\OO_F). \]
\end{itemize}

\subsection{Definition of Hecke algebra}
We reminder the reader the definition of a Hecke algebra.
For this subsection, $G$ will denote \emph{any}
unimodular locally compact topological group,
and $K$ any closed subgroup of $G$.

\begin{definition}
  The \emph{Hecke algebra}
  \[ \HH(G, K) \coloneqq \QQ[K \backslash G \slash K] \]
  is defined as the space of compactly supported $K$-binvariant
  locally constant functions on $G$.

  Given two such functions $\phi_1$ and $\phi_2$ in $\HH(G,K)$,
  one can consider define the convolution
  \[ (\phi_1 \ast \phi_2)(g) \coloneqq \int_G \phi_1(g\inv x) \phi_2(x) \; \odif x \]
  which makes $\HH(G, K)$ into a $\QQ$-algebra,
  whose identity element is $\mathbf{1}_K$.
\end{definition}
In the case where $G$ is a reductive Lie group and
$K$ is the maximal compact subgroup
(or more generally whenever $(G,K)$ is a Gelfand pair),
this Hecke algebra is actually commutative.

\subsection{The specific Hecke algebras for $G' = \GL_n(E)$ and $G = \U(V_0)$}
For our purposes, we define two Hecke algebras:
\begin{align*}
  \HH(G', K') &\coloneqq \HH(\GL_n(E), \GL_n(\OO_E)) \\
  \HH(G, K) &\coloneqq \HH(\U(V_0), \U(V_0) \cap \GL_n(\OO_E))
\end{align*}

\subsection{Two Hecke modules}
Now the symmetric space $S_n(F)$ is not a group,
so it does not make sense to define the same thing here.
Nevertheless, we introduce
\[ \HH(S_n(F), K') \coloneqq \mathcal C_{\mathrm{c}}^{\infty}(S_n(F))^{K'} \]
as the set of smooth compactly supported functions on $S_n(F)$
which are invariant under the action of $K' \subseteq G'$;
this is an $\HH(G', K')$-module, where the action of $f \in \HH(G', K')$ is given by
\[ f \cdot \varphi = {?} \]
\todo{I think I successfully confused myself about the multiplication structure}
This does \textbf{not} have a multiplication structure at the moment, \emph{a priori},
although later we will see how one could be imposed.

Similarly, consider $S_n(F) \times V_n$.
We set
\[ \HH(S_n(F) \times V_n', {?}) \coloneqq \mathcal C_{\mathrm{c}}^{\infty}(S_n(F) \times V_n')^{K'} \]
as the set of smooth compactly supported functions on $S_n(F) \times V_n'$
which are invariant under the action of ${?} \subseteq G'$.
\todo{This should be $\GL_n(\OO_F)$? Does it need another name?}
This is an \dots
