\chapter{Synopsis of the orbital integral
  $\Orb((\gamma, \uu, \vv^\top), \phi \otimes \oneV, s)$
  for $(\gamma, \uu, \vv^\top) \in (S_2(F) \times V'_2(F))\rs$
  and $\phi \in \HH(S_2(F), K')$}
\label{ch:orbitalFJ0}

Throughout this section, $H = \GL_n(F)$ (rather than $H = \GL_{n-1}(F)$)
and $K' = \GL_n(\OO_F)$.
For the concrete calculation, we are mostly interested in the case $n = 2$.

\section{Definition}
We will not work in the generality of a function on all of $S_n(F) \times V_n'(F)$.
Instead, for our conjecture, it will be enough to define the orbital integral
in the case where our function is of the form
\[ \phi \otimes \oneV \]
where $\phi \in \HH(S_n(F), K')$ is the left component, and
the right component is the indicator function defined in the obvious way:
\begin{align*}
  \mathbf{1}_{\OO_F^n \times (\OO_F^n)^\vee} \colon V'_n(F) &\to \{0,1\} \\
  (\uu, \vv^\top) &\mapsto
  \begin{cases}
    1 & \uu \text{ and } \vv^\top \text{ have } \OO_F \text{-entries} \\
    0 & \text{otherwise}.
  \end{cases}
\end{align*}

Then, unsurprisingly from the definition of our action as
\[ h \cdot (\gamma, \uu, \vv^\top) = (h\gamma h\inv, h\uu, \vv^\top h\inv) \]
we analogously define the orbital integral as follows.
\begin{definition}
  For brevity let $\eta(h) \coloneqq \eta(\det h)$ for $h \in H$.
  For $(\gamma, \uu, \vv^\top) \in S_n(F) \times V'_n(F)$,
  $\phi \in \HH(S_n(F), K')$, and $s \in \CC$,
  we define the orbital integral by
  \begin{align*}
    & \Orb((\gamma, \uu, \vv^\top), \phi \otimes \oneV, s) \\
    &\coloneqq
    \int_{h \in H} \phi(h\inv \gamma h)
    \oneV(h \uu, \vv^\top h^{-1})
    \eta(h) \left\lvert \det(h) \right\rvert_F^{-s} \odif h
  \end{align*}
\end{definition}

\section{Basis for the indicator functions in $\HH(S_2(F), K')$}
From now on assume $n = 2$.
This section is almost an exact analog of \Cref{ch:orbital0_hecke_basis},
so we will be slightly terser.
Again set
\[ S_2(F) \coloneqq \left\{ g \in \GL_2(E) \mid g \bar{g} = \id_2 \right\}. \]
We again have a Cartan decomposition indexed by a single integer $r \ge 0$:
\begin{lemma}
  [Cartan decomposition of $S_2(F)$]
  For each integer $r \ge 0$ let
  \[ K'_{S,r} \coloneqq \GL_2(\OO_E) \cdot
    \begin{bmatrix} 0 & \varpi^r \\ \varpi^{-r} & 0 \end{bmatrix} \]
  denote the orbit of
  $\begin{bmatrix} 0 & \varpi^r \\ \varpi^{-r} & 0 \end{bmatrix}$
  under the left action of $\GL_2(\OO_E)$.
  Then we have a decomposition
  \[ S_2(F) = \coprod_{r \geq 0} K'_{S,r}. \]
\end{lemma}
Like last time, $K'_{S,r}$ is the part of $S_2(F)$
for which the most negative valuation among the nine entries is $-r$.
And as before we abbreviate the $r = 0$ term specifically:
\begin{align*}
  K'_S
  &\coloneqq K'_{S,0} \\
  &= \GL_2(\OO_E) \cdot \begin{bmatrix} & 1 \\ 1 \end{bmatrix} \\
  &= \GL_2(\OO_E) \cdot \id_2 = S_2(F) \cap \GL_2(\OO_E).
\end{align*}

Repeating the definition
\[ K'_{S, \le r} \coloneqq S_2(F) \cap \varpi^{-r} \GL_2(\OO_E)
  = K'_{S,0} \sqcup K'_{S,1} \sqcup \dots \sqcup K'_{S,r} \]
we get a basis of indicator functions for the Hecke algebra $\HH(S_2(F), K')$:
\begin{proposition}
  For $r \ge 0$, the indicator functions $\mathbf{1}_{K'_{S, \le r}}$
  form a basis of $\HH(S_2(F), K')$.
\end{proposition}

\section{Parametrization of $\gamma$}
From now on assume $n = 2$,
and that $(\gamma, \uu, \vv^\top) \in (S_2(F) \times V'_2)\rs$ is regular.

\subsection{Identifying an orbit representative}
The orbital integral depends only on the $H$-orbit of $(\gamma, \uu, \vv^\top)$.
Consequently, we may assume without loss of generality
(via multiplication by a suitable change-of-basis $h \in H = \GL_2(F)$) that
\[ \uu = \begin{bmatrix} 0 \\ 1 \end{bmatrix}, \qquad
  \vv^\top = \begin{bmatrix} 0 & e \end{bmatrix} \qquad e \in F. \]
(We know $\uu$ is not the zero vector from the regular condition
applied on $(\gamma, \uu, \vv^\top)$.)

Meanwhile, we will let
$\gamma = \begin{bmatrix} a & b \\ c & d \end{bmatrix} \in \GL_2(F)$
for $a,b,c,d \in F$.
Then, viewed as an element of $\GL_3(F)$ via the embedding we described earlier, we have
\[
  (\gamma, \uu, \vv^\top)
  \mapsto \begin{bmatrix}
    a & b & 0 \\
    c & d & 1 \\
    0 & e & 0
  \end{bmatrix} \in \GL_3(F).
\]
Thus, our definition of regular requires that
$\begin{bmatrix} 0 \\ 1 \end{bmatrix}$
is linearly independent from $\begin{bmatrix} b \\ d \end{bmatrix}$
and
$\begin{bmatrix} 0 & e \end{bmatrix}$
is linearly independent from $\begin{bmatrix} c & d \end{bmatrix}$.
This is just saying that $b$, $c$, $e$ are all nonzero.
We also know that $\gamma \in S_2(F)$, which gives us relations on $a$, $b$, $c$, $d$
(the same as \cite[equation (7.3.2)]{ref:AFLspherical}); we have
\[
  \begin{bmatrix} 1 & 0 \\ 0 & 1 \end{bmatrix}
  = \begin{bmatrix} a & b \\ c & d \end{bmatrix} \begin{bmatrix} \bar a & \bar b \\ \bar c & \bar d \end{bmatrix}
  \implies
  \begin{aligned}
    \bar b c = b \bar c &= 1 - a \bar a = 1 - d \bar d \\
    \text{and } d &= - \bar a c / \bar c = -\bar a b / \bar b.
  \end{aligned}
\]

\subsection{Simplification due to the matching of non-split unitary group}
Like before, we focus on the case where regular $(\gamma, \uu, \vv^\top)$
matches an element in the non-split unitary group.
\todo{I need to ask Wei exactly what's up here}
This is controlled by the parity of $v(\Delta)$, where
\[ \Delta = \det \left[ \vv^\top \gamma^{i+j} \uu \right]_{0 \le i,j \le n-1}. \]
When $n=2$, for the representatives we described before,
we have
\[ \left[ \vv^\top \gamma^{i+j} \uu \right]_{0 \le i,j \le n-1}
  = \begin{bmatrix} e & de \\ de & bce + d^2e \end{bmatrix} \]
so
\[ \Delta = bce^2 = \frac{b}{\bar b}(1-a \bar a) e^2 . \]
Hence, $v(\Delta)$ is odd if and only if $v(1-a \bar a)$ is odd.
Thus, we restrict attention to the following situation:
\begin{assume}
  We will assume that
  \[ v(1-a \bar a) \equiv 1 \pmod 2. \]
  \label{assume:a_odd}
\end{assume}
In particular, $a$ must be a unit.
And since $d = -\bar a c / \bar c$, it follows $d$ is a unit.
In other words, \Cref{assume:a_odd} gives the direct corollary
\[ v(a) = v(d) = 0. \]

\section{Quantities to state the answer in terms of}
