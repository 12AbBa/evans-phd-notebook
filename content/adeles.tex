\chapter{Adeles}
\section{The adele ring}
Let $K$ be a global field.
\begin{definition}
  The \alert{adele ring} of $K$ is defined as the restricted product
  \[ \AA_K = \rprod_v (K_v, \OO_v) \]
  across all the places $v$ of $K$,
  both Archimedean and non-Archimedean.
  Elements of $\AA_K$ are called adeles.
\end{definition}
This means that it consists of tuples $(a_v)_{v}$
for which $a_v \in \OO_v$ for all but finitely many $v$.
\begin{definition}
The \alert{idele group} of $K$ is defined as the restricted product
\[ \AA_K^\times = \rprod_v (K_v^\times, \OO_v^\times) \]
across all the places $v$ of $K$,
both Archimedean and non-Archimedean.
Elements of $\AA_K^\times$ are called ideles.
\end{definition}
(The topology of the idele group is not
the subspace topology of $\AA_K$,
so the inclusion $\AA_K^\times \subseteq \AA_K$
occurs only at the level of sets.)

There are obviously diagonal inclusions
$K \injto \AA_K$ and $K^\times \injto \AA_K^{\times}$.
By abuse of notation we're going to just treat
$K$ and $K^\times$ as subsets of $\AA_K$ and $\AA_K^\times$,
hence e.g.\ $\AA_K/K$ refers to the coimage of the former map.

The adelic absolute value is the map
\[ \AA_K^\times \to \RR_{>0} \]
given by $(a_v) \mapsto \prod_v \left\lvert a_v \right\rvert_v$.
\begin{proposition}
  [Product formula]
  We have $|a|=1$ for $a \in K^\times$.
\end{proposition}

\section{Example}
By ``strong approximation'',
\[ \AA_{\QQ}^\times
  = \QQ_{>0} \times \RR \times \prod_p \ZZ_p^\times. \]
But if $K$ is a number field with class number greater than $1$
then such a clean decomposition is usually impossible.

\section{First properties of the adele ring}
Upshots of using the restricted product include
the following results.
\begin{theorem}
  [$\AA_K$ is LCA Hausdorff and self-dual]
  $\AA_K$ is locally compact and Hausdorff,
  and in fact equal to its own Pontryagin dual.
\end{theorem}

\begin{theorem}
  [$K$ is the Pontryagin dual of $\AA_K/K$]
  $K$ is a discrete co-compact subgroup of $\AA_K$.
  In fact, $K$ and $\AA_K/K$ are Pontryagin duals.
\end{theorem}

\begin{remark*}
In fact, it's also true that $K^\times$
is discrete and co-compact in
\[ \AA_K^{\mathbf 1} \defeq
  \ker \left(
    \left\lvert \bullet \right\rvert \colon
    \AA_K^\times \to \RR^\times
  \right) \]
the set of ideles of adelic absolute value $1$.

In general, it may be shown
there is a short exact sequence
\[
  1 \to \frac{K_\infty^{\mathbf 1} \times
    \prod_{v < \infty} \OO_v^\times}{\OO_K^\times}
    \to \frac{\AA_K^{\mathbf 1}}{K^\times}
    \to \opname{ClassGrp}(K) \to 1
\]
which does not necessarily split,
though when $K$ has trivial class group
it does explicitly characterize $\AA^{\mathbf 1}$
(e.g.\ $K=\QQ$, as we saw in the earlier example).

This short exact sequence is conceptually nice because
it implies both Dirichlet's unit theorem,
and the finiteness of the class group,
solely from the compactness of the center term.
\end{remark*}
\todo{citation needed on Dirichlet unit theorem}

\section{Calculus on adeles}
The following results are included here for completeness,
to signal that they do not depend on any of the
automorphic-flavored stuff to follow.
But you could skip this section for now
and come back to it later when it's quoted, if you prefer.

\subsection{Fourier setup for local fields}
In this section, let $F$ be a local field,
and define the following \emph{standard character} on it:
\[
  \psi_F(x)
  = \begin{cases}
    e^{-2\pi i x} & F = \RR \\
    e^{-2\pi i |x|_p} & F = \QQ_p \\
    e^{2\pi i a_{-1} /p} & F = \FF_p( (t) ) \\
    \psi_0\left( \Tr_{F/F_0}(x) \right) & F = \FF_p( (t) ).
  \end{cases}
\]
These choices are contrived so that later,
if $K$ is a global field, then
$\prod_v \psi_{K_v} \colon \AA \to \CC$ will vanish on $K$.

Moreover, we can choose a Haar measure such that
\begin{itemize}
  \ii $dx$ is the standard Lebesgue measure for $F = \RR$
  \ii $dx$ is twice the standard Lebesgue measure on $F = \CC$
  \ii $dx$ is selected such that $\opname{Vol}(\OO) = (\#\OO/\mathcal D)^{-1/2}$,
  where $\mathcal D$ is the different,
  when $F$ is a non-Archimedean local field.
\end{itemize}

This lets us define a Fourier transform
\[ \wh f = \int_F f(x) \psi_F(xy) \; dx. \]

\subsection{Poisson summation}
\todo{write this}
