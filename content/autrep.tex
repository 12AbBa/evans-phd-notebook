\chapter{Automorphic representations}
Towards the start of the text,
we saw how the classical modular forms
and their counterpart Maass forms naturally arose from
representations of $L^2(\Gamma\backslash\GL_2(\RR))$.

%For $\GL_2(\AA_F)$,
%we are going to \emph{need} the representation theory
%more deeply; that is, we want to first exhibit
%an explicit correspondence between automorphic forms
%and so-called automorphic representations
%(neither of which we have defined yet, sorry).
%The reason is that even the definition of the
%functional equations is going to require
%
%We're going to want to do the same thing here \emph{first}

\section{Automorphic forms}
The definition of an adelic automorphic form was stated earlier,
except for the technical niceness condition.
\begin{itemize}
  \ii Smooth
  \ii $K$-finite
  \ii $\mathcal Z$-finite, where $\mathcal Z$ is the center
  of the universal enveloping algebra of $U(\mathfrak{gl}(n, K_v))$,
  \ii moderate growth.
\end{itemize}
The \alert{cusp forms} are those which obey the additional hypothesis
\[
  \int \phi \left(
    \begin{bmatrix}
      I_r & X \\ & I_s
    \end{bmatrix}
  \right) \; dX = 0.
\]

\section{Automorphic representations, and admissible representations}
\begin{definition}
  An \alert{admissible representation} $(\pi, V)$
  of $\GL_2(\AA)$ is a representation of
  $\GL_2(\Af)$ with a commuting $(\kg_\infty, K_\infty)$-module structure
  with the additional constraint that
  every vector of $V$ is $K$-finite
  and every isotypic part is finite dimensional.
\end{definition}

\begin{definition}
  An \alert{automorphic representation} is a irreducible representation of
  $\GL_2(\Af)$ with a commuting $(\kg_\infty, K_\infty)$-module structure
  which can be realized as the quotient of a submodule
  of the space of automorphic forms of some central quasicharacter.
\end{definition}

\begin{proposition}
  Automorphic representations are admissible.
\end{proposition}

\begin{definition}
  If $F$ is a non-Archimedean local field,
  an admissible representation of $\GL_2(F)$
  is an actual representation such that every vector has open stabilizer
  and every $K = \GL_2(\OO_F)$-isotypic part is finite dimensional.
  We say a spherical representation is one with a nonzero $K$-fixed vector,
  which we also say is spherical.

  If $F$ is an Archimedean local field,
  we instead want a $(\kg_\infty, K_v)$-module
  with finite-dimensional isotypic parts.
\end{definition}

\begin{theorem}
  [Tensor product theorem]
  Let $(V, \pi)$ be an irreducible admissible representation.
  Then we can choose $(V_v, \pi_v)$ for every place $v$ such that
  there exists a nonzero spherical vector $\xi^0_v \in V_v$ for almost all $v$,
  such that \[ V = \wh{\bigotimes} (V_v, \pi_v) \]
  where the restricted tensor product uses $\xi_0^v$.
\end{theorem}


The space of cusp forms contains each representation at most once;
actually following stronger result holds.
\begin{theorem}
  [Strong multiplicity one theorem]
  Let $(\pi, V)$ and $(\pi, V')$ be irreducible admissible subrepresentations
  of the space $\opname{CuspForms}(\GL(n, K) \backslash \GL(n, \AA_k), \omega)$.
  If $\pi_v = \pi'_v$ for almost all $v$ then $V = V'$.
\end{theorem}
