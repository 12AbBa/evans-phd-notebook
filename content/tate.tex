\chapter{Tate's thesis}
\section{Goals moving forward (road map of the future)}
In brief, take a unitary character
$\omega \colon \AA^\times / K^\times \to \CC$,
known as a \alert{Hecke character}
(analogous to the nebentypus Dirichlet character from earlier).
An adelic automorphic form with central quasicharacter
$\omega$ will be a function
\[ \phi \colon \GL_n(\AA_K) \to \CC \]
obeying the conditions
\begin{itemize}
  \ii For all $g \in \GL_n(\AA_K)$
    and $z \in \AA_K^\times$, we have
  \[
    \phi\left(
      \begin{bmatrix}
        z \\ & \ddots \\ && z
      \end{bmatrix}
      g
    \right) = \omega(z) \phi(g).
  \]
  The choice of the letter $z$ comes from the fact
  that the center of the group $\GL(n, \AA_K)$
  is exactly the diagonal matrices appearing above.
  \ii For all $g \in \GL_n(\AA_K)$ and
  $\gamma \in \GL_n( K^\times)$ we have
  \[ \phi(\gamma g) = \phi(g). \]
  The name ``automorphic'' comes from here,
  and means we may equally regard
  $\phi$ as a function on
  $\GL_n(K^\times) \backslash \GL_n(\AA_K)$.
  \ii Four other technical niceness conditions defined later.
\end{itemize}
We are going to have two goals:
\begin{description}
  \ii[First goal --- tying forms to representations]
  Connect automorphic forms to certain
  ``representations'' on some space of functions
  $\GL_n(K^\times) \backslash \GL_n(\AA_K) \to \CC$.
  This is analogous to how we saw that modular forms
  and Maass forms turned out to correspond exactly
  with representations of $L^2(\Gamma \backslash \SL(2,\RR))$.

  \ii[Second goal --- $L$-functions]
  For each such representation, construct an $L$-function,
  give it an analytic continuation, and exhibit a functional equation.
  (One common trend in analytic number theory
  is that $L$-functions are worth their weight in gold.)
\end{description}
There is one place where the analogy is slightly weaker.
Rather than attaching $L$-functions directly to the
automorphic \emph{forms},
it turns out to be more convenient to attach them to the
automorphic \emph{representations} directly.

\todo{make a table?}

\section{Motivation}
Even for $n=2$, this task we described will be rather technical
(let alone replacing $\GL_n$ with a general algebraic group).

So we will first examine the case where $n=1$;
a result widely known now as Tate's thesis.
In this case much of the theory simplifies immensely:
\begin{exercise}
  When $n=1$, in the notation of the earlier section,
  prove that $\phi = c \omega$ for some constant $c$.
\end{exercise}
In other words, there's only a one-dimensional space
of forms anyways (once $\omega$ is fixed)
and they are literally just multiples of $\omega$.
Hence Tate's thesis only involves constructing $L$-functions
for each given Hecke character $\omega$.

It is traditional that for Tate's thesis,
the Hecke character is denoted by $\chi$ rather than $\omega$.

\section{Local functional equation}
Let $F$ be a local field, and $\eta \colon F^\times \to S^1$
a unitary character on it.
Recall that $\eta$ is unramified if it is trivial on $\OO^\times$,
which is equivalent to $\eta(x) = |x|^{i\lambda}$ for $\lambda \in \RR$

\subsection{Non-Archimedean definition}
If $F$ is non-Archimedean,
the local $L$-factors are defined in the following way:
\[
  L(s, \eta) =
  \begin{cases}
    1 & \text{ramified} \\
    \left( 1 - \eta(\varpi) q^{-s} \right)^{-1} & \text{unramified}.
  \end{cases}
\]
More generally, we may define for any Schwartz function $f$ the local zeta integral
\[
  Z(s, \eta, f) = \int_{F^\times} |x|^s \eta(x) f(x) \; dx^\times.
\]
We justify ``more generally'' right away with the following calculation.
\begin{proposition}
  [Zeta integral generalizes $L$-factor, at least in unramified case]
  Suppose $F$ is non-Archimedean
  and $\eta$ is a unramified unitary character on $F$.
  Then we have
  \[ L(s, \eta) = Z(s, \eta, 1_\OO). \]
\end{proposition}
In other words, $L$ is a special case of $Z$ with $f = 1_\OO$.
\begin{proof}
  By condition, $\eta(x) = |x|^{i\lambda}$
  for some $\lambda \in \RR$ with $\eta$ trivial on $\OO$.
  Show both equal to $Z(s + i\lambda, 1, 1_\OO)$.
  Poonen left this as homework so I didn't do it.
\end{proof}
The ramified case is more annoying.
\begin{proposition}
  [Ramified case]
  Suppose $F$ is non-Archimedean
  and $\eta$ is a unitary character on $F$ with conductor $\kp^n$.
  Then
  \[ Z(s, \eta, 1_{1+\kp^n}) = q^{-n}. \]
\end{proposition}
Of course since $L(s,\eta)=1$ in this situation we could also
write $Z(s, \eta, 1_{1+\kp^n}) = q^{-n} L(s, \eta)$,
but that would be silly.

\subsection{Archimedean definition}
If $F = \RR$, we instead have the complicated formula
\[
  L(s, \underbrace{|x|^\lambda \sign(x)^\eps}_{=\eta} )
  = \pi^{-\half(s+\lambda+\eps)/2} \Gamma\left( \frac{s+\lambda+\eps}{2} \right)
\]
and when $F = \CC$ the formula
\[
  L(s, \underbrace{|x|^{2 \lambda} (x/|x|)^n}_{=\eta} )
  = 2(2\pi)^{-(s+\lambda+\half|n|)} \Gamma\left( s + \lambda + \frac{|n|}{2} \right).
\]
Again, these can be viewed as special cases of the zeta integral.
\begin{proposition}
  [Archimedean relation]
  We have
  \begin{align*}
    L(s, \eta)
    &= \begin{cases}
      Z(s, e^{-\pi x^2}, \eta) & \eta = |x|^\lambda, F = \RR \\
      i Z(s, e^{-\pi x^2}, \eta) & \eta = |x|^\lambda \sign(x), F = \RR \\
      (-i)^a \dots Z(s, \dots, \dots) & F = \CC
    \end{cases}
  \end{align*}
\end{proposition}
\begin{proof}
  Bash.
\end{proof}

\subsection{The local functional equation for a test function}
We can coalesce the earlier three results in the following lemma.
\begin{proposition}
  [The local functional equation holds for our test function]
  Let $F$ be a local field, and $\eta$ a unitary character on it.
  Then there exists \emph{single, nice explicit choice}, of function $f$,
  such that
  \[ \frac{Z(1-s, \wh{f}, \ol\eta)}{L(1-s, \ol\eta)}
  = \eps(s, \eta) \cdot \frac{Z(s, f, \eta)}{L(s, \eta)}. \]
  holds for $0 < \sigma < 1$,
  with $\eps$ of exponential type in $s$.
  Moreover, the left-hand side is nonvanishing holomorphic when $\sigma > 0$,
  while the right-hand side is nonvanishing holomorphic when $\sigma < 1$.
\end{proposition}
The choice of $f$ is called a \emph{test function},
and it's just the function we chose in the earlier proofs.
We will see this lemma
verifies the functional equation for a single value of $f$.
\begin{proof}
  Earlier, we saw that each $L(s, \eta)$
  is basically equal (up to some constant) to
  $Z(f, s, \eta)$ for some choice of $f$:
  \begin{itemize}
    \ii $f = \OO$ if $\eta$ is unramified on non-Archimedean $F$
    \ii $f = 1_{1+\kp^n}$ if $\eta$ is ramified on non-Archimedean $F$
    \ii $f = e^{-\pi x^2}$ if $F = \RR$
    \ii \dots if $F = \CC$.
  \end{itemize}
  So naturally, we use this as our function $f$.
  This means the right-hand side is basically known at this point.
  The difficulty is to calculate the left-hand side.

  For the $\RR$ and $\CC$ case, this is not much different.
  It's more difficult in the non-Archimedean case and requires a Gauss sum.
\end{proof}

\subsection{The local functional equation}
The main result is the earlier local functional equation holds
for any function $f$ and the epsilon factor does
\begin{theorem}
  [Local functional equation]
  For any Schwartz function $f$ and unitary character $\eta$,
  \[ \frac{Z(1-s, \wh{f}, \ol\eta)}{L(1-s, \ol\eta)}
  = \eps(s, \eta) \cdot \frac{Z(s, f, \eta)}{L(s, \eta)}. \]
  Moreover, the left-hand side is nonvanishing holomorphic when $\sigma > 0$,
  while the right-hand side is nonvanishing holomorphic when $\sigma < 1$.
\end{theorem}
\begin{proof}
  It suffices to prove that for arbitrary Schwartz functions $f$ and $g$ we have
  \[ Z(s, \eta, f) Z(1-s, \ol\eta, \wh g)
    = Z(s,\eta, g)Z(1-s, \ol\eta, \wh f). \]
  Bash with Fubini theorem.
\end{proof}


\section{Global functional equation}
A Schwartz function on $\AA_K$ is a function $\prod_{v \in K} f_v$
such that $f_v = 1_{\OO_v}$ almost everywhere.
For such a function we may define the global zeta integral
\[ Z(s, f, \eta) = \int_{\AA^\times} |x|^s \eta(x) f(x) \; d^\times x \]
in nearly the same way as before.
In fact,
\[ Z(s, f, \eta) = \prod_v Z(s, f_v, \eta_v). \]

\begin{theorem}
  This integral converges when $\sigma > 1$ and satisfies
  \[ Z(s, f, \eta) = Z(1-s, \wh f, \ol\eta). \]
\end{theorem}
\begin{proof}
  Poisson summation.
\end{proof}

Finally, we may define
\begin{align*}
  L(s, \eta) &= \prod_v L(s, \eta_v) \\
  \eps(s, \eta) &= \prod_v \eps(s, \eta_v)
\end{align*}
and multiply everything together to get that
\[ L(s, \eta) = \eps(s, \eta) L(1-s, \ol \eta). \]
