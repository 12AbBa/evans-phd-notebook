\chapter{Prerequisites}
This chapter covers some theory that is considered ``prerequisite''
for all the fancy number theory to follow, but isn't already covered
in \href{https://web.evanchen.cc/napkin.html}{Napkin}.

\section{Fourier transforms}
As usual, $\ee(t)$ is shorthand for $\exp(2\pi i t)$.

\subsection{Fourier transform of a periodic function}
We'll repeatedly need the following.
\begin{theorem}
[Fourier coefficients of a periodic function]
Suppose that $g \colon \RR \to \CC$
is smooth and satisfies $g(x+1) = g(x) + 1$.
In that case, it can be expressed
in terms of the basis $t \mapsto \ee(nt)$ by
\[
  g(x) = \sum_n a_n \ee(nx)
  \qquad\text{where}\qquad
  a_m = \int_0^1 g(x) \; \ee(-nx) \; dx.
\]
\end{theorem}

Thus if $f \colon \HH \to \CC$ is holomorphic
that satisfies $f(z+1) = f(z)$,
then we will also have the relation
\[ f(z) = \sum_n a_n \ee(nz) = \sum_n a_n q^n
  \qquad\text{where}\qquad
  a_n = \int_w^{w+1} f(z) \; \ee(-nz) \; dz \]
for the same reason.
Here $w$ may be any complex number in $\HH$;
by the contour theorem, the choice doesn't matter.

If we write $z = x+yi$ it's often
more economical to divorce $x$ and $y$:
\[ f(z) = \sum_n a_n(y) \ee(nx)
  \qquad\text{where}\qquad
  a_n(y) = \int_0^1 f(x+yi) \ee(-nx) \; dx \]

\subsection{Fourier transform of a real function}
Now let us suppose $f \colon \RR \to \CC$.
We say it is
\begin{itemize}
  \ii of \alert{moderate decrease}
  if $|f(x)| = O( (1+x^2)^{-1} )$, and
  \ii a \alert{Schwartz function} (or of rapid decrease)
  if all derivatives decay faster than any polynomial.
\end{itemize}
In either case,
one can define the Fourier transform by
\[ \wh f(\xi)
  = \int_\RR f(x) \ee(-\xi \cdot x) \; dx \]
which converges for any $x$.
The advantage of the Schwartz functions is that
$\wh{\bullet}$ is actually a bijection on this space;
whereas $\wh f$ may not be of moderate decrease
even if $f$ is.

We'll repeatedly use the following.
\begin{theorem}
  [Poisson summation formula]
  If both $f$ and $\wh f$ are both of
  moderate decrease then
  \[ \sum_n f(n) = \sum_n \wh f(n). \]
\end{theorem}
% wtf you can't just switch order of sum
\begin{proof}
  More generally, the following is true:
  \[ \sum_n f(x+n) = \sum_n \wh f(n) \ee(nx). \]
  You can prove this by applying
  the previous result to $g(x) = \sum_n f(x+n)$.
  Indeed, the $m$th Fourier coefficients of this $g$ is
  \begin{align*}
    a_m &= \int_0^1 \sum_n f(x+n) \ee(-mx) \; dx \\
    &= \sum_n \int_0^1 f(x+n) \ee(-mx) \; dx \\
    &= \int_\RR f(x) \ee(-m(x-\left\lfloor x \right\rfloor)) \; dx \\
    &= \int_\RR f(x) \ee(-mx) \; dx = \wh f(m).
    \qedhere
  \end{align*}
\end{proof}

We'll also occasionally use:
\begin{theorem}
  [Fourier inversion]
  If $f$ is a Schwartz function then
  \[ \wh{\wh{f}}(x) = f(-x). \]
\end{theorem}

\subsection{Applications of Fourier stuff}
We do now a few calculations which we will use later.

\begin{proposition}
  Fix $t > 0$.
  The Fourier transform of $h_t = \exp(-\pi t x^2)$
  is $\wh{h_t} = \frac{1}{\sqrt{t}} h_{1/t}$.
\end{proposition}
\begin{proof}
  We calculate
  \begin{align*}
    \wh{h}_t(\xi)
    &= \int_\RR \exp(-\pi t x^2) \exp(-2\pi i \xi x) \; dx \\
    &= \int_\RR \exp \left( -\frac{\pi}{t}
      \left[ t x + i \xi \right]^2 \right)
      \exp\left( -\frac{\pi \xi^2}{t} \right) \; dx \\
    &= \exp\left( -\frac{\pi \xi^2}{t} \right)
      \int_{\Im z = c} \exp \left( - \pi (\sqrt t z)^2 \right) \; dz \\
    &= \frac{1}{\sqrt{t}}
      \exp\left( -\frac{\pi \xi^2}{t} \right)
      \int_{\Im z = c} \exp \left( - \pi z^2 \right) \; dz \\
    &= \frac{1}{\sqrt{t}}
      \exp\left( -\frac{\pi \xi^2}{t} \right)
    = \frac{1}{\sqrt{t}} h_{1/t}(\xi).
  \end{align*}
  Here we used the Cauchy residue theorem
  to assert that
  $\int_{\Im z = c} \exp \left( - \pi z^2 \right) \; dz$
  is independent of the choice of $c$,
  and thus we may replace it with $0$
  at which point we get the famous Gaussian integral.
\end{proof}

\begin{proposition}
  Fix $t > 0$.
  The Fourier transform of $h_t(x) = x\exp(-\pi t x^2)$
  is $\wh{h_t} = \frac{i}{t^{3/2}} h_{1/t}$.
  %% TODO might be off by sign
\end{proposition}
\begin{proof}
  Integrate by parts and repeat previous proposition.
\end{proof}

\section{Mellin transform}
\subsection{Generalized Mellin transform}
We'll follow
\href{https://people.mpim-bonn.mpg.de/zagier/files/tex/MellinTransform/fulltext.pdf}{Zagier's appendix}.

Initially,
suppose $\phi \colon (0,\infty) \to \CC$ is a smooth function
which decays rapidly at infinity.
We will not assume $\phi$ decays rapidly at zero
since this restriction is too much;
we will instead just assume it is has an asymptotic expansion
\[ \phi(t) = \sum_n a_n t^{\alpha_n} \qquad t \to 0 \]
where $\alpha_1$, $\alpha_2$, \dots are complex numbers
say with $\Re \alpha_1 \le \Re \alpha_2 \le \cdots$.
We allow this sequence to be finite,
or also infinite if $\lim_{n \to \infty} \Re \alpha_n = \infty$.
In the most common case this will be a
sort of ``Taylor series'' at $0$.
It's not required that this series actually
converges at any point.

Then we define its \alert{Mellin transform}
$\MM\phi \colon \CC \to \CC$ by
\[ \MM(\phi, s)
  = \int_0^\infty t^s \phi(t) \; \frac{dt}{t} \]
which is initially defined
as long as $\Re s > -\Re\alpha_1$.
(For example, $\phi$ is smooth at $0$,
the above equation is okay for $\opname{Re} s > 0$.)
The following result will be indispensable
as a source of meromorphic continuations:

\begin{theorem}
  [Generalized Mellin transform
    for rapidly decaying functions]
  This Mellin transform $\MM(\phi, s)$
  has a meromorphic continuation to $\CC$
  with (at most) simple poles of residue
  $a_n$ at $s=-\alpha_n$ (and no other poles).
\end{theorem}
\begin{proof}
  Let $T > 0$ be arbitrary.
  For each $N > 0$ write
  \begin{align*}
    \MM(\phi, s)
    &= \int_0^\infty t^s \phi(t) \; \frac{dt}{t} \\
    &= \int_0^T t^s \phi(t) \; \frac{dt}{t}
    + \int_T^\infty t^s \phi(t) \; \frac{dt}{t} \\
    &= \int_0^T t^s
      \left( \phi(t) - \sum_{n=1}^N
      a_n t^{\alpha_n} \right)
      \; \frac{dt}{t}
      + \sum_{n \le N} \frac{a_n}{s+\alpha_n} T^{s+\alpha_n}
    + \int_T^\infty t^s \phi(t) \; \frac{dt}{t}
  \end{align*}
  which gives a desired meromorphic
  continuation to $\Re s > - \Re \alpha_N$
  with poles in the specified places,
  of residue $a_n$.
  (Moreover, this choice is independent of $T$;
  a meromorphic continuation is going to be unique, anyways.)
\end{proof}

\subsection{Applications of Mellin}
\begin{itemize}
  \ii Let $\phi(t) = e^{-t}$
  which has a Taylor expansion $1 - t + \frac{t^2}{2} - \cdots$
  near zero.
  Then by definition
  \[ \MM(\phi, s) = \int_0^{\infty} t^s e^{-t} \; \frac{dt}{t}
    = \Gamma(s) \]
  is the famous Gamma function.
  We now immediately know that $\Gamma$
  has meromorphic continuation
  with poles of residue $\frac{(-1)^n}{n!}$
  at $s = -n$, for $n = 0, 1, \dots$.
  Note that integration by parts gives the functional equation
  $\Gamma(s+1) = s \Gamma(s)$,
  and since $\Gamma(1) = 1$,
  we could have deduced the result a similar way.

  \ii More generally, if $\phi(t) = e^{-\lambda t}$
  for some $\lambda > 0$ then
  \[ \MM(\phi, s) = \lambda^{-s} \cdot \Gamma(s) \]
  by a change of variables.
  (Actually, for $\lambda = 0$ we also find that the Mellin
  transform of a constant function is zero.)
\end{itemize}

Strong use: we can take $\phi$ a sum of such exponentials.
Consider
\begin{align*}
\phi(t) &= \frac{1}{e^t-1} \\
&= \frac{1}{t + \frac{t^2}{2!} + \dots}
= \sum_{n \ge -1} \frac{B_{n+1}}{(n+1)!} t^n \\
&= e^{-t} + e^{-2t} + e^{-3t} + \cdots
\end{align*}
where $B_{n+1}$ are the Beronulli numbers.
Taking the Mellin transform of the last expression now gives
\[ \MM(\phi, s) = \sum_{n \ge 1} n^{-s} \Gamma(s)
= \Gamma(s) \zeta(s) \]
where $\zeta$ is the Riemann zeta function.
This gives an identity
\[ \zeta(s) = \frac{\MM(\phi, s)}{\Gamma(s)} \]
which is now the meromorphic continuation of $\zeta$!

Let's see what we can extract about its poles.
For $n \ge -1$,
note that $\MM(\phi, s)$ has simple poles at $s = -n$
of residue $\frac{B_{n+1}}{(n+1)!}$.
For $n \ge 0$, the function $\Gamma$ has simple poles at $s = -n$
of residue $\frac{(-1)^n}{n!}$.
That means $\zeta$ has only a simple pole of residue $1$ at $s=1$
(since $B_0/0! = 1$).
And the values of the zeta function for $n \ge 0$ are now given by
\[ \zeta(-n) = (-1)^n \frac{B_{n+1}}{n+1}. \]
In particular, $n \ge 0$ is odd then $B_{n+1} = 0$.

\subsection{Mellin transforms for functions not decaying at infinity}
This is still not general enough for us.
For example, we will later want to take the Mellin
transform of a certain $\theta$ function
corresponding to the Riemann zeta function.
However, this function behaves like $\frac{1}{2\sqrt t}$
for $t \to 0$ and $1/2$ for $t \to \infty$.
This makes the previous definition fail.

We will now consider a case where $t$ may not decay
rapidly at either $0$ or $\infty$, but having
asymptotic expansions at both
\begin{align*}
  \phi(t) &= \sum_n a_n t^{\alpha_n} \qquad \text{as }t \to 0 \\
  \phi(t) &= \sum_n b_n t^{\beta_n} \qquad \text{as } t \to \infty
\end{align*}
where $\alpha_1$, $\alpha_2$, \dots
and $\beta_1$, $\beta_2$, \dots are complex numbers
with $\Re \alpha_1 \le \Re \alpha_2 \le \cdots$
and  $\Re \beta_1 \ge \Re \beta_2 \ge \cdots$; again we assume
$\lim_{n \to \infty} \Re \alpha_n = \infty$
and $\lim_{n \to \infty} \Re \beta_n = -\infty$
if either sequence is infinite.
In that case, the original integral
$\int_0^\infty t^s \phi(t) \frac{dt}{t}$ is defined
if $\Re s > - \Re \alpha_1$ and $\Re s < - \Re \beta_1$.
The problem is that this might not hold for \emph{any}
values of $s$ at all!

This means even defining the function
which we want to take extend requires some straightforward
but annoying work.
Here is the specification.
Again pick a real number $T > 0$.
The idea is that
$\int_0^\infty t^s \phi(t) \frac{dt}{t}$
maybe split into
$\int_0^T t^s \phi(t) \frac{dt}{t}
+ \int_T^\infty t^s \phi(t) \frac{dt}{t}$
and each of the two halves will be defined somewhere
and can be extended analytically as before.
The explicit definition is to consider
\begin{align*}
  \MM(\phi, s) &=
  \int_0^T \left( \phi(t)
    - \sum_{n=1}^N a_n t^{\alpha_n} \right) \frac{dt}{t}
    + \sum_{n=1}^N \frac{a_n}{s+\alpha_n} T^{s+\alpha_n} \\
  &+   \int_T^\infty \left( \phi(t)
    - \sum_{n=1}^N b_n t^{\beta_n} \right) \frac{dt}{t}
    - \sum_{n=1}^N \frac{b_n}{s+\beta_n} T^{s+\beta_n}
\end{align*}
as $N \to \infty$.
The first line is defined
as long as $\Re s > - \Re \alpha_N$
while the second line is defined
as long as $\Re s < - \Re \beta_N$.
Of course, the overall sum is also independent of $T$.
Thus:
\begin{theorem}
  [Generalized Mellin transform]
  Let $\phi \colon (0,\infty) \to \CC$ be a function
  with asymptotic expansions at $0$ and $\infty$ as above.
  Then $\MM(\phi, s)$ as given above
  is a meromorphic function
  with at most simple poles only at $s = -\alpha_n$ and $s = -\beta_n$,
  of residue $a_n$ and $-b_n$ (respectively, and additively).
\end{theorem}

This means that $\MM(-,s)$ is gives an $\CC$-linear map
from the set of functions $(0,\infty) \to \CC$
with asymptotic expansions at $0$ and $\infty$,
to the space of meromorphic functions.

An important example we can now note:
\begin{example}
  [Kernel of the generalized Mellin transform]
  Suppose $\phi(t) = 1$.
  Then $\MM(1, s) \equiv 0$, because we have $a_1 = b_1 = 1$
  and $\alpha_1 = \beta_1 = 0$, and $a_1 - b_1 = 0$
  means the pole there gets cancelled out.
  More generally, if $\phi(t)$ is any polynomial in $t$,
  or any finite sum of $t^\alpha$ terms,
  then its generalized Mellin transform vanishes.
\end{example}
This means we can shift away any constants when discussing
the Mellin transform.
One just has to keep in mind the definition.

In general, the following result
can be proven by change of variables.
\begin{proposition}
  [What happens if you $u$-sub a Mellin transform]
  \label{prop:usubmellin}
  Let $\phi \colon (0,\infty) \to \CC$ be a function
  with asymptotic expansions at $0$ and $\infty$ as above.
  Then for any $c > 0$, $d \in \RR$ and $\alpha \in \CC$,
  \[ \MM(t^\alpha \phi(ct^d), s)
    = \frac{\MM(\phi, \frac{s+\alpha}{d})}
    {(c^{1/d}|d|)^{s+\alpha}}. \]
\end{proposition}
\begin{proof}
  This is actually a compact way of abbreviating three changes
  of variables; the idea is
  \begin{align*}
    \MM(t^\alpha \phi(ct^d), s)
    &= \MM(\phi(ct^d), s+\alpha)
      = |d|^{-(s+\alpha)}
      \MM\left(\phi(ct), \frac{s+\alpha}{d} \right) \\
    &= c^{\frac{-(s+\alpha)}{d}}
      |d|^{-(s+\alpha)}
      \MM\left(\phi(t), \frac{s+\alpha}{d} \right). \qedhere
  \end{align*}
\end{proof}


\section{Dirichlet characters}
\subsection{Sums involving Dirichlet characters}
Let $\chi \colon (\ZZ/N)^\times \to \CC$
be a primitive Dirichlet character with conductor $N$.

We need two things:
\begin{itemize}
  \ii Recall that the \alert{Gauss sum} is defined by
  \[ \tau(\chi) = \sum_{n \bmod N} \chi(n) \ee(n/N) \]
  which satisfies the famous
  identity $|\tau(\chi)| = \sqrt N$.

  \ii We need the interpolation formula for
  primitive Dirichlet characters:
  \[ \chi(n) = \frac{\chi(-1)\tau(\chi)}{N}
    \sum_{r \bmod N} \ol{\chi}(r) \ee(nr/N).  \]
  The point of this formula is to re-express $\chi(n)$
  as a sum of exponentials in $n$
  with certain coefficients (given by $\ol{\chi}$).

  In particular, this extends $\chi \colon \RR \to \CC$.
  For example, this means we could,
  if we wanted to, speak of the Fourier transform
  of $\chi$ (viewed as a function of period $N$).
\end{itemize}

Here is one application of the interpolation formula.
\begin{corollary}
  [Twisted Poisson summation]
  If $f$ is of moderate decay then
  \[ \sum_n \chi(n) f(n) =
    \frac{\tau(\chi)}{N}
    \sum_n \ol \chi (n) \wh f\left( \frac nN \right).  \]
\end{corollary}
\begin{proof}
  Let $F(n) = \chi(n) f(n)$.
  Then by the interpolation formula, we can extend to
  \[ F(x) = \frac{\chi(-1)\tau(\chi)}{N}
    \sum_{r \bmod N} \ol{\chi(r)} f(x) \ee(xr/N) \]
  Now the Fourier transform of $f(x) \ee(xr/N)$
  is given by
  \[ \int_\RR f(x) \ee(x r /N)
    \ee(-\xi x) \; dx
    = \wh f \left( \xi - \frac rN \right) \]
  Thus, by Poisson summation formula,
  \begin{align*}
    \sum_n \chi(n) f(n) &= \frac{\chi(-1)\tau(\chi)}{N}
      \sum_{r \bmod N} \sum_m \ol{\chi}(r)
      \wh f\left( n - \frac r N \right) \\
    &= \frac{\tau(\chi)}{N}
      \sum_m \sum_{r \bmod N} \ol{\chi}(-r)
      \wh f\left( m - \frac r N \right) \\
    &= \frac{\tau(\chi)}{N} \sum_n \ol{\chi}(n)
      \wh f\left( \frac nN \right). \qedhere
  \end{align*}
\end{proof}

\subsection{The $L$-function of a Dirichlet character}
We then attach the $L$-function defined by
\[
  L(s, \chi) = \sum_n \chi(n) n^{-s}
  = \prod_p \left( 1 - \frac{\chi(p)}{p^{-s}} \right)
\]
for $\Re s > 1$.
For example, if $\chi$ is the trivial character (so $N=1$),
then $L(s,\chi) = \zeta$ is the Riemann zeta function.

Any time we have an $L$-function
our goal will to be get an \emph{analytic continuation}
and a \emph{functional equation}.
For Dirichlet characters, both goals will be achieved
by using the following so-called \emph{theta function}.
\begin{definition}
  Let $\eps = \frac{1-\chi(-1)}{2}$ and define
  \[ \theta_\chi(t)
  = \half \sum_n n^\eps \chi(n) \exp(-\pi n^2 t). \]
\end{definition}
Note that $\chi(0) = 0$ unless $N = 1$,
which would cause $\chi(0) = 1$
(and actually $\chi \equiv 1$).
It is clear that $\theta_\chi(t) - \half \chi(0)$
decays rapidly as $t \to \infty$.
We'll now show the following functional equation:
\begin{proposition}
  [Functional equation of the theta function]
  We have
  \[ \theta_\chi(t) = \frac{(-i)^\eps \tau(\chi)}
    {N^{1+\eps} t^{\eps+1/2}}
    \theta_{\ol{\chi}} \left( \frac{1}{N^2 t} \right). \]
\end{proposition}
\begin{proof}
  By cases on whether $\eps = 0$ or $\eps = 1$;
  just apply twisted Poisson summation
  on $\exp(-\pi x^2t)$ and $x\exp(-\pi x^2t)$ respectively.
\end{proof}
When $N > 1$, this means $\theta_\chi(t)$
decays rapidly to $0$ as well.
Let's assume momentarily that $N > 1$,
and see what falls out when we compute the
Mellin transform of $\theta_\chi$.
It equals
\begin{align*}
  \MM\left(\theta_\chi, s\right)
  &= \sum_{n>0} n^\eps \chi(n)
    \MM(\exp(-\pi n^2t), s) \\
  &= \sum_{n>0} n^\eps \chi(n) (\pi n^2)^{-s} \Gamma(s) \\
  &= \Gamma(s) \pi^{-s} \sum_{n>0} n^\eps \chi(n) n^{-2s} \\
  &= \Gamma(s) \pi^{-s} L(2s-\eps, \chi)
\end{align*}
Replacing $s$ with $\half(s+\eps)$ and rearranging:
\[ \MM\left(\theta_\chi, \frac{s+\eps}{2} \right)
  = \pi^{-\half(s+\eps)}
  \Gamma\left( \frac{s+\eps}{2} \right) L(s, \chi).  \]
Since $\Gamma$ never vanishes,
this gives the analytic continuation of $L(s, \chi)$.

When $N = 1$ and $\chi = \mathbf 1$ is the trivial character we get
$\theta_\chi(t) = \frac{1}{\sqrt t} \theta_{\chi}(1/t)$,
and so $\theta_\chi(t)$ behaves like $\frac{1}{2\sqrt t}$ at zero,
and the Mellin transform is defined.
The same calculation gives
\begin{align*}
  \MM\left(\theta_{\mathbf 1}, s\right)
  &= \MM(1/2, 0) + \sum_{n>0} \MM(\exp(-\pi n^2t), s) \\
  &= 0 + \Gamma(s) \pi^{-s} L(2s-\eps, \mathbf 1)
  = \Gamma(s) \pi^{-s} \zeta(2s)
\end{align*}
so
\[ \MM\left(\theta_{\mathbf 1}, s/2 \right)
  = \Gamma(s) \pi^{-s} L(2s-\eps, \mathbf 1)
  = \Gamma(s/2) \pi^{-s/2} \zeta(s) \]
In this case, $\MM\left(\theta_{\mathbf 1}, s/2 \right)$
is meromorphic except for simple poles at $s=0$ and $s=1$
of residue $1$.

The Mellin transforms we mentioned are usually
called the completed $L$-functions as follows:
\begin{theorem}
  [The Mellin transform of the theta function]
  Define
  \[ \Lambda(s, \chi)
    = \MM\left( \theta_\chi, \frac{s+\eps}{2} \right)
    = \pi^{-\half(s+\eps)}
    \Gamma\left( \frac{s+\eps}{2} \right) L(s, \chi).  \]
  Then $\Lambda(s, \chi)$ is analytic if $\chi \ne \mathbf 1$;
  otherwise it is meromorphic,
  with simple poles at $s = 0$ and $s = 1$ of residue $1$.
  Moreover, we have the functional equation
  \[ \Lambda(s, \chi) = \Lambda(1-s, \chi). \]
\end{theorem}

\begin{proof}
The first half of the theorem follows from Mellin transform properties.
To show the functional equation, use Proposition~\ref{prop:usubmellin}
to clean everything up once the theta function is used.
\begin{align*}
  \Lambda(s, \chi) &= \MM\left(\theta_\chi(t),
    \frac{s+\eps}{2} \right)
  = \MM\left(
    \frac{(-i)^{\eps} \tau(\chi)}{N^{1+\eps} t^{\eps+1/2}}
    \theta_{\ol\chi}
    \left( \frac{1}{N^2} t^{-1} \right),
    \frac{s+\eps}{2} \right) \\
  &= \frac{(-i)^{\eps} \tau(\chi)}{N^{1+\eps}}
    \cdot \MM\left( t^{-(\eps+1/2)}\theta_{\ol\chi}
    \left( \frac{1}{N^2} t^{-1} \right),
    \frac{s+\eps}{2} \right) \\
  &= \frac{(-i)^{\eps} \tau(\chi)}{N^{1+\eps}}
    \cdot \frac{\MM\left( \theta_{\ol\chi},
    \frac{\frac{s+\eps}{2} - (\eps + 1/2)}{-1} \right)}
    {\left( N^2 \right)^{\frac{s+\eps}{2} - (\eps+1/2)}}
  = \frac{(-i)^{\eps} \tau(\chi)}{N^{s}}
    \cdot \MM\left( \theta_{\ol\chi},
    \frac{(1-s)+\eps}{2}\right) \\
  &= \frac{(-i)^{\eps} \tau(\chi)}{N^{s}}
    \Lambda(\theta_{\ol\chi}, 1-s). \qedhere
\end{align*}
\end{proof}

\section{Linear algebraic groups}
\subsection{Reductive groups}
\begin{definition}
  A linear algebraic group $G$ over a field $k$ is a closed subscheme
  of $\GL_n(k)$, i.e.\ a smooth affine group scheme over $k$.
\end{definition}

Recall that a generic group $H$ is
\begin{itemize}
  \ii \textbf{solvable} if the map $X \mapsto [X,X]$ starting from $H$
  eventually stabilizes; equivalently, there needs to be a normal series
  \[ 1 = G_0 \trianglelefteq G_1 \trianglelefteq G_2 \trianglelefteq
    \dots \trianglelefteq G_n = G \]
  where each $G_i / G_{i-1}$ is abelian.

  \ii \textbf{unipotent} if every element of the group is unipotent;
  this implies $H$ is a closed subgroup of $U_n$.
\end{itemize}

\begin{definition}
  Given a LAG $G$, we define
  \begin{itemize}
    \ii its \alert{radical} is the largest connected solvable normal subgroup;
    \ii its \alert{unipotent radical}
    is the largest connected unipotent normal subgroup.
  \end{itemize}
  We say $G$ is
  \begin{itemize}
    \ii \alert{semisimple} if the radical is trivial;
    \ii \alert{reductive} if the unipotent radical is trivial.
  \end{itemize}
\end{definition}

Semisimple groups are considered pretty rigid,
but reductive groups are not too much worse;
from \url{https://mathoverflow.net/a/223895/70654}:
\begin{quote}
From the modern perspective the class of (connected) reductive groups is more
natural than that of (connected) semisimple groups for the purposes of setting
up a robust general theory, due to the fact that Levi factors of parabolics in
reductive groups are always reductive but generally are not semisimple when the
ambient group is semisimple. However, after some development of the basic theory
one learns that reductive groups are just a fattening of semisimple groups via a
central torus (e.g., $\GL_n$ versus $\SL_n$), so Harish-Chandra had no trouble to get by
in the semisimple case by just dragging along some central tori here and there
in the middle of proofs.
\end{quote}

Following non-obvious theorem is another motivation for why reductive
groups are considered really nice:
\begin{theorem}
  A smooth connected affine group over a field of characteristic $0$
  is reductive if and only if all of its algebraic representations
  are completely reducible.
\end{theorem}

\subsection{Parabolic and Borel subgroups}
\begin{definition}
  A \alert{Borel subgroup} of $G$ is
  \begin{itemize}
    \ii a connected solvable subgroup variety $B$ for which $G/B$ is complete;
    \ii or equivalently, a maximal Zariski-closed normal solvable subgroup
    (but not necessarily normal).
  \end{itemize}
\end{definition}
The Borel subgroups are all conjugate to each other.

\begin{definition}
  A \alert{parabolic subgroup} $P$ of $G$ is
  \begin{itemize}
    \ii any subgroup containing a Borel subgroup;
    \ii over an algebraically closed field, equivalently,
    such that $G/P$ is a complete variety.
  \end{itemize}
\end{definition}

Example copied from Wikipedia: for $G = \GL_4(\CC)$, a Borel subgroup is
\[ \left\{A={\begin{bmatrix}a_{11}&a_{12}&a_{13}&a_{14}\\0&a_{22}&a_{23}&a_{24}\\0&0&a_{33}&a_{34}\\0&0&0&a_{44}\end{bmatrix}}:\det(A)\neq 0\right\}
\]
and the maximal proper parabolic subgroups of $G$ containing $B$ are:
\[
{\displaystyle \left\{{\begin{bmatrix}a_{11}&a_{12}&a_{13}&a_{14}\\0&a_{22}&a_{23}&a_{24}\\0&a_{32}&a_{33}&a_{34}\\0&a_{42}&a_{43}&a_{44}\end{bmatrix}}\right\},{\text{ }}\left\{{\begin{bmatrix}a_{11}&a_{12}&a_{13}&a_{14}\\a_{21}&a_{22}&a_{23}&a_{24}\\0&0&a_{33}&a_{34}\\0&0&a_{43}&a_{44}\end{bmatrix}}\right\},{\text{ }}\left\{{\begin{bmatrix}a_{11}&a_{12}&a_{13}&a_{14}\\a_{21}&a_{22}&a_{23}&a_{24}\\a_{31}&a_{32}&a_{33}&a_{34}\\0&0&0&a_{44}\end{bmatrix}}\right\}}
\]
Also, a maximal torus in $B$ is
\[ \left\{{\begin{bmatrix}a_{11}&0&0&0\\0&a_{22}&0&0\\0&0&a_{33}&0\\0&0&0&a_{44}\end{bmatrix}}:a_{11}\cdot a_{22}\cdot a_{33}\cdot a_{44}\neq 0\right\}
  \cong \Gm^4
\]

\subsection{Table}
\begin{center}
  \begin{tabular}[h]{cccc}
    $G$ & Radical & Unipotent radical & Borel subgroup \\ \hline
    $\GL(n)$ & $\CC$ (diagonal matrices) & $1$ & Upper tri \\
    $\SL(n)$ & $1$ & $1$ \\
    $\opname{O}(n)$ & & $1$ \\
    $\opname{SO}(n)$ & & $1$ \\
    $\opname{Sp}(n)$ & & $1$ \\
    $\opname{U}(n)$ & & $1$ \\
    $\Ga$ & $\Ga$ & $\Ga$ & \\
  \end{tabular}
\end{center}
