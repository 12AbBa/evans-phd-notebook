\chapter{Orbital integral}
\section{Symmetric space}
Let $F$ be a finite extension of $\QQ_p$ and
let $E/F$ be a quadratic field extension.
Denote by $\varpi$ a uniformizer of $\OO_F$.

We define the symmetric space
\[ S_3(F) \coloneqq \left\{ s \in \GL_3(E) \mid s \bar s = \id \right\}. \]
We also pay particular attention to the subspace which have $\OO_E$ entries:
\[ K_S \coloneqq S_3(F) \cap \GL_3(\OO_E). \]
\begin{lemma}
  [Cartan decomposition]
  For each integer $m \ge 0$ let
  \[ K_{S,m} \coloneqq K_S \cdot \begin{bmatrix} 0 & 0 & \varpi^m \\ 0 & 1 & 0 \\ \varpi^{-m} & 0 & 0 \end{bmatrix}. \]
  Then we have a decomposition
  \[ S_3(F) = \coprod_{m \geq 0} K_{S,m}. \]
\end{lemma}

\section{Orbital integral}
Define
\[
  H' \coloneqq
  \left\{ \begin{bmatrix} t_1 & t_2 \\ \bar t_2 & \bar t_1 \end{bmatrix} \right\}
  \cong \GL_2(F).
\]
We embed $H'$ into $\GL_3(F)$ by
$h' \mapsto \left[ \begin{smallmatrix} h' & 0 \\ 0 &1 \end{smallmatrix} \right]$,
which allows $H$ to act on $\GL_3(FF)$ and hence $S_3(F)$.

Now we can define the orbital integral.
\begin{definition}
  Let $\eta$ be the quadratic character attached to $E/F$ by class field theory,
  and for brevity let $\eta(h') \coloneqq \eta(\det h')$ for $h' \in H'$.
  For $\gamma \in S_3(F)$ and $s \in \CC$, we define the orbital integral by
  \[ O(\gamma, s) \coloneqq
    \int_{g \in H'} 1_{K_{S,m}}(\bar g\inv \gamma g) \eta(g) |\det(g)|^s \odif g \]
  where
  \[ \odif g = \kappa \cdot \frac{\odif t_1 \odif t_2}
    {\left\lvert t_1 \bar t_1 - t_2 \bar t_2 \right\rvert_F^2} \]
  for the constant $\kappa \coloneqq (1-q\inv)\inv(1-q^{-2})\inv$.
\end{definition}

Indeed, for $h' \in H$ and $\gamma \in S_3(F)$ we have $h' \gamma (\bar h')\inv \in S_3(F)$
and so the indicator function is filtering based on which part of the
Cartan decomposition that $h' \gamma (\bar h')\inv$ falls in.

Evidently $O(\gamma, s)$ only depends on the $H'$-orbit of $\gamma$.
So it makes sense to pick a canonical representative for the $H'$-orbit to compute
the orbital integral in terms of.
For so-called \emph{regular} $\gamma$, the representatives
\[ \gamma(a,b,d) =
  \begin{bmatrix}
    a & 0 & 0 \\
    b & - \bar d & 1 \\
    c & 1 - d \bar d & d
  \end{bmatrix}
  \in S_3(F); \quad \text{where $c = -a \bar b + b d$} \]
cover all the \emph{regular} orbits, which are the ones we care about.

For $m=0$, AFL computes $\pdv{}{s}O(\gamma,s)$ at $s=0$.
Our goal is to compute it for $m > 0$ too.
