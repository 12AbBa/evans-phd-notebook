\chapter{Orbital integral}
\section{Background}
\subsection{Symmetric space}
Let $F$ be a finite extension of $\QQ_p$ and
let $E/F$ be a quadratic field extension.
Denote by $\varpi$ a uniformizer of $\OO_F$, such that $\bar \varpi = \varpi$.

We define the symmetric space
\[ S_3(F) \coloneqq \left\{ s \in \GL_3(E) \mid s \bar s = \id \right\}. \]
We also pay particular attention to the subspace which have $\OO_E$ entries:
\[ K_S \coloneqq S_3(F) \cap \GL_3(\OO_E). \]
\begin{lemma}
  [Cartan decomposition]
  For each integer $m \ge 0$ let
  \[ K_{S,m} \coloneqq K_S \cdot \begin{bmatrix} 0 & 0 & \varpi^m \\ 0 & 1 & 0 \\ \varpi^{-m} & 0 & 0 \end{bmatrix}. \]
  Then we have a decomposition
  \[ S_3(F) = \coprod_{m \geq 0} K_{S,m}. \]
\end{lemma}
For $r \geq 0$, dfeine
\[ \Omega_r \coloneqq S_3(F) \cap \varpi^{-m} \GL_3(\OO_E). \]
We can re-parametrize the problem according to the following claim.
\begin{claim}
  \[ \Omega_r = K_{S,0} \sqcup K_{S,1} \sqcup \dots \sqcup K_{S,r}. \]
\end{claim}
If this claim is true (still need to check it),
then an integral over each $\Omega_r$ lets us extract the integrals over $K_{S,m}$.

\subsection{Orbital integral}
Define
\[ H' \coloneqq
  \left\{ \begin{bmatrix} t_1 & t_2 \\ \bar t_2 & \bar t_1 \end{bmatrix} \right\}
  \cong \GL_2(F). \]
We embed $H'$ into $\GL_3(F)$ by
$h' \mapsto \left[ \begin{smallmatrix} h' & 0 \\ 0 &1 \end{smallmatrix} \right]$,
which allows $H$ to act on $\GL_3(FF)$ and hence $S_3(F)$.

Now we can define the orbital integral.
\begin{definition}
  Let $\eta$ be the quadratic character attached to $E/F$ by class field theory,
  and for brevity let $\eta(h') \coloneqq \eta(\det h')$ for $h' \in H'$.
  For $\gamma \in S_3(F)$ and $s \in \CC$, we define the orbital integral by
  \[ O(\gamma, s) \coloneqq
    \int_{g \in H'} \mathbf{1}_{\Omega_r}(\bar g\inv \gamma g) \eta(g)
    \left\lvert \det(g) \right\rvert_F^{-s} \odif g \]
  where
  \[ \odif g = \kappa \cdot \frac{\odif t_1 \odif t_2}
    {\left\lvert t_1 \bar t_1 - t_2 \bar t_2 \right\rvert_F^2} \]
  for the constant $\kappa \coloneqq (1-q\inv)\inv(1-q^{-2})\inv$.
\end{definition}

Indeed, for $h' \in H$ and $\gamma \in S_3(F)$ we have $h' \gamma (\bar h')\inv \in S_3(F)$
and so the indicator function is filtering based on which part of the
Cartan decomposition that $h' \gamma (\bar h')\inv$ falls in.

Evidently $O(\gamma, s)$ only depends on the $H'$-orbit of $\gamma$.
So it makes sense to pick a canonical representative for the $H'$-orbit to compute
the orbital integral in terms of.
For so-called \emph{regular} $\gamma$, the representatives
\[ \gamma(a,b,d) =
  \begin{bmatrix}
    a & 0 & 0 \\
    b & - \bar d & 1 \\
    c & 1 - d \bar d & d
  \end{bmatrix}
  \in S_3(F); \quad \text{where $c = -a \bar b + b d$} \]
cover all the \emph{regular} orbits, which are the ones we care about.

For $r=0$, \cite{ref:AFL} computes $\pdv{}{s}O(\gamma,s)$ at $s=0$ in terms of $a$, $b$, $d$.
Our goal is to compute it for $r > 0$ too.

\section{Reparametrization in terms of valuations}
\subsection{Computation of value in indicator function}
We are integrating over $t_1 \in E$ and $t_2 \in E$.
Regarding $g \in H'$ as an element of $\GL_3$ as described before, we have
\[ g = \begin{bmatrix}
  t_1 & t_2 & 0  \\
  \bar t_2 & \bar t_1 & 0 \\
  0 & 0 & 1
  \end{bmatrix}. \]
We therefore have
\[ \bar g \inv = \begin{bmatrix}
  \frac{t_1}{t_1 \bar t_1 - t_2 \bar t_2} & \frac{-\bar t_2}{t_1 \bar t_1 - t_2 \bar t_2} & 0 \\
  \frac{-t_2}{t_1 \bar t_1 - t_2 \bar t_2} & \frac{\bar t_1}{t_1 \bar t_1 - t_2 \bar t_2} & 0 \\
  0 & 0 & 1 \end{bmatrix}. \]
Hence
\begin{align*}
  \bar g \inv \gamma g
  &=
  \begin{bmatrix}
  \frac{t_1}{t_1 \bar t_1 - t_2 \bar t_2} & \frac{-\bar t_2}{t_1 \bar t_1 - t_2 \bar t_2} & 0 \\
  \frac{-t_2}{t_1 \bar t_1 - t_2 \bar t_2} & \frac{\bar t_1}{t_1 \bar t_1 - t_2 \bar t_2} & 0 \\
  0 & 0 & 1 \end{bmatrix}
  \begin{bmatrix}
    a & 0 & 0 \\
    b & - \bar d & 1 \\
    c & 1 - d \bar d & d
  \end{bmatrix}
  \begin{bmatrix}
  t_1 & t_2 & 0  \\
  \bar t_2 & \bar t_1 & 0 \\
  0 & 0 & 1
  \end{bmatrix} \\
  &=
  \begin{bmatrix}
  \frac{t_1}{t_1 \bar t_1 - t_2 \bar t_2} & \frac{-\bar t_2}{t_1 \bar t_1 - t_2 \bar t_2} & 0 \\
  \frac{-t_2}{t_1 \bar t_1 - t_2 \bar t_2} & \frac{\bar t_1}{t_1 \bar t_1 - t_2 \bar t_2} & 0 \\
  0 & 0 & 1 \end{bmatrix}
  \begin{bmatrix}
    at_1 & at_2 & 0 \\
    bt_1 - \bar d \bar t_2 & b t_2 - \bar d \bar t_1 & 1 \\
    ct_1 + (1-d\bar d)\bar t_2 & ct_2 + (1-d \bar d) \bar t_1 & d
  \end{bmatrix}
  \\
  &=
  \begin{bmatrix}
    \dfrac{at_1^2 - bt_1 \bar t_2 + d \bar t_2^2}{t_1 \bar t_1 - t_2 \bar t_2}
    & \dfrac{at_1t_2 - bt_2 \bar t_2 + \bar d \bar t_1 \bar t_2}{t_1 \bar t_1 - t_2 \bar t_2}
    & \dfrac{-\bar t_2}{t_1 \bar t_1 - t_2 \bar t_2} \\[2ex]
    \dfrac{-at_1t_2+bt_1\bar t_1-\bar d \bar t_1 \bar t_2}{t_1 \bar t_1 - t_2 \bar t_2}
    & \dfrac{-at_2^2+b\bar t_1 t_2-d\bar t_1^2}{t_1 \bar t_1 - t_2 \bar t_2}
    & \dfrac{\bar t_1}{t_1 \bar t_1 - t_2 \bar t_2} \\[2ex]
    ct_1 + (1-d\bar d)\bar t_2 & ct_2 + (1-d \bar d) \bar t_1 & d
  \end{bmatrix}
\end{align*}
Let us define \[ t = t_2 \bar t_1 \inv \iff t_2 = t \bar t_1. \]
This lets us rewrite everything in terms of the ratio $t$ and $t_1 \in E$:
\[
  \bar g \inv \gamma g
  =
  \begin{bmatrix}
    \dfrac{t_1^2(a-b\bar t+\bar d \bar t^2)}{t_1 \bar t_1(1-t \bar t)}
    & \dfrac{t_1 \bar t_1(at-bt\bar t+\bar d \bar t)}{t_1 \bar t_1(1-t \bar t)}
    & \dfrac{t_1 \cdot (-\bar t)}{t_1 \bar t_1 (1-t \bar t)} \\[2ex]
    \dfrac{t_1\bar t_1(-at+b-\bar d \bar t)}{t_1 \bar t_1(1-t \bar t)}
    & \dfrac{\bar t_1^2(-at^2+bt-\bar d)}{t_1 \bar t_1(1-t \bar t)}
    & \dfrac{-\bar t_1}{t_1 \bar t_1(1-t \bar t)} \\[2ex]
    t_1(c + (1-d\bar d)\bar t) & \bar t_1(ct + (1-d \bar d)) & d
  \end{bmatrix}
\]
This new parametrization is better because $t_1$ only plays the role of
a scale factor on the outside, with ``interesting'' terms only involving $t$.
To make this further explicit, we write
\[ t_1 = \varpi^m \eps \]
for $m \in \ZZ$ and $\eps \in \OO_E^\times$.
Then we actually have
\[
  \begin{bmatrix} \bar\eps \\ & \eps \\ & & 1 \end{bmatrix}
  \bar g \inv \gamma g
  \begin{bmatrix} \eps\inv \\ & \bar\eps\inv \\ & & 1 \end{bmatrix}
  =
  \begin{bmatrix}
    \dfrac{a-b\bar t+\bar d \bar t^2}{1-t \bar t}
    & \dfrac{at-bt\bar t+\bar d \bar t}{1-t \bar t}
    & \dfrac{-\varpi^m \bar t}{1-t \bar t} \\[2ex]
    \dfrac{-at+b-\bar d \bar t}{1-t \bar t}
    & \dfrac{-at^2+bt-\bar d}{1-t \bar t}
    & \dfrac{-\varpi^m}{1-t \bar t} \\[2ex]
    \dfrac{c + (1-d\bar d)\bar t}{\varpi^m} & \dfrac{ct + (1-d \bar d)}{\varpi^m} & d
  \end{bmatrix}
\]
For brevity, we will let $\Gamma(\gamma, t, m)$ denote the right-hand matrix.
The conjugation by
$\left[ \begin{smallmatrix} \eps\inv \\ & \bar\eps\inv \\ & & 1 \end{smallmatrix} \right]$
has no effect on any of the $\Omega_r$,
so we may use the above expression $\mathbf{1}_{\Omega_r}(\Gamma(\gamma, t, m))$
in the work that follows.
By abuse of notation, we abbreviate
\[ \mathbf{1}_r(\gamma, t, m) \coloneqq \mathbf{1}_{\Omega_r}(\Gamma(\gamma, t, m)). \]

\subsection{Reparametrizing the integral in terms of $t$ and $m$}
From now on, we always fix the notation
\begin{align*}
  m &= m(t_1) \coloneqq -v(t_1) \\
  n &= n(t) \coloneqq v(1-t\bar t).
\end{align*}
We need to rewrite the integral, phrased originally via $\odif g$,
in terms of the parameters $t$ (hence $n$), $m$, and $\gamma$.
We start by observing that
\[ \det g = t_1 \bar t_1 - t_2 \bar t_2 = t_1 \bar t_1 (1 - t\bar t) \]
which means that
\[ v(\det g) = -2m + n \]
ergo
\begin{align*}
  \left\lvert \det g \right\rvert_F &= q^{-v(\det g)} = q^{2m-n} \\
  \eta(g) &= (-1)^{v(\det g)} = (-1)^n.
\end{align*}
Meanwhile, from $t_2 = t \bar t_1$ we derive
\[ \odif t_2 = \left\lvert t_1 \right\rvert_E \odif t = q^{2m} \odif t. \]

Bringing this all into the orbital integral gives
\begin{align*}
  O(\gamma, s) &= \kappa \int_{t, t_1 \in E} \mathbf{1}_r(\gamma, t, m)
  (-1)^n \left( q^{2m-n} \right)^{s-2} \odif t_1 \cdot (q^{2m} \odif t) \\
  &= \kappa \int_{t, t_1 \in E} \mathbf{1}_r(\gamma, t, m)
  (-1)^n q^{s(2m-n)} \cdot q^{2n-2m} \odif t \odif t_1.
\end{align*}

\section{Simplifying assumptions}
For the purposes of \cite{ref:AFL},
we will only care about when
\[ v\left( (1-d\bar d)^2 - c \bar c\right) \equiv 1 \pmod 2 \]
and assume so.
This means in particular that
\[ v(1 - d \bar d) = v(c) \neq -\infty \]
are nonzero with the same valuation.
So following \cite{ref:AFL} we will define
\[ u \coloneqq \frac{\bar c}{1-d \bar d} \in \OO_E^\times \]
so that
\begin{equation}
  b = -au - \bar{d} \bar{u}
  \label{eqn:b}
\end{equation}

In what follows, we are going to always assume $v(d) \geq -r$,
because if this $v(d) < -r$ then the integral will always vanish
(because the bottom-right entry of $\Gamma(\gamma, t, m)$ is no-good).
Because of this, it then also follows that $v(b) \geq -r$ from \eqref{eqn:b}.

\section{Evaluation of the integral}
\subsection{Easy cases}
\begin{claim}
  Whenever $n = 0$ (this requires $v(t) \geq 0$),
  \[
    \mathbf{1}(\gamma, t, m) =
    \begin{cases}
      1 & \text{if } -r \le m \le v(1-d\bar d)+r \\
      0 & \text{otherwise.}
    \end{cases}
  \]
\end{claim}
\begin{proof}
  We have to consider the nine entries of $\Gamma(\gamma, t, m)$ in tandem.

  The upper $2 \times 2$ matrix is always in $\omega^{-r}\OO_E$,
  because $v(t) \geq 0$, $v(d) \geq -r$, $v(b) \geq -r$, and $v(a) = 0$ suffices.

  In the right column, since $v(t) \geq 0$ and $n = 0$, the condition is simply $m \ge -r$.

  In the bottom row, we need
  $v\left( c+(1-d\bar d) \bar t \right)-m \geq -r$
  and $v\left( ct +(1-d\bar d) \right)-m \geq -r$.
  Since $v(c) = v(1-d \bar d)$,
  if $v(t) > 0$ this is equivalent to $m-r \leq v(1-d \bar d)$.
  In the case where $v(t) = 0$ we instead use the observation that
  \[ \bar t \left[ ct + (1-d \bar d) \right]
    - \left[ c + (1-d \bar d) \bar t \right] = (1-t\bar t) c \]
  which forces at least one of
  $ct + (1-d \bar d)$
  and $c + (1-d \bar d) \bar t$ to have the valuation equal to $v(c) = v(1-d \bar d)$.
  So the claim follows now.
\end{proof}

\begin{claim}
  Suppose $n = -2j < 0$, equivalently, $v(t) = -j < 0$, for some $j$.
  \[
    \mathbf{1}(\gamma, t, m) =
    \begin{cases}
      1 & \text{if } -r \le m+j \le v(1-d\bar d)+r \\
      0 & \text{otherwise.}
    \end{cases}
  \]
\end{claim}
\begin{proof}
  The proof is similar to the previous claim, but simpler.

  Since $j > 0$, the fraction $\frac{t^2}{1-t \bar t}$ has positive valuation,
  so the upper $2 \times 2$ of $\Gamma(\gamma, t, m)$ is always in $\varpi^{-r}\OO_E$.
  Turning to the right column, the condition reads exactly $m+j \geq -r$.
  Finally, in the bottom row, from $v(t) > 0$ and $v(c) = v(1-d \bar d)$
  the condition is simply $-j+v(1-d \bar d)-m \geq -r$.
\end{proof}
