\chapter{Adelization}
\todo{write}
(how to make a modular form into an automorphic form)
\section{Derivation of modular forms via representation theory}
TODO: I have no memory of writing this section or what it's about

Let $f$ be a classical cusp form of weight $k$.
Then we can associate it to a function $\phi$ on $\Gamma \backslash \SL(2, \RR)$ by
\[ \phi_f(g) = f(g(i)) (c \cdot g(i) + d)^{-k}. \]
This is a bijection between modular forms and functions
satisfying certain conditions.

Similarly, given a Maass form $\phi$ we can define $\phi_f(i) = f(g(i))$.
Again so-and-so bijection.

On the other hand the center of the universal enveloping algebra
of $\mathfrak{sl}(2,\CC)$ is given by
\[ \mathcal Z = \CC[\id, \Delta] \]
where $\Delta$ is the Laplacian (Casimir element) mentioned earlier.
Since $\Delta$ is in the center, it acts as a scalar in every irreducible representation.

% use this to motivate Maass forms as well i guess
% corresponds to chapter 2 of gelbert
